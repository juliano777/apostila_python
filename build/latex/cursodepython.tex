%% Generated by Sphinx.
\def\sphinxdocclass{report}
\documentclass[letterpaper,10pt,brazil]{sphinxmanual}
\ifdefined\pdfpxdimen
   \let\sphinxpxdimen\pdfpxdimen\else\newdimen\sphinxpxdimen
\fi \sphinxpxdimen=.75bp\relax

\PassOptionsToPackage{warn}{textcomp}
\usepackage[utf8]{inputenc}
\ifdefined\DeclareUnicodeCharacter
% support both utf8 and utf8x syntaxes
  \ifdefined\DeclareUnicodeCharacterAsOptional
    \def\sphinxDUC#1{\DeclareUnicodeCharacter{"#1}}
  \else
    \let\sphinxDUC\DeclareUnicodeCharacter
  \fi
  \sphinxDUC{00A0}{\nobreakspace}
  \sphinxDUC{2500}{\sphinxunichar{2500}}
  \sphinxDUC{2502}{\sphinxunichar{2502}}
  \sphinxDUC{2514}{\sphinxunichar{2514}}
  \sphinxDUC{251C}{\sphinxunichar{251C}}
  \sphinxDUC{2572}{\textbackslash}
\fi
\usepackage{cmap}
\usepackage[T1]{fontenc}
\usepackage{amsmath,amssymb,amstext}
\usepackage{babel}



\usepackage{times}
\expandafter\ifx\csname T@LGR\endcsname\relax
\else
% LGR was declared as font encoding
  \substitutefont{LGR}{\rmdefault}{cmr}
  \substitutefont{LGR}{\sfdefault}{cmss}
  \substitutefont{LGR}{\ttdefault}{cmtt}
\fi
\expandafter\ifx\csname T@X2\endcsname\relax
  \expandafter\ifx\csname T@T2A\endcsname\relax
  \else
  % T2A was declared as font encoding
    \substitutefont{T2A}{\rmdefault}{cmr}
    \substitutefont{T2A}{\sfdefault}{cmss}
    \substitutefont{T2A}{\ttdefault}{cmtt}
  \fi
\else
% X2 was declared as font encoding
  \substitutefont{X2}{\rmdefault}{cmr}
  \substitutefont{X2}{\sfdefault}{cmss}
  \substitutefont{X2}{\ttdefault}{cmtt}
\fi


\usepackage[Sonny]{fncychap}
\ChNameVar{\Large\normalfont\sffamily}
\ChTitleVar{\Large\normalfont\sffamily}
\usepackage{sphinx}

\fvset{fontsize=\small}
\usepackage{geometry}


% Include hyperref last.
\usepackage{hyperref}
% Fix anchor placement for figures with captions.
\usepackage{hypcap}% it must be loaded after hyperref.
% Set up styles of URL: it should be placed after hyperref.
\urlstyle{same}
\addto\captionsbrazil{\renewcommand{\contentsname}{Contents:}}

\usepackage{sphinxmessages}
\setcounter{tocdepth}{3}
\setcounter{secnumdepth}{3}


\title{Curso de Python}
\date{12 mar. 2020}
\release{2019-08-28}
\author{Juliano Atanazio}
\newcommand{\sphinxlogo}{\vbox{}}
\renewcommand{\releasename}{Release}
\makeindex
\begin{document}

\ifdefined\shorthandoff
  \ifnum\catcode`\=\string=\active\shorthandoff{=}\fi
  \ifnum\catcode`\"=\active\shorthandoff{"}\fi
\fi

\pagestyle{empty}
\sphinxmaketitle
\pagestyle{plain}
\sphinxtableofcontents
\pagestyle{normal}
\phantomsection\label{\detokenize{index::doc}}



\chapter{Sobre Python}
\label{\detokenize{content/about:sobre-python}}\label{\detokenize{content/about::doc}}

\section{O que é Python}
\label{\detokenize{content/about:o-que-e-python}}\begin{quote}

Python é uma linguagem de programação criada pelo holandês Guido van Rossum no começo dos anos 90 na Stichting Mathematisch Centrum (\sphinxurl{http://www.cwi.nl/}), na Holanda, com o objetivo de ser uma sucessora de uma linguagem chamada ABC.
O nome linguagem foi inspirado na série humorística Monty Python’s Flying Circus, do grupo humorístico britânico Monty Python.
É muito atrativa para desenvolvimento ágil de aplicações.

Site oficial: www.python.org
Site da comunidade brasileira: www.python.org.br
\end{quote}


\subsection{Características}
\label{\detokenize{content/about:caracteristicas}}\begin{itemize}
\item {} 
Linguagem de altíssimo nível;

\item {} 
Fácil de aprender;

\item {} 
Simples;

\item {} 
Objetiva;

\item {} 
Preza pela legibilidade;

\item {} 
Case sensitive;

\item {} 
Mutiparadigma: orientada a objeto e procedural;

\item {} 
Interpretada;

\item {} 
Compilada (transparente ao usuário);

\item {} 
Tipagem forte;

\item {} 
Tipagem dinâmica;

\item {} 
Suporte a módulos e packages encorajando modularidade e reúso de código;

\item {} 
Multiplataforma (Linux, MacOS, BSDs, Windows e outros);

\item {} 
Não suporte sobrecarga de funções;

\item {} 
Blocos delimitados por endentação (ou espaços ou tabs: não misturar!);

\item {} 
Um comando por linha (podem ser colocados mais que um após um “;”, mas não é uma saída elegante);

\item {} 
Possui modo interativo;

\item {} 
Software Livre.

\end{itemize}


\subsection{Licença}
\label{\detokenize{content/about:licenca}}\begin{quote}

Python utiliza uma licença similar à licença BSD, a PSF License (Python Software Foundation License).
É um tipo de licença muito permissiva e compatível com a licença GPL.
É tão permissiva que permite ao usuário escolher em manter ou não o código\sphinxhyphen{}fonte aberto.
Mesmo empresas que pegam o código\sphinxhyphen{}fonte de um projeto sob essa licença, quando o fecham e derivam então gerando um software proprietário, acabam colaborando com o projeto original firmando assim uma parceria em que ambos saem ganhando.
É o mesmo tipo de licença adotado por outros projetos de software livre como: PostgreSQL, FreeBSD, OpenBSD, NetBSD.
\end{quote}


\section{O Interpretador Python}
\label{\detokenize{content/about:o-interpretador-python}}\begin{quote}

Como o próprio nome diz, é o interpretador que faz a análise sintática e executa as instruções Python.
\end{quote}


\subsection{Modo Interativo do Interpretador Python}
\label{\detokenize{content/about:modo-interativo-do-interpretador-python}}\begin{quote}

O modo interativo de Python é um recurso interessante que facilita o trabalho do desenvolvedor de forma que se possa testar algo que esteja fazendo assim que o comando é finalizado.
É um recurso muito útil para debugging, quick hacking e testes.
\end{quote}


\subsection{Invocando o Modo Interativo do Interpretador}
\label{\detokenize{content/about:invocando-o-modo-interativo-do-interpretador}}\begin{quote}

A maioria das distribuições Linux tem já o Python instalado, de forma que para ter acesso ao prompt interativo bastar digitar no terminal o comando python.
Para sair do interpretador basta digitar o caractere de fim de arquivo (EOF \sphinxhyphen{} End\sphinxhyphen{}Of\sphinxhyphen{}File) (\textless{}Ctrl + D\textgreater{} em Unix ou \textless{}Ctrl + Z\textgreater{} no Windows). Também pode\sphinxhyphen{}se usar as funções exit() ou quit().
\end{quote}


\section{A Filosofia de Python}
\label{\detokenize{content/about:a-filosofia-de-python}}\begin{quote}

Há um easter egg em Python bem divertido que exprime bem a filosofia de Python, que é conhecido como The Zen Of Python.
The Zen Of Python é a PEP 20, PEP que significa Python Enhancement Proposals (Propostas de Melhoria para Python).
Para visualizar esse easter egg basta que no shell interativo dê o seguinte comando:
\end{quote}

\begin{sphinxVerbatim}[commandchars=\\\{\}]
\PYG{k+kn}{import} \PYG{n+nn}{this}
\end{sphinxVerbatim}

\begin{sphinxVerbatim}[commandchars=\\\{\}]
\PYG{g+go}{The Zen of Python, by Tim Peters}

\PYG{g+go}{Beautiful is better than ugly.}
\PYG{g+go}{Explicit is better than implicit.}
\PYG{g+go}{Simple is better than complex.}
\PYG{g+go}{Complex is better than complicated.}
\PYG{g+go}{Flat is better than nested.}
\PYG{g+go}{Sparse is better than dense.}
\PYG{g+go}{Readability counts.}
\PYG{g+go}{Special cases aren\PYGZsq{}t special enough to break the rules.}
\PYG{g+go}{Although practicality beats purity.}
\PYG{g+go}{Errors should never pass silently.}
\PYG{g+go}{Unless explicitly silenced.}
\PYG{g+go}{In the face of ambiguity, refuse the temptation to guess.}
\PYG{g+go}{There should be one\PYGZhy{}\PYGZhy{} and preferably only one \PYGZhy{}\PYGZhy{}obvious way to do it.}
\PYG{g+go}{Although that way may not be obvious at first unless you\PYGZsq{}re Dutch.}
\PYG{g+go}{Now is better than never.}
\PYG{g+go}{Although never is often better than *right* now.}
\PYG{g+go}{If the implementation is hard to explain, it\PYGZsq{}s a bad idea.}
\PYG{g+go}{If the implementation is easy to explain, it may be a good idea.}
\PYG{g+go}{Namespaces are one honking great idea \PYGZhy{}\PYGZhy{} let\PYGZsq{}s do more of those!}
\end{sphinxVerbatim}

Tradução:

\begin{sphinxVerbatim}[commandchars=\\\{\}]
\PYG{g+go}{O Zen de Python, por Tim Peters}

\PYG{g+go}{Bonito é melhor do que feio.}
\PYG{g+go}{Explícito é melhor do que implícito.}
\PYG{g+go}{Simples é melhor do que complexo.}
\PYG{g+go}{Complexo é melhor do que complicado.}
\PYG{g+go}{Plano é melhor do que aninhado.}
\PYG{g+go}{Disperso é melhor do que denso.}
\PYG{g+go}{Legibilidade conta.}
\PYG{g+go}{Casos especiais não são especiais o bastante para quebrar as regras.}
\PYG{g+go}{Embora a praticidade vença a pureza.}
\PYG{g+go}{Erros não devem passar silenciosamente.}
\PYG{g+go}{A não ser que sejam explicitamente silenciados.}
\PYG{g+go}{Diante a ambigüidade, recuse a tentação de adivinhar.}
\PYG{g+go}{Deve haver um\PYGZhy{}\PYGZhy{} e preferencialmente apenas um \PYGZhy{}\PYGZhy{}modo óbvio de fazer isso.}
\PYG{g+go}{Embora a maneira não seja óbvia à primeira vista, a menos que seja holandês.}
\PYG{g+go}{Agora é melhor do que nunca.}
\PYG{g+go}{Embora nunca é muitas vezes melhor do que *agora* mesmo}
\PYG{g+go}{Se a implementação é difícil de explicar, é uma má idéia.}
\PYG{g+go}{Se a implementação é fácil de explicar, deve ser uma boa idéia.}
\PYG{g+go}{Namespaces são uma idéia fantástica \textendash{} vamos fazer mais desses!}
\end{sphinxVerbatim}


\chapter{Particularidades da Linguagem}
\label{\detokenize{content/language_particularities:particularidades-da-linguagem}}\label{\detokenize{content/language_particularities::doc}}

\section{Palavras Reservadas}
\label{\detokenize{content/language_particularities:palavras-reservadas}}
Palavras reservadas ou em inglês \sphinxstyleemphasis{keywords} são

\begin{sphinxVerbatim}[commandchars=\\\{\}]
False   class     finally  is        return
None    continue  for      lambda    try
True    def       from     nonlocal  while
and     del       global   not       with
as      elif      if       or        yield
assert  else      import   pass
break   except    in       raise
\end{sphinxVerbatim}

Podemos verificar as palavras reservdas de Python usando código:

\begin{sphinxVerbatim}[commandchars=\\\{\}]
\PYG{k+kn}{from} \PYG{n+nn}{keyword} \PYG{k+kn}{import} \PYG{n}{kwlist} \PYG{k}{as} \PYG{n}{keywords}

\PYG{k}{for} \PYG{n}{i} \PYG{o+ow}{in} \PYG{n}{keywords}\PYG{p}{:}
    \PYG{n+nb}{print}\PYG{p}{(}\PYG{n}{i}\PYG{p}{)}
\end{sphinxVerbatim}

\begin{sphinxVerbatim}[commandchars=\\\{\}]
\PYG{g+go}{. . .}
\end{sphinxVerbatim}


\section{Definição do Interpretador de Comandos}
\label{\detokenize{content/language_particularities:definicao-do-interpretador-de-comandos}}\begin{quote}

Para scripts Python, na primeira linha podemos especificar o interpretador de comandos a ser utilizado, também conhecido como shebang;
\end{quote}

\begin{sphinxVerbatim}[commandchars=\\\{\},numbers=left,firstnumber=1,stepnumber=1]
\PYG{c+ch}{\PYGZsh{}!/usr/bin/env python3}
\end{sphinxVerbatim}

\begin{sphinxadmonition}{note}{Nota:}
Em ambientes Windows não é preciso espeficar, isso é pertinente a ambientes Unix.
\end{sphinxadmonition}


\subsection{Codificação (Conjunto de Caracteres \sphinxhyphen{} Encoding)}
\label{\detokenize{content/language_particularities:codificacao-conjunto-de-caracteres-encoding}}\begin{quote}

É possível utilizar codificações diferentes de ASCII em arquivos de código\sphinxhyphen{}fonte.
A melhor maneira de se fazer isso é colocar uma linha especial no começo do arquivo, se foi especificado o interpretador, deve vir logo depois dele:
\end{quote}

\begin{sphinxVerbatim}[commandchars=\\\{\},numbers=left,firstnumber=1,stepnumber=1]
\PYG{c+c1}{\PYGZsh{} \PYGZus{}*\PYGZus{} encoding: utf\PYGZhy{}8 \PYGZus{}*\PYGZus{}}
\end{sphinxVerbatim}

\begin{sphinxVerbatim}[commandchars=\\\{\}]
vim /tmp/hello.py
\end{sphinxVerbatim}

\begin{sphinxVerbatim}[commandchars=\\\{\},numbers=left,firstnumber=1,stepnumber=1]
\PYG{c+ch}{\PYGZsh{}!/usr/bin/env python}

\PYG{n+nb}{print}\PYG{p}{(}\PYG{l+s+s1}{\PYGZsq{}}\PYG{l+s+s1}{Olá!}\PYG{l+s+s1}{\PYGZsq{}}\PYG{p}{)}
\end{sphinxVerbatim}

\begin{sphinxVerbatim}[commandchars=\\\{\}]
\PYGZdl{} python /tmp/hello.py
\end{sphinxVerbatim}

\begin{sphinxVerbatim}[commandchars=\\\{\}]
\PYG{g+go}{File \PYGZdq{}/tmp/hello.py\PYGZdq{}, line 4}
\PYG{g+go}{SyntaxError: Non\PYGZhy{}ASCII character \PYGZsq{}\PYGZbs{}xc3\PYGZsq{} in file /tmp/hello.py on line 4, but no encoding declared; see http://www.python.org/peps/pep\PYGZhy{}0263.html for details}
\end{sphinxVerbatim}

Editar o script Python:

\begin{sphinxVerbatim}[commandchars=\\\{\},numbers=left,firstnumber=1,stepnumber=1]
\PYG{c+ch}{\PYGZsh{}!/usr/bin/env python}
\PYG{c+c1}{\PYGZsh{}\PYGZus{}*\PYGZus{} coding: utf\PYGZhy{}8 \PYGZus{}*\PYGZus{}}

\PYG{n+nb}{print}\PYG{p}{(}\PYG{l+s+s1}{\PYGZsq{}}\PYG{l+s+s1}{Olá!}\PYG{l+s+s1}{\PYGZsq{}}\PYG{p}{)}
\end{sphinxVerbatim}

Executar o script:

\begin{sphinxVerbatim}[commandchars=\\\{\}]
python /tmp/hello.py
\end{sphinxVerbatim}

\begin{sphinxVerbatim}[commandchars=\\\{\}]
\PYG{g+go}{Olá!}
\end{sphinxVerbatim}


\section{Case Sensitive}
\label{\detokenize{content/language_particularities:case-sensitive}}\begin{quote}

Python é case sensitive, ou seja, letras maiúsculas e minúsculas são interpretadas de formas diferentes.
\end{quote}

\begin{sphinxVerbatim}[commandchars=\\\{\}]
\PYG{n}{foo} \PYG{o}{=} \PYG{l+s+s1}{\PYGZsq{}}\PYG{l+s+s1}{bar}\PYG{l+s+s1}{\PYGZsq{}}
\PYG{n}{Foo} \PYG{o}{=} \PYG{l+s+s1}{\PYGZsq{}}\PYG{l+s+s1}{foo}\PYG{l+s+s1}{\PYGZsq{}}
\PYG{n+nb}{print}\PYG{p}{(}\PYG{n}{foo}\PYG{p}{)}
\end{sphinxVerbatim}

\begin{sphinxVerbatim}[commandchars=\\\{\}]
\PYG{g+go}{bar}
\end{sphinxVerbatim}

\begin{sphinxVerbatim}[commandchars=\\\{\}]
\PYG{n+nb}{print}\PYG{p}{(}\PYG{n}{Foo}\PYG{p}{)}
\end{sphinxVerbatim}

\begin{sphinxVerbatim}[commandchars=\\\{\}]
\PYG{g+go}{foo}
\end{sphinxVerbatim}


\section{Não Suporta Sobrecarga de Funções / Métodos}
\label{\detokenize{content/language_particularities:nao-suporta-sobrecarga-de-funcoes-metodos}}\begin{quote}

Àqueles que vêm de Java deve estranhar dentre outras coisas o fato de Python não suportar sobrecarga de funções e métodos.
Quando uma mesma função é escrita duas ou mais vezes, o que prevalece é a última definição.
\end{quote}

\begin{sphinxVerbatim}[commandchars=\\\{\}]
\PYG{c+c1}{\PYGZsh{} Definição da função sem parâmetro}
\PYG{k}{def} \PYG{n+nf}{hello\PYGZus{}world}\PYG{p}{(}\PYG{p}{)}\PYG{p}{:}
    \PYG{n+nb}{print}\PYG{p}{(}\PYG{l+s+s1}{\PYGZsq{}}\PYG{l+s+s1}{Hello World}\PYG{l+s+s1}{\PYGZsq{}}\PYG{p}{)}

\PYG{c+c1}{\PYGZsh{} (Re)Definição da função com um parâmetro}
\PYG{k}{def} \PYG{n+nf}{hello\PYGZus{}world}\PYG{p}{(}\PYG{n}{string}\PYG{p}{)}\PYG{p}{:}
    \PYG{n+nb}{print}\PYG{p}{(}\PYG{n}{string}\PYG{p}{)}

\PYG{c+c1}{\PYGZsh{} Tentativa de execução sem parâmetro}
\PYG{n}{hello\PYGZus{}world}\PYG{p}{(}\PYG{p}{)}
\end{sphinxVerbatim}

\begin{sphinxVerbatim}[commandchars=\\\{\}]
\PYG{g+go}{Traceback (most recent call last):}
\PYG{g+go}{  File \PYGZdq{}\PYGZlt{}input\PYGZgt{}\PYGZdq{}, line 1, in \PYGZlt{}module\PYGZgt{}}
\PYG{g+go}{    hello\PYGZus{}world()}
\PYG{g+go}{TypeError: hello\PYGZus{}world() missing 1 required positional argument: \PYGZsq{}string\PYGZsq{}}
\end{sphinxVerbatim}

\begin{sphinxVerbatim}[commandchars=\\\{\}]
\PYG{n}{hello\PYGZus{}world}\PYG{p}{(}\PYG{l+s+s1}{\PYGZsq{}}\PYG{l+s+s1}{foo}\PYG{l+s+s1}{\PYGZsq{}}\PYG{p}{)}
\end{sphinxVerbatim}

\begin{sphinxVerbatim}[commandchars=\\\{\}]
\PYG{g+go}{foo}
\end{sphinxVerbatim}


\section{Orientada a Objetos}
\label{\detokenize{content/language_particularities:orientada-a-objetos}}\begin{quote}

Em Python tudo é objeto.
Ainda fazendo comparação com o mundo Java, em Python não existem tipos primitivos.
Até mesmo um número inteiro é uma instância de int e tem seus atributos e métodos.
\end{quote}

\begin{sphinxVerbatim}[commandchars=\\\{\}]
\PYG{g+go}{\PYGZsq{}0x2d\PYGZsq{}}
\end{sphinxVerbatim}

\begin{sphinxVerbatim}[commandchars=\\\{\}]
\PYG{n}{x}\PYG{o}{.}\PYG{n}{real}
\end{sphinxVerbatim}

\begin{sphinxVerbatim}[commandchars=\\\{\}]
\PYG{g+go}{45}
\end{sphinxVerbatim}

\begin{sphinxVerbatim}[commandchars=\\\{\}]
\PYG{n}{x}\PYG{o}{.}\PYG{n}{imag}
\end{sphinxVerbatim}

\begin{sphinxVerbatim}[commandchars=\\\{\}]
\PYG{g+go}{0}
\end{sphinxVerbatim}

A criação de classes em Python é extremamente simples, sendo que uma classe primária a herança é
feita da classe object e cada classe pode herdar mais de uma. Ou seja, também é aceita herança múltipla.

\begin{sphinxVerbatim}[commandchars=\\\{\}]
\PYG{k}{class} \PYG{n+nc}{Carro}\PYG{p}{(}\PYG{n+nb}{object}\PYG{p}{)}\PYG{p}{:}
    \PYG{n}{marca} \PYG{o}{=} \PYG{l+s+s1}{\PYGZsq{}}\PYG{l+s+s1}{\PYGZsq{}}
    \PYG{n}{modelo} \PYG{o}{=} \PYG{l+s+s1}{\PYGZsq{}}\PYG{l+s+s1}{\PYGZsq{}}
    \PYG{n}{ano} \PYG{o}{=} \PYG{l+m+mi}{0}

\PYG{n}{c1} \PYG{o}{=} \PYG{n}{Carro}\PYG{p}{(}\PYG{p}{)}

\PYG{n}{c1}\PYG{o}{.}\PYG{n}{marca} \PYG{o}{=} \PYG{l+s+s1}{\PYGZsq{}}\PYG{l+s+s1}{Porsche}\PYG{l+s+s1}{\PYGZsq{}}

\PYG{n}{c1}\PYG{o}{.}\PYG{n}{modelo} \PYG{o}{=} \PYG{l+s+s1}{\PYGZsq{}}\PYG{l+s+s1}{Carrera}\PYG{l+s+s1}{\PYGZsq{}}

\PYG{n}{c1}\PYG{o}{.}\PYG{n}{ano} \PYG{o}{=} \PYG{l+m+mi}{1995}

\PYG{n+nb}{print}\PYG{p}{(}\PYG{l+s+s1}{\PYGZsq{}}\PYG{l+s+s1}{O }\PYG{l+s+si}{\PYGZob{}\PYGZcb{}}\PYG{l+s+s1}{ }\PYG{l+s+si}{\PYGZob{}\PYGZcb{}}\PYG{l+s+s1}{ fabricado em }\PYG{l+s+si}{\PYGZob{}\PYGZcb{}}\PYG{l+s+s1}{ estava estacionado.}\PYG{l+s+s1}{\PYGZsq{}}\PYG{o}{.}\PYG{n}{format}\PYG{p}{(}\PYG{n}{c1}\PYG{o}{.}\PYG{n}{marca}\PYG{p}{,} \PYG{n}{c1}\PYG{o}{.}\PYG{n}{modelo}\PYG{p}{,} \PYG{n}{c1}\PYG{o}{.}\PYG{n}{ano}\PYG{p}{)}\PYG{p}{)}
\end{sphinxVerbatim}

\begin{sphinxVerbatim}[commandchars=\\\{\}]
\PYG{g+go}{O Porsche Carrera fabricado em 1995 estava estacionado.}
\end{sphinxVerbatim}

\begin{sphinxVerbatim}[commandchars=\\\{\}]
\PYG{k}{class} \PYG{n+nc}{Animal}\PYG{p}{(}\PYG{n+nb}{object}\PYG{p}{)}\PYG{p}{:}
    \PYG{n}{peso} \PYG{o}{=} \PYG{l+m+mf}{0.0}

\PYG{k}{class} \PYG{n+nc}{Humano}\PYG{p}{(}\PYG{n}{Animal}\PYG{p}{)}\PYG{p}{:}
    \PYG{n}{quoficiente\PYGZus{}inteligencia} \PYG{o}{=} \PYG{l+m+mf}{0.0}

\PYG{k}{class} \PYG{n+nc}{Touro}\PYG{p}{(}\PYG{n}{Animal}\PYG{p}{)}\PYG{p}{:}
    \PYG{n}{envergadura\PYGZus{}chifre} \PYG{o}{=} \PYG{l+m+mf}{0.0}

\PYG{k}{class} \PYG{n+nc}{Minotauro}\PYG{p}{(}\PYG{n}{Humano}\PYG{p}{,} \PYG{n}{Animal}\PYG{p}{)}\PYG{p}{:}
    \PYG{k}{pass}
\end{sphinxVerbatim}


\section{Tipagem Dinâmica}
\label{\detokenize{content/language_particularities:tipagem-dinamica}}\begin{quote}

O interpretador define o tipo de acordo com o valor atribuído à variável.
A mesma variável pode ter seu tipo mudado de acordo com valores a ela atribuídos ao longo do código\sphinxhyphen{}fonte e em seu tempo de execução.
\end{quote}

\begin{sphinxVerbatim}[commandchars=\\\{\}]
\PYG{n}{foo} \PYG{o}{=} \PYG{l+s+s1}{\PYGZsq{}}\PYG{l+s+s1}{bar}\PYG{l+s+s1}{\PYGZsq{}}
\PYG{n+nb}{type}\PYG{p}{(}\PYG{n}{foo}\PYG{p}{)}
\end{sphinxVerbatim}

\begin{sphinxVerbatim}[commandchars=\\\{\}]
\PYG{g+go}{str}
\end{sphinxVerbatim}

\begin{sphinxVerbatim}[commandchars=\\\{\}]
\PYG{n}{foo} \PYG{o}{=} \PYG{l+m+mi}{123}
\PYG{n+nb}{type}\PYG{p}{(}\PYG{n}{foo}\PYG{p}{)}
\end{sphinxVerbatim}

\begin{sphinxVerbatim}[commandchars=\\\{\}]
\PYG{g+go}{int}
\end{sphinxVerbatim}

\begin{sphinxVerbatim}[commandchars=\\\{\}]
\PYG{n}{foo} \PYG{o}{=} \PYG{l+m+mf}{7.0}
\PYG{n+nb}{type}\PYG{p}{(}\PYG{n}{foo}\PYG{p}{)}
\end{sphinxVerbatim}

\begin{sphinxVerbatim}[commandchars=\\\{\}]
\PYG{g+go}{float}
\end{sphinxVerbatim}


\section{Tipagem Forte}
\label{\detokenize{content/language_particularities:tipagem-forte}}\begin{quote}

O interpretador verifica se a operação é válida e não faz coerção automática entre tipos incompatíveis. Caso haja operações de tipos incompatíveis é preciso fazer a conversão explícita da variável ou variáveis antes da operação.
\end{quote}

\begin{sphinxVerbatim}[commandchars=\\\{\}]
\PYG{n}{foo} \PYG{o}{=} \PYG{l+s+s1}{\PYGZsq{}}\PYG{l+s+s1}{2}\PYG{l+s+s1}{\PYGZsq{}}
\PYG{n}{bar} \PYG{o}{=} \PYG{l+m+mi}{5}
\PYG{n+nb}{type}\PYG{p}{(}\PYG{n}{foo}\PYG{p}{)}
\end{sphinxVerbatim}

\begin{sphinxVerbatim}[commandchars=\\\{\}]
\PYG{g+go}{\PYGZlt{}class \PYGZsq{}str\PYGZsq{}\PYGZgt{}}
\end{sphinxVerbatim}

\begin{sphinxVerbatim}[commandchars=\\\{\}]
\PYG{n+nb}{type}\PYG{p}{(}\PYG{n}{bar}\PYG{p}{)}
\end{sphinxVerbatim}

\begin{sphinxVerbatim}[commandchars=\\\{\}]
\PYG{g+go}{\PYGZlt{}class \PYGZsq{}int\PYGZsq{}\PYGZgt{}}
\end{sphinxVerbatim}

\begin{sphinxVerbatim}[commandchars=\\\{\}]
\PYG{n}{foobar} \PYG{o}{=} \PYG{n}{foo} \PYG{o}{+} \PYG{n}{bar}
\end{sphinxVerbatim}

\begin{sphinxVerbatim}[commandchars=\\\{\}]
\PYG{g+go}{Traceback (most recent call last):}
\PYG{g+go}{    File \PYGZdq{}\PYGZlt{}input\PYGZgt{}\PYGZdq{}, line 1, in \PYGZlt{}module\PYGZgt{}}
\PYG{g+go}{        foobar = foo + bar}
\PYG{g+go}{TypeError: can only concatenate str (not \PYGZdq{}int\PYGZdq{}) to str}
\end{sphinxVerbatim}

\begin{sphinxVerbatim}[commandchars=\\\{\}]
\PYG{n}{foobar} \PYG{o}{=} \PYG{n+nb}{int}\PYG{p}{(}\PYG{n}{foo}\PYG{p}{)} \PYG{o}{+} \PYG{n}{bar}
\PYG{n+nb}{print}\PYG{p}{(}\PYG{n}{foobar}\PYG{p}{)}
\end{sphinxVerbatim}

\begin{sphinxVerbatim}[commandchars=\\\{\}]
\PYG{g+go}{7}
\end{sphinxVerbatim}

\begin{sphinxVerbatim}[commandchars=\\\{\}]
\PYG{n}{foo} \PYG{o}{=} \PYG{l+m+mf}{2.0}
\PYG{n+nb}{type}\PYG{p}{(}\PYG{n}{foo}\PYG{p}{)}
\end{sphinxVerbatim}

\begin{sphinxVerbatim}[commandchars=\\\{\}]
\PYG{g+go}{\PYGZlt{}class \PYGZsq{}float\PYGZsq{}\PYGZgt{}}
\end{sphinxVerbatim}

\begin{sphinxVerbatim}[commandchars=\\\{\}]
\PYG{n}{bar} \PYG{o}{=} \PYG{l+m+mi}{5}
\PYG{n+nb}{type}\PYG{p}{(}\PYG{n}{bar}\PYG{p}{)}
\end{sphinxVerbatim}

\begin{sphinxVerbatim}[commandchars=\\\{\}]
\PYG{g+go}{\PYGZlt{}class \PYGZsq{}int\PYGZsq{}\PYGZgt{}}
\end{sphinxVerbatim}

\begin{sphinxVerbatim}[commandchars=\\\{\}]
\PYG{n}{foobar} \PYG{o}{=} \PYG{n}{foo} \PYG{o}{+} \PYG{n}{bar}
\PYG{n+nb}{print}\PYG{p}{(}\PYG{n}{foobar}\PYG{p}{)}
\end{sphinxVerbatim}

\begin{sphinxVerbatim}[commandchars=\\\{\}]
\PYG{g+go}{7.0}
\end{sphinxVerbatim}


\section{Bytecode}
\label{\detokenize{content/language_particularities:bytecode}}\begin{quote}

Formato binário multiplataforma resultante da compilação de um código Python.
\end{quote}

Criação de estrutura de diretórios para teste de pacote e bytecode:

\begin{sphinxVerbatim}[commandchars=\\\{\}]
mkdir \PYGZhy{}p /tmp/python/PacoteA/PacoteA1
\end{sphinxVerbatim}

Editar o módulo “Modulo1” que está dentro do pacote “PacoteA”:

\begin{sphinxVerbatim}[commandchars=\\\{\}]
vim /tmp/python/PacoteA/Modulo1.py
\end{sphinxVerbatim}

\begin{sphinxVerbatim}[commandchars=\\\{\},numbers=left,firstnumber=1,stepnumber=1]
\PYG{k}{def} \PYG{n+nf}{funcao}\PYG{p}{(}\PYG{p}{)}\PYG{p}{:}
    \PYG{n+nb}{print}\PYG{p}{(}\PYG{l+s+s1}{\PYGZsq{}}\PYG{l+s+s1}{Hello World!!!}\PYG{l+s+s1}{\PYGZsq{}}\PYG{p}{)}
\end{sphinxVerbatim}

Editar o módulo “Modulo2” que está dentro do pacote “PacoteA”:

\begin{sphinxVerbatim}[commandchars=\\\{\}]
vim /tmp/python/PacoteA/PacoteA1/Modulo2.py
\end{sphinxVerbatim}

\begin{sphinxVerbatim}[commandchars=\\\{\},numbers=left,firstnumber=1,stepnumber=1]
\PYG{k}{def} \PYG{n+nf}{funcao}\PYG{p}{(}\PYG{n}{numero}\PYG{p}{)}\PYG{p}{:}
    \PYG{n+nb}{print}\PYG{p}{(}\PYG{n}{numero} \PYG{o}{*}\PYG{o}{*} \PYG{l+m+mi}{3}\PYG{p}{)}
\end{sphinxVerbatim}

Edição de script de exemplo:

\begin{sphinxVerbatim}[commandchars=\\\{\}]
vim /tmp/python/foo.py
\end{sphinxVerbatim}

\begin{sphinxVerbatim}[commandchars=\\\{\},numbers=left,firstnumber=1,stepnumber=1]
\PYG{c+ch}{\PYGZsh{}!/usr/bin/env python}
\PYG{c+c1}{\PYGZsh{} \PYGZus{}*\PYGZus{} encoding: utf\PYGZhy{}8 \PYGZus{}*\PYGZus{}}

\PYG{k+kn}{from} \PYG{n+nn}{PacoteA}\PYG{n+nn}{.}\PYG{n+nn}{Modulo1} \PYG{k+kn}{import} \PYG{n}{funcao}
\PYG{k+kn}{from} \PYG{n+nn}{PacoteA}\PYG{n+nn}{.}\PYG{n+nn}{PacoteA1} \PYG{k+kn}{import} \PYG{n}{Modulo2}

\PYG{n+nb}{print}\PYG{p}{(}\PYG{l+s+s1}{\PYGZsq{}}\PYG{l+s+se}{\PYGZbs{}n}\PYG{l+s+s1}{Atenção!!!}\PYG{l+s+se}{\PYGZbs{}n}\PYG{l+s+s1}{\PYGZsq{}}\PYG{p}{)}
\PYG{n+nb}{print}\PYG{p}{(}\PYG{l+s+s1}{\PYGZsq{}}\PYG{l+s+s1}{O teste vai começar...}\PYG{l+s+se}{\PYGZbs{}n}\PYG{l+s+s1}{\PYGZsq{}}\PYG{p}{)}

\PYG{n}{funcao}\PYG{p}{(}\PYG{p}{)}

\PYG{n}{Modulo2}\PYG{o}{.}\PYG{n}{funcao}\PYG{p}{(}\PYG{l+m+mi}{3}\PYG{p}{)}
\end{sphinxVerbatim}

Execução do script:

\begin{sphinxVerbatim}[commandchars=\\\{\}]
python3 /tmp/python/foo.py
\end{sphinxVerbatim}

\begin{sphinxVerbatim}[commandchars=\\\{\}]
\PYG{g+go}{Atenção!!!}

\PYG{g+go}{O teste vai começar...}

\PYG{g+go}{Hello World!!!}
\PYG{g+go}{27}
\end{sphinxVerbatim}

Quando um módulo é carregado pela primeira vez ou se seu código é mais novo do que o arquivo binário ele é compilado e então gera ou gera novamente o arquivo binário .pyc.

Listar o conteúdo de “PacoteA”:

\begin{sphinxVerbatim}[commandchars=\\\{\}]
ls /tmp/python/PacoteA/
\end{sphinxVerbatim}

\begin{sphinxVerbatim}[commandchars=\\\{\}]
\PYG{g+go}{Modulo1.py  PacoteA1  \PYGZus{}\PYGZus{}pycache\PYGZus{}\PYGZus{}}
\end{sphinxVerbatim}

Listar o conteúdo de \_\_pycache\_\_:

\begin{sphinxVerbatim}[commandchars=\\\{\}]
ls /tmp/python/PacoteA/\PYGZus{}\PYGZus{}pycache\PYGZus{}\PYGZus{}/
\end{sphinxVerbatim}

\begin{sphinxVerbatim}[commandchars=\\\{\}]
\PYG{g+go}{Modulo1.cpython\PYGZhy{}36.pyc}
\end{sphinxVerbatim}

Com o comando “file” verificar informações de tipo de arquivo:

\begin{sphinxVerbatim}[commandchars=\\\{\}]
file /tmp/python/PacoteA/\PYGZus{}\PYGZus{}pycache\PYGZus{}\PYGZus{}/Modulo1.cpython\PYGZhy{}36.pyc
\end{sphinxVerbatim}

\begin{sphinxVerbatim}[commandchars=\\\{\}]
\PYG{g+go}{/tmp/python/PacoteA/\PYGZus{}\PYGZus{}pycache\PYGZus{}\PYGZus{}/Modulo1.cpython\PYGZhy{}36.pyc: python 3.6 byte\PYGZhy{}compiled}
\end{sphinxVerbatim}


\section{Quebra de linhas}
\label{\detokenize{content/language_particularities:quebra-de-linhas}}
Pode ser usada a barra invertida ou por vírgula.

\begin{sphinxVerbatim}[commandchars=\\\{\}]
\PYG{n}{varTeste} \PYG{o}{=} \PYG{l+m+mi}{3} \PYG{o}{*} \PYG{l+m+mi}{5} \PYG{o}{+} \PYGZbs{}
\PYG{p}{(}\PYG{l+m+mi}{10} \PYG{o}{+} \PYG{l+m+mi}{7}\PYG{p}{)}

\PYG{n}{varLista} \PYG{o}{=} \PYG{p}{[}\PYG{l+m+mi}{7}\PYG{p}{,}\PYG{l+m+mi}{14}\PYG{p}{,}\PYG{l+m+mi}{25}\PYG{p}{,}
            \PYG{l+m+mi}{81}\PYG{p}{,}\PYG{l+m+mi}{121}\PYG{p}{]}
\end{sphinxVerbatim}


\section{Blocos}
\label{\detokenize{content/language_particularities:blocos}}\begin{quote}

São delimitados por endentação e a linha anterior ao bloco sempre termina com dois pontos.
\end{quote}

\begin{sphinxVerbatim}[commandchars=\\\{\}]
\PYG{c+c1}{\PYGZsh{}Definição de uma classe}
\PYG{k}{class} \PYG{n+nc}{Carro}\PYG{p}{(}\PYG{n+nb}{object}\PYG{p}{)}\PYG{p}{:}
    \PYG{n}{ano} \PYG{o}{=} \PYG{l+m+mi}{0}
    \PYG{n}{marca} \PYG{o}{=} \PYG{l+s+s1}{\PYGZsq{}}\PYG{l+s+s1}{\PYGZsq{}}
    \PYG{n}{estado\PYGZus{}farois} \PYG{o}{=} \PYG{k+kc}{False}

    \PYG{c+c1}{\PYGZsh{}Definição de um método da classe}
    \PYG{k}{def} \PYG{n+nf}{interruptor\PYGZus{}farois}\PYG{p}{(}\PYG{n+nb+bp}{self}\PYG{p}{)}\PYG{p}{:}
        \PYG{c+c1}{\PYGZsh{}Bloco if}
            \PYG{k}{if}\PYG{p}{(}\PYG{n+nb+bp}{self}\PYG{o}{.}\PYG{n}{estado\PYGZus{}farois}\PYG{p}{)}\PYG{p}{:}
                \PYG{n+nb}{print}\PYG{p}{(}\PYG{l+s+s1}{\PYGZsq{}}\PYG{l+s+s1}{Apagando faróis}\PYG{l+s+s1}{\PYGZsq{}}\PYG{p}{)}
                \PYG{n+nb+bp}{self}\PYG{o}{.}\PYG{n}{estado\PYGZus{}farois} \PYG{o}{=} \PYG{k+kc}{False}
            \PYG{k}{else}\PYG{p}{:}
                \PYG{n+nb}{print}\PYG{p}{(}\PYG{l+s+s1}{\PYGZsq{}}\PYG{l+s+s1}{Acendendo faróis}\PYG{l+s+s1}{\PYGZsq{}}\PYG{p}{)}
                \PYG{n+nb+bp}{self}\PYG{o}{.}\PYG{n}{estado\PYGZus{}farois} \PYG{o}{=} \PYG{k+kc}{True}
\end{sphinxVerbatim}


\section{Comentários}
\label{\detokenize{content/language_particularities:comentarios}}\begin{quote}

Inicia\sphinxhyphen{}se com o caractere “\#” em cada linha.
\end{quote}

\begin{sphinxVerbatim}[commandchars=\\\{\}]
\PYG{c+c1}{\PYGZsh{} um simples comentário}

\PYG{c+c1}{\PYGZsh{} A seguir uma soma}

\PYG{n}{x} \PYG{o}{=} \PYG{l+m+mi}{5} \PYG{o}{+} \PYG{l+m+mi}{2}

\PYG{n+nb}{print}\PYG{p}{(}\PYG{n}{x}\PYG{p}{)}  \PYG{c+c1}{\PYGZsh{} Imprime o valor de x}
\end{sphinxVerbatim}


\section{Docstrings ou Strings de Múltiplas Linhas}
\label{\detokenize{content/language_particularities:docstrings-ou-strings-de-multiplas-linhas}}\begin{quote}

Feitos dentro de funções e classes, que geram documentação automaticamente que pode ser acessado pela função help().
São usados três pares de apóstrofos (‘) ou três pares de aspas (“), 3 (três) no início e 3 (três) no fim.
\end{quote}

\begin{sphinxVerbatim}[commandchars=\\\{\}]
\PYG{c+c1}{\PYGZsh{} Com apóstrofos}

\PYG{l+s+sd}{\PYGZsq{}\PYGZsq{}\PYGZsq{}Esta função faz isso de forma}
\PYG{l+s+sd}{x, y e z além de bla bla bla bla\PYGZsq{}\PYGZsq{}\PYGZsq{}}

\PYG{c+c1}{\PYGZsh{} Com aspas}

\PYG{l+s+sd}{\PYGZdq{}\PYGZdq{}\PYGZdq{}Esta função faz isso de forma}
\PYG{l+s+sd}{x, y e z além de bla bla bla bla\PYGZdq{}\PYGZdq{}\PYGZdq{}}
\end{sphinxVerbatim}

Recurso para criar documentação automaticamente:

\begin{sphinxVerbatim}[commandchars=\\\{\}]
\PYG{k}{def} \PYG{n+nf}{funcao}\PYG{p}{(}\PYG{p}{)}\PYG{p}{:}
    \PYG{l+s+sd}{\PYGZsq{}\PYGZsq{}\PYGZsq{}Esta função não faz absolutamente nada\PYGZsq{}\PYGZsq{}\PYGZsq{}}

\PYG{n}{help}\PYG{p}{(}\PYG{n}{funcao}\PYG{p}{)}
\end{sphinxVerbatim}

\begin{sphinxVerbatim}[commandchars=\\\{\}]
\PYG{g+go}{Help on function funcao in module \PYGZus{}\PYGZus{}main\PYGZus{}\PYGZus{}:}

\PYG{g+go}{funcao()}
\PYG{g+go}{    Esta função não faz absolutamente nada}
\end{sphinxVerbatim}


\section{Operadores}
\label{\detokenize{content/language_particularities:operadores}}
Python suporta operadores dos tipos:
\begin{itemize}
\item {} 
Aritiméticos: +, \sphinxhyphen{}, \sphinxstyleemphasis{, /, //, *}, \%;

\item {} 
Relacionais: \textgreater{}, \textless{}, \textgreater{}=, \textgreater{}=, ==, !=;

\item {} 
Atribuição: =, +=, +=, \sphinxhyphen{}=, /=, {\color{red}\bfseries{}*}=, \%=, {\color{red}\bfseries{}**}=, //=;

\item {} 
Lógicos: and, or, not;

\item {} 
Associação: in, not in;

\item {} 
Identidade: is, is not;

\item {} 
Bitwise: \&, {\color{red}\bfseries{}|}, \textasciicircum{}, \textasciitilde{}, \textless{}\textless{}, \textgreater{}\textgreater{}.

\end{itemize}

Operadores serão discutidos em mais detalhes em um capítulo posterior.


\section{O Comando del}
\label{\detokenize{content/language_particularities:o-comando-del}}\begin{quote}

Este comando tem como objetivo remover a referência de um objeto.
Se esse objeto não tiver outra referência, o garbage collector atuará liberando recursos.
\end{quote}

\begin{sphinxVerbatim}[commandchars=\\\{\}]
\PYG{n}{sogra} \PYG{o}{=} \PYG{l+s+s1}{\PYGZsq{}}\PYG{l+s+s1}{Edelbarina}\PYG{l+s+s1}{\PYGZsq{}}
\PYG{n+nb}{print}\PYG{p}{(}\PYG{n}{sogra}\PYG{p}{)}
\end{sphinxVerbatim}

\begin{sphinxVerbatim}[commandchars=\\\{\}]
\PYG{g+go}{Edelbarina}
\end{sphinxVerbatim}

\begin{sphinxVerbatim}[commandchars=\\\{\}]
\PYG{k}{del} \PYG{n}{sogra}
\PYG{n+nb}{print}\PYG{p}{(}\PYG{n}{sogra}\PYG{p}{)}
\end{sphinxVerbatim}

\begin{sphinxVerbatim}[commandchars=\\\{\}]
\PYG{g+go}{Traceback (most recent call last):}
\PYG{g+go}{    File \PYGZdq{}\PYGZlt{}stdin\PYGZgt{}\PYGZdq{}, line 1, in \PYGZlt{}module\PYGZgt{}}
\PYG{g+go}{NameError: name \PYGZsq{}sogra\PYGZsq{} is not defined}
\end{sphinxVerbatim}

\begin{sphinxVerbatim}[commandchars=\\\{\}]
\PYG{n}{a} \PYG{o}{=} \PYG{p}{[}\PYG{l+s+s1}{\PYGZsq{}}\PYG{l+s+s1}{Z}\PYG{l+s+s1}{\PYGZsq{}}\PYG{p}{,} \PYG{l+m+mi}{1}\PYG{p}{,} \PYG{l+m+mi}{5}\PYG{p}{,} \PYG{l+s+s1}{\PYGZsq{}}\PYG{l+s+s1}{m}\PYG{l+s+s1}{\PYGZsq{}}\PYG{p}{]}
\PYG{k}{del} \PYG{n}{a}\PYG{p}{[}\PYG{l+m+mi}{2}\PYG{p}{]}
\PYG{n+nb}{print}\PYG{p}{(}\PYG{n}{a}\PYG{p}{)}
\end{sphinxVerbatim}

\begin{sphinxVerbatim}[commandchars=\\\{\}]
\PYG{g+go}{[\PYGZsq{}Z\PYGZsq{}, 1, \PYGZsq{}m\PYGZsq{}]}
\end{sphinxVerbatim}


\section{print}
\label{\detokenize{content/language_particularities:print}}\begin{quote}

Antes era somente um comando, a partir da série 3.X será apenas interpretado como função.
\end{quote}

\begin{sphinxVerbatim}[commandchars=\\\{\}]
\PYG{n+nb}{print}\PYG{p}{(}\PYG{l+s+s1}{\PYGZsq{}}\PYG{l+s+s1}{Teste}\PYG{l+s+s1}{\PYGZsq{}}\PYG{p}{)}
\end{sphinxVerbatim}

\begin{sphinxVerbatim}[commandchars=\\\{\}]
\PYG{g+go}{Teste}
\end{sphinxVerbatim}


\section{Referência de Identificadores}
\label{\detokenize{content/language_particularities:referencia-de-identificadores}}
\begin{sphinxVerbatim}[commandchars=\\\{\}]
\PYG{n}{x} \PYG{o}{=} \PYG{l+m+mi}{7}
\PYG{n}{y} \PYG{o}{=} \PYG{n}{x}
\PYG{n}{z} \PYG{o}{=} \PYG{n}{x}

\PYG{n+nb}{id}\PYG{p}{(}\PYG{n}{x}\PYG{p}{)}
\end{sphinxVerbatim}

\begin{sphinxVerbatim}[commandchars=\\\{\}]
\PYG{g+go}{29786312}
\end{sphinxVerbatim}

\begin{sphinxVerbatim}[commandchars=\\\{\}]
\PYG{n+nb}{id}\PYG{p}{(}\PYG{n}{y}\PYG{p}{)}
\end{sphinxVerbatim}

\begin{sphinxVerbatim}[commandchars=\\\{\}]
\PYG{g+go}{29786312}
\end{sphinxVerbatim}

\begin{sphinxVerbatim}[commandchars=\\\{\}]
\PYG{n+nb}{id}\PYG{p}{(}\PYG{n}{z}\PYG{p}{)}
\end{sphinxVerbatim}

\begin{sphinxVerbatim}[commandchars=\\\{\}]
\PYG{g+go}{29786312}
\end{sphinxVerbatim}

3 (três) referências ao mesmo objeto

\begin{sphinxVerbatim}[commandchars=\\\{\}]
\PYG{k}{del} \PYG{n}{x}
\end{sphinxVerbatim}

Agora são 2 (duas) referências…

\begin{sphinxVerbatim}[commandchars=\\\{\}]
\PYG{n+nb}{print}\PYG{p}{(}\PYG{n}{y}\PYG{p}{)}
\end{sphinxVerbatim}

\begin{sphinxVerbatim}[commandchars=\\\{\}]
\PYG{g+go}{7}
\end{sphinxVerbatim}

\begin{sphinxVerbatim}[commandchars=\\\{\}]
\PYG{k}{del} \PYG{n}{y}
\end{sphinxVerbatim}

Resta apenas 1 (uma) referência…

\begin{sphinxVerbatim}[commandchars=\\\{\}]
\PYG{n+nb}{print}\PYG{p}{(}\PYG{n}{z}\PYG{p}{)}
\end{sphinxVerbatim}

\begin{sphinxVerbatim}[commandchars=\\\{\}]
\PYG{g+go}{7}
\end{sphinxVerbatim}

\begin{sphinxVerbatim}[commandchars=\\\{\}]
\PYG{k}{del} \PYG{n}{z}
\end{sphinxVerbatim}

O contador de referências chegou a 0 (zero), ou seja, não há mais referência para o objeto.
Então entra em ação o Garbage Collector para limpar a memória.


\chapter{Comandos e Funções Importantes}
\label{\detokenize{content/built-ins:comandos-e-funcoes-importantes}}\label{\detokenize{content/built-ins::doc}}\begin{quote}

Neste capítulo são abordados comandos e funções interessantes e / ou imprescindíveis para a linguagem.
\end{quote}


\section{print()}
\label{\detokenize{content/built-ins:print}}\begin{quote}

Em Python 2 era um comando e em Python 3 passou a ser exclusivamente uma função.
Seu objetivo é imprimir uma mensagem que por padrão é STDOUT.
\end{quote}

\begin{sphinxVerbatim}[commandchars=\\\{\}]
\PYG{c+c1}{\PYGZsh{} Um simples \PYGZdq{}Hello, world!\PYGZdq{}:}

\PYG{n+nb}{print}\PYG{p}{(}\PYG{l+s+s1}{\PYGZsq{}}\PYG{l+s+s1}{Hello, world!}\PYG{l+s+s1}{\PYGZsq{}}\PYG{p}{)}
\end{sphinxVerbatim}

\begin{sphinxVerbatim}[commandchars=\\\{\}]
\PYG{g+go}{Hello, world!}
\end{sphinxVerbatim}

\begin{sphinxVerbatim}[commandchars=\\\{\}]
\PYG{c+c1}{\PYGZsh{} Pode\PYGZhy{}se passar mais de uma string como parâmetro:}

\PYG{n+nb}{print}\PYG{p}{(}\PYG{l+s+s1}{\PYGZsq{}}\PYG{l+s+s1}{foo}\PYG{l+s+s1}{\PYGZsq{}}\PYG{p}{,} \PYG{l+s+s1}{\PYGZsq{}}\PYG{l+s+s1}{bar}\PYG{l+s+s1}{\PYGZsq{}}\PYG{p}{,} \PYG{l+s+s1}{\PYGZsq{}}\PYG{l+s+s1}{baz}\PYG{l+s+s1}{\PYGZsq{}}\PYG{p}{)}
\end{sphinxVerbatim}

\begin{sphinxVerbatim}[commandchars=\\\{\}]
\PYG{g+go}{foo bar baz}
\end{sphinxVerbatim}

\begin{sphinxVerbatim}[commandchars=\\\{\}]
\PYG{c+c1}{\PYGZsh{} Colocando como separador uma nova linha para cada string passada como parâmetro}

\PYG{n+nb}{print}\PYG{p}{(}\PYG{l+s+s1}{\PYGZsq{}}\PYG{l+s+s1}{foo}\PYG{l+s+s1}{\PYGZsq{}}\PYG{p}{,} \PYG{l+s+s1}{\PYGZsq{}}\PYG{l+s+s1}{bar}\PYG{l+s+s1}{\PYGZsq{}}\PYG{p}{,} \PYG{l+s+s1}{\PYGZsq{}}\PYG{l+s+s1}{baz}\PYG{l+s+s1}{\PYGZsq{}}\PYG{p}{,} \PYG{n}{sep}\PYG{o}{=}\PYG{l+s+s1}{\PYGZsq{}}\PYG{l+s+se}{\PYGZbs{}n}\PYG{l+s+s1}{\PYGZsq{}}\PYG{p}{)}
\end{sphinxVerbatim}

\begin{sphinxVerbatim}[commandchars=\\\{\}]
\PYG{g+go}{foo}
\PYG{g+go}{bar}
\PYG{g+go}{baz}
\end{sphinxVerbatim}


\section{len()}
\label{\detokenize{content/built-ins:len}}\begin{quote}

Função que retorna a quantidade de itens de um contêiner.
\end{quote}

\begin{sphinxVerbatim}[commandchars=\\\{\}]
\PYG{c+c1}{\PYGZsh{} Criação de um objeto contêiner e verificação da quantidade de elementos}
\PYG{n}{foo} \PYG{o}{=} \PYG{p}{(}\PYG{l+s+s1}{\PYGZsq{}}\PYG{l+s+s1}{x}\PYG{l+s+s1}{\PYGZsq{}}\PYG{p}{,} \PYG{l+s+s1}{\PYGZsq{}}\PYG{l+s+s1}{y}\PYG{l+s+s1}{\PYGZsq{}}\PYG{p}{,} \PYG{l+s+s1}{\PYGZsq{}}\PYG{l+s+s1}{z}\PYG{l+s+s1}{\PYGZsq{}}\PYG{p}{,} \PYG{l+m+mi}{123}\PYG{p}{,} \PYG{l+m+mf}{5.7}\PYG{p}{)}

\PYG{c+c1}{\PYGZsh{} len(foo)}
\end{sphinxVerbatim}

\begin{sphinxVerbatim}[commandchars=\\\{\}]
\PYG{g+go}{5}
\end{sphinxVerbatim}

\begin{sphinxVerbatim}[commandchars=\\\{\}]
\PYG{c+c1}{\PYGZsh{} Tamanho de uma string}
\PYG{n+nb}{len}\PYG{p}{(}\PYG{l+s+s1}{\PYGZsq{}}\PYG{l+s+s1}{Heavy Metal}\PYG{l+s+s1}{\PYGZsq{}}\PYG{p}{)}
\end{sphinxVerbatim}

\begin{sphinxVerbatim}[commandchars=\\\{\}]
\PYG{g+go}{11}
\end{sphinxVerbatim}


\section{range()}
\label{\detokenize{content/built-ins:range}}\begin{quote}

É uma função que retorna um objeto com uma faixa inteiros (range object).
Muito útil para uso em loops.
\end{quote}

Sintaxe:
\begin{quote}

range(stop)
range(start, stop{[}, step{]})

start: Valor inicial da sequência, por padrão é 0 (zero).
stop:  Valor final da sequẽncia \sphinxhyphen{} 1.
step:  Valor de incremento, cujo padrão é 1 (um), quando start é maior que stop, ou seja, para se fazer uma sequência regressiva é preciso um número negativo.
\end{quote}

\begin{sphinxVerbatim}[commandchars=\\\{\}]
\PYG{c+c1}{\PYGZsh{} Um parâmetro (stop)}
\PYG{k}{for} \PYG{n}{i} \PYG{o+ow}{in} \PYG{n+nb}{range}\PYG{p}{(}\PYG{l+m+mi}{5}\PYG{p}{)}\PYG{p}{:}
    \PYG{n+nb}{print}\PYG{p}{(}\PYG{n}{i}\PYG{p}{)}
\end{sphinxVerbatim}

\begin{sphinxVerbatim}[commandchars=\\\{\}]
\PYG{g+go}{0}
\PYG{g+go}{1}
\PYG{g+go}{2}
\PYG{g+go}{3}
\PYG{g+go}{4}
\end{sphinxVerbatim}

\begin{sphinxVerbatim}[commandchars=\\\{\}]
\PYG{c+c1}{\PYGZsh{} Dois parâmetros (start e stop)}
\PYG{k}{for} \PYG{n}{i} \PYG{o+ow}{in} \PYG{n+nb}{range}\PYG{p}{(}\PYG{l+m+mi}{3}\PYG{p}{,} \PYG{l+m+mi}{10}\PYG{p}{)}\PYG{p}{:}
    \PYG{n+nb}{print}\PYG{p}{(}\PYG{n}{i}\PYG{p}{)}
\end{sphinxVerbatim}

\begin{sphinxVerbatim}[commandchars=\\\{\}]
\PYG{g+go}{3}
\PYG{g+go}{4}
\PYG{g+go}{5}
\PYG{g+go}{6}
\PYG{g+go}{7}
\PYG{g+go}{8}
\PYG{g+go}{9}
\end{sphinxVerbatim}

\begin{sphinxVerbatim}[commandchars=\\\{\}]
\PYG{c+c1}{\PYGZsh{} Três parâmetros (start, stop e step)}
\PYG{k}{for} \PYG{n}{i} \PYG{o+ow}{in} \PYG{n+nb}{range}\PYG{p}{(}\PYG{l+m+mi}{1}\PYG{p}{,} \PYG{l+m+mi}{10}\PYG{p}{,} \PYG{l+m+mi}{2}\PYG{p}{)}\PYG{p}{:}
    \PYG{n+nb}{print}\PYG{p}{(}\PYG{n}{i}\PYG{p}{)}
\end{sphinxVerbatim}

\begin{sphinxVerbatim}[commandchars=\\\{\}]
\PYG{g+go}{1}
\PYG{g+go}{3}
\PYG{g+go}{5}
\PYG{g+go}{7}
\PYG{g+go}{9}
\end{sphinxVerbatim}

\begin{sphinxVerbatim}[commandchars=\\\{\}]
\PYG{c+c1}{\PYGZsh{} Sequẽncia regressiva}
\PYG{k}{for} \PYG{n}{i} \PYG{o+ow}{in} \PYG{n+nb}{range}\PYG{p}{(}\PYG{l+m+mi}{20}\PYG{p}{,} \PYG{l+m+mi}{1}\PYG{p}{,} \PYG{o}{\PYGZhy{}}\PYG{l+m+mi}{5}\PYG{p}{)}\PYG{p}{:}
    \PYG{n+nb}{print}\PYG{p}{(}\PYG{n}{i}\PYG{p}{)}
\end{sphinxVerbatim}

\begin{sphinxVerbatim}[commandchars=\\\{\}]
\PYG{g+go}{20}
\PYG{g+go}{15}
\PYG{g+go}{10}
\PYG{g+go}{5}
\end{sphinxVerbatim}


\section{filter()}
\label{\detokenize{content/built-ins:filter}}\begin{quote}

Função que retorna um iterador produzindo os itens iteráveis para os quais a função(item) for True.
\end{quote}

\begin{sphinxVerbatim}[commandchars=\\\{\}]
\PYG{c+c1}{\PYGZsh{} Criação de uma função que retorna True se o objeto for ímpar}
\PYG{k}{def} \PYG{n+nf}{impar}\PYG{p}{(}\PYG{n}{x}\PYG{p}{)}\PYG{p}{:}
    \PYG{k}{return} \PYG{n}{x} \PYG{o}{\PYGZpc{}} \PYG{l+m+mi}{2} \PYG{o}{!=} \PYG{l+m+mi}{0}

\PYG{c+c1}{\PYGZsh{} Testando a função}
\PYG{n}{impar}\PYG{p}{(}\PYG{l+m+mi}{7}\PYG{p}{)}
\end{sphinxVerbatim}

\begin{sphinxVerbatim}[commandchars=\\\{\}]
\PYG{g+go}{True}
\end{sphinxVerbatim}

\begin{sphinxVerbatim}[commandchars=\\\{\}]
\PYG{n}{impar}\PYG{p}{(}\PYG{l+m+mi}{6}\PYG{p}{)}
\end{sphinxVerbatim}

\begin{sphinxVerbatim}[commandchars=\\\{\}]
\PYG{g+go}{False}
\end{sphinxVerbatim}

Em uma sequência de 0 a 19, pela função impar criar um objeto filter somente com os elementos ímpares e posteriormente ser convertido para lista:

\begin{sphinxVerbatim}[commandchars=\\\{\}]
\PYG{n}{f} \PYG{o}{=} \PYG{n+nb}{filter}\PYG{p}{(}\PYG{n}{impar}\PYG{p}{,} \PYG{n+nb}{range}\PYG{p}{(}\PYG{l+m+mi}{0}\PYG{p}{,} \PYG{l+m+mi}{20}\PYG{p}{)}\PYG{p}{)}
\PYG{n+nb}{print}\PYG{p}{(}\PYG{n+nb}{list}\PYG{p}{(}\PYG{n}{f}\PYG{p}{)}\PYG{p}{)}
\end{sphinxVerbatim}

\begin{sphinxVerbatim}[commandchars=\\\{\}]
\PYG{g+go}{[1, 3, 5, 7, 9, 11, 13, 15, 17, 19]}
\end{sphinxVerbatim}


\section{map()}
\label{\detokenize{content/built-ins:map}}\begin{quote}

Cria um iterador que aplica uma função para cada elemento do iterável.
\end{quote}

\begin{sphinxVerbatim}[commandchars=\\\{\}]
\PYG{c+c1}{\PYGZsh{} Dada uma tupla com várias strings, criar uma lista}
\PYG{c+c1}{\PYGZsh{} com o tamanho de cada string respectivamente}
\PYG{n}{m} \PYG{o}{=} \PYG{n+nb}{map}\PYG{p}{(}\PYG{n+nb}{len}\PYG{p}{,} \PYG{p}{(}\PYG{l+s+s1}{\PYGZsq{}}\PYG{l+s+s1}{spam}\PYG{l+s+s1}{\PYGZsq{}}\PYG{p}{,} \PYG{l+s+s1}{\PYGZsq{}}\PYG{l+s+s1}{foo}\PYG{l+s+s1}{\PYGZsq{}}\PYG{p}{,} \PYG{l+s+s1}{\PYGZsq{}}\PYG{l+s+s1}{bar}\PYG{l+s+s1}{\PYGZsq{}}\PYG{p}{,} \PYG{l+s+s1}{\PYGZsq{}}\PYG{l+s+s1}{eggs}\PYG{l+s+s1}{\PYGZsq{}}\PYG{p}{,} \PYG{l+s+s1}{\PYGZsq{}}\PYG{l+s+s1}{Python}\PYG{l+s+s1}{\PYGZsq{}}\PYG{p}{)}\PYG{p}{)}
\PYG{n+nb}{print}\PYG{p}{(}\PYG{n+nb}{list}\PYG{p}{(}\PYG{n}{m}\PYG{p}{)}\PYG{p}{)}
\end{sphinxVerbatim}

\begin{sphinxVerbatim}[commandchars=\\\{\}]
\PYG{g+go}{[4, 3, 3, 4, 6]}
\end{sphinxVerbatim}

\begin{sphinxVerbatim}[commandchars=\\\{\}]
\PYG{c+c1}{\PYGZsh{} Para cada item da lista, criar uma nova lista}
\PYG{c+c1}{\PYGZsh{} com seus respectivos tipos}
\PYG{n}{m} \PYG{o}{=} \PYG{n+nb}{map}\PYG{p}{(}\PYG{n+nb}{type}\PYG{p}{,} \PYG{p}{[}\PYG{l+s+s1}{\PYGZsq{}}\PYG{l+s+s1}{foo}\PYG{l+s+s1}{\PYGZsq{}}\PYG{p}{,} \PYG{l+m+mf}{1.4}\PYG{p}{,} \PYG{l+m+mi}{2} \PYG{o}{+} \PYG{l+m+mi}{5}\PYG{n}{j}\PYG{p}{,} \PYG{l+m+mi}{1000}\PYG{p}{]}\PYG{p}{)}
\PYG{n+nb}{print}\PYG{p}{(}\PYG{n+nb}{list}\PYG{p}{(}\PYG{n}{m}\PYG{p}{)}\PYG{p}{)}
\end{sphinxVerbatim}

\begin{sphinxVerbatim}[commandchars=\\\{\}]
\PYG{g+go}{[str, float, complex, int]}
\end{sphinxVerbatim}

\begin{sphinxVerbatim}[commandchars=\\\{\}]
\PYG{c+c1}{\PYGZsh{} Para cada item da primeira lista elevar (potência)}
\PYG{c+c1}{\PYGZsh{} ao elemento respectivo na segunda lista e criar uma}
\PYG{c+c1}{\PYGZsh{} nova lista com os resultados}
\PYG{n}{m} \PYG{o}{=} \PYG{n+nb}{map}\PYG{p}{(}\PYG{n+nb}{pow}\PYG{p}{,} \PYG{p}{[}\PYG{l+m+mi}{3}\PYG{p}{,} \PYG{l+m+mi}{7}\PYG{p}{,} \PYG{l+m+mi}{5}\PYG{p}{,} \PYG{l+m+mi}{10}\PYG{p}{]}\PYG{p}{,} \PYG{p}{[}\PYG{l+m+mi}{2}\PYG{p}{,} \PYG{l+m+mi}{1}\PYG{p}{,} \PYG{l+m+mi}{7}\PYG{p}{,} \PYG{l+m+mi}{3}\PYG{p}{]}\PYG{p}{)}
\PYG{n+nb}{print}\PYG{p}{(}\PYG{n+nb}{list}\PYG{p}{(}\PYG{n}{m}\PYG{p}{)}\PYG{p}{)}
\end{sphinxVerbatim}

\begin{sphinxVerbatim}[commandchars=\\\{\}]
\PYG{g+go}{[9, 7, 78125, 1000]}
\end{sphinxVerbatim}


\section{reduce()}
\label{\detokenize{content/built-ins:reduce}}\begin{quote}

Em Python 2 estava disponível sem a necessidade de fazer importação, hoje em Python 3 está no módulo functools.
\end{quote}

\begin{sphinxVerbatim}[commandchars=\\\{\}]
\PYG{c+c1}{\PYGZsh{} Via loop somar todos elementos de uma tupla}
\PYG{n}{soma} \PYG{o}{=} \PYG{l+m+mi}{0}  \PYG{c+c1}{\PYGZsh{} Variável que terá o valor da soma após o loop}

\PYG{k}{for} \PYG{n}{i} \PYG{o+ow}{in} \PYG{p}{(}\PYG{l+m+mi}{2}\PYG{p}{,} \PYG{l+m+mi}{1}\PYG{p}{,} \PYG{l+m+mi}{4}\PYG{p}{,} \PYG{l+m+mi}{3}\PYG{p}{)}\PYG{p}{:}  \PYG{c+c1}{\PYGZsh{} Loop e incrementação}
    \PYG{n}{soma} \PYG{o}{+}\PYG{o}{=} \PYG{n}{i}

\PYG{n+nb}{print}\PYG{p}{(}\PYG{n}{soma}\PYG{p}{)}  \PYG{c+c1}{\PYGZsh{} Exibe o resultado}
\end{sphinxVerbatim}

\begin{sphinxVerbatim}[commandchars=\\\{\}]
\PYG{g+go}{10}
\end{sphinxVerbatim}

\begin{sphinxVerbatim}[commandchars=\\\{\}]
\PYG{c+c1}{\PYGZsh{} Importando reduce de functools}
\PYG{k+kn}{from} \PYG{n+nn}{functools} \PYG{k+kn}{import} \PYG{n}{reduce}

\PYG{c+c1}{\PYGZsh{} Função reduce para executar a mesma}
\PYG{c+c1}{\PYGZsh{} tarefa anterior com apenas um comando}
\PYG{n}{reduce}\PYG{p}{(}\PYG{n+nb}{int}\PYG{o}{.}\PYG{n+nf+fm}{\PYGZus{}\PYGZus{}add\PYGZus{}\PYGZus{}}\PYG{p}{,} \PYG{p}{(}\PYG{l+m+mi}{2}\PYG{p}{,} \PYG{l+m+mi}{1}\PYG{p}{,} \PYG{l+m+mi}{4}\PYG{p}{,} \PYG{l+m+mi}{3}\PYG{p}{)}\PYG{p}{)}
\end{sphinxVerbatim}

\begin{sphinxVerbatim}[commandchars=\\\{\}]
\PYG{g+go}{10}
\end{sphinxVerbatim}


\section{del}
\label{\detokenize{content/built-ins:del}}\begin{quote}

Pode ser tanto um comando como uma função cuja finalidade é remover a referência de um objeto.
Também apaga elemento de uma coleção.
\end{quote}

\begin{sphinxVerbatim}[commandchars=\\\{\}]
\PYG{c+c1}{\PYGZsh{} Teste de del em um objeto mutável (lista)}
\PYG{n}{lista} \PYG{o}{=} \PYG{p}{[}\PYG{l+s+s1}{\PYGZsq{}}\PYG{l+s+s1}{a}\PYG{l+s+s1}{\PYGZsq{}}\PYG{p}{,} \PYG{l+s+s1}{\PYGZsq{}}\PYG{l+s+s1}{b}\PYG{l+s+s1}{\PYGZsq{}}\PYG{p}{,} \PYG{l+s+s1}{\PYGZsq{}}\PYG{l+s+s1}{c}\PYG{l+s+s1}{\PYGZsq{}}\PYG{p}{]}  \PYG{c+c1}{\PYGZsh{} Definição da lista}
\PYG{k}{del} \PYG{n}{lista}\PYG{p}{[}\PYG{l+m+mi}{1}\PYG{p}{]}  \PYG{c+c1}{\PYGZsh{} Apaga o segundo elemento da lista}
\end{sphinxVerbatim}

ou

\begin{sphinxVerbatim}[commandchars=\\\{\}]
\PYG{k}{del}\PYG{p}{(}\PYG{n}{lista}\PYG{p}{[}\PYG{l+m+mi}{1}\PYG{p}{]}\PYG{p}{)}  \PYG{c+c1}{\PYGZsh{} Equivalência ao comando anterior em forma de função}
\PYG{n+nb}{print}\PYG{p}{(}\PYG{n}{lista}\PYG{p}{)}  \PYG{c+c1}{\PYGZsh{} Exibe a lista após o elemento ser retirado da mesma}
\end{sphinxVerbatim}

\begin{sphinxVerbatim}[commandchars=\\\{\}]
\PYG{g+go}{[\PYGZsq{}a\PYGZsq{}, \PYGZsq{}c\PYGZsq{}]}
\end{sphinxVerbatim}

\begin{sphinxVerbatim}[commandchars=\\\{\}]
\PYG{c+c1}{\PYGZsh{} Teste de del para desalocar um objeto criado}
\PYG{n}{foo} \PYG{o}{=} \PYG{l+s+s1}{\PYGZsq{}}\PYG{l+s+s1}{bar}\PYG{l+s+s1}{\PYGZsq{}}  \PYG{c+c1}{\PYGZsh{} Objeto string criado}
\PYG{n+nb}{print}\PYG{p}{(}\PYG{n}{foo}\PYG{p}{)}  \PYG{c+c1}{\PYGZsh{} Verificando o valor da string}
\end{sphinxVerbatim}

\begin{sphinxVerbatim}[commandchars=\\\{\}]
\PYG{g+go}{bar}
\end{sphinxVerbatim}

\begin{sphinxVerbatim}[commandchars=\\\{\}]
\PYG{k}{del} \PYG{n}{foo}  \PYG{c+c1}{\PYGZsh{} Apagando o objeto string}
\PYG{n+nb}{print}\PYG{p}{(}\PYG{n}{foo}\PYG{p}{)}  \PYG{c+c1}{\PYGZsh{} Tentativa de imprimir o valor do objeto desalocado}
\end{sphinxVerbatim}

\begin{sphinxVerbatim}[commandchars=\\\{\}]
\PYG{g+go}{NameError: name \PYGZsq{}foo\PYGZsq{} is not defined}
\end{sphinxVerbatim}

Nota\sphinxhyphen{}se que após o del não é possível mais fazer referência ao objeto.


\section{ord e chr}
\label{\detokenize{content/built-ins:ord-e-chr}}\begin{quote}

A função ord retorna o código Unicode de um caractere.
A função chr faz o caminho inverso, ou seja, retorna um caractere dado um código Unicode. Em Python 2 chr era unichr.
\end{quote}

\begin{sphinxVerbatim}[commandchars=\\\{\}]
\PYG{n+nb}{ord}\PYG{p}{(}\PYG{l+s+s1}{\PYGZsq{}}\PYG{l+s+se}{\PYGZbs{}n}\PYG{l+s+s1}{\PYGZsq{}}\PYG{p}{)}  \PYG{c+c1}{\PYGZsh{} Qual é o código Unicode para new line?}
\end{sphinxVerbatim}

\begin{sphinxVerbatim}[commandchars=\\\{\}]
\PYG{g+go}{10}
\end{sphinxVerbatim}

\begin{sphinxVerbatim}[commandchars=\\\{\}]
\PYG{n+nb}{chr}\PYG{p}{(}\PYG{l+m+mi}{10}\PYG{p}{)}  \PYG{c+c1}{\PYGZsh{} Qual caractere Unicode corresponde ao código 10?}
\end{sphinxVerbatim}

\begin{sphinxVerbatim}[commandchars=\\\{\}]
\PYG{g+go}{\PYGZsq{}\PYGZbs{}n\PYGZsq{}}
\end{sphinxVerbatim}

\begin{sphinxVerbatim}[commandchars=\\\{\}]
\PYG{n+nb}{ord}\PYG{p}{(}\PYG{l+s+s1}{\PYGZsq{}}\PYG{l+s+se}{\PYGZbs{}r}\PYG{l+s+s1}{\PYGZsq{}}\PYG{p}{)}  \PYG{c+c1}{\PYGZsh{} Qual é o código unicode para carriage return?}
\end{sphinxVerbatim}

\begin{sphinxVerbatim}[commandchars=\\\{\}]
\PYG{g+go}{13}
\end{sphinxVerbatim}

\begin{sphinxVerbatim}[commandchars=\\\{\}]
\PYG{n+nb}{chr}\PYG{p}{(}\PYG{l+m+mi}{13}\PYG{p}{)}  \PYG{c+c1}{\PYGZsh{} Qual caractere Unicode corresponde ao código 13?}
\end{sphinxVerbatim}

\begin{sphinxVerbatim}[commandchars=\\\{\}]
\PYG{g+go}{\PYGZsq{}\PYGZbs{}r\PYGZsq{}}
\end{sphinxVerbatim}

\begin{sphinxVerbatim}[commandchars=\\\{\}]
\PYG{n+nb}{chr}\PYG{p}{(}\PYG{l+m+mi}{97}\PYG{p}{)}  \PYG{c+c1}{\PYGZsh{} Qual caractere Unicode corresponde ao código 97?}
\end{sphinxVerbatim}

\begin{sphinxVerbatim}[commandchars=\\\{\}]
\PYG{g+go}{\PYGZsq{}a\PYGZsq{}}
\end{sphinxVerbatim}

\begin{sphinxVerbatim}[commandchars=\\\{\}]
\PYG{n+nb}{ord}\PYG{p}{(}\PYG{l+s+s1}{\PYGZsq{}}\PYG{l+s+s1}{a}\PYG{l+s+s1}{\PYGZsq{}}\PYG{p}{)}  \PYG{c+c1}{\PYGZsh{} Qual é o código unicode para o caractere \PYGZdq{}a\PYGZdq{}?}
\end{sphinxVerbatim}

\begin{sphinxVerbatim}[commandchars=\\\{\}]
\PYG{g+go}{97}
\end{sphinxVerbatim}

\begin{sphinxVerbatim}[commandchars=\\\{\}]
\PYG{n+nb}{chr}\PYG{p}{(}\PYG{l+m+mi}{120}\PYG{p}{)}  \PYG{c+c1}{\PYGZsh{} Qual caractere Unicode corresponde ao código 120?}
\end{sphinxVerbatim}

\begin{sphinxVerbatim}[commandchars=\\\{\}]
\PYG{g+go}{\PYGZsq{}x\PYGZsq{}}
\end{sphinxVerbatim}

\begin{sphinxVerbatim}[commandchars=\\\{\}]
\PYG{n+nb}{chr}\PYG{p}{(}\PYG{l+m+mi}{981}\PYG{p}{)}  \PYG{c+c1}{\PYGZsh{} Qual caractere Unicode corresponde ao código 981?}
\end{sphinxVerbatim}

\begin{sphinxVerbatim}[commandchars=\\\{\}]
\PYG{g+go}{\PYGZsq{}ϕ\PYGZsq{}}
\end{sphinxVerbatim}


\section{dir()}
\label{\detokenize{content/built-ins:dir}}\begin{quote}
\begin{quote}

Função que lista atributos e métodos de um elemento.
\end{quote}

Se chamada sem nenhum argumento retorna os nomes do escopo atual.
A chamada dessa função é correspondente ao executar o método \_\_dir\_\_.
\end{quote}

\begin{sphinxVerbatim}[commandchars=\\\{\}]
\PYG{c+c1}{\PYGZsh{} Definição de variáeis}
\PYG{n}{x} \PYG{o}{=} \PYG{l+m+mi}{0}
\PYG{n}{y} \PYG{o}{=} \PYG{l+m+mi}{1}
\PYG{n}{z} \PYG{o}{=} \PYG{l+m+mi}{2}
\end{sphinxVerbatim}

\begin{sphinxVerbatim}[commandchars=\\\{\}]
\PYG{c+c1}{\PYGZsh{} Execução da função dir sem parâmetros}
\PYG{n+nb}{dir}\PYG{p}{(}\PYG{p}{)}
\end{sphinxVerbatim}

\begin{sphinxVerbatim}[commandchars=\\\{\}]
\PYG{g+go}{[\PYGZsq{}In\PYGZsq{},}
\PYG{g+go}{ \PYGZsq{}Out\PYGZsq{},}
\PYG{g+go}{ . . .}
\PYG{g+go}{ \PYGZsq{}x\PYGZsq{},}
\PYG{g+go}{ \PYGZsq{}y\PYGZsq{},}
\PYG{g+go}{ \PYGZsq{}z\PYGZsq{}]}
\end{sphinxVerbatim}

\begin{sphinxVerbatim}[commandchars=\\\{\}]
\PYG{c+c1}{\PYGZsh{} A variável foi declarada no escopo?}
\PYG{l+s+s1}{\PYGZsq{}}\PYG{l+s+s1}{x}\PYG{l+s+s1}{\PYGZsq{}} \PYG{o+ow}{and} \PYG{l+s+s1}{\PYGZsq{}}\PYG{l+s+s1}{y}\PYG{l+s+s1}{\PYGZsq{}} \PYG{o+ow}{and} \PYG{l+s+s1}{\PYGZsq{}}\PYG{l+s+s1}{y}\PYG{l+s+s1}{\PYGZsq{}} \PYG{o+ow}{and} \PYG{l+s+s1}{\PYGZsq{}}\PYG{l+s+s1}{z}\PYG{l+s+s1}{\PYGZsq{}} \PYG{o+ow}{in} \PYG{n+nb}{dir}\PYG{p}{(}\PYG{p}{)}
\end{sphinxVerbatim}

\begin{sphinxVerbatim}[commandchars=\\\{\}]
\PYG{g+go}{True}
\end{sphinxVerbatim}

\begin{sphinxVerbatim}[commandchars=\\\{\}]
\PYG{l+s+s1}{\PYGZsq{}}\PYG{l+s+s1}{w}\PYG{l+s+s1}{\PYGZsq{}} \PYG{o+ow}{in} \PYG{n+nb}{dir}\PYG{p}{(}\PYG{p}{)}
\end{sphinxVerbatim}

\begin{sphinxVerbatim}[commandchars=\\\{\}]
\PYG{g+go}{False}
\end{sphinxVerbatim}

\begin{sphinxVerbatim}[commandchars=\\\{\}]
\PYG{c+c1}{\PYGZsh{} Criação de uma classe}

\PYG{k}{class} \PYG{n+nc}{Pessoa}\PYG{p}{(}\PYG{n+nb}{object}\PYG{p}{)}\PYG{p}{:}
    \PYG{c+c1}{\PYGZsh{} Atributos}
    \PYG{n}{nome} \PYG{o}{=} \PYG{l+s+s1}{\PYGZsq{}}\PYG{l+s+s1}{\PYGZsq{}}
    \PYG{n}{rg} \PYG{o}{=} \PYG{l+s+s1}{\PYGZsq{}}\PYG{l+s+s1}{\PYGZsq{}}
    \PYG{n}{cpf} \PYG{o}{=} \PYG{l+m+mi}{0}
    \PYG{n}{email} \PYG{o}{=} \PYG{l+s+s1}{\PYGZsq{}}\PYG{l+s+s1}{\PYGZsq{}}

    \PYG{c+c1}{\PYGZsh{} Métodos}
    \PYG{k}{def} \PYG{n+nf}{saudacao}\PYG{p}{(}\PYG{n+nb+bp}{self}\PYG{p}{)}\PYG{p}{:}
        \PYG{n+nb}{print}\PYG{p}{(}\PYG{l+s+s1}{\PYGZsq{}}\PYG{l+s+s1}{Olá}\PYG{l+s+s1}{\PYGZsq{}}\PYG{p}{)}

    \PYG{k}{def} \PYG{n+nf}{dizer\PYGZus{}nome}\PYG{p}{(}\PYG{n+nb+bp}{self}\PYG{p}{)}\PYG{p}{:}
        \PYG{n+nb}{print}\PYG{p}{(}\PYG{l+s+s1}{\PYGZsq{}}\PYG{l+s+s1}{Meu nome é }\PYG{l+s+si}{\PYGZob{}\PYGZcb{}}\PYG{l+s+s1}{\PYGZsq{}}\PYG{o}{.}\PYG{n}{format}\PYG{p}{(}\PYG{n+nb+bp}{self}\PYG{o}{.}\PYG{n}{nome}\PYG{p}{)}\PYG{p}{)}
\end{sphinxVerbatim}

\begin{sphinxVerbatim}[commandchars=\\\{\}]
\PYG{c+c1}{\PYGZsh{} Verificando o conteúdo da classe (atributos e métodos)}
 \PYG{n+nb}{dir}\PYG{p}{(}\PYG{n}{Pessoa}\PYG{p}{)}
\end{sphinxVerbatim}

\begin{sphinxVerbatim}[commandchars=\\\{\}]
\PYG{g+go}{[\PYGZsq{}\PYGZus{}\PYGZus{}class\PYGZus{}\PYGZus{}\PYGZsq{},}
\PYG{g+go}{ \PYGZsq{}\PYGZus{}\PYGZus{}delattr\PYGZus{}\PYGZus{}\PYGZsq{},}
\PYG{g+go}{ \PYGZsq{}\PYGZus{}\PYGZus{}dict\PYGZus{}\PYGZus{}\PYGZsq{},}
\PYG{g+go}{ . . .}
\PYG{g+go}{ \PYGZsq{}cpf\PYGZsq{},}
\PYG{g+go}{ \PYGZsq{}dizer\PYGZus{}nome\PYGZsq{},}
\PYG{g+go}{ \PYGZsq{}email\PYGZsq{},}
\PYG{g+go}{ \PYGZsq{}nome\PYGZsq{},}
\PYG{g+go}{ \PYGZsq{}rg\PYGZsq{},}
\PYG{g+go}{ \PYGZsq{}saudacao\PYGZsq{}]}
\end{sphinxVerbatim}

\begin{sphinxVerbatim}[commandchars=\\\{\}]
\PYG{c+c1}{\PYGZsh{} Criação de um objeto da classe e definição de atributos}
\PYG{n}{p} \PYG{o}{=} \PYG{n}{Pessoa}\PYG{p}{(}\PYG{p}{)}
\PYG{n}{p}\PYG{o}{.}\PYG{n}{nome} \PYG{o}{=} \PYG{l+s+s1}{\PYGZsq{}}\PYG{l+s+s1}{Chiquinho}\PYG{l+s+s1}{\PYGZsq{}}
\PYG{n}{p}\PYG{o}{.}\PYG{n}{rg} \PYG{o}{=} \PYG{l+s+s1}{\PYGZsq{}}\PYG{l+s+s1}{00000000}\PYG{l+s+s1}{\PYGZsq{}}
\PYG{n}{p}\PYG{o}{.}\PYG{n}{cpf} \PYG{o}{=} \PYG{l+m+mi}{12345678901}
\PYG{n}{p}\PYG{o}{.}\PYG{n}{email} \PYG{o}{=} \PYG{l+s+s1}{\PYGZsq{}}\PYG{l+s+s1}{chiquinho@chiquinhodasilva.xx}\PYG{l+s+s1}{\PYGZsq{}}
\end{sphinxVerbatim}

\begin{sphinxVerbatim}[commandchars=\\\{\}]
\PYG{c+c1}{\PYGZsh{} Atributo \PYGZus{}\PYGZus{}dict\PYGZus{}\PYGZus{}, é um dicionário que contém os atributos do objeto}
\PYG{n}{p}\PYG{o}{.}\PYG{n+nv+vm}{\PYGZus{}\PYGZus{}dict\PYGZus{}\PYGZus{}}
\end{sphinxVerbatim}

\begin{sphinxVerbatim}[commandchars=\\\{\}]
\PYG{g+go}{\PYGZob{}\PYGZsq{}nome\PYGZsq{}: \PYGZsq{}Chiquinho\PYGZsq{}, \PYGZsq{}rg\PYGZsq{}: \PYGZsq{}00000000\PYGZsq{}, \PYGZsq{}cpf\PYGZsq{}: 12345678901, \PYGZsq{}email\PYGZsq{}: \PYGZsq{}chiquinho@chiquinhodasilva.xx\PYGZsq{}\PYGZcb{}}
\end{sphinxVerbatim}

\begin{sphinxVerbatim}[commandchars=\\\{\}]
\PYG{c+c1}{\PYGZsh{} Pegando o valor do atributo pelo dicionário}
\PYG{n}{p}\PYG{o}{.}\PYG{n+nv+vm}{\PYGZus{}\PYGZus{}dict\PYGZus{}\PYGZus{}}\PYG{p}{[}\PYG{l+s+s1}{\PYGZsq{}}\PYG{l+s+s1}{nome}\PYG{l+s+s1}{\PYGZsq{}}\PYG{p}{]}
\end{sphinxVerbatim}

\begin{sphinxVerbatim}[commandchars=\\\{\}]
\PYG{g+go}{\PYGZsq{}Chiquinho\PYGZsq{}}
\end{sphinxVerbatim}

\begin{sphinxVerbatim}[commandchars=\\\{\}]
\PYG{c+c1}{\PYGZsh{} Com o auxílio da função dir, listar todos os métodos do objeto}
\PYG{k}{def} \PYG{n+nf}{is\PYGZus{}dunder}\PYG{p}{(}\PYG{n}{s}\PYG{p}{)}\PYG{p}{:}
    \PYG{l+s+sd}{\PYGZsq{}\PYGZsq{}\PYGZsq{}}
\PYG{l+s+sd}{    Função auxiliar que retorna True para dunder strings}
\PYG{l+s+sd}{    \PYGZsq{}\PYGZsq{}\PYGZsq{}}

    \PYG{c+c1}{\PYGZsh{} Se começar e terminar com \PYGZdq{}\PYGZus{}\PYGZus{}\PYGZdq{} retornar True}
    \PYG{k}{if} \PYG{n}{s}\PYG{o}{.}\PYG{n}{startswith}\PYG{p}{(}\PYG{l+s+s1}{\PYGZsq{}}\PYG{l+s+s1}{\PYGZus{}\PYGZus{}}\PYG{l+s+s1}{\PYGZsq{}}\PYG{p}{)} \PYG{o+ow}{and} \PYG{n}{s}\PYG{o}{.}\PYG{n}{endswith}\PYG{p}{(}\PYG{l+s+s1}{\PYGZsq{}}\PYG{l+s+s1}{\PYGZus{}\PYGZus{}}\PYG{l+s+s1}{\PYGZsq{}}\PYG{p}{)}\PYG{p}{:}
        \PYG{k}{return} \PYG{k+kc}{True}
    \PYG{k}{else}\PYG{p}{:}
        \PYG{k}{return} \PYG{k+kc}{False}
\end{sphinxVerbatim}

\begin{sphinxVerbatim}[commandchars=\\\{\}]
\PYG{c+c1}{\PYGZsh{} Utilizando a função auxiliar criada criar uma nova função}
\PYG{k}{def} \PYG{n+nf}{mostra\PYGZus{}metodos}\PYG{p}{(}\PYG{n}{objeto}\PYG{p}{)}\PYG{p}{:}
    \PYG{l+s+sd}{\PYGZsq{}\PYGZsq{}\PYGZsq{}}
\PYG{l+s+sd}{    Função que mostra em tela todos os nomes de métodos de um objeto}
\PYG{l+s+sd}{    \PYGZsq{}\PYGZsq{}\PYGZsq{}}

    \PYG{c+c1}{\PYGZsh{} Generator que conterá os nomes dos métodos por tuple comprehension}
    \PYG{n}{metodos} \PYG{o}{=} \PYG{p}{(}\PYG{n}{i} \PYG{k}{for} \PYG{n}{i} \PYG{o+ow}{in} \PYG{n+nb}{dir}\PYG{p}{(}\PYG{n}{objeto}\PYG{p}{)}
               \PYG{k}{if} \PYG{n}{callable}\PYG{p}{(}\PYG{n+nb}{getattr}\PYG{p}{(}\PYG{n}{objeto}\PYG{p}{,} \PYG{n}{i}\PYG{p}{)}\PYG{p}{)}
               \PYG{o+ow}{and} \PYG{p}{(}\PYG{o+ow}{not} \PYG{n}{is\PYGZus{}dunder}\PYG{p}{(}\PYG{n}{i}\PYG{p}{)}\PYG{p}{)}
              \PYG{p}{)}

    \PYG{k}{for} \PYG{n}{i} \PYG{o+ow}{in} \PYG{n}{metodos}\PYG{p}{:}
        \PYG{n+nb}{print}\PYG{p}{(}\PYG{n}{i}\PYG{p}{)}
\end{sphinxVerbatim}

\begin{sphinxVerbatim}[commandchars=\\\{\}]
\PYG{c+c1}{\PYGZsh{} Chamando a função criada para imprimir em tela os nomes dos métodos:}
\PYG{n}{mostra\PYGZus{}metodos}\PYG{p}{(}\PYG{n}{p}\PYG{p}{)}
\end{sphinxVerbatim}

\begin{sphinxVerbatim}[commandchars=\\\{\}]
\PYG{g+go}{dizer\PYGZus{}nome}
\PYG{g+go}{saudacao}
\end{sphinxVerbatim}


\section{pass}
\label{\detokenize{content/built-ins:pass}}\begin{quote}

É um comando de  operação nula, ou seja, quando é executado nada acontece. É útil como um marcador quando um statement é requerido sintaticamente, mas não tem necessidade de um código a ser executado.
\end{quote}

\begin{sphinxVerbatim}[commandchars=\\\{\}]
\PYG{c+c1}{\PYGZsh{} Função que nada faz:}
\PYG{k}{def} \PYG{n+nf}{nula}\PYG{p}{(}\PYG{p}{)}\PYG{p}{:}
    \PYG{k}{pass}
\end{sphinxVerbatim}


\section{assert}
\label{\detokenize{content/built-ins:assert}}\begin{quote}

Um statements assert é uma maneira conveniente para inserir asserções de debug.
O comando assert verifica em tempo de execução uma determinada condição e se a mesma não for verdadira uma exceção AssertionError é lançada e se essa exceção não for tratada, o programa pára.
\end{quote}

\begin{sphinxVerbatim}[commandchars=\\\{\}]
\PYG{c+c1}{\PYGZsh{} Criação do script com assert sem tratamento de exceção}
vim /tmp/assert\PYGZus{}sem\PYGZus{}try.py
\end{sphinxVerbatim}

\begin{sphinxVerbatim}[commandchars=\\\{\},numbers=left,firstnumber=1,stepnumber=1]
\PYG{n+nb}{print}\PYG{p}{(}\PYG{l+s+s1}{\PYGZsq{}}\PYG{l+s+s1}{Começo}\PYG{l+s+s1}{\PYGZsq{}}\PYG{p}{)}

\PYG{k}{assert} \PYG{l+m+mi}{1} \PYG{o}{==} \PYG{l+m+mi}{1}  \PYG{c+c1}{\PYGZsh{} OK}
\PYG{k}{assert} \PYG{l+m+mi}{2} \PYG{o}{==} \PYG{l+m+mi}{1}  \PYG{c+c1}{\PYGZsh{} Ops...}

\PYG{n+nb}{print}\PYG{p}{(}\PYG{l+s+s1}{\PYGZsq{}}\PYG{l+s+s1}{Fim}\PYG{l+s+s1}{\PYGZsq{}}\PYG{p}{)}
\end{sphinxVerbatim}

\begin{sphinxVerbatim}[commandchars=\\\{\}]
\PYG{c+c1}{\PYGZsh{} Execução}
\PYGZdl{} python3 /tmp/assert\PYGZus{}sem\PYGZus{}try.py
\end{sphinxVerbatim}

\begin{sphinxVerbatim}[commandchars=\\\{\}]
\PYG{g+go}{Começo}
\PYG{g+go}{Traceback (most recent call last):}
\PYG{g+go}{. . .}
\PYG{g+go}{AssertionError}
\end{sphinxVerbatim}

Nota\sphinxhyphen{}se que a execução do script não chegou até o fim.

\begin{sphinxVerbatim}[commandchars=\\\{\}]
\PYGZdl{} \PYG{c+c1}{\PYGZsh{} Criação do script com assert com tratamento de exceção}
vim /tmp/assert\PYGZus{}com\PYGZus{}try.py
\end{sphinxVerbatim}

\begin{sphinxVerbatim}[commandchars=\\\{\},numbers=left,firstnumber=1,stepnumber=1]
\PYG{n+nb}{print}\PYG{p}{(}\PYG{l+s+s1}{\PYGZsq{}}\PYG{l+s+s1}{Começo}\PYG{l+s+s1}{\PYGZsq{}}\PYG{p}{)}

\PYG{k}{try}\PYG{p}{:}
    \PYG{k}{assert} \PYG{l+m+mi}{1} \PYG{o}{==} \PYG{l+m+mi}{1}  \PYG{c+c1}{\PYGZsh{} OK}
    \PYG{k}{assert} \PYG{l+m+mi}{2} \PYG{o}{==} \PYG{l+m+mi}{1}  \PYG{c+c1}{\PYGZsh{} Ops...}
\PYG{k}{except} \PYG{n+ne}{AssertionError}\PYG{p}{:}
    \PYG{n+nb}{print}\PYG{p}{(}\PYG{l+s+s1}{\PYGZsq{}}\PYG{l+s+s1}{Teve erro...}\PYG{l+s+s1}{\PYGZsq{}}\PYG{p}{)}

\PYG{n+nb}{print}\PYG{p}{(}\PYG{l+s+s1}{\PYGZsq{}}\PYG{l+s+s1}{Fim}\PYG{l+s+s1}{\PYGZsq{}}\PYG{p}{)}
\end{sphinxVerbatim}

\begin{sphinxVerbatim}[commandchars=\\\{\}]
\PYGZdl{} \PYG{c+c1}{\PYGZsh{} Execução}
python3.7 /tmp/assert\PYGZus{}com\PYGZus{}try.py
\end{sphinxVerbatim}

\begin{sphinxVerbatim}[commandchars=\\\{\}]
\PYG{g+go}{Começo}
\PYG{g+go}{Teve erro...}
\PYG{g+go}{Fim}
\end{sphinxVerbatim}


\section{abs()}
\label{\detokenize{content/built-ins:abs}}\begin{quote}

Retorna o valor absoluto do argumento.
\end{quote}

\begin{sphinxVerbatim}[commandchars=\\\{\}]
\PYG{n+nb}{abs}\PYG{p}{(}\PYG{l+m+mi}{3}\PYG{p}{)}
\end{sphinxVerbatim}

\begin{sphinxVerbatim}[commandchars=\\\{\}]
\PYG{g+go}{3}
\end{sphinxVerbatim}

\begin{sphinxVerbatim}[commandchars=\\\{\}]
\PYG{n+nb}{abs}\PYG{p}{(}\PYG{o}{\PYGZhy{}}\PYG{l+m+mi}{3}\PYG{p}{)}
\end{sphinxVerbatim}

\begin{sphinxVerbatim}[commandchars=\\\{\}]
\PYG{g+go}{3}
\end{sphinxVerbatim}


\section{divmod()}
\label{\detokenize{content/built-ins:divmod}}\begin{quote}

Função que retorna uma tupla de dois elementos no formato (x//y, x\%y), respectivamente resultado da divisão inteira e resto da divisão:
\end{quote}

\begin{sphinxVerbatim}[commandchars=\\\{\}]
\PYG{n+nb}{divmod}\PYG{p}{(}\PYG{l+m+mi}{11}\PYG{p}{,} \PYG{l+m+mi}{4}\PYG{p}{)}  \PYG{c+c1}{\PYGZsh{} Equivalente: 11 // 4, 11 \PYGZpc{} 4}
\end{sphinxVerbatim}

\begin{sphinxVerbatim}[commandchars=\\\{\}]
\PYG{g+gp+gpVirtualEnv}{(2, 3)}
\end{sphinxVerbatim}


\section{round()}
\label{\detokenize{content/built-ins:round}}\begin{quote}

Função que retorna um número de forma arredondada dada uma precisão em dígitos decimais.
O valor de retorno é um inteiro se o número de dígitos for omitido ou None. Caso contrário, o valor de retorno terá o mesmo tipo do número. O número de dígitos pode ser negativo.
\end{quote}

\begin{sphinxVerbatim}[commandchars=\\\{\}]
\PYG{c+c1}{\PYGZsh{} Arredondamento sem especificar o número de dígitos (segundo parâmetro)}
\PYG{n+nb}{round}\PYG{p}{(}\PYG{l+m+mf}{3.333333}\PYG{p}{)}
\end{sphinxVerbatim}

\begin{sphinxVerbatim}[commandchars=\\\{\}]
\PYG{g+go}{3}
\end{sphinxVerbatim}

\begin{sphinxVerbatim}[commandchars=\\\{\}]
\PYG{c+c1}{\PYGZsh{} Arredondamento com quatro dígitos de precisão:}
\PYG{n+nb}{round}\PYG{p}{(}\PYG{l+m+mf}{3.333333}\PYG{p}{,} \PYG{l+m+mi}{4}\PYG{p}{)}
\end{sphinxVerbatim}

\begin{sphinxVerbatim}[commandchars=\\\{\}]
\PYG{g+go}{3.3333}
\end{sphinxVerbatim}

\begin{sphinxVerbatim}[commandchars=\\\{\}]
\PYG{c+c1}{\PYGZsh{} Precisão variando de 1 a \PYGZhy{}3:}
\PYG{n+nb}{round}\PYG{p}{(}\PYG{l+m+mf}{1237.87431}\PYG{p}{,} \PYG{l+m+mi}{1}\PYG{p}{)}
\end{sphinxVerbatim}

\begin{sphinxVerbatim}[commandchars=\\\{\}]
\PYG{g+go}{1237.9}
\end{sphinxVerbatim}

\begin{sphinxVerbatim}[commandchars=\\\{\}]
\PYG{c+c1}{\PYGZsh{}}
\PYG{n+nb}{round}\PYG{p}{(}\PYG{l+m+mf}{1237.87431}\PYG{p}{,} \PYG{l+m+mi}{0}\PYG{p}{)}
\end{sphinxVerbatim}

\begin{sphinxVerbatim}[commandchars=\\\{\}]
\PYG{g+go}{1238.0}
\end{sphinxVerbatim}

\begin{sphinxVerbatim}[commandchars=\\\{\}]
\PYG{c+c1}{\PYGZsh{}}
\PYG{n+nb}{round}\PYG{p}{(}\PYG{l+m+mf}{1237.87431}\PYG{p}{,} \PYG{o}{\PYGZhy{}}\PYG{l+m+mi}{1}\PYG{p}{)}
\end{sphinxVerbatim}

\begin{sphinxVerbatim}[commandchars=\\\{\}]
\PYG{g+go}{1240.0}
\end{sphinxVerbatim}

\begin{sphinxVerbatim}[commandchars=\\\{\}]
\PYG{c+c1}{\PYGZsh{}}
\PYG{n+nb}{round}\PYG{p}{(}\PYG{l+m+mf}{1237.87431}\PYG{p}{,} \PYG{o}{\PYGZhy{}}\PYG{l+m+mi}{2}\PYG{p}{)}
\end{sphinxVerbatim}

\begin{sphinxVerbatim}[commandchars=\\\{\}]
\PYG{g+go}{1200.0}
\end{sphinxVerbatim}

\begin{sphinxVerbatim}[commandchars=\\\{\}]
\PYG{c+c1}{\PYGZsh{}}
\PYG{n+nb}{round}\PYG{p}{(}\PYG{l+m+mf}{1237.87431}\PYG{p}{,} \PYG{o}{\PYGZhy{}}\PYG{l+m+mi}{3}\PYG{p}{)}
\end{sphinxVerbatim}

\begin{sphinxVerbatim}[commandchars=\\\{\}]
\PYG{g+go}{1000.0}
\end{sphinxVerbatim}


\section{callable()}
\label{\detokenize{content/built-ins:callable}}\begin{quote}

Função que retorna True se o objeto é “chamável” (callable) (i. e., algum tipo de função).
Vale lembrar que classes também são chamáveis, bem como objetos de classes que implementam o método \_\_call\_\_().
\end{quote}

\begin{sphinxVerbatim}[commandchars=\\\{\}]
\PYG{c+c1}{\PYGZsh{} Criação de uma função}
\PYG{k}{def} \PYG{n+nf}{myfunction}\PYG{p}{(}\PYG{p}{)}\PYG{p}{:}
    \PYG{k}{pass}

\PYG{c+c1}{\PYGZsh{} Criação de uma classe sem o método \PYGZus{}\PYGZus{}call\PYGZus{}\PYGZus{}()}
\PYG{k}{class} \PYG{n+nc}{Foo}\PYG{p}{:}
    \PYG{k}{pass}

\PYG{c+c1}{\PYGZsh{} Criação de uma classe com o método \PYGZus{}\PYGZus{}call\PYGZus{}\PYGZus{}()}
\PYG{k}{class} \PYG{n+nc}{Bar}\PYG{p}{:}
    \PYG{k}{def} \PYG{n+nf+fm}{\PYGZus{}\PYGZus{}call\PYGZus{}\PYGZus{}}\PYG{p}{(}\PYG{n+nb+bp}{self}\PYG{p}{)}\PYG{p}{:}
        \PYG{k}{pass}

\PYG{c+c1}{\PYGZsh{} Instância da classe sem o método \PYGZus{}\PYGZus{}call\PYGZus{}\PYGZus{}()}
\PYG{n}{f} \PYG{o}{=} \PYG{n}{Foo}\PYG{p}{(}\PYG{p}{)}

\PYG{c+c1}{\PYGZsh{} Instância da classe com o método \PYGZus{}\PYGZus{}call\PYGZus{}\PYGZus{}()}
\PYG{n}{b} \PYG{o}{=} \PYG{n}{Bar}\PYG{p}{(}\PYG{p}{)}

\PYG{c+c1}{\PYGZsh{} Execuções de callable}
\PYG{n}{callable}\PYG{p}{(}\PYG{l+s+s1}{\PYGZsq{}}\PYG{l+s+s1}{foo}\PYG{l+s+s1}{\PYGZsq{}}\PYG{p}{)}
\end{sphinxVerbatim}

\begin{sphinxVerbatim}[commandchars=\\\{\}]
\PYG{g+go}{False}
\end{sphinxVerbatim}

\begin{sphinxVerbatim}[commandchars=\\\{\}]
\PYG{c+c1}{\PYGZsh{}}
\PYG{n}{callable}\PYG{p}{(}\PYG{n}{myfunction}\PYG{p}{)}
\end{sphinxVerbatim}

\begin{sphinxVerbatim}[commandchars=\\\{\}]
\PYG{g+go}{True}
\end{sphinxVerbatim}

\begin{sphinxVerbatim}[commandchars=\\\{\}]
\PYG{c+c1}{\PYGZsh{}}
\PYG{n}{callable}\PYG{p}{(}\PYG{n}{Foo}\PYG{p}{)}
\end{sphinxVerbatim}

\begin{sphinxVerbatim}[commandchars=\\\{\}]
\PYG{g+go}{True}
\end{sphinxVerbatim}

\begin{sphinxVerbatim}[commandchars=\\\{\}]
\PYG{c+c1}{\PYGZsh{}}
\PYG{n}{callable}\PYG{p}{(}\PYG{n}{Bar}\PYG{p}{)}
\end{sphinxVerbatim}

\begin{sphinxVerbatim}[commandchars=\\\{\}]
\PYG{g+go}{True}
\end{sphinxVerbatim}

\begin{sphinxVerbatim}[commandchars=\\\{\}]
\PYG{c+c1}{\PYGZsh{}}
\PYG{n}{callable}\PYG{p}{(}\PYG{n}{f}\PYG{p}{)}
\end{sphinxVerbatim}

\begin{sphinxVerbatim}[commandchars=\\\{\}]
\PYG{g+go}{False}
\end{sphinxVerbatim}

\begin{sphinxVerbatim}[commandchars=\\\{\}]
\PYG{c+c1}{\PYGZsh{}}
\PYG{n}{callable}\PYG{p}{(}\PYG{n}{b}\PYG{p}{)}
\end{sphinxVerbatim}

True


\section{oct()}
\label{\detokenize{content/built-ins:oct}}\begin{quote}

Função que retorna a representação octal de um inteiro.
\end{quote}

\begin{sphinxVerbatim}[commandchars=\\\{\}]
\PYG{c+c1}{\PYGZsh{}}
\PYG{n+nb}{oct}\PYG{p}{(}\PYG{l+m+mi}{9}\PYG{p}{)}
\end{sphinxVerbatim}

\begin{sphinxVerbatim}[commandchars=\\\{\}]
\PYG{g+go}{\PYGZsq{}0o11\PYGZsq{}}
\end{sphinxVerbatim}

\begin{sphinxVerbatim}[commandchars=\\\{\}]
\PYG{c+c1}{\PYGZsh{}}
\PYG{n+nb}{oct}\PYG{p}{(}\PYG{l+m+mi}{10}\PYG{p}{)}
\end{sphinxVerbatim}

\begin{sphinxVerbatim}[commandchars=\\\{\}]
\PYG{g+go}{\PYGZsq{}0o12\PYGZsq{}}
\end{sphinxVerbatim}


\section{hash()}
\label{\detokenize{content/built-ins:hash}}\begin{quote}

Função que retorna o valor hash de um dado objeto.
Dois objetos que são comparados também devem ter o mesmo valor de hash.
\end{quote}

\begin{sphinxVerbatim}[commandchars=\\\{\}]
\PYG{c+c1}{\PYGZsh{} Testes com a função hash}
\PYG{n+nb}{hash}\PYG{p}{(}\PYG{l+m+mi}{1}\PYG{p}{)}  \PYG{c+c1}{\PYGZsh{} O hash de um inteiro vai ser seu próprio valor}
\end{sphinxVerbatim}

\begin{sphinxVerbatim}[commandchars=\\\{\}]
\PYG{g+go}{1}
\end{sphinxVerbatim}

\begin{sphinxVerbatim}[commandchars=\\\{\}]
\PYG{n+nb}{hash}\PYG{p}{(}\PYG{l+m+mi}{2}\PYG{p}{)}
\end{sphinxVerbatim}

\begin{sphinxVerbatim}[commandchars=\\\{\}]
\PYG{g+go}{2}
\end{sphinxVerbatim}

\begin{sphinxVerbatim}[commandchars=\\\{\}]
\PYG{c+c1}{\PYGZsh{} Hash de uma string}
\PYG{n}{x} \PYG{o}{=} \PYG{l+s+s1}{\PYGZsq{}}\PYG{l+s+s1}{foo}\PYG{l+s+s1}{\PYGZsq{}}
\end{sphinxVerbatim}

\begin{sphinxVerbatim}[commandchars=\\\{\}]
\PYG{c+c1}{\PYGZsh{} Hash de uma string}
\PYG{n+nb}{hash}\PYG{p}{(}\PYG{n}{x}\PYG{p}{)}
\end{sphinxVerbatim}

\begin{sphinxVerbatim}[commandchars=\\\{\}]
\PYG{g+go}{8540844669962366372}
\end{sphinxVerbatim}

\begin{sphinxVerbatim}[commandchars=\\\{\}]
\PYG{c+c1}{\PYGZsh{} Nova variável y igual a x}
\PYG{n}{y} \PYG{o}{=} \PYG{n}{x}

\PYG{c+c1}{\PYGZsh{} Por terem o mesmo valor, o hash será igual}
\PYG{n+nb}{hash}\PYG{p}{(}\PYG{n}{x}\PYG{p}{)} \PYG{o}{==} \PYG{n+nb}{hash}\PYG{p}{(}\PYG{n}{y}\PYG{p}{)}
\end{sphinxVerbatim}

\begin{sphinxVerbatim}[commandchars=\\\{\}]
\PYG{g+go}{True}
\end{sphinxVerbatim}

\begin{sphinxVerbatim}[commandchars=\\\{\}]
\PYG{c+c1}{\PYGZsh{} Alguns tipos como list, dict e set são unhashable}
\PYG{n+nb}{hash}\PYG{p}{(}\PYG{p}{[}\PYG{l+m+mi}{1}\PYG{p}{,} \PYG{l+m+mi}{2}\PYG{p}{,} \PYG{l+m+mi}{3}\PYG{p}{]}\PYG{p}{)}
\end{sphinxVerbatim}

\begin{sphinxVerbatim}[commandchars=\\\{\}]
\PYG{g+go}{TypeError: unhashable type: \PYGZsq{}list\PYGZsq{}}
\end{sphinxVerbatim}

\begin{sphinxVerbatim}[commandchars=\\\{\}]
\PYG{c+c1}{\PYGZsh{} Quando um número é muito grande seu hash será diferente de seu valor}
\PYG{n+nb}{hash}\PYG{p}{(}\PYG{l+m+mi}{9999999999999999999}\PYG{p}{)}
\end{sphinxVerbatim}

\begin{sphinxVerbatim}[commandchars=\\\{\}]
\PYG{g+go}{776627963145224195}
\end{sphinxVerbatim}


\section{id()}
\label{\detokenize{content/built-ins:id}}\begin{quote}

É uma função que retorna a identidade de um objeto.
É a garantia que o objeto será único dentre outros.
\end{quote}

\begin{sphinxVerbatim}[commandchars=\\\{\}]
\PYG{c+c1}{\PYGZsh{} Criação de duas tuplas}
\PYG{n}{foo} \PYG{o}{=} \PYG{p}{(}\PYG{l+s+s1}{\PYGZsq{}}\PYG{l+s+s1}{x}\PYG{l+s+s1}{\PYGZsq{}}\PYG{p}{,} \PYG{l+s+s1}{\PYGZsq{}}\PYG{l+s+s1}{y}\PYG{l+s+s1}{\PYGZsq{}}\PYG{p}{)}
\PYG{n}{bar} \PYG{o}{=} \PYG{p}{(}\PYG{l+s+s1}{\PYGZsq{}}\PYG{l+s+s1}{x}\PYG{l+s+s1}{\PYGZsq{}}\PYG{p}{,} \PYG{l+s+s1}{\PYGZsq{}}\PYG{l+s+s1}{y}\PYG{l+s+s1}{\PYGZsq{}}\PYG{p}{)}

\PYG{c+c1}{\PYGZsh{} Comparando as tuplas criadas}
\PYG{n}{foo} \PYG{o}{==} \PYG{n}{bar}
\end{sphinxVerbatim}

\begin{sphinxVerbatim}[commandchars=\\\{\}]
\PYG{g+go}{True}
\end{sphinxVerbatim}

\begin{sphinxVerbatim}[commandchars=\\\{\}]
\PYG{c+c1}{\PYGZsh{} Verificando o a identidade das tuplas criadas}
\PYG{n+nb}{id}\PYG{p}{(}\PYG{n}{foo}\PYG{p}{)}
\end{sphinxVerbatim}

\begin{sphinxVerbatim}[commandchars=\\\{\}]
\PYG{g+go}{139651439554952}
\end{sphinxVerbatim}

\begin{sphinxVerbatim}[commandchars=\\\{\}]
\PYG{c+c1}{\PYGZsh{} id(bar)}
\end{sphinxVerbatim}

\begin{sphinxVerbatim}[commandchars=\\\{\}]
\PYG{g+go}{139651403802056}
\end{sphinxVerbatim}

\begin{sphinxVerbatim}[commandchars=\\\{\}]
\PYG{c+c1}{\PYGZsh{} É o mesmo objeto?}
\PYG{n}{foo} \PYG{o+ow}{is} \PYG{n}{bar}
\end{sphinxVerbatim}

\begin{sphinxVerbatim}[commandchars=\\\{\}]
\PYG{g+go}{False}
\end{sphinxVerbatim}

\begin{sphinxVerbatim}[commandchars=\\\{\}]
\PYG{c+c1}{\PYGZsh{} Criação de uma nova variável atribuindo com base em um objeto pré\PYGZhy{}existente}
\PYG{n}{baz} \PYG{o}{=} \PYG{n}{bar}

\PYG{c+c1}{\PYGZsh{} Comparando as variáveis}
\PYG{n}{baz} \PYG{o}{==} \PYG{n}{bar}
\end{sphinxVerbatim}

\begin{sphinxVerbatim}[commandchars=\\\{\}]
\PYG{g+go}{True}
\end{sphinxVerbatim}

\begin{sphinxVerbatim}[commandchars=\\\{\}]
\PYG{c+c1}{\PYGZsh{} É o mesmo objeto?}
\PYG{n}{baz} \PYG{o+ow}{is} \PYG{n}{bar}
\end{sphinxVerbatim}

\begin{sphinxVerbatim}[commandchars=\\\{\}]
\PYG{g+go}{True}
\end{sphinxVerbatim}

\begin{sphinxVerbatim}[commandchars=\\\{\}]
\PYG{c+c1}{\PYGZsh{}}
\PYG{n+nb}{id}\PYG{p}{(}\PYG{n}{bar}\PYG{p}{)} \PYG{o}{==} \PYG{n+nb}{id}\PYG{p}{(}\PYG{n}{baz}\PYG{p}{)}
\end{sphinxVerbatim}

\begin{sphinxVerbatim}[commandchars=\\\{\}]
\PYG{g+go}{True}
\end{sphinxVerbatim}

Aqui fica demonstrado que quando se cria uma nova variável simplesmente por atribuição é na verdade a criação de uma nova referência (apontamento) para o mesmo objeto.


\section{input}
\label{\detokenize{content/built-ins:input}}\begin{quote}

É uma função de entrada de dados pelo teclado (STDIN), cujos dados são interpretados como string.
Opcionalmente podemos colocar uma mensagem para pedir uma entrada de teclado.
\end{quote}

\begin{sphinxVerbatim}[commandchars=\\\{\}]
\PYG{c+c1}{\PYGZsh{} Entrada de dados sem prompt}
\PYG{n}{foo} \PYG{o}{=} \PYG{n+nb}{input}\PYG{p}{(}\PYG{p}{)}  \PYG{c+c1}{\PYGZsh{} Digite algo...}

\PYG{c+c1}{\PYGZsh{} Imprimindo o valor da variável}
\PYG{n+nb}{print}\PYG{p}{(}\PYG{n}{foo}\PYG{p}{)}
\end{sphinxVerbatim}

\begin{sphinxVerbatim}[commandchars=\\\{\}]
\PYG{g+go}{. . .}
\end{sphinxVerbatim}

\begin{sphinxVerbatim}[commandchars=\\\{\}]
\PYG{c+c1}{\PYGZsh{} Entrada de dados com prompt}
\PYG{n}{foo} \PYG{o}{=} \PYG{n+nb}{input}\PYG{p}{(}\PYG{l+s+s1}{\PYGZsq{}}\PYG{l+s+s1}{Digite uma string qualquer... }\PYG{l+s+s1}{\PYGZsq{}}\PYG{p}{)}
\end{sphinxVerbatim}

\begin{sphinxVerbatim}[commandchars=\\\{\}]
\PYG{g+go}{Digite uma string qualquer...}
\end{sphinxVerbatim}

\begin{sphinxVerbatim}[commandchars=\\\{\}]
\PYG{c+c1}{\PYGZsh{} Imprimindo o valor da variável}
\PYG{n+nb}{print}\PYG{p}{(}\PYG{n}{foo}\PYG{p}{)}
\end{sphinxVerbatim}

\begin{sphinxVerbatim}[commandchars=\\\{\}]
\PYG{g+go}{. . .}
\end{sphinxVerbatim}


\section{min e max}
\label{\detokenize{content/built-ins:min-e-max}}\begin{quote}

Dada uma coleção, seja ela uma lista, tupla, conjunto ou string, as funções min e max trazem, respectivamente, o valor mínimo e o máximo.
\end{quote}

\begin{sphinxVerbatim}[commandchars=\\\{\}]
\PYG{c+c1}{\PYGZsh{} Valor mínimo entre inteiros}
\PYG{n+nb}{min}\PYG{p}{(}\PYG{l+m+mi}{0}\PYG{p}{,} \PYG{l+m+mi}{2}\PYG{p}{,} \PYG{o}{\PYGZhy{}}\PYG{l+m+mi}{50}\PYG{p}{,} \PYG{l+m+mi}{7}\PYG{p}{)}
\end{sphinxVerbatim}

\begin{sphinxVerbatim}[commandchars=\\\{\}]
\PYG{g+go}{\PYGZhy{}50}
\end{sphinxVerbatim}

\begin{sphinxVerbatim}[commandchars=\\\{\}]
\PYG{c+c1}{\PYGZsh{} Valor máximo entre inteiros}
\PYG{n+nb}{max}\PYG{p}{(}\PYG{l+m+mi}{0}\PYG{p}{,} \PYG{l+m+mi}{2}\PYG{p}{,} \PYG{o}{\PYGZhy{}}\PYG{l+m+mi}{50}\PYG{p}{,} \PYG{l+m+mi}{7}\PYG{p}{)}
\end{sphinxVerbatim}

\begin{sphinxVerbatim}[commandchars=\\\{\}]
\PYG{g+go}{7}
\end{sphinxVerbatim}

\begin{sphinxVerbatim}[commandchars=\\\{\}]
\PYG{c+c1}{\PYGZsh{} Para caracteres a ordem alfabética é levada em conta}
\PYG{n+nb}{max}\PYG{p}{(}\PYG{l+s+s1}{\PYGZsq{}}\PYG{l+s+s1}{c}\PYG{l+s+s1}{\PYGZsq{}}\PYG{p}{,} \PYG{l+s+s1}{\PYGZsq{}}\PYG{l+s+s1}{x}\PYG{l+s+s1}{\PYGZsq{}}\PYG{p}{,} \PYG{l+s+s1}{\PYGZsq{}}\PYG{l+s+s1}{k}\PYG{l+s+s1}{\PYGZsq{}}\PYG{p}{)}
\end{sphinxVerbatim}

\begin{sphinxVerbatim}[commandchars=\\\{\}]
\PYG{g+go}{\PYGZsq{}x\PYGZsq{}}
\end{sphinxVerbatim}


\section{enumerate}
\label{\detokenize{content/built-ins:enumerate}}\begin{quote}

Função que retorna um objeto iterável.
\end{quote}

\begin{sphinxVerbatim}[commandchars=\\\{\}]
\PYG{c+c1}{\PYGZsh{} Criação de uma tupla}
\PYG{n}{x} \PYG{o}{=} \PYG{p}{(}\PYG{l+s+s1}{\PYGZsq{}}\PYG{l+s+s1}{verde}\PYG{l+s+s1}{\PYGZsq{}}\PYG{p}{,} \PYG{l+s+s1}{\PYGZsq{}}\PYG{l+s+s1}{azul}\PYG{l+s+s1}{\PYGZsq{}}\PYG{p}{,} \PYG{l+s+s1}{\PYGZsq{}}\PYG{l+s+s1}{amarelo}\PYG{l+s+s1}{\PYGZsq{}}\PYG{p}{)}

\PYG{c+c1}{\PYGZsh{} Criação de um objeto iterável com base na tupla criada anteriormente}
\PYG{n}{y} \PYG{o}{=} \PYG{n+nb}{enumerate}\PYG{p}{(}\PYG{n}{x}\PYG{p}{)}

\PYG{c+c1}{\PYGZsh{} Exibindo o tipo de y}
\PYG{n+nb}{type}\PYG{p}{(}\PYG{n}{y}\PYG{p}{)}
\end{sphinxVerbatim}

\begin{sphinxVerbatim}[commandchars=\\\{\}]
\PYG{g+go}{enumerate}
\end{sphinxVerbatim}

\begin{sphinxVerbatim}[commandchars=\\\{\}]
\PYG{c+c1}{\PYGZsh{} Loop sobre o iterável}
\PYG{k}{for} \PYG{n}{i}\PYG{p}{,} \PYG{n}{j} \PYG{o+ow}{in} \PYG{n}{y}\PYG{p}{:}
    \PYG{n+nb}{print}\PYG{p}{(}\PYG{l+s+s1}{\PYGZsq{}}\PYG{l+s+si}{\PYGZob{}\PYGZcb{}}\PYG{l+s+s1}{ \PYGZhy{} }\PYG{l+s+si}{\PYGZob{}\PYGZcb{}}\PYG{l+s+s1}{\PYGZsq{}}\PYG{o}{.}\PYG{n}{format}\PYG{p}{(}\PYG{n}{i}\PYG{p}{,} \PYG{n}{j}\PYG{p}{)}\PYG{p}{)}
\end{sphinxVerbatim}

\begin{sphinxVerbatim}[commandchars=\\\{\}]
\PYG{g+go}{0 \PYGZhy{} verde}
\PYG{g+go}{1 \PYGZhy{} azul}
\PYG{g+go}{2 \PYGZhy{} amarelo}
\end{sphinxVerbatim}

\begin{sphinxVerbatim}[commandchars=\\\{\}]
\PYG{c+c1}{\PYGZsh{} Criar o iterável novamente}
\PYG{n}{y} \PYG{o}{=} \PYG{n+nb}{enumerate}\PYG{p}{(}\PYG{n}{x}\PYG{p}{)}

\PYG{c+c1}{\PYGZsh{} Método \PYGZus{}\PYGZus{}next\PYGZus{}\PYGZus{}() que traz uma tupla com o índice e o valor}
\PYG{n}{y}\PYG{o}{.} \PYG{n+nf+fm}{\PYGZus{}\PYGZus{}next\PYGZus{}\PYGZus{}}\PYG{p}{(}\PYG{p}{)}
\end{sphinxVerbatim}

\begin{sphinxVerbatim}[commandchars=\\\{\}]
\PYG{g+gp+gpVirtualEnv}{(0, \PYGZsq{}verde\PYGZsq{})}
\end{sphinxVerbatim}

\begin{sphinxVerbatim}[commandchars=\\\{\}]
\PYG{c+c1}{\PYGZsh{}}
\PYG{n}{y}\PYG{o}{.} \PYG{n+nf+fm}{\PYGZus{}\PYGZus{}next\PYGZus{}\PYGZus{}}\PYG{p}{(}\PYG{p}{)}
\end{sphinxVerbatim}

\begin{sphinxVerbatim}[commandchars=\\\{\}]
\PYG{g+gp+gpVirtualEnv}{(1, \PYGZsq{}azul\PYGZsq{})}
\end{sphinxVerbatim}

\begin{sphinxVerbatim}[commandchars=\\\{\}]
\PYG{c+c1}{\PYGZsh{}}
\PYG{n}{y}\PYG{o}{.} \PYG{n+nf+fm}{\PYGZus{}\PYGZus{}next\PYGZus{}\PYGZus{}}\PYG{p}{(}\PYG{p}{)}
\end{sphinxVerbatim}

\begin{sphinxVerbatim}[commandchars=\\\{\}]
\PYG{g+gp+gpVirtualEnv}{(2, \PYGZsq{}amarelo\PYGZsq{})}
\end{sphinxVerbatim}


\chapter{Tipos de Dados}
\label{\detokenize{content/data_types:tipos-de-dados}}\label{\detokenize{content/data_types::doc}}\begin{quote}
\begin{quote}

Os tipos em Python podem ser mutáveis ou imutáveis, ou seja, permitem ou não alterar seu conteúdo.
Todos os tipos de dados são objetos, pois não existem tipos primitivos em Python.
\end{quote}

O tipo de dado de um objeto é determinado em tempo de execução e não há uma declaração explícita como se vê em outras linguagens.
\end{quote}

Criação de um objeto e verificando seu tipo:

\begin{sphinxVerbatim}[commandchars=\\\{\}]
\PYG{c+c1}{\PYGZsh{} Atribuir um inteiro à variável (objeto) \PYGZdq{}x\PYGZdq{}:}
\PYG{n}{x} \PYG{o}{=} \PYG{l+m+mi}{7}

\PYG{c+c1}{\PYGZsh{} Verificar o tipo de \PYGZdq{}x\PYGZdq{}:}
\PYG{n+nb}{type}\PYG{p}{(}\PYG{n}{x}\PYG{p}{)}
\end{sphinxVerbatim}

\begin{sphinxVerbatim}[commandchars=\\\{\}]
\PYG{g+go}{int}
\end{sphinxVerbatim}


\section{Variáveis}
\label{\detokenize{content/data_types:variaveis}}\begin{quote}

São criadas através de atribuição de valor e quando não existem mais referência a elas são destruídas pelo garbage colector.
Seus nomes devem começar por uma letra (não acentuadas) ou por underline “\_”.
\end{quote}


\subsection{Tipagem Dinâmica}
\label{\detokenize{content/data_types:tipagem-dinamica}}\begin{quote}

No mesmo código pode ter objetos diferentes com o mesmo nome.
Como dito anteriormente, o tipo é determinado na execução e um mesmo nome pode ter tipos diferentes ao longo do código, porém na verdade será outro objeto.
\end{quote}

\begin{sphinxVerbatim}[commandchars=\\\{\}]
\PYG{c+c1}{\PYGZsh{} Criação de dois objetos de mesmo nome}
\PYG{n}{x} \PYG{o}{=} \PYG{l+m+mf}{3.7}

\PYG{c+c1}{\PYGZsh{} Exibir o id de \PYGZdq{}x\PYGZdq{}:}
\PYG{n+nb}{id}\PYG{p}{(}\PYG{n}{x}\PYG{p}{)}
\end{sphinxVerbatim}

\begin{sphinxVerbatim}[commandchars=\\\{\}]
\PYG{g+go}{140291958334736}
\end{sphinxVerbatim}

\begin{sphinxVerbatim}[commandchars=\\\{\}]
\PYG{c+c1}{\PYGZsh{} Qual é o tipo de \PYGZdq{}x\PYGZdq{}:}
\PYG{n+nb}{type}\PYG{p}{(}\PYG{n}{x}\PYG{p}{)}
\end{sphinxVerbatim}

\begin{sphinxVerbatim}[commandchars=\\\{\}]
\PYG{g+go}{float}
\end{sphinxVerbatim}

\begin{sphinxVerbatim}[commandchars=\\\{\}]
\PYG{c+c1}{\PYGZsh{} Atribuir uma string à variável \PYGZdq{}x\PYGZdq{}:}
\PYG{n}{x} \PYG{o}{=} \PYG{l+s+s1}{\PYGZsq{}}\PYG{l+s+s1}{foo}\PYG{l+s+s1}{\PYGZsq{}}

\PYG{c+c1}{\PYGZsh{} Verificar a id de \PYGZdq{}x\PYGZdq{}:}
\PYG{n+nb}{id}\PYG{p}{(}\PYG{n}{x}\PYG{p}{)}
\end{sphinxVerbatim}

\begin{sphinxVerbatim}[commandchars=\\\{\}]
\PYG{g+go}{140292017787768}
\end{sphinxVerbatim}

\begin{sphinxVerbatim}[commandchars=\\\{\}]
\PYG{c+c1}{\PYGZsh{} Qual é o tipo de \PYGZdq{}x\PYGZdq{}:}
\PYG{n+nb}{type}\PYG{p}{(}\PYG{n}{x}\PYG{p}{)}
\end{sphinxVerbatim}

\begin{sphinxVerbatim}[commandchars=\\\{\}]
\PYG{g+go}{str}
\end{sphinxVerbatim}

Foram criados dois objetos “x”, sendo o primeiro float e o segundo uma string.
Nota\sphinxhyphen{}se ao redefinir o valor do objeto o mesmo deixou de existir (garbagem collector) criando um novo objeto.


\subsection{Tipagem Forte}
\label{\detokenize{content/data_types:tipagem-forte}}\begin{quote}

A tipagem em Python além de dinâmica ela é forte.
Em casos de operações matemáticas, por exemplo, é necessário fazer um cast para ser possível quando os tipos são incompatíveis.
\end{quote}

\begin{sphinxVerbatim}[commandchars=\\\{\}]
\PYG{c+c1}{\PYGZsh{} Soma entre um um número de ponto flutuante e um inteiro}
\PYG{l+m+mf}{5.0} \PYG{o}{+} \PYG{l+m+mi}{2}
\end{sphinxVerbatim}

\begin{sphinxVerbatim}[commandchars=\\\{\}]
\PYG{g+go}{7.0}
\end{sphinxVerbatim}

\begin{sphinxVerbatim}[commandchars=\\\{\}]
\PYG{c+c1}{\PYGZsh{} Tentativa de soma entre uma string e um inteiro}
\PYG{l+s+s1}{\PYGZsq{}}\PYG{l+s+s1}{5}\PYG{l+s+s1}{\PYGZsq{}} \PYG{o}{+} \PYG{l+m+mi}{2}
\end{sphinxVerbatim}

\begin{sphinxVerbatim}[commandchars=\\\{\}]
\PYG{g+go}{TypeError: must be str, not int}
\end{sphinxVerbatim}

Ocorreu um erro, pois foi feita uma tentativa de somar uma string com um inteiro.
Para isso ser possível é necessário fazer um cast da string para um valor numérico (quando for compatível, é claro).

\begin{sphinxVerbatim}[commandchars=\\\{\}]
\PYG{c+c1}{\PYGZsh{} Soma utilizando cast}
\PYG{n+nb}{int}\PYG{p}{(}\PYG{l+s+s1}{\PYGZsq{}}\PYG{l+s+s1}{5}\PYG{l+s+s1}{\PYGZsq{}}\PYG{p}{)} \PYG{o}{+} \PYG{l+m+mi}{2}
\end{sphinxVerbatim}

\begin{sphinxVerbatim}[commandchars=\\\{\}]
\PYG{g+go}{7}
\end{sphinxVerbatim}


\chapter{Strings}
\label{\detokenize{content/str:strings}}\label{\detokenize{content/str::doc}}\begin{quote}

Em Ciências da Computação chamamos de string um texto, também conhecido como cadeia de caracteres.
Strings representam textos, frases ou palavras.
É um recurso muito caro em termos de recursos computacionais (processamento e memória) e que portanto deve ser utilizado com cuidado, pois escalabilidade é algo que deve ser sempre um fator a ser levado em conta.

\sphinxurl{https://docs.python.org/3/library/string.html}
\end{quote}


\section{Strings em Python}
\label{\detokenize{content/str:strings-em-python}}\begin{quote}

Pode\sphinxhyphen{}se usar tanto entre aspas como entre apóstrofos.
\end{quote}

\begin{sphinxVerbatim}[commandchars=\\\{\}]
\PYG{c+c1}{\PYGZsh{} Declaração de uma variável string utilizando apóstrofos}
\PYG{n}{s1} \PYG{o}{=} \PYG{l+s+s1}{\PYGZsq{}}\PYG{l+s+s1}{string}\PYG{l+s+s1}{\PYGZsq{}}
\end{sphinxVerbatim}

\begin{sphinxVerbatim}[commandchars=\\\{\}]
\PYG{c+c1}{\PYGZsh{} Declaração de uma variável string utilizando aspas}
\PYG{n}{s2} \PYG{o}{=} \PYG{l+s+s2}{\PYGZdq{}}\PYG{l+s+s2}{string}\PYG{l+s+s2}{\PYGZdq{}}
\end{sphinxVerbatim}

\begin{sphinxVerbatim}[commandchars=\\\{\}]
\PYG{c+c1}{\PYGZsh{} Declaração de uma variável string utilizando a função str}
\PYG{n}{s} \PYG{o}{=} \PYG{n+nb}{str}\PYG{p}{(}\PYG{l+s+s1}{\PYGZsq{}}\PYG{l+s+s1}{foo}\PYG{l+s+s1}{\PYGZsq{}}\PYG{p}{)}
\end{sphinxVerbatim}


\section{Apóstrofos ou aspas? Qual devo utilizar?}
\label{\detokenize{content/str:apostrofos-ou-aspas-qual-devo-utilizar}}\begin{quote}

Se não tivessem essas duas opções, se fosse apenas aspas como em outras linguagens, em uma string que precisa ter aspas, seria preciso escapar com a contrabarra desta maneira: “. O que também funcionaria.
Fazer uso de contrabarra para escapar por muitas vezes pode ser um tanto confuso e tornar o código menos legível.
Com a facilidade de se poder utilizar ambos torna o escape desnecessário para a maioria dos casos em que aspas ou apóstrofos façam parte de uma string.
\end{quote}

Dois exemplos com print de strings com aspas e apóstrofos dentro:

\begin{sphinxVerbatim}[commandchars=\\\{\}]
\PYG{n+nb}{print}\PYG{p}{(}\PYG{l+s+s1}{\PYGZsq{}}\PYG{l+s+s1}{Uma string que contém }\PYG{l+s+s1}{\PYGZdq{}}\PYG{l+s+s1}{aspas}\PYG{l+s+s1}{\PYGZdq{}}\PYG{l+s+s1}{ em si}\PYG{l+s+s1}{\PYGZsq{}}\PYG{p}{)}
\end{sphinxVerbatim}

\begin{sphinxVerbatim}[commandchars=\\\{\}]
\PYG{g+go}{Uma string que contém \PYGZdq{}aspas\PYGZdq{} em si}
\end{sphinxVerbatim}

\begin{sphinxVerbatim}[commandchars=\\\{\}]
\PYG{n+nb}{print}\PYG{p}{(}\PYG{l+s+s2}{\PYGZdq{}}\PYG{l+s+s2}{Uma string que contém }\PYG{l+s+s2}{\PYGZsq{}}\PYG{l+s+s2}{apóstrofos}\PYG{l+s+s2}{\PYGZsq{}}\PYG{l+s+s2}{ em si}\PYG{l+s+s2}{\PYGZdq{}}\PYG{p}{)}
\end{sphinxVerbatim}

\begin{sphinxVerbatim}[commandchars=\\\{\}]
\PYG{g+go}{Uma string que contém \PYGZsq{}apóstrofos\PYGZsq{} em si}
\end{sphinxVerbatim}

\begin{sphinxVerbatim}[commandchars=\\\{\}]
\PYG{c+c1}{\PYGZsh{} Um caso clássico é em strings com um comando SQL}
\PYG{n}{sql} \PYG{o}{=} \PYG{l+s+s2}{\PYGZdq{}}\PYG{l+s+s2}{SELECT * FROM tb\PYGZus{}musica WHERE artista = }\PYG{l+s+s2}{\PYGZsq{}}\PYG{l+s+s2}{Mozart}\PYG{l+s+s2}{\PYGZsq{}}\PYG{l+s+s2}{;}\PYG{l+s+s2}{\PYGZdq{}}

\PYG{c+c1}{\PYGZsh{} Exibindo o conteúdo da variável:}
\PYG{n+nb}{print}\PYG{p}{(}\PYG{n}{sql}\PYG{p}{)}
\end{sphinxVerbatim}

\begin{sphinxVerbatim}[commandchars=\\\{\}]
\PYG{g+go}{SELECT * FROM tb\PYGZus{}musica WHERE artista = \PYGZsq{}Mozart\PYGZsq{};}
\end{sphinxVerbatim}


\section{Strings de Múltiplas Linhas}
\label{\detokenize{content/str:strings-de-multiplas-linhas}}
É possível se fazer uma string de múltiplas linhas quando colocamos como fechamento e abertura três apóstrofos ou aspas.

\begin{sphinxVerbatim}[commandchars=\\\{\}]
\PYG{c+c1}{\PYGZsh{} String de múltiplas linhas com triplos apóstrofos:}
\PYG{n}{s1} \PYG{o}{=} \PYG{l+s+s1}{\PYGZsq{}\PYGZsq{}\PYGZsq{}}
\PYG{l+s+s1}{Um}
\PYG{l+s+s1}{exemplo}
\PYG{l+s+s1}{de string}
\PYG{l+s+s1}{de várias}
\PYG{l+s+s1}{linhas}
\PYG{l+s+s1}{\PYGZsq{}\PYGZsq{}\PYGZsq{}}

\PYG{c+c1}{\PYGZsh{} String de múltiplas linhas com triplas aspas:}
\PYG{n}{s2} \PYG{o}{=} \PYG{l+s+s2}{\PYGZdq{}\PYGZdq{}\PYGZdq{}}
\PYG{l+s+s2}{Um}
\PYG{l+s+s2}{exemplo}
\PYG{l+s+s2}{de string}
\PYG{l+s+s2}{de várias}
\PYG{l+s+s2}{linhas}
\PYG{l+s+s2}{\PYGZdq{}\PYGZdq{}\PYGZdq{}}

\PYG{c+c1}{\PYGZsh{} String de múltiplas linhas entre parênteses:}
\PYG{n}{s3} \PYG{o}{=} \PYG{p}{(}\PYG{l+s+s1}{\PYGZsq{}}\PYG{l+s+s1}{Um exemplo de string feito para não ultrapassar os setenta }\PYG{l+s+se}{\PYGZbs{}n}\PYG{l+s+s1}{\PYGZsq{}}
      \PYG{l+s+s1}{\PYGZsq{}}\PYG{l+s+s1}{e nove caracteres da PEP8 (Python Enhancement Proposal), }\PYG{l+s+se}{\PYGZbs{}n}\PYG{l+s+s1}{\PYGZsq{}}
      \PYG{l+s+s1}{\PYGZsq{}}\PYG{l+s+s1}{Proposta de aprimoramento do Python, que visa boas práticas}\PYG{l+s+s1}{\PYGZsq{}}
      \PYG{l+s+s1}{\PYGZsq{}}\PYG{l+s+s1}{ de programação.}\PYG{l+s+s1}{\PYGZsq{}}\PYG{p}{)}

\PYG{c+c1}{\PYGZsh{} Exibindo s3:}
\PYG{n+nb}{print}\PYG{p}{(}\PYG{n}{s3}\PYG{p}{)}
\end{sphinxVerbatim}

\begin{sphinxVerbatim}[commandchars=\\\{\}]
\PYG{g+go}{Um exemplo de string feito para não ultrapassar os setenta}
\PYG{g+go}{e nove caracteres da PEP8 (Python Enhancement Proposal),}
\PYG{g+go}{Proposta de aprimoramento do Python, que visa boas práticas de programação.}
\end{sphinxVerbatim}


\section{Caracteres Especiais}
\label{\detokenize{content/str:caracteres-especiais}}

\begin{savenotes}\sphinxattablestart
\centering
\begin{tabular}[t]{|*{4}{\X{1}{4}|}}
\hline
\sphinxstyletheadfamily 
\begin{DUlineblock}{0em}
\item[] Sequência
\item[] de Escape
\end{DUlineblock}
&\sphinxstyletheadfamily 
Descrição
&\sphinxstyletheadfamily 
\begin{DUlineblock}{0em}
\item[] Exemplo
\item[] (print)
\end{DUlineblock}
&\sphinxstyletheadfamily 
Saída
\\
\hline
\sphinxcode{\sphinxupquote{\textbackslash{}\textbackslash{}}}
&
Imprime uma contrabarra
&
\sphinxcode{\sphinxupquote{\textquotesingle{}\textbackslash{}\textbackslash{}\textquotesingle{}}}
&
\begin{sphinxVerbatimintable}[commandchars=\\\{\}]
\PYG{g+go}{\PYGZbs{}}
\end{sphinxVerbatimintable}
\\
\hline
\sphinxcode{\sphinxupquote{\textbackslash{}\textquotesingle{}}}
&
Imprime um apóstrofo
&
\sphinxcode{\sphinxupquote{\textquotesingle{}\textbackslash{}\textquotesingle{}\textquotesingle{}}}
&
\begin{sphinxVerbatimintable}[commandchars=\\\{\}]
\PYG{g+go}{\PYGZsq{}}
\end{sphinxVerbatimintable}
\\
\hline
\sphinxcode{\sphinxupquote{\textbackslash{}"}}
&
Imprime uma aspa
&
\sphinxcode{\sphinxupquote{"\textbackslash{}""}}
&
\begin{sphinxVerbatimintable}[commandchars=\\\{\}]
\PYG{g+go}{\PYGZdq{}}
\end{sphinxVerbatimintable}
\\
\hline
\sphinxcode{\sphinxupquote{\textbackslash{}a}}
&
ASCII bell (beep)
&
\sphinxcode{\sphinxupquote{\textquotesingle{}\textbackslash{}a\textquotesingle{}}}
&
\begin{sphinxVerbatimintable}[commandchars=\\\{\}]

\end{sphinxVerbatimintable}
\\
\hline
\sphinxcode{\sphinxupquote{\textbackslash{}b}}
&
ASCII backspace (BS) remove o caractere anterior
&
\sphinxcode{\sphinxupquote{\textquotesingle{}Casas\textbackslash{}b\textquotesingle{}}}
&
\begin{sphinxVerbatimintable}[commandchars=\\\{\}]
\PYG{g+go}{Casa}
\end{sphinxVerbatimintable}
\\
\hline
\sphinxcode{\sphinxupquote{\textbackslash{}f}}
&
ASCII formfeed (FF)
&
\sphinxcode{\sphinxupquote{\textquotesingle{}foo\textbackslash{}fbar\textquotesingle{}}}
&
\begin{sphinxVerbatimintable}[commandchars=\\\{\}]
\PYG{g+go}{foo}
\PYG{g+go}{   bar}
\end{sphinxVerbatimintable}
\\
\hline
\sphinxcode{\sphinxupquote{\textbackslash{}n}}
&
ASCII linefeed (LF)
&
\sphinxcode{\sphinxupquote{\textquotesingle{}foo\textbackslash{}nbar\textquotesingle{}}}
&
\begin{sphinxVerbatimintable}[commandchars=\\\{\}]
\PYG{g+go}{foo}
\PYG{g+go}{bar}
\end{sphinxVerbatimintable}
\\
\hline
\sphinxcode{\sphinxupquote{\textbackslash{}r}}
&
ASCII carriage return (CR)
&
\sphinxcode{\sphinxupquote{\textquotesingle{}foo\textbackslash{}rbar\textquotesingle{}}}
&
\begin{sphinxVerbatimintable}[commandchars=\\\{\}]
\PYG{g+go}{bar}
\end{sphinxVerbatimintable}
\\
\hline
\sphinxcode{\sphinxupquote{\textbackslash{}t}}
&
ASCII horizontal tab (TAB) Imprime Tab
&
\sphinxcode{\sphinxupquote{\textquotesingle{}foo\textbackslash{}tbar\textquotesingle{}}}
&
\begin{sphinxVerbatimintable}[commandchars=\\\{\}]
\PYG{g+go}{foo    bar}
\end{sphinxVerbatimintable}
\\
\hline
\sphinxcode{\sphinxupquote{\textbackslash{}v}}
&
ASCII vertical tab (VT)
&
\sphinxcode{\sphinxupquote{\textquotesingle{}foo\textbackslash{}vbar\textquotesingle{}}}
&
\begin{sphinxVerbatimintable}[commandchars=\\\{\}]
\PYG{g+go}{foo}
\PYG{g+go}{   bar}
\end{sphinxVerbatimintable}
\\
\hline
\sphinxcode{\sphinxupquote{\textbackslash{}N\{name\}}}
&
Imprime um caractere da base de dados Unicode
&
\sphinxcode{\sphinxupquote{\textquotesingle{}\textbackslash{}N\{DAGGER\}\textquotesingle{}}}
&
\begin{sphinxVerbatimintable}[commandchars=\\\{\}]
\PYG{g+go}{†}
\end{sphinxVerbatimintable}
\\
\hline
\sphinxcode{\sphinxupquote{\textbackslash{}uxxxx}}
&
Imprime 16\sphinxhyphen{}bit valor hexadecimal de caractere Unicode
&
\sphinxcode{\sphinxupquote{\textquotesingle{}\textbackslash{}u041b\textquotesingle{}}}
&
\begin{sphinxVerbatimintable}[commandchars=\\\{\}]
\PYG{g+go}{Л}
\end{sphinxVerbatimintable}
\\
\hline
\sphinxcode{\sphinxupquote{\textbackslash{}Uxxxxxxxx}}
&
Imprime 16\sphinxhyphen{}bit valor hexadecimal de caractere Unicode
&
\sphinxcode{\sphinxupquote{\textquotesingle{}\textbackslash{}u041b\textquotesingle{}}}
&
\begin{sphinxVerbatimintable}[commandchars=\\\{\}]
\PYG{g+go}{Л}
\end{sphinxVerbatimintable}
\\
\hline
\sphinxcode{\sphinxupquote{\textbackslash{}}}
&
Imprime 32\sphinxhyphen{}bit valor hexadecimal de caractere Unicode
&
\sphinxcode{\sphinxupquote{\textquotesingle{}\textbackslash{}U000001a9\textquotesingle{}}}
&
\begin{sphinxVerbatimintable}[commandchars=\\\{\}]
\PYG{g+go}{Ʃ}
\end{sphinxVerbatimintable}
\\
\hline
\sphinxcode{\sphinxupquote{\textbackslash{}ooo}}
&
Imprime o character baseado em seu valor octal
&
\sphinxcode{\sphinxupquote{\textquotesingle{}\textbackslash{}077\textquotesingle{}}}
&
\begin{sphinxVerbatimintable}[commandchars=\\\{\}]
\PYG{g+go}{?}
\end{sphinxVerbatimintable}
\\
\hline
\sphinxcode{\sphinxupquote{\textbackslash{}xhh}}
&
Imprime o character baseado em seu valor hexadecimal
&
\sphinxcode{\sphinxupquote{\textquotesingle{}1\textbackslash{}xaa\textquotesingle{}}}
&
\begin{sphinxVerbatimintable}[commandchars=\\\{\}]
\PYG{g+go}{1ª}
\end{sphinxVerbatimintable}
\\
\hline
\end{tabular}
\par
\sphinxattableend\end{savenotes}


\section{Formatação}
\label{\detokenize{content/str:formatacao}}\begin{quote}

Há casos que é necessário fazer formatação de strings colocando uma string como um template.
Inicialmente tinha\sphinxhyphen{}se a interpolação que utiliza o sinal de porcentagem (\%), posteriormente foi adicionado o método format.
\end{quote}

\begin{sphinxVerbatim}[commandchars=\\\{\}]
\PYG{c+c1}{\PYGZsh{} Interpolação}
\PYG{l+s+s1}{\PYGZsq{}}\PYG{l+s+si}{\PYGZpc{}s}\PYG{l+s+s1}{ }\PYG{l+s+si}{\PYGZpc{}s}\PYG{l+s+s1}{\PYGZsq{}} \PYG{o}{\PYGZpc{}} \PYG{p}{(}\PYG{l+s+s1}{\PYGZsq{}}\PYG{l+s+s1}{foo}\PYG{l+s+s1}{\PYGZsq{}}\PYG{p}{,} \PYG{l+s+s1}{\PYGZsq{}}\PYG{l+s+s1}{bar}\PYG{l+s+s1}{\PYGZsq{}}\PYG{p}{)}
\end{sphinxVerbatim}

ou

\begin{sphinxVerbatim}[commandchars=\\\{\}]
\PYG{c+c1}{\PYGZsh{} Método format}
\PYG{l+s+s1}{\PYGZsq{}}\PYG{l+s+si}{\PYGZob{}\PYGZcb{}}\PYG{l+s+s1}{ }\PYG{l+s+si}{\PYGZob{}\PYGZcb{}}\PYG{l+s+s1}{\PYGZsq{}}\PYG{o}{.}\PYG{n}{format}\PYG{p}{(}\PYG{l+s+s1}{\PYGZsq{}}\PYG{l+s+s1}{foo}\PYG{l+s+s1}{\PYGZsq{}}\PYG{p}{,} \PYG{l+s+s1}{\PYGZsq{}}\PYG{l+s+s1}{bar}\PYG{l+s+s1}{\PYGZsq{}}\PYG{p}{)}
\end{sphinxVerbatim}

\begin{sphinxVerbatim}[commandchars=\\\{\}]
\PYG{g+go}{\PYGZsq{}foo bar\PYGZsq{}}
\end{sphinxVerbatim}

\begin{sphinxVerbatim}[commandchars=\\\{\}]
\PYG{c+c1}{\PYGZsh{} Valores numéricos decimais (interpolação)}
\PYG{l+s+s1}{\PYGZsq{}}\PYG{l+s+si}{\PYGZpc{}d}\PYG{l+s+s1}{ }\PYG{l+s+si}{\PYGZpc{}d}\PYG{l+s+s1}{\PYGZsq{}} \PYG{o}{\PYGZpc{}} \PYG{p}{(}\PYG{l+m+mi}{70}\PYG{p}{,} \PYG{l+m+mi}{90}\PYG{p}{)}
\end{sphinxVerbatim}

ou

\begin{sphinxVerbatim}[commandchars=\\\{\}]
\PYG{c+c1}{\PYGZsh{} Valores numéricos decimais (método format)}
\PYG{l+s+s1}{\PYGZsq{}}\PYG{l+s+si}{\PYGZob{}\PYGZcb{}}\PYG{l+s+s1}{ }\PYG{l+s+si}{\PYGZob{}\PYGZcb{}}\PYG{l+s+s1}{\PYGZsq{}}\PYG{o}{.}\PYG{n}{format}\PYG{p}{(}\PYG{l+m+mi}{70}\PYG{p}{,} \PYG{l+m+mi}{90}\PYG{p}{)}
\end{sphinxVerbatim}

\begin{sphinxVerbatim}[commandchars=\\\{\}]
\PYG{g+go}{\PYGZsq{}70 90\PYGZsq{}}
\end{sphinxVerbatim}

\begin{sphinxVerbatim}[commandchars=\\\{\}]
\PYG{c+c1}{\PYGZsh{} Interpolação pegando o valor de um dicionário}
\PYG{n+nb}{print}\PYG{p}{(}\PYG{l+s+s1}{\PYGZsq{}}\PYG{l+s+si}{\PYGZpc{}(variavel)s}\PYG{l+s+s1}{\PYGZsq{}} \PYG{o}{\PYGZpc{}} \PYG{p}{\PYGZob{}}\PYG{l+s+s1}{\PYGZsq{}}\PYG{l+s+s1}{variavel}\PYG{l+s+s1}{\PYGZsq{}}\PYG{p}{:} \PYG{l+s+s1}{\PYGZsq{}}\PYG{l+s+s1}{valor}\PYG{l+s+s1}{\PYGZsq{}}\PYG{p}{\PYGZcb{}}\PYG{p}{)}
\end{sphinxVerbatim}

\begin{sphinxVerbatim}[commandchars=\\\{\}]
\PYG{g+go}{valor}
\end{sphinxVerbatim}

\begin{sphinxVerbatim}[commandchars=\\\{\}]
\PYG{c+c1}{\PYGZsh{} Variável que vai receber os valores formatados}
\PYG{n}{foo} \PYG{o}{=} \PYG{l+s+s1}{\PYGZsq{}\PYGZsq{}\PYGZsq{}}\PYG{l+s+s1}{Produto: }\PYG{l+s+si}{\PYGZpc{}(prod)s}
\PYG{l+s+s1}{      Preco: R\PYGZdl{} }\PYG{l+s+si}{\PYGZpc{}(preco).2f}
\PYG{l+s+s1}{      Cód: }\PYG{l+s+si}{\PYGZpc{}(cod)05d}
\PYG{l+s+s1}{      }\PYG{l+s+s1}{\PYGZsq{}\PYGZsq{}\PYGZsq{}}

\PYG{c+c1}{\PYGZsh{} Declaração de um dicionário que conterá as chaves e valores desejados}
\PYG{n}{d} \PYG{o}{=} \PYG{p}{\PYGZob{}}\PYG{l+s+s1}{\PYGZsq{}}\PYG{l+s+s1}{prod}\PYG{l+s+s1}{\PYGZsq{}}\PYG{p}{:} \PYG{l+s+s1}{\PYGZsq{}}\PYG{l+s+s1}{Pente}\PYG{l+s+s1}{\PYGZsq{}}\PYG{p}{,} \PYG{l+s+s1}{\PYGZsq{}}\PYG{l+s+s1}{preco}\PYG{l+s+s1}{\PYGZsq{}}\PYG{p}{:} \PYG{l+m+mf}{3.5}\PYG{p}{,} \PYG{l+s+s1}{\PYGZsq{}}\PYG{l+s+s1}{cod}\PYG{l+s+s1}{\PYGZsq{}}\PYG{p}{:} \PYG{l+m+mi}{157}\PYG{p}{\PYGZcb{}}

\PYG{c+c1}{\PYGZsh{} Exibindo o resultado via interpolação}
\PYG{n+nb}{print}\PYG{p}{(}\PYG{n}{foo} \PYG{o}{\PYGZpc{}} \PYG{n}{d}\PYG{p}{)}
\end{sphinxVerbatim}

\begin{sphinxVerbatim}[commandchars=\\\{\}]
\PYG{g+go}{Produto: Pente}
\PYG{g+go}{Preco: R\PYGZdl{} 3.50}
\PYG{g+go}{Cód: 00157}
\end{sphinxVerbatim}

\begin{sphinxVerbatim}[commandchars=\\\{\}]
\PYG{c+c1}{\PYGZsh{} Exibindo o resultado via método format}
\PYG{n+nb}{print}\PYG{p}{(}\PYG{n}{foo}\PYG{o}{.}\PYG{n}{format}\PYG{p}{(}\PYG{o}{*}\PYG{o}{*}\PYG{n}{d}\PYG{p}{)}\PYG{p}{)}
\end{sphinxVerbatim}

\begin{sphinxVerbatim}[commandchars=\\\{\}]
\PYG{g+go}{Produto: Pente}
\PYG{g+go}{Preco: R\PYGZdl{} 3.50}
\PYG{g+go}{Cód: 00157}
\end{sphinxVerbatim}

\begin{sphinxVerbatim}[commandchars=\\\{\}]
\PYG{c+c1}{\PYGZsh{} String com índice posicional}
\PYG{l+s+s1}{\PYGZsq{}}\PYG{l+s+s1}{O }\PYG{l+s+si}{\PYGZob{}1\PYGZcb{}}\PYG{l+s+s1}{ }\PYG{l+s+si}{\PYGZob{}2\PYGZcb{}}\PYG{l+s+s1}{ quando é }\PYG{l+s+si}{\PYGZob{}0\PYGZcb{}}\PYG{l+s+s1}{.}\PYG{l+s+s1}{\PYGZsq{}}\PYG{o}{.}\PYG{n}{format}\PYG{p}{(}\PYG{l+s+s1}{\PYGZsq{}}\PYG{l+s+s1}{compartilhado}\PYG{l+s+s1}{\PYGZsq{}}\PYG{p}{,} \PYG{l+s+s1}{\PYGZsq{}}\PYG{l+s+s1}{conhecimento}\PYG{l+s+s1}{\PYGZsq{}}\PYG{p}{,} \PYG{l+s+s1}{\PYGZsq{}}\PYG{l+s+s1}{aumenta}\PYG{l+s+s1}{\PYGZsq{}}\PYG{p}{)}
\end{sphinxVerbatim}

\begin{sphinxVerbatim}[commandchars=\\\{\}]
\PYG{g+go}{\PYGZsq{}O Conhecimento aumenta quando se compartilhado\PYGZsq{}}
\end{sphinxVerbatim}


\section{Métodos String e de Representação}
\label{\detokenize{content/str:metodos-string-e-de-representacao}}\begin{quote}

Em objetos temos os dunders str e repr (“\_\_str\_\_” e “\_\_repr\_\_”) que podem ser usados em uma string.
\end{quote}

\begin{sphinxVerbatim}[commandchars=\\\{\}]
\PYG{c+c1}{\PYGZsh{} Criação de uma classe de exemplo}
\PYG{k}{class} \PYG{n+nc}{Foo}\PYG{p}{(}\PYG{n+nb}{object}\PYG{p}{)}\PYG{p}{:}

    \PYG{k}{def} \PYG{n+nf+fm}{\PYGZus{}\PYGZus{}str\PYGZus{}\PYGZus{}}\PYG{p}{(}\PYG{n+nb+bp}{self}\PYG{p}{)}\PYG{p}{:}
        \PYG{k}{return} \PYG{l+s+s1}{\PYGZsq{}}\PYG{l+s+s1}{STRING}\PYG{l+s+s1}{\PYGZsq{}}

    \PYG{k}{def} \PYG{n+nf+fm}{\PYGZus{}\PYGZus{}repr\PYGZus{}\PYGZus{}}\PYG{p}{(}\PYG{n+nb+bp}{self}\PYG{p}{)}\PYG{p}{:}
        \PYG{k}{return} \PYG{l+s+s1}{\PYGZsq{}}\PYG{l+s+s1}{REPRESENTAÇÃO}\PYG{l+s+s1}{\PYGZsq{}}


\PYG{c+c1}{\PYGZsh{} Valores dos métodos \PYGZus{}\PYGZus{}str\PYGZus{}\PYGZus{} e \PYGZus{}\PYGZus{}repr\PYGZus{}\PYGZus{} da classe Foo}
\PYG{l+s+s1}{\PYGZsq{}}\PYG{l+s+si}{\PYGZpc{}s}\PYG{l+s+s1}{ }\PYG{l+s+si}{\PYGZpc{}r}\PYG{l+s+s1}{\PYGZsq{}} \PYG{o}{\PYGZpc{}} \PYG{p}{(}\PYG{n}{Foo}\PYG{p}{(}\PYG{p}{)}\PYG{p}{,} \PYG{n}{Foo}\PYG{p}{(}\PYG{p}{)}\PYG{p}{)}  \PYG{c+c1}{\PYGZsh{} interpolação}
\PYG{l+s+s1}{\PYGZsq{}}\PYG{l+s+si}{\PYGZob{}0!s\PYGZcb{}}\PYG{l+s+s1}{ }\PYG{l+s+si}{\PYGZob{}0!r\PYGZcb{}}\PYG{l+s+s1}{\PYGZsq{}}\PYG{o}{.}\PYG{n}{format}\PYG{p}{(}\PYG{n}{Foo}\PYG{p}{(}\PYG{p}{)}\PYG{p}{)}  \PYG{c+c1}{\PYGZsh{} format}
\end{sphinxVerbatim}

\begin{sphinxVerbatim}[commandchars=\\\{\}]
\PYG{g+go}{\PYGZsq{}STRING REPRESENTAÇÃO\PYGZsq{}}
\end{sphinxVerbatim}

\begin{sphinxVerbatim}[commandchars=\\\{\}]
\PYG{c+c1}{\PYGZsh{} Método de representação e em caracteres ASCII}
\PYG{l+s+s1}{\PYGZsq{}}\PYG{l+s+si}{\PYGZpc{}r}\PYG{l+s+s1}{ }\PYG{l+s+si}{\PYGZpc{}a}\PYG{l+s+s1}{\PYGZsq{}} \PYG{o}{\PYGZpc{}} \PYG{p}{(}\PYG{n}{Foo}\PYG{p}{(}\PYG{p}{)}\PYG{p}{,} \PYG{n}{Foo}\PYG{p}{(}\PYG{p}{)}\PYG{p}{)}  \PYG{c+c1}{\PYGZsh{} interpolação}
\PYG{l+s+s1}{\PYGZsq{}}\PYG{l+s+si}{\PYGZob{}0!r\PYGZcb{}}\PYG{l+s+s1}{ }\PYG{l+s+si}{\PYGZob{}0!a\PYGZcb{}}\PYG{l+s+s1}{\PYGZsq{}}\PYG{o}{.}\PYG{n}{format}\PYG{p}{(}\PYG{n}{Foo}\PYG{p}{(}\PYG{p}{)}\PYG{p}{)}  \PYG{c+c1}{\PYGZsh{} format}
\end{sphinxVerbatim}

\begin{sphinxVerbatim}[commandchars=\\\{\}]
\PYG{g+go}{\PYGZsq{}REPRESENTAÇÃO REPRESENTA\PYGZbs{}\PYGZbs{}xc7\PYGZbs{}\PYGZbs{}xc3O\PYGZsq{}}
\end{sphinxVerbatim}


\section{Preenchimento (padding) e Alinhamento de Strings}
\label{\detokenize{content/str:preenchimento-padding-e-alinhamento-de-strings}}
\begin{sphinxVerbatim}[commandchars=\\\{\}]
\PYG{c+c1}{\PYGZsh{} Alinhamento à direita dentro de 7 colunas}
\PYG{l+s+s1}{\PYGZsq{}}\PYG{l+s+si}{\PYGZpc{}7s}\PYG{l+s+s1}{\PYGZsq{}} \PYG{o}{\PYGZpc{}} \PYG{l+s+s1}{\PYGZsq{}}\PYG{l+s+s1}{foo}\PYG{l+s+s1}{\PYGZsq{}}  \PYG{c+c1}{\PYGZsh{} interpolação}
\PYG{l+s+s1}{\PYGZsq{}}\PYG{l+s+si}{\PYGZob{}:\PYGZgt{}7\PYGZcb{}}\PYG{l+s+s1}{\PYGZsq{}}\PYG{o}{.}\PYG{n}{format}\PYG{p}{(}\PYG{l+s+s1}{\PYGZsq{}}\PYG{l+s+s1}{foo}\PYG{l+s+s1}{\PYGZsq{}}\PYG{p}{)}  \PYG{c+c1}{\PYGZsh{} format}
\end{sphinxVerbatim}

\begin{sphinxVerbatim}[commandchars=\\\{\}]
\PYG{g+go}{\PYGZsq{}    foo\PYGZsq{}}
\end{sphinxVerbatim}

\begin{sphinxVerbatim}[commandchars=\\\{\}]
\PYG{c+c1}{\PYGZsh{} Alinhamento à esquerda dentro de 7 colunas}
\PYG{l+s+s1}{\PYGZsq{}}\PYG{l+s+si}{\PYGZpc{}\PYGZhy{}7s}\PYG{l+s+s1}{\PYGZsq{}} \PYG{o}{\PYGZpc{}} \PYG{l+s+s1}{\PYGZsq{}}\PYG{l+s+s1}{foo}\PYG{l+s+s1}{\PYGZsq{}}  \PYG{c+c1}{\PYGZsh{} interpolação}
\PYG{l+s+s1}{\PYGZsq{}}\PYG{l+s+si}{\PYGZob{}:7\PYGZcb{}}\PYG{l+s+s1}{\PYGZsq{}}\PYG{o}{.}\PYG{n}{format}\PYG{p}{(}\PYG{l+s+s1}{\PYGZsq{}}\PYG{l+s+s1}{foo}\PYG{l+s+s1}{\PYGZsq{}}\PYG{p}{)}  \PYG{c+c1}{\PYGZsh{} format}
\PYG{l+s+s1}{\PYGZsq{}}\PYG{l+s+si}{\PYGZob{}:\PYGZlt{}7\PYGZcb{}}\PYG{l+s+s1}{\PYGZsq{}}\PYG{o}{.}\PYG{n}{format}\PYG{p}{(}\PYG{l+s+s1}{\PYGZsq{}}\PYG{l+s+s1}{foo}\PYG{l+s+s1}{\PYGZsq{}}\PYG{p}{)}
\end{sphinxVerbatim}

\begin{sphinxVerbatim}[commandchars=\\\{\}]
\PYG{g+go}{\PYGZsq{}foo    \PYGZsq{}}
\end{sphinxVerbatim}

\begin{sphinxVerbatim}[commandchars=\\\{\}]
\PYG{c+c1}{\PYGZsh{} Alinhamento centralizado dentro de 7 colunas}
\PYG{l+s+s1}{\PYGZsq{}}\PYG{l+s+si}{\PYGZob{}:\PYGZca{}7\PYGZcb{}}\PYG{l+s+s1}{\PYGZsq{}}\PYG{o}{.}\PYG{n}{format}\PYG{p}{(}\PYG{l+s+s1}{\PYGZsq{}}\PYG{l+s+s1}{foo}\PYG{l+s+s1}{\PYGZsq{}}\PYG{p}{)}
\end{sphinxVerbatim}

\begin{sphinxVerbatim}[commandchars=\\\{\}]
\PYG{g+go}{\PYGZsq{}  foo  \PYGZsq{}}
\end{sphinxVerbatim}

\begin{sphinxVerbatim}[commandchars=\\\{\}]
\PYG{c+c1}{\PYGZsh{} Alinhamento à esquerda dentro de 7 colunas preenchendo com o caractere \PYGZdq{}\PYGZus{}\PYGZdq{}}
\PYG{l+s+s1}{\PYGZsq{}}\PYG{l+s+si}{\PYGZob{}:\PYGZus{}\PYGZlt{}7\PYGZcb{}}\PYG{l+s+s1}{\PYGZsq{}}\PYG{o}{.}\PYG{n}{format}\PYG{p}{(}\PYG{l+s+s1}{\PYGZsq{}}\PYG{l+s+s1}{foo}\PYG{l+s+s1}{\PYGZsq{}}\PYG{p}{)}
\end{sphinxVerbatim}

\begin{sphinxVerbatim}[commandchars=\\\{\}]
\PYG{g+go}{\PYGZsq{}foo\PYGZus{}\PYGZus{}\PYGZus{}\PYGZus{}\PYGZsq{}}
\end{sphinxVerbatim}

\begin{sphinxVerbatim}[commandchars=\\\{\}]
\PYG{c+c1}{\PYGZsh{} Alinhamento à direita dentro de 7 colunas preenchendo com o caractere \PYGZdq{}\PYGZus{}\PYGZdq{}}
\end{sphinxVerbatim}

\textgreater{} ‘\{:\_\textgreater{}7\}’.format(‘foo’)

\begin{sphinxVerbatim}[commandchars=\\\{\}]
\PYG{g+go}{\PYGZsq{}\PYGZus{}\PYGZus{}\PYGZus{}\PYGZus{}foo\PYGZsq{}}
\end{sphinxVerbatim}

\begin{sphinxVerbatim}[commandchars=\\\{\}]
\PYG{c+c1}{\PYGZsh{} Alinhamento centralizado dentro de 7 colunas preenchendo com o caractere \PYGZdq{}\PYGZus{}\PYGZdq{}}
\PYG{l+s+s1}{\PYGZsq{}}\PYG{l+s+si}{\PYGZob{}:\PYGZca{}7\PYGZcb{}}\PYG{l+s+s1}{\PYGZsq{}}\PYG{o}{.}\PYG{n}{format}\PYG{p}{(}\PYG{l+s+s1}{\PYGZsq{}}\PYG{l+s+s1}{foo}\PYG{l+s+s1}{\PYGZsq{}}\PYG{p}{)}
\end{sphinxVerbatim}

\begin{sphinxVerbatim}[commandchars=\\\{\}]
\PYG{g+go}{\PYGZsq{}\PYGZus{}\PYGZus{}foo\PYGZus{}\PYGZus{}\PYGZsq{}}
\end{sphinxVerbatim}

\begin{sphinxVerbatim}[commandchars=\\\{\}]
\PYG{c+c1}{\PYGZsh{} Número decimal}
\PYG{l+s+s1}{\PYGZsq{}}\PYG{l+s+si}{\PYGZob{}:.3f\PYGZcb{}}\PYG{l+s+s1}{\PYGZsq{}}\PYG{o}{.}\PYG{n}{format}\PYG{p}{(}\PYG{l+m+mf}{93.85741}\PYG{p}{)}
\end{sphinxVerbatim}

\begin{sphinxVerbatim}[commandchars=\\\{\}]
\PYG{g+go}{\PYGZsq{}93.857\PYGZsq{}}
\end{sphinxVerbatim}

\begin{sphinxVerbatim}[commandchars=\\\{\}]
\PYG{c+c1}{\PYGZsh{}}
\PYG{l+s+s1}{\PYGZsq{}}\PYG{l+s+si}{\PYGZob{}:.3f\PYGZcb{}}\PYG{l+s+s1}{\PYGZsq{}}\PYG{o}{.}\PYG{n}{format}\PYG{p}{(}\PYG{l+m+mi}{70000}\PYG{p}{)}
\end{sphinxVerbatim}

\begin{sphinxVerbatim}[commandchars=\\\{\}]
\PYG{g+go}{\PYGZsq{}70000.000\PYGZsq{}}
\end{sphinxVerbatim}


\section{Representações de Inteiros}
\label{\detokenize{content/str:representacoes-de-inteiros}}
\begin{sphinxVerbatim}[commandchars=\\\{\}]
\PYG{c+c1}{\PYGZsh{} b) Formato binário; número de saída na base 2}
\PYG{n+nb}{format}\PYG{p}{(}\PYG{l+m+mi}{10}\PYG{p}{,} \PYG{l+s+s1}{\PYGZsq{}}\PYG{l+s+s1}{\PYGZsh{}05b}\PYG{l+s+s1}{\PYGZsq{}}\PYG{p}{)}
\end{sphinxVerbatim}

\begin{sphinxVerbatim}[commandchars=\\\{\}]
\PYG{g+go}{\PYGZsq{}0b1010\PYGZsq{}}
\end{sphinxVerbatim}

\begin{sphinxVerbatim}[commandchars=\\\{\}]
\PYG{c+c1}{\PYGZsh{} c) Caractere; converte o inteiro para o caractere unicode correspondente}
\PYG{n+nb}{format}\PYG{p}{(}\PYG{l+m+mi}{93}\PYG{p}{,} \PYG{l+s+s1}{\PYGZsq{}}\PYG{l+s+s1}{c}\PYG{l+s+s1}{\PYGZsq{}}\PYG{p}{)}
\end{sphinxVerbatim}

\begin{sphinxVerbatim}[commandchars=\\\{\}]
\PYG{g+go}{\PYGZsq{}]\PYGZsq{}}
\end{sphinxVerbatim}

\begin{sphinxVerbatim}[commandchars=\\\{\}]
\PYG{c+c1}{\PYGZsh{} d) Inteiro Decimal; saída numérica na base 10 (decimal)}
\PYG{n+nb}{format}\PYG{p}{(}\PYG{l+m+mb}{0b111}\PYG{p}{,} \PYG{l+s+s1}{\PYGZsq{}}\PYG{l+s+s1}{\PYGZsh{}05d}\PYG{l+s+s1}{\PYGZsq{}}\PYG{p}{)}
\end{sphinxVerbatim}

\begin{sphinxVerbatim}[commandchars=\\\{\}]
\PYG{g+go}{\PYGZsq{}00007\PYGZsq{}}
\end{sphinxVerbatim}

\begin{sphinxVerbatim}[commandchars=\\\{\}]
\PYG{c+c1}{\PYGZsh{} o) Formato Octal; saída numérica na base 8 (octal)}
\PYG{n+nb}{format}\PYG{p}{(}\PYG{l+m+mi}{9}\PYG{p}{,} \PYG{l+s+s1}{\PYGZsq{}}\PYG{l+s+s1}{\PYGZsh{}05o}\PYG{l+s+s1}{\PYGZsq{}}\PYG{p}{)}
\end{sphinxVerbatim}

\begin{sphinxVerbatim}[commandchars=\\\{\}]
\PYG{g+go}{\PYGZsq{}0o011\PYGZsq{}}
\end{sphinxVerbatim}

\begin{sphinxVerbatim}[commandchars=\\\{\}]
\PYG{c+c1}{\PYGZsh{} x ou X) Formato Hexadecimal; saída numérica na base 16 (hexadecimal),}
\PYG{c+c1}{\PYGZsh{} a saída é conforme o \PYGZdq{}x\PYGZdq{} maiúsculo ou minúsculo}
\PYG{n+nb}{format}\PYG{p}{(}\PYG{l+m+mi}{200}\PYG{p}{,} \PYG{l+s+s1}{\PYGZsq{}}\PYG{l+s+s1}{\PYGZsh{}05x}\PYG{l+s+s1}{\PYGZsq{}}\PYG{p}{)}
\end{sphinxVerbatim}

\begin{sphinxVerbatim}[commandchars=\\\{\}]
\PYG{g+go}{\PYGZsq{}0x0c8\PYGZsq{}}
\end{sphinxVerbatim}

\begin{sphinxVerbatim}[commandchars=\\\{\}]
\PYG{n+nb}{format}\PYG{p}{(}\PYG{l+m+mi}{200}\PYG{p}{,} \PYG{l+s+s1}{\PYGZsq{}}\PYG{l+s+s1}{\PYGZsh{}05X}\PYG{l+s+s1}{\PYGZsq{}}\PYG{p}{)}
\end{sphinxVerbatim}

\begin{sphinxVerbatim}[commandchars=\\\{\}]
\PYG{g+go}{\PYGZsq{}0X0C8\PYGZsq{}}
\end{sphinxVerbatim}

\begin{sphinxVerbatim}[commandchars=\\\{\}]
\PYG{c+c1}{\PYGZsh{} n) Numérico; o mesmo que \PYGZdq{}d\PYGZdq{}, exceto que ele usa as configurações}
\PYG{c+c1}{\PYGZsh{} de idioma (locale) para exibir caracteres}
\PYG{n+nb}{format}\PYG{p}{(}\PYG{l+m+mf}{31259.74}\PYG{p}{,} \PYG{l+s+s1}{\PYGZsq{}}\PYG{l+s+s1}{n}\PYG{l+s+s1}{\PYGZsq{}}\PYG{p}{)}
\end{sphinxVerbatim}

\begin{sphinxVerbatim}[commandchars=\\\{\}]
\PYG{g+go}{\PYGZsq{}31259.7\PYGZsq{}}
\end{sphinxVerbatim}

\begin{sphinxVerbatim}[commandchars=\\\{\}]
\PYG{n+nb}{format}\PYG{p}{(}\PYG{l+m+mf}{31259.75}\PYG{p}{,} \PYG{l+s+s1}{\PYGZsq{}}\PYG{l+s+s1}{n}\PYG{l+s+s1}{\PYGZsq{}}\PYG{p}{)}
\end{sphinxVerbatim}

\begin{sphinxVerbatim}[commandchars=\\\{\}]
\PYG{g+go}{\PYGZsq{}31259.8\PYGZsq{}}
\end{sphinxVerbatim}

\begin{sphinxVerbatim}[commandchars=\\\{\}]
\PYG{c+c1}{\PYGZsh{} None) Nulo; o mesmo que \PYGZdq{}d\PYGZdq{}}
\PYG{n+nb}{format}\PYG{p}{(}\PYG{l+m+mb}{0b111}\PYG{p}{)}
\end{sphinxVerbatim}

\begin{sphinxVerbatim}[commandchars=\\\{\}]
\PYG{g+go}{\PYGZsq{}7\PYGZsq{}}
\end{sphinxVerbatim}

\begin{sphinxVerbatim}[commandchars=\\\{\}]
\PYG{c+c1}{\PYGZsh{} Para representação exponencial pode\PYGZhy{}se utilizar tanto \PYGZdq{}e\PYGZdq{} ou \PYGZdq{}E\PYGZdq{},}
\PYG{c+c1}{\PYGZsh{} cuja precisão padrão é 6}
\PYG{n+nb}{format}\PYG{p}{(}\PYG{l+m+mi}{1000}\PYG{p}{,} \PYG{l+s+s1}{\PYGZsq{}}\PYG{l+s+s1}{.3e}\PYG{l+s+s1}{\PYGZsq{}}\PYG{p}{)}
\end{sphinxVerbatim}

\begin{sphinxVerbatim}[commandchars=\\\{\}]
\PYG{g+go}{\PYGZsq{}1.000e+03\PYGZsq{}}
\end{sphinxVerbatim}

\begin{sphinxVerbatim}[commandchars=\\\{\}]
\PYG{n+nb}{format}\PYG{p}{(}\PYG{l+m+mi}{1000}\PYG{p}{,} \PYG{l+s+s1}{\PYGZsq{}}\PYG{l+s+s1}{.3E}\PYG{l+s+s1}{\PYGZsq{}}\PYG{p}{)}
\end{sphinxVerbatim}

\begin{sphinxVerbatim}[commandchars=\\\{\}]
\PYG{g+go}{\PYGZsq{}1.000E+03\PYGZsq{}}
\end{sphinxVerbatim}

\begin{sphinxVerbatim}[commandchars=\\\{\}]
\PYG{c+c1}{\PYGZsh{} \PYGZdq{}f\PYGZdq{} ou \PYGZdq{}F\PYGZdq{} faz exibição de número com ponto flutuante podendo determinar}
\PYG{c+c1}{\PYGZsh{} a precisão, cujo padrão é 6.}
\end{sphinxVerbatim}

\begin{sphinxVerbatim}[commandchars=\\\{\}]
\PYG{c+c1}{\PYGZsh{} format(1000, \PYGZsq{}10.2f\PYGZsq{})}
\end{sphinxVerbatim}

\begin{sphinxVerbatim}[commandchars=\\\{\}]
\PYG{g+go}{\PYGZsq{}   1000.00\PYGZsq{}}
\end{sphinxVerbatim}

\begin{sphinxVerbatim}[commandchars=\\\{\}]
\PYG{n+nb}{format}\PYG{p}{(}\PYG{l+m+mi}{1000}\PYG{p}{,} \PYG{l+s+s1}{\PYGZsq{}}\PYG{l+s+s1}{F}\PYG{l+s+s1}{\PYGZsq{}}\PYG{p}{)}
\end{sphinxVerbatim}

\begin{sphinxVerbatim}[commandchars=\\\{\}]
\PYG{g+go}{\PYGZsq{}1000.000000\PYGZsq{}}
\end{sphinxVerbatim}

\begin{sphinxVerbatim}[commandchars=\\\{\}]
\PYG{c+c1}{\PYGZsh{} \PYGZdq{}g\PYGZdq{} ou \PYGZdq{}G\PYGZdq{}; formato geral. Para uma dada precisão, sendo essa precisão}
\PYG{c+c1}{\PYGZsh{} maior ou igual a \PYGZsq{} (um), arredonda o número para p (precisão) de dígitos significantes}
\PYG{n+nb}{format}\PYG{p}{(}\PYG{l+m+mi}{1000}\PYG{p}{,} \PYG{l+s+s1}{\PYGZsq{}}\PYG{l+s+s1}{10.2G}\PYG{l+s+s1}{\PYGZsq{}}\PYG{p}{)}
\end{sphinxVerbatim}

\begin{sphinxVerbatim}[commandchars=\\\{\}]
\PYG{g+go}{\PYGZsq{}     1E+03\PYGZsq{}}
\end{sphinxVerbatim}

\begin{sphinxVerbatim}[commandchars=\\\{\}]
\PYG{n+nb}{format}\PYG{p}{(}\PYG{l+m+mi}{1000}\PYG{p}{,} \PYG{l+s+s1}{\PYGZsq{}}\PYG{l+s+s1}{10.3G}\PYG{l+s+s1}{\PYGZsq{}}\PYG{p}{)}
\end{sphinxVerbatim}

\begin{sphinxVerbatim}[commandchars=\\\{\}]
\PYG{g+go}{\PYGZsq{}     1e+03\PYGZsq{}}
\end{sphinxVerbatim}

\begin{sphinxVerbatim}[commandchars=\\\{\}]
\PYG{n+nb}{format}\PYG{p}{(}\PYG{l+m+mi}{100000}\PYG{p}{,} \PYG{l+s+s1}{\PYGZsq{}}\PYG{l+s+s1}{g}\PYG{l+s+s1}{\PYGZsq{}}\PYG{p}{)}
\end{sphinxVerbatim}

\begin{sphinxVerbatim}[commandchars=\\\{\}]
\PYG{g+go}{\PYGZsq{}100000\PYGZsq{}}
\end{sphinxVerbatim}

\begin{sphinxVerbatim}[commandchars=\\\{\}]
\PYG{n+nb}{format}\PYG{p}{(}\PYG{l+m+mi}{1000000}\PYG{p}{,} \PYG{l+s+s1}{\PYGZsq{}}\PYG{l+s+s1}{g}\PYG{l+s+s1}{\PYGZsq{}}\PYG{p}{)}
\end{sphinxVerbatim}

\begin{sphinxVerbatim}[commandchars=\\\{\}]
\PYG{g+go}{\PYGZsq{}1e+06\PYGZsq{}}
\end{sphinxVerbatim}

\begin{sphinxVerbatim}[commandchars=\\\{\}]
\PYG{n+nb}{format}\PYG{p}{(}\PYG{l+m+mf}{999.5}\PYG{p}{,} \PYG{l+s+s1}{\PYGZsq{}}\PYG{l+s+s1}{10.4G}\PYG{l+s+s1}{\PYGZsq{}}\PYG{p}{)}
\end{sphinxVerbatim}

\begin{sphinxVerbatim}[commandchars=\\\{\}]
\PYG{g+go}{\PYGZsq{}     999.5\PYGZsq{}}
\end{sphinxVerbatim}

\begin{sphinxVerbatim}[commandchars=\\\{\}]
\PYG{c+c1}{\PYGZsh{} format(999.5, \PYGZsq{}10.3G\PYGZsq{})}
\end{sphinxVerbatim}

\begin{sphinxVerbatim}[commandchars=\\\{\}]
\PYG{g+go}{\PYGZsq{}     1E+03\PYGZsq{}}
\end{sphinxVerbatim}


\section{Tipos de Strings em Python}
\label{\detokenize{content/str:tipos-de-strings-em-python}}\begin{quote}

Em Python temos algumas variações de strings, cada qual é designada por um prefixo, que é uma letra que representa o tipo de string e por omissão é unicode.
Cada tipo de string tem um prefixo, que são “b” bytes, “f” format, “r” raw e “u” unicode.
\end{quote}

\begin{sphinxVerbatim}[commandchars=\\\{\}]
\PYG{c+c1}{\PYGZsh{} Como unicode é padrão, podemos omitir o prefixo}
\PYG{n+nb}{print}\PYG{p}{(}\PYG{l+s+sa}{u}\PYG{l+s+s1}{\PYGZsq{}}\PYG{l+s+s1}{Foo}\PYG{l+s+s1}{\PYGZsq{}} \PYG{o}{==} \PYG{l+s+s1}{\PYGZsq{}}\PYG{l+s+s1}{Foo}\PYG{l+s+s1}{\PYGZsq{}}\PYG{p}{)}
\end{sphinxVerbatim}

\begin{sphinxVerbatim}[commandchars=\\\{\}]
\PYG{g+go}{True}
\end{sphinxVerbatim}


\subsection{Bytes (b)}
\label{\detokenize{content/str:bytes-b}}\begin{quote}

Strings de bytes utilizam o prefixo “b” e quando contém caracteres especiais, esses são representados pelo código hexadecimal da codificação utilizada.
\end{quote}

\begin{sphinxVerbatim}[commandchars=\\\{\}]
\PYG{c+c1}{\PYGZsh{} Criação de 3 (três) strings comuns}
\PYG{n}{s1} \PYG{o}{=} \PYG{l+s+s1}{\PYGZsq{}}\PYG{l+s+s1}{Sem caracteres especiais}\PYG{l+s+s1}{\PYGZsq{}}
\PYG{n}{s2} \PYG{o}{=} \PYG{l+s+s1}{\PYGZsq{}}\PYG{l+s+s1}{Macarrão}\PYG{l+s+s1}{\PYGZsq{}}
\PYG{n}{s3} \PYG{o}{=} \PYG{l+s+s1}{\PYGZsq{}}\PYG{l+s+s1}{Ação}\PYG{l+s+s1}{\PYGZsq{}}

\PYG{c+c1}{\PYGZsh{} A partir das três strings criadas anteriormente, criar outras três strings,}
\PYG{c+c1}{\PYGZsh{} mas strings de bytes}
\PYG{n}{sb1} \PYG{o}{=} \PYG{n}{s1}\PYG{o}{.}\PYG{n}{encode}\PYG{p}{(}\PYG{l+s+s1}{\PYGZsq{}}\PYG{l+s+s1}{utf\PYGZhy{}8}\PYG{l+s+s1}{\PYGZsq{}}\PYG{p}{)}
\PYG{n}{sb2} \PYG{o}{=} \PYG{n}{s2}\PYG{o}{.}\PYG{n}{encode}\PYG{p}{(}\PYG{l+s+s1}{\PYGZsq{}}\PYG{l+s+s1}{utf\PYGZhy{}8}\PYG{l+s+s1}{\PYGZsq{}}\PYG{p}{)}
\PYG{n}{sb3} \PYG{o}{=} \PYG{n}{s3}\PYG{o}{.}\PYG{n}{encode}\PYG{p}{(}\PYG{l+s+s1}{\PYGZsq{}}\PYG{l+s+s1}{utf\PYGZhy{}8}\PYG{l+s+s1}{\PYGZsq{}}\PYG{p}{)}
\end{sphinxVerbatim}

O método encode, utilizando a codificação UTF\sphinxhyphen{}8 faz a codificação de cada caractere para bytes.

\begin{sphinxVerbatim}[commandchars=\\\{\}]
\PYG{c+c1}{\PYGZsh{} Exibir o conteúdo das strings de bytes}
\PYG{n+nb}{print}\PYG{p}{(}\PYG{n}{sb1}\PYG{p}{)}
\end{sphinxVerbatim}

\begin{sphinxVerbatim}[commandchars=\\\{\}]
\PYG{g+go}{b\PYGZsq{}Sem caracteres especiais\PYGZsq{}}
\end{sphinxVerbatim}

\begin{sphinxVerbatim}[commandchars=\\\{\}]
\PYG{n+nb}{print}\PYG{p}{(}\PYG{n}{sb2}\PYG{p}{)}
\end{sphinxVerbatim}

\begin{sphinxVerbatim}[commandchars=\\\{\}]
\PYG{g+go}{b\PYGZsq{}Macarr\PYGZbs{}xc3\PYGZbs{}xa3o\PYGZsq{}}
\end{sphinxVerbatim}

\begin{sphinxVerbatim}[commandchars=\\\{\}]
\PYG{n+nb}{print}\PYG{p}{(}\PYG{n}{sb3}\PYG{p}{)}
\end{sphinxVerbatim}

\begin{sphinxVerbatim}[commandchars=\\\{\}]
\PYG{g+go}{b\PYGZsq{}A\PYGZbs{}xc3\PYGZbs{}xa7\PYGZbs{}xc3\PYGZbs{}xa3o\PYGZsq{}}
\end{sphinxVerbatim}

As strings que tinham caracteres especiais ficaram um tanto “estranhas”…

\begin{DUlineblock}{0em}
\item[] \sphinxcode{\sphinxupquote{\textbackslash{}xc3\textbackslash{}xa3 \sphinxhyphen{}\textgreater{} ã}}
\item[] \sphinxcode{\sphinxupquote{\textbackslash{}xc3\textbackslash{}xa7 \sphinxhyphen{}\textgreater{} ç}}
\end{DUlineblock}

\begin{sphinxVerbatim}[commandchars=\\\{\}]
\PYG{c+c1}{\PYGZsh{} Conversão de bytes}
\PYG{n+nb}{print}\PYG{p}{(}\PYG{l+s+sa}{b}\PYG{l+s+s1}{\PYGZsq{}}\PYG{l+s+se}{\PYGZbs{}xc3}\PYG{l+s+se}{\PYGZbs{}xa3}\PYG{l+s+s1}{\PYGZsq{}}\PYG{o}{.}\PYG{n}{decode}\PYG{p}{(}\PYG{l+s+s1}{\PYGZsq{}}\PYG{l+s+s1}{utf\PYGZhy{}8}\PYG{l+s+s1}{\PYGZsq{}}\PYG{p}{)}\PYG{p}{)}
\end{sphinxVerbatim}

\begin{sphinxVerbatim}[commandchars=\\\{\}]
\PYG{g+go}{ã}
\end{sphinxVerbatim}

\begin{sphinxVerbatim}[commandchars=\\\{\}]
\PYG{n+nb}{print}\PYG{p}{(}\PYG{l+s+sa}{b}\PYG{l+s+s1}{\PYGZsq{}}\PYG{l+s+se}{\PYGZbs{}xc3}\PYG{l+s+se}{\PYGZbs{}xa7}\PYG{l+s+s1}{\PYGZsq{}}\PYG{o}{.}\PYG{n}{decode}\PYG{p}{(}\PYG{l+s+s1}{\PYGZsq{}}\PYG{l+s+s1}{utf\PYGZhy{}8}\PYG{l+s+s1}{\PYGZsq{}}\PYG{p}{)}\PYG{p}{)}
\end{sphinxVerbatim}

\begin{sphinxVerbatim}[commandchars=\\\{\}]
\PYG{g+go}{ç}
\end{sphinxVerbatim}

\begin{sphinxVerbatim}[commandchars=\\\{\}]
\PYG{c+c1}{\PYGZsh{} A letra grega sigma é considerada como um caractere especial}
\PYG{n+nb}{print}\PYG{p}{(}\PYG{l+s+s1}{\PYGZsq{}}\PYG{l+s+s1}{∑}\PYG{l+s+s1}{\PYGZsq{}}\PYG{o}{.}\PYG{n}{encode}\PYG{p}{(}\PYG{l+s+s1}{\PYGZsq{}}\PYG{l+s+s1}{utf\PYGZhy{}8}\PYG{l+s+s1}{\PYGZsq{}}\PYG{p}{)}\PYG{p}{)}
\end{sphinxVerbatim}

\begin{sphinxVerbatim}[commandchars=\\\{\}]
\PYG{g+go}{b\PYGZsq{}\PYGZbs{}xe2\PYGZbs{}x88\PYGZbs{}x91\PYGZsq{}}
\end{sphinxVerbatim}

\begin{sphinxVerbatim}[commandchars=\\\{\}]
\PYG{c+c1}{\PYGZsh{} Caminho reverso}
\PYG{n+nb}{print}\PYG{p}{(}\PYG{l+s+sa}{b}\PYG{l+s+s1}{\PYGZsq{}}\PYG{l+s+se}{\PYGZbs{}xe2}\PYG{l+s+se}{\PYGZbs{}x88}\PYG{l+s+se}{\PYGZbs{}x91}\PYG{l+s+s1}{\PYGZsq{}}\PYG{o}{.}\PYG{n}{decode}\PYG{p}{(}\PYG{l+s+s1}{\PYGZsq{}}\PYG{l+s+s1}{utf\PYGZhy{}8}\PYG{l+s+s1}{\PYGZsq{}}\PYG{p}{)}\PYG{p}{)}
\end{sphinxVerbatim}

\begin{sphinxVerbatim}[commandchars=\\\{\}]
\PYG{g+go}{∑}
\end{sphinxVerbatim}

\begin{sphinxVerbatim}[commandchars=\\\{\}]
\PYG{c+c1}{\PYGZsh{} A partir das strings de bytes obter o texto}
\PYG{n+nb}{print}\PYG{p}{(}\PYG{n}{sb1}\PYG{o}{.}\PYG{n}{decode}\PYG{p}{(}\PYG{l+s+s1}{\PYGZsq{}}\PYG{l+s+s1}{utf\PYGZhy{}8}\PYG{l+s+s1}{\PYGZsq{}}\PYG{p}{)}\PYG{p}{)}
\end{sphinxVerbatim}

\begin{sphinxVerbatim}[commandchars=\\\{\}]
\PYG{g+go}{Sem caracteres especiais}
\end{sphinxVerbatim}

\begin{sphinxVerbatim}[commandchars=\\\{\}]
\PYG{c+c1}{\PYGZsh{} Decodificação UTF\PYGZhy{}8:}
\PYG{n+nb}{print}\PYG{p}{(}\PYG{n}{sb2}\PYG{o}{.}\PYG{n}{decode}\PYG{p}{(}\PYG{l+s+s1}{\PYGZsq{}}\PYG{l+s+s1}{utf\PYGZhy{}8}\PYG{l+s+s1}{\PYGZsq{}}\PYG{p}{)}\PYG{p}{)}
\end{sphinxVerbatim}

\begin{sphinxVerbatim}[commandchars=\\\{\}]
\PYG{g+go}{Macarrão}
\end{sphinxVerbatim}

\begin{sphinxVerbatim}[commandchars=\\\{\}]
\PYG{n+nb}{print}\PYG{p}{(}\PYG{n}{sb3}\PYG{o}{.}\PYG{n}{decode}\PYG{p}{(}\PYG{l+s+s1}{\PYGZsq{}}\PYG{l+s+s1}{utf\PYGZhy{}8}\PYG{l+s+s1}{\PYGZsq{}}\PYG{p}{)}\PYG{p}{)}
\end{sphinxVerbatim}

\begin{sphinxVerbatim}[commandchars=\\\{\}]
\PYG{g+go}{Ação}
\end{sphinxVerbatim}

\begin{sphinxVerbatim}[commandchars=\\\{\}]
\PYG{c+c1}{\PYGZsh{} Pode\PYGZhy{}se também criar um objeto bytes a partir da classe}
\PYG{n}{b} \PYG{o}{=} \PYG{n+nb}{bytes}\PYG{p}{(}\PYG{l+s+s1}{\PYGZsq{}}\PYG{l+s+s1}{∑}\PYG{l+s+s1}{\PYGZsq{}}\PYG{o}{.}\PYG{n}{encode}\PYG{p}{(}\PYG{l+s+s1}{\PYGZsq{}}\PYG{l+s+s1}{utf\PYGZhy{}8}\PYG{l+s+s1}{\PYGZsq{}}\PYG{p}{)}\PYG{p}{)}
\end{sphinxVerbatim}

\begin{sphinxVerbatim}[commandchars=\\\{\}]
\PYG{c+c1}{\PYGZsh{} Verificando o tipo}
\PYG{n+nb}{print}\PYG{p}{(}\PYG{n+nb}{type}\PYG{p}{(}\PYG{n}{b}\PYG{p}{)}\PYG{p}{)}
\end{sphinxVerbatim}

\begin{sphinxVerbatim}[commandchars=\\\{\}]
\PYG{g+go}{\PYGZlt{}class \PYGZsq{}bytes\PYGZsq{}\PYGZgt{}}
\end{sphinxVerbatim}

Uma byte string tem um tipo específico, bytes.

\begin{sphinxVerbatim}[commandchars=\\\{\}]
\PYG{c+c1}{\PYGZsh{} Exibindo a byte string}
\PYG{n+nb}{print}\PYG{p}{(}\PYG{n}{b}\PYG{p}{)}
\end{sphinxVerbatim}

\begin{sphinxVerbatim}[commandchars=\\\{\}]
\PYG{g+go}{b\PYGZsq{}\PYGZbs{}xe2\PYGZbs{}x88\PYGZbs{}x91\PYGZsq{}}
\end{sphinxVerbatim}

\begin{sphinxVerbatim}[commandchars=\\\{\}]
\PYG{c+c1}{\PYGZsh{} Decodificando para unicode}
\PYG{n+nb}{print}\PYG{p}{(}\PYG{n}{b}\PYG{o}{.}\PYG{n}{decode}\PYG{p}{(}\PYG{l+s+s1}{\PYGZsq{}}\PYG{l+s+s1}{utf\PYGZhy{}8}\PYG{l+s+s1}{\PYGZsq{}}\PYG{p}{)}\PYG{p}{)}
\end{sphinxVerbatim}

\begin{sphinxVerbatim}[commandchars=\\\{\}]
\PYG{g+go}{∑}
\end{sphinxVerbatim}

\begin{sphinxVerbatim}[commandchars=\\\{\}]
\PYG{c+c1}{\PYGZsh{} Verificando o tipo quando o objeto é decodificado}
\PYG{n+nb}{type}\PYG{p}{(}\PYG{n}{b}\PYG{o}{.}\PYG{n}{decode}\PYG{p}{(}\PYG{l+s+s1}{\PYGZsq{}}\PYG{l+s+s1}{utf\PYGZhy{}8}\PYG{l+s+s1}{\PYGZsq{}}\PYG{p}{)}\PYG{p}{)}
\end{sphinxVerbatim}

\begin{sphinxVerbatim}[commandchars=\\\{\}]
\PYG{g+go}{str}
\end{sphinxVerbatim}

Ao ser decodificado passa a ser uma string.


\subsection{Format Strings}
\label{\detokenize{content/str:format-strings}}\begin{quote}

Ou também conhecidas como “f strings” foi um recurso adicionado à versão 3.6 de Python.
\end{quote}

\begin{sphinxVerbatim}[commandchars=\\\{\}]
\PYG{c+c1}{\PYGZsh{} Definição de variáveis}
\PYG{n}{marca} \PYG{o}{=} \PYG{l+s+s1}{\PYGZsq{}}\PYG{l+s+s1}{Fiat}\PYG{l+s+s1}{\PYGZsq{}}
\PYG{n}{modelo} \PYG{o}{=} \PYG{l+s+s1}{\PYGZsq{}}\PYG{l+s+s1}{147}\PYG{l+s+s1}{\PYGZsq{}}
\PYG{n}{ano} \PYG{o}{=} \PYG{l+m+mi}{1985}
\PYG{n}{cor} \PYG{o}{=} \PYG{l+s+s1}{\PYGZsq{}}\PYG{l+s+s1}{azul}\PYG{l+s+s1}{\PYGZsq{}}
\end{sphinxVerbatim}

\begin{sphinxVerbatim}[commandchars=\\\{\}]
\PYG{c+c1}{\PYGZsh{} Exibir mensagem com uma f string}
\PYG{n+nb}{print}\PYG{p}{(}\PYG{l+s+sa}{f}\PYG{l+s+s1}{\PYGZsq{}}\PYG{l+s+s1}{Comprei um }\PYG{l+s+si}{\PYGZob{}}\PYG{n}{marca}\PYG{l+s+si}{\PYGZcb{}}\PYG{l+s+s1}{ }\PYG{l+s+si}{\PYGZob{}}\PYG{n}{modelo}\PYG{l+s+si}{\PYGZcb{}}\PYG{l+s+s1}{ }\PYG{l+s+si}{\PYGZob{}}\PYG{n}{cor}\PYG{l+s+si}{\PYGZcb{}}\PYG{l+s+s1}{ ano }\PYG{l+s+si}{\PYGZob{}}\PYG{n}{ano}\PYG{l+s+si}{\PYGZcb{}}\PYG{l+s+s1}{\PYGZsq{}}\PYG{p}{)}
\end{sphinxVerbatim}

\begin{sphinxVerbatim}[commandchars=\\\{\}]
\PYG{g+go}{Comprei um Fiat 147 azul ano 1985}
\end{sphinxVerbatim}

\begin{sphinxVerbatim}[commandchars=\\\{\}]
\PYG{c+c1}{\PYGZsh{} Uma f string também permite que se use expressões}
\PYG{n+nb}{print}\PYG{p}{(}\PYG{l+s+sa}{f}\PYG{l+s+s1}{\PYGZsq{}}\PYG{l+s+si}{\PYGZob{}}\PYG{l+m+mi}{5} \PYG{o}{+} \PYG{l+m+mi}{2}\PYG{l+s+si}{\PYGZcb{}}\PYG{l+s+s1}{\PYGZsq{}}\PYG{p}{)}
\end{sphinxVerbatim}

\begin{sphinxVerbatim}[commandchars=\\\{\}]
\PYG{g+go}{7}
\end{sphinxVerbatim}

\begin{sphinxVerbatim}[commandchars=\\\{\}]
\PYG{c+c1}{\PYGZsh{} Métodos e funções também são permitidos}
\PYG{n+nb}{print}\PYG{p}{(}\PYG{l+s+sa}{f}\PYG{l+s+s1}{\PYGZsq{}}\PYG{l+s+si}{\PYGZob{}}\PYG{n}{cor}\PYG{o}{.}\PYG{n}{upper}\PYG{p}{(}\PYG{p}{)}\PYG{l+s+si}{\PYGZcb{}}\PYG{l+s+s1}{\PYGZsq{}}\PYG{p}{)}
\end{sphinxVerbatim}

\begin{sphinxVerbatim}[commandchars=\\\{\}]
\PYG{g+go}{AZUL}
\end{sphinxVerbatim}

\begin{sphinxVerbatim}[commandchars=\\\{\}]
\PYG{c+c1}{\PYGZsh{} Criação de uma classe de exemplo que recebe quatro parâmetros}
\PYG{k}{class} \PYG{n+nc}{Carro}\PYG{p}{(}\PYG{n+nb}{object}\PYG{p}{)}\PYG{p}{:}
    \PYG{c+c1}{\PYGZsh{} Método de inicialização (construtor)}
    \PYG{k}{def} \PYG{n+nf+fm}{\PYGZus{}\PYGZus{}init\PYGZus{}\PYGZus{}}\PYG{p}{(}\PYG{n+nb+bp}{self}\PYG{p}{,} \PYG{n}{marca}\PYG{p}{,} \PYG{n}{modelo}\PYG{p}{,} \PYG{n}{ano}\PYG{p}{,} \PYG{n}{cor}\PYG{p}{)}\PYG{p}{:}
        \PYG{n+nb+bp}{self}\PYG{o}{.}\PYG{n}{marca} \PYG{o}{=} \PYG{n}{marca}
        \PYG{n+nb+bp}{self}\PYG{o}{.}\PYG{n}{modelo} \PYG{o}{=} \PYG{n}{modelo}
        \PYG{n+nb+bp}{self}\PYG{o}{.}\PYG{n}{ano} \PYG{o}{=} \PYG{n}{ano}
        \PYG{n+nb+bp}{self}\PYG{o}{.}\PYG{n}{cor} \PYG{o}{=} \PYG{n}{cor}

    \PYG{c+c1}{\PYGZsh{} Método string}
    \PYG{k}{def} \PYG{n+nf+fm}{\PYGZus{}\PYGZus{}str\PYGZus{}\PYGZus{}}\PYG{p}{(}\PYG{n+nb+bp}{self}\PYG{p}{)}\PYG{p}{:}
        \PYG{k}{return} \PYG{l+s+sa}{f}\PYG{l+s+s1}{\PYGZsq{}}\PYG{l+s+si}{\PYGZob{}}\PYG{n}{marca}\PYG{l+s+si}{\PYGZcb{}}\PYG{l+s+s1}{ }\PYG{l+s+si}{\PYGZob{}}\PYG{n}{modelo}\PYG{l+s+si}{\PYGZcb{}}\PYG{l+s+s1}{ / }\PYG{l+s+si}{\PYGZob{}}\PYG{n}{cor}\PYG{l+s+si}{\PYGZcb{}}\PYG{l+s+s1}{ / }\PYG{l+s+si}{\PYGZob{}}\PYG{n}{ano}\PYG{l+s+si}{\PYGZcb{}}\PYG{l+s+s1}{\PYGZsq{}}

    \PYG{c+c1}{\PYGZsh{} Método de representação}
    \PYG{k}{def} \PYG{n+nf+fm}{\PYGZus{}\PYGZus{}repr\PYGZus{}\PYGZus{}}\PYG{p}{(}\PYG{n+nb+bp}{self}\PYG{p}{)}\PYG{p}{:}
        \PYG{k}{return} \PYG{l+s+sa}{f}\PYG{l+s+s1}{\PYGZsq{}}\PYG{l+s+si}{\PYGZob{}}\PYG{n}{marca}\PYG{l+s+si}{\PYGZcb{}}\PYG{l+s+s1}{ }\PYG{l+s+si}{\PYGZob{}}\PYG{n}{modelo}\PYG{l+s+si}{\PYGZcb{}}\PYG{l+s+s1}{ | }\PYG{l+s+si}{\PYGZob{}}\PYG{n}{cor}\PYG{l+s+si}{\PYGZcb{}}\PYG{l+s+s1}{ | }\PYG{l+s+si}{\PYGZob{}}\PYG{n}{ano}\PYG{l+s+si}{\PYGZcb{}}\PYG{l+s+s1}{\PYGZsq{}}
\end{sphinxVerbatim}

\begin{sphinxVerbatim}[commandchars=\\\{\}]
\PYG{c+c1}{\PYGZsh{} Criação de um objeto Carro}
\PYG{n}{c} \PYG{o}{=} \PYG{n}{Carro}\PYG{p}{(}\PYG{n}{marca}\PYG{p}{,} \PYG{n}{modelo}\PYG{p}{,} \PYG{n}{ano}\PYG{p}{,} \PYG{n}{cor}\PYG{p}{)}
\end{sphinxVerbatim}

\begin{sphinxVerbatim}[commandchars=\\\{\}]
\PYG{c+c1}{\PYGZsh{} Print do método \PYGZus{}\PYGZus{}str\PYGZus{}\PYGZus{} do objeto:}
\PYG{n+nb}{print}\PYG{p}{(}\PYG{l+s+sa}{f}\PYG{l+s+s1}{\PYGZsq{}}\PYG{l+s+si}{\PYGZob{}}\PYG{n}{c}\PYG{l+s+si}{\PYGZcb{}}\PYG{l+s+s1}{\PYGZsq{}}\PYG{p}{)}
\end{sphinxVerbatim}

\begin{sphinxVerbatim}[commandchars=\\\{\}]
\PYG{g+go}{Fiat 147 / azul / 1985}
\end{sphinxVerbatim}

\begin{sphinxVerbatim}[commandchars=\\\{\}]
\PYG{c+c1}{\PYGZsh{} Print do método \PYGZus{}\PYGZus{}repr\PYGZus{}\PYGZus{} do objeto}
\PYG{n+nb}{print}\PYG{p}{(}\PYG{l+s+sa}{f}\PYG{l+s+s1}{\PYGZsq{}}\PYG{l+s+si}{\PYGZob{}}\PYG{n}{c}\PYG{l+s+si}{!r\PYGZcb{}}\PYG{l+s+s1}{\PYGZsq{}}\PYG{p}{)}
\end{sphinxVerbatim}

\begin{sphinxVerbatim}[commandchars=\\\{\}]
\PYG{g+go}{Fiat 147 | azul | 1985}
\end{sphinxVerbatim}

\begin{sphinxVerbatim}[commandchars=\\\{\}]
\PYG{c+c1}{\PYGZsh{} f string de múltiplas linhas}
\PYG{n}{msg} \PYG{o}{=} \PYG{l+s+sa}{f}\PYG{l+s+s1}{\PYGZsq{}}\PYG{l+s+s1}{Marca: }\PYG{l+s+si}{\PYGZob{}}\PYG{n}{marca}\PYG{l+s+si}{\PYGZcb{}}\PYG{l+s+se}{\PYGZbs{}n}\PYG{l+s+s1}{\PYGZsq{}}\PYGZbs{}
    \PYG{l+s+sa}{f}\PYG{l+s+s1}{\PYGZsq{}}\PYG{l+s+s1}{Modelo: }\PYG{l+s+si}{\PYGZob{}}\PYG{n}{modelo}\PYG{l+s+si}{\PYGZcb{}}\PYG{l+s+se}{\PYGZbs{}n}\PYG{l+s+s1}{\PYGZsq{}}\PYGZbs{}
    \PYG{l+s+sa}{f}\PYG{l+s+s1}{\PYGZsq{}}\PYG{l+s+s1}{Ano: }\PYG{l+s+si}{\PYGZob{}}\PYG{n}{ano}\PYG{l+s+si}{\PYGZcb{}}\PYG{l+s+se}{\PYGZbs{}n}\PYG{l+s+s1}{\PYGZsq{}}\PYGZbs{}
    \PYG{l+s+sa}{f}\PYG{l+s+s1}{\PYGZsq{}}\PYG{l+s+s1}{Cor: }\PYG{l+s+si}{\PYGZob{}}\PYG{n}{cor}\PYG{l+s+si}{\PYGZcb{}}\PYG{l+s+s1}{\PYGZsq{}}

\PYG{c+c1}{\PYGZsh{} Exibir a mensagem}
\PYG{n+nb}{print}\PYG{p}{(}\PYG{n}{msg}\PYG{p}{)}
\end{sphinxVerbatim}

\begin{sphinxVerbatim}[commandchars=\\\{\}]
\PYG{g+go}{Marca: Fiat}
\PYG{g+go}{Modelo: 147}
\PYG{g+go}{Ano: 1985}
\PYG{g+go}{Cor: azul}
\end{sphinxVerbatim}

\begin{sphinxVerbatim}[commandchars=\\\{\}]
\PYG{c+c1}{\PYGZsh{} f String entre parênteses}
\PYG{n}{msg} \PYG{o}{=} \PYG{p}{(}\PYG{l+s+sa}{f}\PYG{l+s+s1}{\PYGZsq{}}\PYG{l+s+s1}{Marca: }\PYG{l+s+si}{\PYGZob{}}\PYG{n}{marca}\PYG{l+s+si}{\PYGZcb{}}\PYG{l+s+s1}{ \PYGZhy{} }\PYG{l+s+s1}{\PYGZsq{}}
     \PYG{l+s+sa}{f}\PYG{l+s+s1}{\PYGZsq{}}\PYG{l+s+s1}{Modelo: }\PYG{l+s+si}{\PYGZob{}}\PYG{n}{modelo}\PYG{l+s+si}{\PYGZcb{}}\PYG{l+s+s1}{ \PYGZhy{} }\PYG{l+s+s1}{\PYGZsq{}}
     \PYG{l+s+sa}{f}\PYG{l+s+s1}{\PYGZsq{}}\PYG{l+s+s1}{Ano: }\PYG{l+s+si}{\PYGZob{}}\PYG{n}{ano}\PYG{l+s+si}{\PYGZcb{}}\PYG{l+s+s1}{ \PYGZhy{} }\PYG{l+s+s1}{\PYGZsq{}}
     \PYG{l+s+sa}{f}\PYG{l+s+s1}{\PYGZsq{}}\PYG{l+s+s1}{Cor: }\PYG{l+s+si}{\PYGZob{}}\PYG{n}{cor}\PYG{l+s+si}{\PYGZcb{}}\PYG{l+s+s1}{\PYGZsq{}}\PYG{p}{)}

\PYG{c+c1}{\PYGZsh{} Exibir a mensagem}
\PYG{n+nb}{print}\PYG{p}{(}\PYG{n}{msg}\PYG{p}{)}
\end{sphinxVerbatim}

\begin{sphinxVerbatim}[commandchars=\\\{\}]
\PYG{g+go}{Marca: Fiat \PYGZhy{} Modelo: 147 \PYGZhy{} Ano: 1985 \PYGZhy{} Cor: azul}
\end{sphinxVerbatim}


\subsection{Raw Strings (r)}
\label{\detokenize{content/str:raw-strings-r}}\begin{quote}

É o tipo de string cujo conteúdo é interpretado literalmente.
\end{quote}

\begin{sphinxVerbatim}[commandchars=\\\{\}]
\PYG{c+c1}{\PYGZsh{} Exemplo de print com raw string}
\PYG{n+nb}{print}\PYG{p}{(}\PYG{l+s+sa}{r}\PYG{l+s+s1}{\PYGZsq{}}\PYG{l+s+s1}{foo}\PYG{l+s+s1}{\PYGZbs{}}\PYG{l+s+s1}{tbar}\PYG{l+s+s1}{\PYGZsq{}}\PYG{p}{)}
\end{sphinxVerbatim}

\begin{sphinxVerbatim}[commandchars=\\\{\}]
\PYG{g+go}{foo\PYGZbs{}tbar}
\end{sphinxVerbatim}

É de se notar que a string não teve interpretação do caractere especial de tab (t), ou seja, não houve qualquer interpretação.


\subsection{Unicode Strings (u)}
\label{\detokenize{content/str:unicode-strings-u}}\begin{quote}

É o padrão para uma string em Python, não há a necessidade de adicionar o sufixo “u” antes do apóstrofo ou aspas.
\end{quote}

\begin{sphinxVerbatim}[commandchars=\\\{\}]
\PYG{c+c1}{\PYGZsh{} Comparação de strings}
\PYG{l+s+sa}{u}\PYG{l+s+s1}{\PYGZsq{}}\PYG{l+s+s1}{Foo}\PYG{l+s+s1}{\PYGZsq{}} \PYG{o}{==} \PYG{l+s+s1}{\PYGZsq{}}\PYG{l+s+s1}{Foo}\PYG{l+s+s1}{\PYGZsq{}}
\end{sphinxVerbatim}

\begin{sphinxVerbatim}[commandchars=\\\{\}]
\PYG{g+go}{True}
\end{sphinxVerbatim}

Das duas strings, somente a primeira tem o sufixo “u”.


\section{Operações de Strings}
\label{\detokenize{content/str:operacoes-de-strings}}

\subsection{Concatenação}
\label{\detokenize{content/str:concatenacao}}
\begin{sphinxVerbatim}[commandchars=\\\{\}]
\PYG{c+c1}{\PYGZsh{}}
\PYG{n+nb}{print}\PYG{p}{(}\PYG{l+s+s2}{\PYGZdq{}}\PYG{l+s+s2}{Curso}\PYG{l+s+s2}{\PYGZdq{}} \PYG{o}{+} \PYG{l+s+s2}{\PYGZdq{}}\PYG{l+s+s2}{ de }\PYG{l+s+s2}{\PYGZdq{}} \PYG{o}{+} \PYG{l+s+s2}{\PYGZdq{}}\PYG{l+s+s2}{Python}\PYG{l+s+s2}{\PYGZdq{}}\PYG{p}{)}
\end{sphinxVerbatim}

\begin{sphinxVerbatim}[commandchars=\\\{\}]
\PYG{g+go}{Curso de Python}
\end{sphinxVerbatim}

\begin{sphinxVerbatim}[commandchars=\\\{\}]
\PYG{c+c1}{\PYGZsh{}}
\PYG{n}{spam} \PYG{o}{=} \PYG{l+s+s2}{\PYGZdq{}}\PYG{l+s+s2}{Curso}\PYG{l+s+s2}{\PYGZdq{}}\PYG{o}{.}\PYG{n+nf+fm}{\PYGZus{}\PYGZus{}add\PYGZus{}\PYGZus{}}\PYG{p}{(}\PYG{l+s+s2}{\PYGZdq{}}\PYG{l+s+s2}{ de }\PYG{l+s+s2}{\PYGZdq{}}\PYG{o}{.}\PYG{n+nf+fm}{\PYGZus{}\PYGZus{}add\PYGZus{}\PYGZus{}}\PYG{p}{(}\PYG{l+s+s2}{\PYGZdq{}}\PYG{l+s+s2}{Python}\PYG{l+s+s2}{\PYGZdq{}}\PYG{p}{)}\PYG{p}{)}

\PYG{c+c1}{\PYGZsh{}}
\PYG{n+nb}{print}\PYG{p}{(}\PYG{n}{spam}\PYG{p}{)}
\end{sphinxVerbatim}

\begin{sphinxVerbatim}[commandchars=\\\{\}]
\PYG{g+go}{Curso de Python}
\end{sphinxVerbatim}


\subsection{Multiplicação}
\label{\detokenize{content/str:multiplicacao}}
\begin{sphinxVerbatim}[commandchars=\\\{\}]
\PYG{c+c1}{\PYGZsh{}}
\PYG{n+nb}{print}\PYG{p}{(}\PYG{l+s+s1}{\PYGZsq{}}\PYG{l+s+s1}{\PYGZlt{}}\PYG{l+s+s1}{\PYGZsq{}} \PYG{o}{+} \PYG{l+s+s1}{\PYGZsq{}}\PYG{l+s+s1}{Python}\PYG{l+s+s1}{\PYGZsq{}} \PYG{o}{*} \PYG{l+m+mi}{3} \PYG{o}{+} \PYG{l+s+s1}{\PYGZsq{}}\PYG{l+s+s1}{\PYGZgt{}}\PYG{l+s+s1}{\PYGZsq{}}\PYG{p}{)}
\end{sphinxVerbatim}

\begin{sphinxVerbatim}[commandchars=\\\{\}]
\PYG{g+go}{\PYGZlt{}PythonPythonPython\PYGZgt{}}
\end{sphinxVerbatim}

\begin{sphinxVerbatim}[commandchars=\\\{\}]
\PYG{c+c1}{\PYGZsh{}}
\PYG{n+nb}{print}\PYG{p}{(}\PYG{l+s+s1}{\PYGZsq{}}\PYG{l+s+s1}{\PYGZlt{}}\PYG{l+s+s1}{\PYGZsq{}} \PYG{o}{+} \PYG{l+s+s1}{\PYGZsq{}}\PYG{l+s+s1}{Python}\PYG{l+s+s1}{\PYGZsq{}}\PYG{o}{.}\PYG{n+nf+fm}{\PYGZus{}\PYGZus{}mul\PYGZus{}\PYGZus{}}\PYG{p}{(}\PYG{l+m+mi}{3}\PYG{p}{)} \PYG{o}{+} \PYG{l+s+s1}{\PYGZsq{}}\PYG{l+s+s1}{\PYGZgt{}}\PYG{l+s+s1}{\PYGZsq{}}\PYG{p}{)}
\end{sphinxVerbatim}

\begin{sphinxVerbatim}[commandchars=\\\{\}]
\PYG{g+go}{\PYGZsq{}\PYGZlt{}PythonPythonPython\PYGZgt{}\PYGZsq{}}
\end{sphinxVerbatim}


\subsection{Split}
\label{\detokenize{content/str:split}}\begin{quote}

Quebra a string em palavras formando uma lista.
\end{quote}

\begin{sphinxVerbatim}[commandchars=\\\{\}]
\PYG{c+c1}{\PYGZsh{}}
\PYG{n+nb}{print}\PYG{p}{(}\PYG{l+s+s1}{\PYGZsq{}}\PYG{l+s+s1}{Curso de Python}\PYG{l+s+s1}{\PYGZsq{}}\PYG{o}{.}\PYG{n}{split}\PYG{p}{(}\PYG{p}{)}\PYG{p}{)}
\end{sphinxVerbatim}

\begin{sphinxVerbatim}[commandchars=\\\{\}]
\PYG{g+go}{[\PYGZsq{}Curso\PYGZsq{}, \PYGZsq{}de\PYGZsq{}, \PYGZsq{}Python\PYGZsq{}]}
\end{sphinxVerbatim}

\begin{sphinxVerbatim}[commandchars=\\\{\}]
\PYG{c+c1}{\PYGZsh{}}
\PYG{n+nb}{print}\PYG{p}{(}\PYG{l+s+s1}{\PYGZsq{}}\PYG{l+s+s1}{Curso de Python}\PYG{l+s+s1}{\PYGZsq{}}\PYG{o}{.}\PYG{n}{split}\PYG{p}{(}\PYG{l+s+s1}{\PYGZsq{}}\PYG{l+s+s1}{de}\PYG{l+s+s1}{\PYGZsq{}}\PYG{p}{)}\PYG{p}{)}
\end{sphinxVerbatim}

\begin{sphinxVerbatim}[commandchars=\\\{\}]
\PYG{g+go}{[\PYGZsq{}Curso \PYGZsq{}, \PYGZsq{} Python\PYGZsq{}]}
\end{sphinxVerbatim}


\subsection{Slice}
\label{\detokenize{content/str:slice}}
\begin{DUlineblock}{0em}
\item[] Corte de string \sphinxhyphen{} \sphinxcode{\sphinxupquote{\textquotesingle{}string\textquotesingle{}{[}inicio:fim \sphinxhyphen{} 1:incremento{]}}}.
É importante salientar que no intervalo início:fim começam por zero, o ínício é fechado e o fim é aberto {[}início:fim).
Por padrão o incremento é 1.
\end{DUlineblock}

\begin{sphinxVerbatim}[commandchars=\\\{\}]
\PYG{c+c1}{\PYGZsh{} Primeira posição (começa com zero) da string:}
\PYG{n+nb}{print}\PYG{p}{(}\PYG{l+s+s2}{\PYGZdq{}}\PYG{l+s+s2}{Curso de Python”[0])}
\end{sphinxVerbatim}

\begin{sphinxVerbatim}[commandchars=\\\{\}]
\PYG{g+go}{\PYGZsq{}C\PYGZsq{}}
\end{sphinxVerbatim}

\begin{sphinxVerbatim}[commandchars=\\\{\}]
\PYG{c+c1}{\PYGZsh{} Da segunda à quarta posição da string:}
\PYG{n+nb}{print}\PYG{p}{(}\PYG{l+s+s2}{\PYGZdq{}}\PYG{l+s+s2}{Curso de Python}\PYG{l+s+s2}{\PYGZdq{}}\PYG{p}{[}\PYG{l+m+mi}{1}\PYG{p}{:}\PYG{l+m+mi}{5}\PYG{p}{]}\PYG{p}{)}
\end{sphinxVerbatim}

\begin{sphinxVerbatim}[commandchars=\\\{\}]
\PYG{g+go}{\PYGZsq{}urso\PYGZsq{}}
\end{sphinxVerbatim}

\begin{sphinxVerbatim}[commandchars=\\\{\}]
\PYG{c+c1}{\PYGZsh{} Da segunda à quarta posição com incremento 2:}
\PYG{n+nb}{print}\PYG{p}{(}\PYG{l+s+s2}{\PYGZdq{}}\PYG{l+s+s2}{Curso de Python}\PYG{l+s+s2}{\PYGZdq{}}\PYG{p}{[}\PYG{l+m+mi}{1}\PYG{p}{:}\PYG{l+m+mi}{5}\PYG{p}{:}\PYG{l+m+mi}{2}\PYG{p}{]}\PYG{p}{)}
\end{sphinxVerbatim}

\begin{sphinxVerbatim}[commandchars=\\\{\}]
\PYG{g+go}{\PYGZsq{}us\PYGZsq{}}
\end{sphinxVerbatim}

\begin{sphinxVerbatim}[commandchars=\\\{\}]
\PYG{c+c1}{\PYGZsh{} Da posição 9 em diante:}
\PYG{n+nb}{print}\PYG{p}{(}\PYG{l+s+s2}{\PYGZdq{}}\PYG{l+s+s2}{Curso de Python}\PYG{l+s+s2}{\PYGZdq{}}\PYG{p}{[}\PYG{l+m+mi}{9}\PYG{p}{:}\PYG{p}{]}\PYG{p}{)}
\end{sphinxVerbatim}

\begin{sphinxVerbatim}[commandchars=\\\{\}]
\PYG{g+go}{\PYGZsq{}Python\PYGZsq{}}
\end{sphinxVerbatim}

\begin{sphinxVerbatim}[commandchars=\\\{\}]
\PYG{c+c1}{\PYGZsh{} Até a posição 4:}
\PYG{n+nb}{print}\PYG{p}{(}\PYG{l+s+s2}{\PYGZdq{}}\PYG{l+s+s2}{Curso de Python}\PYG{l+s+s2}{\PYGZdq{}}\PYG{p}{[}\PYG{p}{:}\PYG{l+m+mi}{5}\PYG{p}{]}\PYG{p}{)}
\end{sphinxVerbatim}

\begin{sphinxVerbatim}[commandchars=\\\{\}]
\PYG{g+go}{\PYGZsq{}Curso\PYGZsq{}}
\end{sphinxVerbatim}

\begin{sphinxVerbatim}[commandchars=\\\{\}]
\PYG{c+c1}{\PYGZsh{} Padrão...:}
\PYG{n+nb}{print}\PYG{p}{(}\PYG{l+s+s2}{\PYGZdq{}}\PYG{l+s+s2}{Curso de Python}\PYG{l+s+s2}{\PYGZdq{}}\PYG{p}{[}\PYG{p}{:}\PYG{p}{:}\PYG{p}{]}\PYG{p}{)}
\end{sphinxVerbatim}

\begin{sphinxVerbatim}[commandchars=\\\{\}]
\PYG{g+go}{\PYGZsq{}Curso de Python\PYGZsq{}}
\end{sphinxVerbatim}

\begin{sphinxVerbatim}[commandchars=\\\{\}]
\PYG{c+c1}{\PYGZsh{} String reversa, incremento negativo:}
\PYG{n+nb}{print}\PYG{p}{(}\PYG{l+s+s2}{\PYGZdq{}}\PYG{l+s+s2}{Curso de Python}\PYG{l+s+s2}{\PYGZdq{}}\PYG{p}{[}\PYG{p}{:}\PYG{p}{:}\PYG{o}{\PYGZhy{}}\PYG{l+m+mi}{1}\PYG{p}{]}\PYG{p}{)}
\end{sphinxVerbatim}

\begin{sphinxVerbatim}[commandchars=\\\{\}]
\PYG{g+go}{\PYGZsq{}nohtyP ed osruC\PYGZsq{}}
\end{sphinxVerbatim}


\section{Docstrings}
\label{\detokenize{content/str:docstrings}}\begin{quote}

São strings que vêm logo após a definição de uma função, de um método ou de uma classe.
É muito útil para fins de documentação.
Para visualizar o conteúdo dessa string utiliza\sphinxhyphen{}se o atributo mágico \_\_doc\_\_ ou a função help.
\end{quote}

\begin{sphinxVerbatim}[commandchars=\\\{\}]
\PYG{c+c1}{\PYGZsh{} Criação de uma função}
\PYG{k}{def} \PYG{n+nf}{foo}\PYG{p}{(}\PYG{p}{)}\PYG{p}{:}
    \PYG{l+s+s1}{\PYGZsq{}}\PYG{l+s+s1}{Uma simples função}\PYG{l+s+s1}{\PYGZsq{}}
\end{sphinxVerbatim}

\begin{sphinxVerbatim}[commandchars=\\\{\}]
\PYG{c+c1}{\PYGZsh{} Exibe a docstring da função}
\PYG{n+nb}{print}\PYG{p}{(}\PYG{n}{foo}\PYG{o}{.}\PYG{n+nv+vm}{\PYGZus{}\PYGZus{}doc\PYGZus{}\PYGZus{}}\PYG{p}{)}
\end{sphinxVerbatim}

\begin{sphinxVerbatim}[commandchars=\\\{\}]
\PYG{g+go}{Uma simples função}
\end{sphinxVerbatim}

\begin{sphinxVerbatim}[commandchars=\\\{\}]
\PYG{c+c1}{\PYGZsh{} Criação de função}
\PYG{k}{def} \PYG{n+nf}{bar}\PYG{p}{(}\PYG{p}{)}\PYG{p}{:}
    \PYG{l+s+sd}{\PYGZsq{}\PYGZsq{}\PYGZsq{}}
\PYG{l+s+sd}{    Mais outra}
\PYG{l+s+sd}{    função}
\PYG{l+s+sd}{    que não faz}
\PYG{l+s+sd}{    nada}
\PYG{l+s+sd}{    \PYGZsq{}\PYGZsq{}\PYGZsq{}}
\end{sphinxVerbatim}

\begin{sphinxVerbatim}[commandchars=\\\{\}]
\PYG{c+c1}{\PYGZsh{} Exibe a docstring da função}
\PYG{n+nb}{print}\PYG{p}{(}\PYG{n}{bar}\PYG{o}{.}\PYG{n+nv+vm}{\PYGZus{}\PYGZus{}doc\PYGZus{}\PYGZus{}}\PYG{p}{)}
\end{sphinxVerbatim}

\begin{sphinxVerbatim}[commandchars=\\\{\}]
\PYG{g+go}{Mais outra}
\PYG{g+go}{função}
\PYG{g+go}{que não faz}
\PYG{g+go}{nada}
\end{sphinxVerbatim}

\begin{sphinxVerbatim}[commandchars=\\\{\}]
\PYG{c+c1}{\PYGZsh{} Criação de uma classe}
\PYG{k}{class} \PYG{n+nc}{Foo}\PYG{p}{(}\PYG{n+nb}{object}\PYG{p}{)}\PYG{p}{:}
    \PYG{l+s+sd}{\PYGZsq{}\PYGZsq{}\PYGZsq{}}
\PYG{l+s+sd}{    Uma classe}
\PYG{l+s+sd}{    de teste}
\PYG{l+s+sd}{    \PYGZsq{}\PYGZsq{}\PYGZsq{}}
\end{sphinxVerbatim}

\begin{sphinxVerbatim}[commandchars=\\\{\}]
\PYG{c+c1}{\PYGZsh{} Exibe a docstring da classe}
\PYG{n+nb}{print}\PYG{p}{(}\PYG{n}{Foo}\PYG{o}{.}\PYG{n+nv+vm}{\PYGZus{}\PYGZus{}doc\PYGZus{}\PYGZus{}}\PYG{p}{)}
\end{sphinxVerbatim}

\begin{sphinxVerbatim}[commandchars=\\\{\}]
\PYG{g+go}{Uma classe}
\PYG{g+go}{de teste}
\end{sphinxVerbatim}

\begin{sphinxVerbatim}[commandchars=\\\{\}]
\PYG{c+c1}{\PYGZsh{} Help da classe:}
\PYG{n}{help}\PYG{p}{(}\PYG{n}{Foo}\PYG{p}{)}
\end{sphinxVerbatim}

\begin{sphinxVerbatim}[commandchars=\\\{\}]
\PYG{g+go}{Help on class Foo in module \PYGZus{}\PYGZus{}main\PYGZus{}\PYGZus{}:}

\PYG{g+go}{class Foo(\PYGZus{}\PYGZus{}builtin\PYGZus{}\PYGZus{}.object)}

\PYG{g+go}{|  Uma classe}
\PYG{g+go}{|  de teste}
\PYG{g+go}{|}
\PYG{g+go}{|  Data descriptors defined here:}
\PYG{g+go}{|}
\PYG{g+go}{|  \PYGZus{}\PYGZus{}dict\PYGZus{}\PYGZus{}}
\PYG{g+go}{|      dictionary for instance variables (if defined)}
\PYG{g+go}{|}
\PYG{g+go}{|  \PYGZus{}\PYGZus{}weakref\PYGZus{}\PYGZus{}}
\PYG{g+go}{|      list of weak references to the object (if defined)}
\end{sphinxVerbatim}


\section{Imutabilidade}
\label{\detokenize{content/str:imutabilidade}}\begin{quote}

Strings em Python são imutáveis.
\end{quote}

\begin{sphinxVerbatim}[commandchars=\\\{\}]
\PYG{c+c1}{\PYGZsh{} Criação de uma string}
\PYG{n}{foo} \PYG{o}{=} \PYG{l+s+s1}{\PYGZsq{}}\PYG{l+s+s1}{bar}\PYG{l+s+s1}{\PYGZsq{}}

\PYG{c+c1}{\PYGZsh{} Primeiro elemento da string}
\PYG{n}{foo}\PYG{p}{[}\PYG{l+m+mi}{0}\PYG{p}{]}
\end{sphinxVerbatim}

\begin{sphinxVerbatim}[commandchars=\\\{\}]
\PYG{g+go}{\PYGZsq{}b\PYGZsq{}}
\end{sphinxVerbatim}

\begin{sphinxVerbatim}[commandchars=\\\{\}]
\PYG{c+c1}{\PYGZsh{} Tentativa de redefinição do primeiro elemento da string}
\PYG{n}{foo}\PYG{p}{[}\PYG{l+m+mi}{0}\PYG{p}{]} \PYG{o}{=} \PYG{l+s+s1}{\PYGZsq{}}\PYG{l+s+s1}{B}\PYG{l+s+s1}{\PYGZsq{}}
\end{sphinxVerbatim}

\begin{sphinxVerbatim}[commandchars=\\\{\}]
\PYG{g+go}{TypeError: \PYGZsq{}str\PYGZsq{} object does not support item assignment}
\end{sphinxVerbatim}

\begin{sphinxVerbatim}[commandchars=\\\{\}]
\PYG{c+c1}{\PYGZsh{} Id da string}
\PYG{n+nb}{id}\PYG{p}{(}\PYG{n}{foo}\PYG{p}{)}
\end{sphinxVerbatim}

\begin{sphinxVerbatim}[commandchars=\\\{\}]
\PYG{g+go}{139876439773904}
\end{sphinxVerbatim}

\begin{sphinxVerbatim}[commandchars=\\\{\}]
\PYG{c+c1}{\PYGZsh{} Criação de uma string com o mesmo nome da anterior}
\PYG{c+c1}{\PYGZsh{} utilizando concatenação e slice}
\PYG{n}{foo} \PYG{o}{=} \PYG{l+s+s1}{\PYGZsq{}}\PYG{l+s+s1}{B}\PYG{l+s+s1}{\PYGZsq{}} \PYG{o}{+} \PYG{n}{foo}\PYG{p}{[}\PYG{l+m+mi}{1}\PYG{p}{:}\PYG{p}{]}
\end{sphinxVerbatim}

\begin{sphinxVerbatim}[commandchars=\\\{\}]
\PYG{c+c1}{\PYGZsh{} Verificando o Id da variável}
\PYG{n+nb}{id}\PYG{p}{(}\PYG{n}{foo}\PYG{p}{)}
\end{sphinxVerbatim}

\begin{sphinxVerbatim}[commandchars=\\\{\}]
\PYG{g+go}{140159122071800}
\end{sphinxVerbatim}

Nota\sphinxhyphen{}se que o Id é diferente, pois agora é outro objeto.

\begin{sphinxVerbatim}[commandchars=\\\{\}]
\PYG{c+c1}{\PYGZsh{} Exibindo o valor da variável}
\PYG{n+nb}{print}\PYG{p}{(}\PYG{n}{foo}\PYG{p}{)}
\end{sphinxVerbatim}

\begin{sphinxVerbatim}[commandchars=\\\{\}]
\PYG{g+go}{Bar}
\end{sphinxVerbatim}

\begin{sphinxVerbatim}[commandchars=\\\{\}]
\PYG{c+c1}{\PYGZsh{} Criação de uma nova string}
\PYG{n}{s} \PYG{o}{=} \PYG{l+s+s1}{\PYGZsq{}}\PYG{l+s+s1}{Black}\PYG{l+s+s1}{\PYGZsq{}}

\PYG{c+c1}{\PYGZsh{} Id da string}
 \PYG{n+nb}{id}\PYG{p}{(}\PYG{n}{s}\PYG{p}{)}
\end{sphinxVerbatim}

\begin{sphinxVerbatim}[commandchars=\\\{\}]
\PYG{g+go}{140159159537600}
\end{sphinxVerbatim}

\begin{sphinxVerbatim}[commandchars=\\\{\}]
\PYG{c+c1}{\PYGZsh{} Criando uma nova string com o mesmo nome}
\PYG{c+c1}{\PYGZsh{} da anterior via concatenação}
\end{sphinxVerbatim}

\begin{sphinxVerbatim}[commandchars=\\\{\}]
\PYG{c+c1}{\PYGZsh{}}
\PYG{n}{s} \PYG{o}{+}\PYG{o}{=} \PYG{l+s+s1}{\PYGZsq{}}\PYG{l+s+s1}{ Sabbath}\PYG{l+s+s1}{\PYGZsq{}}
\end{sphinxVerbatim}

\begin{sphinxVerbatim}[commandchars=\\\{\}]
\PYG{c+c1}{\PYGZsh{} Id da nova variável}
\PYG{n+nb}{id}\PYG{p}{(}\PYG{n}{s}\PYG{p}{)}
\end{sphinxVerbatim}

\begin{sphinxVerbatim}[commandchars=\\\{\}]
\PYG{g+go}{140159122296368}
\end{sphinxVerbatim}

Novamente nota\sphinxhyphen{}se que o Id é diferente, pois é na verdade um novo objeto.

\begin{sphinxVerbatim}[commandchars=\\\{\}]
\PYG{c+c1}{\PYGZsh{} Exibindo a string}
\PYG{n+nb}{print}\PYG{p}{(}\PYG{n}{s}\PYG{p}{)}
\end{sphinxVerbatim}

\begin{sphinxVerbatim}[commandchars=\\\{\}]
\PYG{g+go}{Black Sabbath}
\end{sphinxVerbatim}


\section{Concatenação de Strings em Loops}
\label{\detokenize{content/str:concatenacao-de-strings-em-loops}}

\subsection{Método 1 \sphinxhyphen{} Ineficaz}
\label{\detokenize{content/str:metodo-1-ineficaz}}
\begin{sphinxVerbatim}[commandchars=\\\{\}]
\PYG{c+c1}{\PYGZsh{} Criação de uma string vazia}
\PYG{n}{s} \PYG{o}{=} \PYG{l+s+s1}{\PYGZsq{}}\PYG{l+s+s1}{\PYGZsq{}}

\PYG{c+c1}{\PYGZsh{} Loop de concatenação}
\PYG{k}{for} \PYG{n}{i} \PYG{o+ow}{in} \PYG{n+nb}{range}\PYG{p}{(}\PYG{l+m+mi}{50}\PYG{p}{)}\PYG{p}{:}
    \PYG{n}{s} \PYG{o}{+}\PYG{o}{=} \PYG{n+nb}{str}\PYG{p}{(}\PYG{n}{i}\PYG{p}{)}

\PYG{c+c1}{\PYGZsh{} String pronta}
\PYG{n+nb}{print}\PYG{p}{(}\PYG{n}{s}\PYG{p}{)}
\end{sphinxVerbatim}

\begin{sphinxVerbatim}[commandchars=\\\{\}]
\PYG{g+go}{012345678910111213141516171819202122232425262728293031323334353637383940414243444546474849}
\end{sphinxVerbatim}

Para cada iteração a referência do objeto antigo é retirada e sendo criado um novo a partir do resultado da concatenação do valor antigo com o valor de do atual e o garbage collector é acionado.
Isso faz muita alocação de memória, o que torna o desempenho horrível para coisas maiores.


\subsection{Método 2 \sphinxhyphen{} Eficaz}
\label{\detokenize{content/str:metodo-2-eficaz}}
\begin{sphinxVerbatim}[commandchars=\\\{\}]
\PYG{c+c1}{\PYGZsh{} Criação de uma lista vazia}
\PYG{n}{s} \PYG{o}{=} \PYG{p}{[}\PYG{p}{]}

\PYG{c+c1}{\PYGZsh{} Loop de concatenação}
\PYG{k}{for} \PYG{n}{i} \PYG{o+ow}{in} \PYG{n+nb}{range}\PYG{p}{(}\PYG{l+m+mi}{50}\PYG{p}{)}\PYG{p}{:}
    \PYG{n}{s}\PYG{o}{.}\PYG{n}{append}\PYG{p}{(}\PYG{n+nb}{str}\PYG{p}{(}\PYG{n}{i}\PYG{p}{)}\PYG{p}{)}

\PYG{c+c1}{\PYGZsh{} Fazendo a junção de uma string vazia com a lista criada}
\PYG{c+c1}{\PYGZsh{} com seus elementos via método append}
\PYG{n+nb}{print}\PYG{p}{(}\PYG{l+s+s1}{\PYGZsq{}}\PYG{l+s+s1}{\PYGZsq{}}\PYG{o}{.}\PYG{n}{join}\PYG{p}{(}\PYG{n}{s}\PYG{p}{)}\PYG{p}{)}
\end{sphinxVerbatim}

\begin{sphinxVerbatim}[commandchars=\\\{\}]
\PYG{g+go}{012345678910111213141516171819202122232425262728293031323334353637383940414243444546474849}
\end{sphinxVerbatim}

\begin{sphinxVerbatim}[commandchars=\\\{\}]
\PYG{c+c1}{\PYGZsh{} Criando uma string via método join da lista de mesmo nome}
\PYG{n}{s} \PYG{o}{=} \PYG{l+s+s1}{\PYGZsq{}}\PYG{l+s+s1}{\PYGZsq{}}\PYG{o}{.}\PYG{n}{join}\PYG{p}{(}\PYG{n}{s}\PYG{p}{)}

\PYG{c+c1}{\PYGZsh{} Exibindo o valor da variável}
\PYG{n+nb}{print}\PYG{p}{(}\PYG{n}{s}\PYG{p}{)}
\end{sphinxVerbatim}

\begin{sphinxVerbatim}[commandchars=\\\{\}]
\PYG{g+go}{012345678910111213141516171819202122232425262728293031323334353637383940414243444546474849}
\end{sphinxVerbatim}

Foi criada uma lista de strings no loop em que a cada iteração é utilizado o método append da lista para adicionar o item atual.
No final é utilizado o método de string join que utiliza como separador uma string vazia (‘’) juntando em uma string (o novo s) todos os valores da lista.
A estrutura de dados de uma lista Python é mais eficiente para crescer, pois o método append apenas adiciona um novo elemento, de forma rápida e eficiente. O método join, que é escrito em C, que faz a junção de todos elementos concatenando em um único passo.Muito melhor do que o método anterior em que um novo objeto é criado a cada iteração.


\section{Métodos de Strings}
\label{\detokenize{content/str:metodos-de-strings}}\begin{itemize}
\item {} 
join; junta elementos de uma lista ou tupla utlizando uma string.

\end{itemize}

\begin{sphinxVerbatim}[commandchars=\\\{\}]
\PYG{c+c1}{\PYGZsh{} Criação de uma lista}
\PYG{n}{foo} \PYG{o}{=} \PYG{n+nb}{list}\PYG{p}{(}\PYG{l+s+s1}{\PYGZsq{}}\PYG{l+s+s1}{Python}\PYG{l+s+s1}{\PYGZsq{}}\PYG{p}{)}

\PYG{c+c1}{\PYGZsh{} Exibe a lista}
\PYG{n+nb}{print}\PYG{p}{(}\PYG{n}{foo}\PYG{p}{)}
\end{sphinxVerbatim}

\begin{sphinxVerbatim}[commandchars=\\\{\}]
\PYG{g+go}{[\PYGZsq{}P\PYGZsq{}, \PYGZsq{}y\PYGZsq{}, \PYGZsq{}t\PYGZsq{}, \PYGZsq{}h\PYGZsq{}, \PYGZsq{}o\PYGZsq{}, \PYGZsq{}n\PYGZsq{}]}
\end{sphinxVerbatim}

\begin{sphinxVerbatim}[commandchars=\\\{\}]
\PYG{c+c1}{\PYGZsh{} Criação de uma nova variável juntando os elementos}
\PYG{c+c1}{\PYGZsh{} da lista com uma string vazia}
\PYG{n}{bar} \PYG{o}{=} \PYG{l+s+s1}{\PYGZsq{}}\PYG{l+s+s1}{\PYGZsq{}}\PYG{o}{.}\PYG{n}{join}\PYG{p}{(}\PYG{n}{foo}\PYG{p}{)}

\PYG{c+c1}{\PYGZsh{} Exibindo a nova string}
\PYG{n+nb}{print}\PYG{p}{(}\PYG{n}{bar}\PYG{p}{)}
\end{sphinxVerbatim}

\begin{sphinxVerbatim}[commandchars=\\\{\}]
\PYG{g+go}{Python}
\end{sphinxVerbatim}

\begin{sphinxVerbatim}[commandchars=\\\{\}]
\PYG{c+c1}{\PYGZsh{} Criando uma tupla}
\PYG{n}{foo} \PYG{o}{=} \PYG{n+nb}{tuple}\PYG{p}{(}\PYG{l+s+s1}{\PYGZsq{}}\PYG{l+s+s1}{Python}\PYG{l+s+s1}{\PYGZsq{}}\PYG{p}{)}

\PYG{c+c1}{\PYGZsh{} Exibindo os elementos da tupla}
\PYG{n+nb}{print}\PYG{p}{(}\PYG{n}{foo}\PYG{p}{)}
\end{sphinxVerbatim}

\begin{sphinxVerbatim}[commandchars=\\\{\}]
\PYG{g+gp+gpVirtualEnv}{(\PYGZsq{}P\PYGZsq{}, \PYGZsq{}y\PYGZsq{}, \PYGZsq{}t\PYGZsq{}, \PYGZsq{}h\PYGZsq{}, \PYGZsq{}o\PYGZsq{}, \PYGZsq{}n\PYGZsq{})}
\end{sphinxVerbatim}

\begin{sphinxVerbatim}[commandchars=\\\{\}]
\PYG{c+c1}{\PYGZsh{} Criação de uma nova variável juntando os elementos}
\PYG{c+c1}{\PYGZsh{} da tupla com uma string vazia}
\PYG{n}{bar} \PYG{o}{=} \PYG{l+s+s1}{\PYGZsq{}}\PYG{l+s+s1}{\PYGZsq{}}\PYG{o}{.}\PYG{n}{join}\PYG{p}{(}\PYG{n}{foo}\PYG{p}{)}

\PYG{c+c1}{\PYGZsh{} Exibindo o valor da variável}
\PYG{n+nb}{print}\PYG{p}{(}\PYG{n}{bar}\PYG{p}{)}
\end{sphinxVerbatim}

\begin{sphinxVerbatim}[commandchars=\\\{\}]
\PYG{g+go}{Python}
\end{sphinxVerbatim}
\begin{itemize}
\item {} 
find \& index \sphinxhyphen{} Qual é a diferença entre eles?

\end{itemize}

\begin{sphinxVerbatim}[commandchars=\\\{\}]
\PYG{c+c1}{\PYGZsh{} Dada a seguinte string}
\PYG{n}{foo} \PYG{o}{=} \PYG{l+s+s1}{\PYGZsq{}}\PYG{l+s+s1}{Python FreeBSD PostgreSQL}\PYG{l+s+s1}{\PYGZsq{}}
\end{sphinxVerbatim}

Temos seus caracteres e suas respectivas posições:

\begin{sphinxVerbatim}[commandchars=\\\{\}]
\PYG{n}{P}\PYG{o}{|}\PYG{n}{y}\PYG{o}{|}\PYG{n}{t}\PYG{o}{|}\PYG{n}{h}\PYG{o}{|}\PYG{n}{o}\PYG{o}{|}\PYG{n}{n}\PYG{o}{|} \PYG{o}{|}\PYG{n}{F}\PYG{o}{|}\PYG{n}{r}\PYG{o}{|}\PYG{n}{e}\PYG{o}{|}\PYG{n}{e} \PYG{o}{|}\PYG{n}{B} \PYG{o}{|}\PYG{n}{S} \PYG{o}{|}\PYG{n}{D} \PYG{o}{|}  \PYG{o}{|}\PYG{n}{P} \PYG{o}{|}\PYG{n}{o} \PYG{o}{|}\PYG{n}{s} \PYG{o}{|}\PYG{n}{t} \PYG{o}{|}\PYG{n}{g} \PYG{o}{|}\PYG{n}{r} \PYG{o}{|}\PYG{n}{e} \PYG{o}{|}\PYG{n}{S} \PYG{o}{|}\PYG{n}{Q} \PYG{o}{|}\PYG{n}{L}
\PYG{l+m+mi}{0}\PYG{o}{|}\PYG{l+m+mi}{1}\PYG{o}{|}\PYG{l+m+mi}{2}\PYG{o}{|}\PYG{l+m+mi}{3}\PYG{o}{|}\PYG{l+m+mi}{4}\PYG{o}{|}\PYG{l+m+mi}{5}\PYG{o}{|}\PYG{l+m+mi}{6}\PYG{o}{|}\PYG{l+m+mi}{7}\PYG{o}{|}\PYG{l+m+mi}{8}\PYG{o}{|}\PYG{l+m+mi}{9}\PYG{o}{|}\PYG{l+m+mi}{10}\PYG{o}{|}\PYG{l+m+mi}{11}\PYG{o}{|}\PYG{l+m+mi}{12}\PYG{o}{|}\PYG{l+m+mi}{13}\PYG{o}{|}\PYG{l+m+mi}{14}\PYG{o}{|}\PYG{l+m+mi}{15}\PYG{o}{|}\PYG{l+m+mi}{16}\PYG{o}{|}\PYG{l+m+mi}{17}\PYG{o}{|}\PYG{l+m+mi}{18}\PYG{o}{|}\PYG{l+m+mi}{19}\PYG{o}{|}\PYG{l+m+mi}{20}\PYG{o}{|}\PYG{l+m+mi}{21}\PYG{o}{|}\PYG{l+m+mi}{22}\PYG{o}{|}\PYG{l+m+mi}{23}\PYG{o}{|}\PYG{l+m+mi}{24}
\end{sphinxVerbatim}

\begin{sphinxVerbatim}[commandchars=\\\{\}]
\PYG{c+c1}{\PYGZsh{} A partir de qual posição aparece a string?}
\PYG{n}{foo}\PYG{o}{.}\PYG{n}{index}\PYG{p}{(}\PYG{l+s+s1}{\PYGZsq{}}\PYG{l+s+s1}{FreeBSD}\PYG{l+s+s1}{\PYGZsq{}}\PYG{p}{)}
\PYG{n}{foo}\PYG{o}{.}\PYG{n}{find}\PYG{p}{(}\PYG{l+s+s1}{\PYGZsq{}}\PYG{l+s+s1}{FreeBSD}\PYG{l+s+s1}{\PYGZsq{}}\PYG{p}{)}
\end{sphinxVerbatim}

\begin{sphinxVerbatim}[commandchars=\\\{\}]
\PYG{g+go}{7}
\end{sphinxVerbatim}

No exemplo dado o texto existe na string. E se não existisse?

\begin{sphinxVerbatim}[commandchars=\\\{\}]
\PYG{c+c1}{\PYGZsh{} Buscando um texto que não existe dentro da string}
\PYG{n}{foo}\PYG{o}{.}\PYG{n}{index}\PYG{p}{(}\PYG{l+s+s1}{\PYGZsq{}}\PYG{l+s+s1}{Linux}\PYG{l+s+s1}{\PYGZsq{}}\PYG{p}{)}
\end{sphinxVerbatim}

\begin{sphinxVerbatim}[commandchars=\\\{\}]
\PYG{g+go}{ValueError: substring not found}
\end{sphinxVerbatim}

\begin{sphinxVerbatim}[commandchars=\\\{\}]
\PYG{c+c1}{\PYGZsh{}}
\PYG{n}{foo}\PYG{o}{.}\PYG{n}{find}\PYG{p}{(}\PYG{l+s+s1}{\PYGZsq{}}\PYG{l+s+s1}{Linux}\PYG{l+s+s1}{\PYGZsq{}}\PYG{p}{)}
\end{sphinxVerbatim}

\begin{sphinxVerbatim}[commandchars=\\\{\}]
\PYG{g+go}{\PYGZhy{}1}
\end{sphinxVerbatim}

Nota\sphinxhyphen{}se que que index lança uma exceção, enquanto find retorna \sphinxhyphen{}1 ao não encontrar o que foi pedido.
O \sphinxhyphen{}1 não deve ser confundido como último elemento.
\begin{itemize}
\item {} 
count

\end{itemize}

\begin{sphinxVerbatim}[commandchars=\\\{\}]
\PYG{c+c1}{\PYGZsh{} Na frase em latim abaixo, quantas vezes aparece a letra \PYGZdq{}u\PYGZdq{}?}
\PYG{l+s+s1}{\PYGZsq{}}\PYG{l+s+s1}{sic mundus creatus est}\PYG{l+s+s1}{\PYGZsq{}}\PYG{o}{.}\PYG{n}{count}\PYG{p}{(}\PYG{l+s+s1}{\PYGZsq{}}\PYG{l+s+s1}{u}\PYG{l+s+s1}{\PYGZsq{}}\PYG{p}{)}
\end{sphinxVerbatim}

\begin{sphinxVerbatim}[commandchars=\\\{\}]
\PYG{g+go}{3}
\end{sphinxVerbatim}

\begin{sphinxVerbatim}[commandchars=\\\{\}]
\PYG{c+c1}{\PYGZsh{} Quantas vezes aparece a sequência de caracteres \PYGZdq{}foo\PYGZdq{}?}
\PYG{l+s+s1}{\PYGZsq{}}\PYG{l+s+s1}{XXXfooXXXfooXXXbar}\PYG{l+s+s1}{\PYGZsq{}}\PYG{o}{.}\PYG{n}{count}\PYG{p}{(}\PYG{l+s+s1}{\PYGZsq{}}\PYG{l+s+s1}{foo}\PYG{l+s+s1}{\PYGZsq{}}\PYG{p}{)}
\end{sphinxVerbatim}

\begin{sphinxVerbatim}[commandchars=\\\{\}]
\PYG{g+go}{2}
\end{sphinxVerbatim}


\chapter{Tipos Numéricos \sphinxhyphen{} int, float e complex}
\label{\detokenize{content/numeric_data_types:tipos-numericos-int-float-e-complex}}\label{\detokenize{content/numeric_data_types::doc}}

\section{int}
\label{\detokenize{content/numeric_data_types:int}}\begin{quote}

Para representação de todos números inteiros é o tipo int.
A princípio utilizamos para números inteiros o tipo int. O número máximo de int que é aceito pode variar de uma máquina para outra.
Para sabermos qual é o número máximo do tipo int fazemos:
\end{quote}

\begin{sphinxVerbatim}[commandchars=\\\{\}]
\PYG{c+c1}{\PYGZsh{} Objeto inteiro:}
\PYG{n}{i} \PYG{o}{=} \PYG{n+nb}{int}\PYG{p}{(}\PYG{l+m+mi}{7}\PYG{p}{)}
\end{sphinxVerbatim}

ou

\begin{sphinxVerbatim}[commandchars=\\\{\}]
\PYG{n}{i} \PYG{o}{=} \PYG{l+m+mi}{7}
\end{sphinxVerbatim}

\begin{sphinxVerbatim}[commandchars=\\\{\}]
\PYG{c+c1}{\PYGZsh{} Verificando seu valor:}
\PYG{n}{i}
\end{sphinxVerbatim}

\begin{sphinxVerbatim}[commandchars=\\\{\}]
\PYG{g+go}{7}
\end{sphinxVerbatim}

\begin{sphinxVerbatim}[commandchars=\\\{\}]
\PYG{c+c1}{\PYGZsh{} Verificando seu tipo:}
\PYG{n+nb}{type}\PYG{p}{(}\PYG{n}{i}\PYG{p}{)}
\end{sphinxVerbatim}

\begin{sphinxVerbatim}[commandchars=\\\{\}]
\PYG{g+go}{int}
\end{sphinxVerbatim}

\begin{sphinxVerbatim}[commandchars=\\\{\}]
\PYG{c+c1}{\PYGZsh{} Tipo de \PYGZdq{}bar\PYGZdq{}:}
\PYG{n+nb}{type}\PYG{p}{(}\PYG{n}{bar}\PYG{p}{)}
\end{sphinxVerbatim}

\begin{sphinxVerbatim}[commandchars=\\\{\}]
\PYG{g+go}{long}
\end{sphinxVerbatim}

\begin{sphinxVerbatim}[commandchars=\\\{\}]
\PYG{c+c1}{\PYGZsh{} Representação hexadecimal de 178}
\PYG{l+m+mh}{0xb2}
\end{sphinxVerbatim}

\begin{sphinxVerbatim}[commandchars=\\\{\}]
\PYG{g+go}{178}
\end{sphinxVerbatim}

\begin{sphinxVerbatim}[commandchars=\\\{\}]
\PYG{c+c1}{\PYGZsh{} Representação octal de 8:}
\PYG{l+m+mo}{0o10}
\end{sphinxVerbatim}

\begin{sphinxVerbatim}[commandchars=\\\{\}]
\PYG{g+go}{8}
\end{sphinxVerbatim}

\begin{sphinxVerbatim}[commandchars=\\\{\}]
\PYG{c+c1}{\PYGZsh{} Representação binária de 14:}
\PYG{l+m+mb}{0b1110}
\end{sphinxVerbatim}

\begin{sphinxVerbatim}[commandchars=\\\{\}]
\PYG{g+go}{14}
\end{sphinxVerbatim}

\begin{sphinxVerbatim}[commandchars=\\\{\}]
\PYG{c+c1}{\PYGZsh{} Número 7 (sete) convertido para as bases}
\PYG{c+c1}{\PYGZsh{} binária, hexadecimal e octal respectivamente:}
\PYG{n+nb}{bin}\PYG{p}{(}\PYG{l+m+mi}{7}\PYG{p}{)}
\end{sphinxVerbatim}

\begin{sphinxVerbatim}[commandchars=\\\{\}]
\PYG{g+go}{\PYGZsq{}0b111\PYGZsq{}}
\end{sphinxVerbatim}

\begin{sphinxVerbatim}[commandchars=\\\{\}]
\PYG{c+c1}{\PYGZsh{} Retorna a forma octal de 7:}
\PYG{n+nb}{oct}\PYG{p}{(}\PYG{l+m+mi}{7}\PYG{p}{)}
\end{sphinxVerbatim}

\begin{sphinxVerbatim}[commandchars=\\\{\}]
\PYG{g+go}{\PYGZsq{}0o7\PYGZsq{}}
\end{sphinxVerbatim}

\begin{sphinxVerbatim}[commandchars=\\\{\}]
\PYG{c+c1}{\PYGZsh{} Retorna a versão hexadecimal de 7:}
\PYG{n+nb}{hex}\PYG{p}{(}\PYG{l+m+mi}{7}\PYG{p}{)}
\end{sphinxVerbatim}

\begin{sphinxVerbatim}[commandchars=\\\{\}]
\PYG{g+go}{\PYGZsq{}0x7\PYGZsq{}}
\end{sphinxVerbatim}

\begin{sphinxVerbatim}[commandchars=\\\{\}]
\PYG{c+c1}{\PYGZsh{} Descobrir o decimal dada uma base:}
\PYG{n+nb}{int}\PYG{p}{(}\PYG{l+s+s1}{\PYGZsq{}}\PYG{l+s+s1}{facada}\PYG{l+s+s1}{\PYGZsq{}}\PYG{p}{,} \PYG{n}{base}\PYG{o}{=}\PYG{l+m+mi}{16}\PYG{p}{)}
\end{sphinxVerbatim}

\begin{sphinxVerbatim}[commandchars=\\\{\}]
\PYG{g+go}{16435930}
\end{sphinxVerbatim}

\begin{sphinxVerbatim}[commandchars=\\\{\}]
\PYG{c+c1}{\PYGZsh{} 25 octal em decimal é:}
\PYG{n+nb}{int}\PYG{p}{(}\PYG{l+s+s1}{\PYGZsq{}}\PYG{l+s+s1}{25}\PYG{l+s+s1}{\PYGZsq{}}\PYG{p}{,} \PYG{n}{base}\PYG{o}{=}\PYG{l+m+mi}{8}\PYG{p}{)}
\end{sphinxVerbatim}

\begin{sphinxVerbatim}[commandchars=\\\{\}]
\PYG{g+go}{21}
\end{sphinxVerbatim}

\begin{sphinxVerbatim}[commandchars=\\\{\}]
\PYG{c+c1}{\PYGZsh{} 1111 binário é em decimal:}
\PYG{n+nb}{int}\PYG{p}{(}\PYG{l+s+s1}{\PYGZsq{}}\PYG{l+s+s1}{1111}\PYG{l+s+s1}{\PYGZsq{}}\PYG{p}{,} \PYG{n}{base}\PYG{o}{=}\PYG{l+m+mi}{2}\PYG{p}{)}
\end{sphinxVerbatim}

\begin{sphinxVerbatim}[commandchars=\\\{\}]
\PYG{g+go}{15}
\end{sphinxVerbatim}


\section{float}
\label{\detokenize{content/numeric_data_types:float}}\begin{quote}

Ponto flutuante; não tem precisão absoluta, sua precisão é relativa.
Para uma maior precisão com números que tenham ponto flutuante, utilizar o módulo decimal.
\end{quote}

\begin{sphinxVerbatim}[commandchars=\\\{\}]
\PYG{c+c1}{\PYGZsh{} Criação de um float:}
\PYG{n}{f} \PYG{o}{=} \PYG{n+nb}{float}\PYG{p}{(}\PYG{l+m+mi}{3}\PYG{p}{)}
\end{sphinxVerbatim}

ou

\begin{sphinxVerbatim}[commandchars=\\\{\}]
\PYG{n}{f} \PYG{o}{=} \PYG{l+m+mf}{3.0}
\end{sphinxVerbatim}

\begin{sphinxVerbatim}[commandchars=\\\{\}]
\PYG{n}{f}  \PYG{c+c1}{\PYGZsh{} Veririca o valor}
\end{sphinxVerbatim}

\begin{sphinxVerbatim}[commandchars=\\\{\}]
\PYG{g+go}{3.0}
\end{sphinxVerbatim}

Formas de se definir um float:

\begin{sphinxVerbatim}[commandchars=\\\{\}]
\PYG{n}{x} \PYG{o}{=} \PYG{l+m+mf}{0.5000000000}
\end{sphinxVerbatim}

ou

\begin{sphinxVerbatim}[commandchars=\\\{\}]
\PYG{n}{x} \PYG{o}{=} \PYG{l+m+mf}{0.5}
\end{sphinxVerbatim}

ou

\begin{sphinxVerbatim}[commandchars=\\\{\}]
\PYG{n}{x} \PYG{o}{=} \PYG{o}{.}\PYG{l+m+mi}{5}

\PYG{n}{x}  \PYG{c+c1}{\PYGZsh{} Exibe o valor}
\end{sphinxVerbatim}

\begin{sphinxVerbatim}[commandchars=\\\{\}]
\PYG{g+go}{0.5}
\end{sphinxVerbatim}

\begin{sphinxVerbatim}[commandchars=\\\{\}]
\PYG{n+nb}{type}\PYG{p}{(}\PYG{n}{x}\PYG{p}{)}  \PYG{c+c1}{\PYGZsh{} Tipo}
\end{sphinxVerbatim}

\begin{sphinxVerbatim}[commandchars=\\\{\}]
\PYG{g+go}{float}
\end{sphinxVerbatim}

\begin{sphinxVerbatim}[commandchars=\\\{\}]
\PYG{n}{x} \PYG{o}{=} \PYG{l+m+mf}{2.}

\PYG{n}{x}  \PYG{c+c1}{\PYGZsh{} Verifica o valor}
\end{sphinxVerbatim}

\begin{sphinxVerbatim}[commandchars=\\\{\}]
\PYG{g+go}{2.0}
\end{sphinxVerbatim}

\begin{sphinxVerbatim}[commandchars=\\\{\}]
\PYG{c+c1}{\PYGZsh{} Que tipo resulta de da soma de um inteiro e um float?:}
    \PYG{n+nb}{type}\PYG{p}{(}\PYG{l+m+mi}{7} \PYG{o}{+} \PYG{l+m+mf}{3.0}\PYG{p}{)}
\end{sphinxVerbatim}

\begin{sphinxVerbatim}[commandchars=\\\{\}]
\PYG{g+go}{float}
\end{sphinxVerbatim}

\begin{sphinxVerbatim}[commandchars=\\\{\}]
\PYG{c+c1}{\PYGZsh{} Resultado:}
\PYG{l+m+mi}{7} \PYG{o}{+} \PYG{l+m+mf}{3.0}
\end{sphinxVerbatim}

\begin{sphinxVerbatim}[commandchars=\\\{\}]
\PYG{g+go}{10.0}
\end{sphinxVerbatim}

\begin{sphinxVerbatim}[commandchars=\\\{\}]
\PYG{c+c1}{\PYGZsh{} Divisão:}
\PYG{l+m+mi}{3} \PYG{o}{/} \PYG{l+m+mi}{2}
\end{sphinxVerbatim}

ou

\begin{sphinxVerbatim}[commandchars=\\\{\}]
\PYG{l+m+mi}{3} \PYG{o}{/} \PYG{l+m+mf}{2.0}
\end{sphinxVerbatim}

ou

\begin{sphinxVerbatim}[commandchars=\\\{\}]
\PYG{l+m+mf}{3.0} \PYG{o}{/} \PYG{l+m+mi}{2}
\end{sphinxVerbatim}

ou

\begin{sphinxVerbatim}[commandchars=\\\{\}]
\PYG{l+m+mf}{3.0} \PYG{o}{/} \PYG{l+m+mf}{2.0}
\end{sphinxVerbatim}

\begin{sphinxVerbatim}[commandchars=\\\{\}]
\PYG{g+go}{1.5}
\end{sphinxVerbatim}

\begin{sphinxVerbatim}[commandchars=\\\{\}]
\PYG{c+c1}{\PYGZsh{} Divisão Inteira:}
\PYG{l+m+mi}{3} \PYG{o}{/}\PYG{o}{/} \PYG{l+m+mf}{2.0}
\end{sphinxVerbatim}

\begin{sphinxVerbatim}[commandchars=\\\{\}]
\PYG{g+go}{1.0}
\end{sphinxVerbatim}

\begin{sphinxVerbatim}[commandchars=\\\{\}]
\PYG{c+c1}{\PYGZsh{} Notação Científica:}
\PYG{l+m+mf}{1e+2}
\end{sphinxVerbatim}

\begin{sphinxVerbatim}[commandchars=\\\{\}]
\PYG{g+go}{100.0}
\end{sphinxVerbatim}

\begin{sphinxVerbatim}[commandchars=\\\{\}]
\PYG{c+c1}{\PYGZsh{} Notação científica com expoente negativo:}
\PYG{l+m+mf}{1e\PYGZhy{}3}
\end{sphinxVerbatim}

\begin{sphinxVerbatim}[commandchars=\\\{\}]
\PYG{g+go}{0.001}
\end{sphinxVerbatim}


\section{complex}
\label{\detokenize{content/numeric_data_types:complex}}\begin{quote}

É o tipo de dados em Python que trata de números complexos, que são muito utilizados em engenharia elétrica.
\end{quote}

\begin{sphinxVerbatim}[commandchars=\\\{\}]
\PYG{c+c1}{\PYGZsh{} Número complexo somente com a parte real:}
\PYG{n}{c} \PYG{o}{=} \PYG{n+nb}{complex}\PYG{p}{(}\PYG{l+m+mi}{1}\PYG{p}{)}
\PYG{n+nb}{print}\PYG{p}{(}\PYG{n}{c}\PYG{p}{)}
\end{sphinxVerbatim}

\begin{sphinxVerbatim}[commandchars=\\\{\}]
\PYG{g+gp+gpVirtualEnv}{(1+0j)}
\end{sphinxVerbatim}

\begin{sphinxVerbatim}[commandchars=\\\{\}]
\PYG{c+c1}{\PYGZsh{} Verificando seu valor e seu tipo:}
\PYG{n+nb}{type}\PYG{p}{(}\PYG{n}{c}\PYG{p}{)}
\end{sphinxVerbatim}

\begin{sphinxVerbatim}[commandchars=\\\{\}]
\PYG{g+go}{complex}
\end{sphinxVerbatim}

\begin{sphinxVerbatim}[commandchars=\\\{\}]
\PYG{c+c1}{\PYGZsh{} Novo valor do número complexo com parte real e imaginária:}
\PYG{n}{c} \PYG{o}{=} \PYG{n+nb}{complex}\PYG{p}{(}\PYG{l+m+mi}{5}\PYG{p}{,} \PYG{l+m+mi}{3}\PYG{p}{)}
\end{sphinxVerbatim}

\begin{sphinxVerbatim}[commandchars=\\\{\}]
\PYG{n}{c}  \PYG{c+c1}{\PYGZsh{} Verificando o valor}
\end{sphinxVerbatim}

\begin{sphinxVerbatim}[commandchars=\\\{\}]
\PYG{g+gp+gpVirtualEnv}{(5+3j)}
\end{sphinxVerbatim}

\begin{sphinxVerbatim}[commandchars=\\\{\}]
\PYG{c+c1}{\PYGZsh{} Número complexo somente com a parte imaginária:}
\PYG{n}{c} \PYG{o}{=} \PYG{n+nb}{complex}\PYG{p}{(}\PYG{l+m+mi}{0}\PYG{p}{,} \PYG{l+m+mi}{3}\PYG{p}{)}
\end{sphinxVerbatim}

\begin{sphinxVerbatim}[commandchars=\\\{\}]
\PYG{n}{c}  \PYG{c+c1}{\PYGZsh{} Verificando seu valor}
\end{sphinxVerbatim}

\begin{sphinxVerbatim}[commandchars=\\\{\}]
\PYG{g+go}{3j}
\end{sphinxVerbatim}

\begin{sphinxVerbatim}[commandchars=\\\{\}]
\PYG{n}{c}\PYG{o}{.}\PYG{n}{imag}  \PYG{c+c1}{\PYGZsh{} Extraindo somente a parte imaginária}
\end{sphinxVerbatim}

\begin{sphinxVerbatim}[commandchars=\\\{\}]
\PYG{g+go}{3.0}
\end{sphinxVerbatim}

\begin{sphinxVerbatim}[commandchars=\\\{\}]
\PYG{n}{c}\PYG{o}{.}\PYG{n}{real}  \PYG{c+c1}{\PYGZsh{} Extraindo somente a parte real}
\end{sphinxVerbatim}

\begin{sphinxVerbatim}[commandchars=\\\{\}]
\PYG{g+go}{0.0}
\end{sphinxVerbatim}

\begin{sphinxVerbatim}[commandchars=\\\{\}]
\PYG{n}{c} \PYG{o}{+} \PYG{l+m+mi}{1}  \PYG{c+c1}{\PYGZsh{} Somando o número com a parte real}
\end{sphinxVerbatim}

\begin{sphinxVerbatim}[commandchars=\\\{\}]
\PYG{g+gp+gpVirtualEnv}{(1+3j)}
\end{sphinxVerbatim}

\begin{sphinxVerbatim}[commandchars=\\\{\}]
\PYG{n}{c} \PYG{o}{+} \PYG{n+nb}{complex}\PYG{p}{(}\PYG{l+s+s1}{\PYGZsq{}}\PYG{l+s+s1}{7j}\PYG{l+s+s1}{\PYGZsq{}}\PYG{p}{)}  \PYG{c+c1}{\PYGZsh{} Somando o número com a parte imaginária}
\end{sphinxVerbatim}

\begin{sphinxVerbatim}[commandchars=\\\{\}]
\PYG{g+go}{10j}
\end{sphinxVerbatim}

\begin{sphinxVerbatim}[commandchars=\\\{\}]
\PYG{n}{c} \PYG{o}{+} \PYG{n+nb}{complex}\PYG{p}{(}\PYG{l+m+mi}{2}\PYG{p}{,} \PYG{l+m+mi}{17}\PYG{p}{)}  \PYG{c+c1}{\PYGZsh{} somando o número complexo com outro complexo}
\end{sphinxVerbatim}

\begin{sphinxVerbatim}[commandchars=\\\{\}]
\PYG{g+gp+gpVirtualEnv}{(2+20j)}
\end{sphinxVerbatim}


\chapter{Booleanos}
\label{\detokenize{content/boolean:booleanos}}\label{\detokenize{content/boolean::doc}}\begin{quote}

Qualquer objeto pode ser testado como um valor verdadeiro, para uso como condição em um if ou um while ou como um operando de operações “booleanas”. Os seguintes valores são considerados como False:
\begin{quote}

\sphinxcode{\sphinxupquote{False}}, \sphinxcode{\sphinxupquote{0}}, \sphinxcode{\sphinxupquote{0.0}}, \sphinxcode{\sphinxupquote{0j}}, \sphinxcode{\sphinxupquote{{[}{]}}}, \sphinxcode{\sphinxupquote{()}}, \sphinxcode{\sphinxupquote{\{\}}}, \sphinxcode{\sphinxupquote{set({[}{]})}}, \sphinxcode{\sphinxupquote{None}}, \sphinxcode{\sphinxupquote{""}} ou \sphinxcode{\sphinxupquote{\textquotesingle{}\textquotesingle{}}}.

Instâncias de classes definidas por usuários, se a classe define um método \_\_bool\_\_() ou \_\_len\_\_(), quando este método retornar o inteiro zero ou um valor booleano False.
Todos outros valores são considerados verdadeiros, então objetos de muitos tipos são considerados como True.
\end{quote}
\end{quote}

\begin{sphinxVerbatim}[commandchars=\\\{\}]
\PYG{c+c1}{\PYGZsh{} Definindo a variável como True de forma indireta:}
\PYG{n}{b} \PYG{o}{=} \PYG{n+nb}{bool}\PYG{p}{(}\PYG{l+m+mi}{1}\PYG{p}{)}

\PYG{c+c1}{\PYGZsh{} Verificando o valor da variável:}
\PYG{n+nb}{print}\PYG{p}{(}\PYG{n}{b}\PYG{p}{)}
\end{sphinxVerbatim}

\begin{sphinxVerbatim}[commandchars=\\\{\}]
\PYG{g+go}{True}
\end{sphinxVerbatim}

\begin{sphinxVerbatim}[commandchars=\\\{\}]
\PYG{c+c1}{\PYGZsh{} Definindo a variável como False de forma indireta:}
\PYG{n}{b} \PYG{o}{=} \PYG{n+nb}{bool}\PYG{p}{(}\PYG{l+m+mi}{0}\PYG{p}{)}

\PYG{c+c1}{\PYGZsh{} Verificando o valor da variável:}
\PYG{n+nb}{print}\PYG{p}{(}\PYG{n}{b}\PYG{p}{)}
\end{sphinxVerbatim}

\begin{sphinxVerbatim}[commandchars=\\\{\}]
\PYG{g+go}{False}
\end{sphinxVerbatim}

\begin{sphinxVerbatim}[commandchars=\\\{\}]
\PYG{c+c1}{\PYGZsh{} Teste usando a lógica OR:}
\PYG{k+kc}{True} \PYG{o}{|} \PYG{k+kc}{False}
\end{sphinxVerbatim}

\begin{sphinxVerbatim}[commandchars=\\\{\}]
\PYG{g+go}{True}
\end{sphinxVerbatim}

\begin{sphinxVerbatim}[commandchars=\\\{\}]
\PYG{c+c1}{\PYGZsh{} Teste usando a lógica AND:}
\PYG{k+kc}{True} \PYG{o}{\PYGZam{}} \PYG{k+kc}{False}
\end{sphinxVerbatim}

\begin{sphinxVerbatim}[commandchars=\\\{\}]
\PYG{g+go}{False}
\end{sphinxVerbatim}

\begin{sphinxVerbatim}[commandchars=\\\{\}]
\PYG{c+c1}{\PYGZsh{} Negação de True:}
\PYG{o+ow}{not} \PYG{k+kc}{True}
\end{sphinxVerbatim}

\begin{sphinxVerbatim}[commandchars=\\\{\}]
\PYG{g+go}{False}
\end{sphinxVerbatim}

\begin{sphinxVerbatim}[commandchars=\\\{\}]
\PYG{c+c1}{\PYGZsh{} Negação de False:}
\PYG{o+ow}{not} \PYG{k+kc}{False}
\end{sphinxVerbatim}

\begin{sphinxVerbatim}[commandchars=\\\{\}]
\PYG{g+go}{True}
\end{sphinxVerbatim}

\begin{sphinxVerbatim}[commandchars=\\\{\}]
\PYG{c+c1}{\PYGZsh{} Criação de classes de teste:}
\PYG{k}{class} \PYG{n+nc}{Foo}\PYG{p}{(}\PYG{n+nb}{object}\PYG{p}{)}\PYG{p}{:}
    \PYG{k}{def} \PYG{n+nf+fm}{\PYGZus{}\PYGZus{}len\PYGZus{}\PYGZus{}}\PYG{p}{(}\PYG{n+nb+bp}{self}\PYG{p}{)}\PYG{p}{:}
        \PYG{k}{return} \PYG{l+m+mi}{1}

\PYG{k}{class} \PYG{n+nc}{Bar}\PYG{p}{(}\PYG{n+nb}{object}\PYG{p}{)}\PYG{p}{:}
    \PYG{k}{def} \PYG{n+nf+fm}{\PYGZus{}\PYGZus{}len\PYGZus{}\PYGZus{}}\PYG{p}{(}\PYG{n+nb+bp}{self}\PYG{p}{)}\PYG{p}{:}
        \PYG{k}{return} \PYG{l+m+mi}{0}
\end{sphinxVerbatim}

\begin{sphinxVerbatim}[commandchars=\\\{\}]
\PYG{c+c1}{\PYGZsh{} Criação de objetos:}
\PYG{n}{foo} \PYG{o}{=} \PYG{n}{Foo}\PYG{p}{(}\PYG{p}{)}
\PYG{n}{bar} \PYG{o}{=} \PYG{n}{Bar}\PYG{p}{(}\PYG{p}{)}

\PYG{c+c1}{\PYGZsh{} Verificando o valor booleano dos objetos:}
\PYG{n+nb}{bool}\PYG{p}{(}\PYG{n}{foo}\PYG{p}{)}
\end{sphinxVerbatim}

\begin{sphinxVerbatim}[commandchars=\\\{\}]
\PYG{g+go}{True}
\end{sphinxVerbatim}

\begin{sphinxVerbatim}[commandchars=\\\{\}]
\PYG{n+nb}{bool}\PYG{p}{(}\PYG{n}{bar}\PYG{p}{)}
\end{sphinxVerbatim}

\begin{sphinxVerbatim}[commandchars=\\\{\}]
\PYG{g+go}{False}
\end{sphinxVerbatim}

\begin{sphinxVerbatim}[commandchars=\\\{\}]
\PYG{c+c1}{\PYGZsh{} True AND (NOT False):}
\PYG{n+nb}{bool}\PYG{p}{(}\PYG{n}{foo}\PYG{p}{)} \PYG{o}{\PYGZam{}} \PYG{p}{(}\PYG{o+ow}{not} \PYG{n+nb}{bool}\PYG{p}{(}\PYG{n}{bar}\PYG{p}{)}\PYG{p}{)}
\end{sphinxVerbatim}

\begin{sphinxVerbatim}[commandchars=\\\{\}]
\PYG{g+go}{True}
\end{sphinxVerbatim}

\begin{sphinxVerbatim}[commandchars=\\\{\}]
\PYG{c+c1}{\PYGZsh{} True AND False:}
\PYG{n+nb}{bool}\PYG{p}{(}\PYG{n}{foo}\PYG{p}{)} \PYG{o}{\PYGZam{}} \PYG{n+nb}{bool}\PYG{p}{(}\PYG{n}{bar}\PYG{p}{)}
\end{sphinxVerbatim}

\begin{sphinxVerbatim}[commandchars=\\\{\}]
\PYG{g+go}{False}
\end{sphinxVerbatim}

\begin{sphinxVerbatim}[commandchars=\\\{\}]
\PYG{c+c1}{\PYGZsh{} Classe para testar os métodos \PYGZus{}\PYGZus{}bool\PYGZus{}\PYGZus{} e \PYGZus{}\PYGZus{}len\PYGZus{}\PYGZus{}:}
\PYG{k}{class} \PYG{n+nc}{Spam}\PYG{p}{(}\PYG{n+nb}{object}\PYG{p}{)}\PYG{p}{:}
    \PYG{k}{def} \PYG{n+nf+fm}{\PYGZus{}\PYGZus{}bool\PYGZus{}\PYGZus{}}\PYG{p}{(}\PYG{n+nb+bp}{self}\PYG{p}{)}\PYG{p}{:}
        \PYG{k}{return} \PYG{k+kc}{True}

    \PYG{k}{def} \PYG{n+nf+fm}{\PYGZus{}\PYGZus{}len\PYGZus{}\PYGZus{}}\PYG{p}{(}\PYG{n+nb+bp}{self}\PYG{p}{)}\PYG{p}{:}
        \PYG{k}{return} \PYG{l+m+mi}{0}
\end{sphinxVerbatim}

O método \_\_bool\_\_ retorna um valor verdadeiro e o método \_\_len\_\_ um falso.
Com ambos declarados na mesma classe, um objeto dela o que retornará?

\begin{sphinxVerbatim}[commandchars=\\\{\}]
\PYG{c+c1}{\PYGZsh{} Criação de objeto:}
\PYG{n}{spam} \PYG{o}{=} \PYG{n}{Spam}\PYG{p}{(}\PYG{p}{)}

\PYG{c+c1}{\PYGZsh{} Verificando o valor booleano:}
\PYG{n+nb}{bool}\PYG{p}{(}\PYG{n}{spam}\PYG{p}{)}
\end{sphinxVerbatim}

\begin{sphinxVerbatim}[commandchars=\\\{\}]
\PYG{g+go}{True}
\end{sphinxVerbatim}

O retorno foi verdadeiro, o método \sphinxcode{\sphinxupquote{\_\_bool\_\_}} prevalece.


\chapter{O Módulo decimal}
\label{\detokenize{content/decimal_module:o-modulo-decimal}}\label{\detokenize{content/decimal_module::doc}}\begin{quote}
\begin{quote}

O módulo decimal fornece suporte para aritmética de ponto flutuante decimal.
Ele oferece muitas vantagens sobre o tipo float.
\end{quote}

O módulo decimal “é baseado em um modelo de ponto flutuante que foi projetado com pessoas em mente e necessariamente tem um princípio orientador fundamental \sphinxhyphen{} computadores devem fornecer uma aritmética que funciona da mesma maneira que a aritmética que as pessoas aprendem na escola” \sphinxhyphen{} extraindo da especificação da aritmética decimal.
Números decimais podem ser representados exatamente. Em contraste, números como 1.1 e 2.2 não têm representações exatas em ponto flutuante binário. Usuários finais não esperam que 1.1 + 2.2 seja exibido como 3.3000000000000003 como é feito como o ponto flutuante binário.
\begin{quote}

A exatidão leva à aritmética. Com ponto flutuante decimal, 0.1 + 0.1 + 0.1 \sphinxhyphen{} 0.3 é exatamente igual a zero. Com ponto flutuante binário, o resultado é 5.5511151231257827e\sphinxhyphen{}017.
\end{quote}

Enquanto próximo a zero, as diferenças evitam testes de igualdade confiáveis e diferenças podem acumular. Por essa razão, decimal é preferido em aplicações de contabilidade que tem invariações de igualdade rigorosas.
O módulo decimal incorpora a noção de posições (casas decimais) significantes em que 1.30 + 1.20 é 2.50. O zero no fim é mantido para indicar a significância. Esta é a apresentação habitual para aplicações monetárias. Para multiplicação, a abordagem do “livro de escola” usa todos os números nos multiplicandos. Por exemplo, 1.3 * 1.2 dá 1.56 enquanto que 1.30 * 1.20 dá 1.5600.
Ao contrário de hardware baseado em ponto flutuante binário, o módulo decimal tem uma precisão alterável (padrão 28 casas decimais).
\end{quote}

\begin{sphinxVerbatim}[commandchars=\\\{\}]
\PYG{c+c1}{\PYGZsh{} Importação do módulo:}
\PYG{k+kn}{import} \PYG{n+nn}{decimal}

\PYG{c+c1}{\PYGZsh{} Criação do objeto Decimal:}
\PYG{n}{d} \PYG{o}{=} \PYG{n}{decimal}\PYG{o}{.}\PYG{n}{Decimal}\PYG{p}{(}\PYG{l+s+s1}{\PYGZsq{}}\PYG{l+s+s1}{0.777}\PYG{l+s+s1}{\PYGZsq{}}\PYG{p}{)}

\PYG{c+c1}{\PYGZsh{} Exibindo o valor do objeto decimal:}
\PYG{n}{d}
\end{sphinxVerbatim}

\begin{sphinxVerbatim}[commandchars=\\\{\}]
\PYG{g+go}{Decimal(\PYGZsq{}0.777\PYGZsq{})}
\end{sphinxVerbatim}

\begin{sphinxVerbatim}[commandchars=\\\{\}]
\PYG{c+c1}{\PYGZsh{} Exibe o valor de \PYGZdq{}d\PYGZdq{}:}
\PYG{n+nb}{print}\PYG{p}{(}\PYG{n}{d}\PYG{p}{)}
\end{sphinxVerbatim}

\begin{sphinxVerbatim}[commandchars=\\\{\}]
\PYG{g+go}{0.777}
\end{sphinxVerbatim}

Algumas coisas “estranhas”:

\begin{sphinxVerbatim}[commandchars=\\\{\}]
\PYG{c+c1}{\PYGZsh{} Somando dois floats:}
\PYG{l+m+mf}{1.1} \PYG{o}{+} \PYG{l+m+mf}{2.2}
\end{sphinxVerbatim}

\begin{sphinxVerbatim}[commandchars=\\\{\}]
\PYG{g+go}{3.3000000000000003}
\end{sphinxVerbatim}

\begin{sphinxVerbatim}[commandchars=\\\{\}]
\PYG{c+c1}{\PYGZsh{} Outra operação com floats:}
\PYG{l+m+mf}{0.1} \PYG{o}{+} \PYG{l+m+mf}{0.1} \PYG{o}{+} \PYG{l+m+mf}{0.1} \PYG{o}{\PYGZhy{}} \PYG{l+m+mf}{0.3}
\end{sphinxVerbatim}

\begin{sphinxVerbatim}[commandchars=\\\{\}]
\PYG{g+go}{5.551115123125783e\PYGZhy{}17}
\end{sphinxVerbatim}

\begin{sphinxVerbatim}[commandchars=\\\{\}]
\PYG{c+c1}{\PYGZsh{} Importando apenas a classe Decimal:}
\PYG{k+kn}{from} \PYG{n+nn}{decimal} \PYG{k+kn}{import} \PYG{n}{Decimal}

\PYG{c+c1}{\PYGZsh{} Fazendo as mesmas operações anteriores com o módulo decimal:}
\PYG{n}{Decimal}\PYG{p}{(}\PYG{l+s+s1}{\PYGZsq{}}\PYG{l+s+s1}{1.1}\PYG{l+s+s1}{\PYGZsq{}}\PYG{p}{)} \PYG{o}{+} \PYG{n}{Decimal}\PYG{p}{(}\PYG{l+s+s1}{\PYGZsq{}}\PYG{l+s+s1}{2.2}\PYG{l+s+s1}{\PYGZsq{}}\PYG{p}{)}
\end{sphinxVerbatim}

\begin{sphinxVerbatim}[commandchars=\\\{\}]
\PYG{g+go}{Decimal(\PYGZsq{}3.3\PYGZsq{})}
\end{sphinxVerbatim}

\begin{sphinxVerbatim}[commandchars=\\\{\}]
\PYG{c+c1}{\PYGZsh{} Importando a função getcontext:}
\PYG{k+kn}{from} \PYG{n+nn}{decimal} \PYG{k+kn}{import} \PYG{n}{getcontext}

\PYG{c+c1}{\PYGZsh{} Operações de multiplicação:}
\PYG{n}{Decimal}\PYG{p}{(}\PYG{l+s+s1}{\PYGZsq{}}\PYG{l+s+s1}{1.3}\PYG{l+s+s1}{\PYGZsq{}}\PYG{p}{)} \PYG{o}{*} \PYG{n}{Decimal}\PYG{p}{(}\PYG{l+s+s1}{\PYGZsq{}}\PYG{l+s+s1}{1.2}\PYG{l+s+s1}{\PYGZsq{}}\PYG{p}{)}
\end{sphinxVerbatim}

\begin{sphinxVerbatim}[commandchars=\\\{\}]
\PYG{g+go}{Decimal(\PYGZsq{}1.56\PYGZsq{})}
\end{sphinxVerbatim}

\begin{sphinxVerbatim}[commandchars=\\\{\}]
\PYG{c+c1}{\PYGZsh{} Multiplicação de decimais:}
\PYG{n}{Decimal}\PYG{p}{(}\PYG{l+s+s1}{\PYGZsq{}}\PYG{l+s+s1}{1.30}\PYG{l+s+s1}{\PYGZsq{}}\PYG{p}{)} \PYG{o}{*} \PYG{n}{Decimal}\PYG{p}{(}\PYG{l+s+s1}{\PYGZsq{}}\PYG{l+s+s1}{1.20}\PYG{l+s+s1}{\PYGZsq{}}\PYG{p}{)}
\end{sphinxVerbatim}

\begin{sphinxVerbatim}[commandchars=\\\{\}]
\PYG{g+go}{Decimal(\PYGZsq{}1.5600\PYGZsq{})}
\end{sphinxVerbatim}

O último retorno foi com um número com 4 (quatro) casas decimais.

A função getcontext, com o atributo “prec” (precision) ajusta a quantidade máxima de dígitos (antes e depois do ponto flutuante) para 3 (três):

\begin{sphinxVerbatim}[commandchars=\\\{\}]
\PYG{c+c1}{\PYGZsh{} Alterar o nível de precisão para 3:}
\PYG{n}{getcontext}\PYG{p}{(}\PYG{p}{)}\PYG{o}{.}\PYG{n}{prec} \PYG{o}{=} \PYG{l+m+mi}{3}
\end{sphinxVerbatim}

A precisão vai ser refletida em operações com o módulo decimal.
Caso seja necessário o número será arredondado.

\begin{sphinxVerbatim}[commandchars=\\\{\}]
\PYG{c+c1}{\PYGZsh{} Operação de multiplicação de números com ponto flutuante}
\PYG{n}{Decimal}\PYG{p}{(}\PYG{l+s+s1}{\PYGZsq{}}\PYG{l+s+s1}{1.300}\PYG{l+s+s1}{\PYGZsq{}}\PYG{p}{)} \PYG{o}{*} \PYG{n}{Decimal}\PYG{p}{(}\PYG{l+s+s1}{\PYGZsq{}}\PYG{l+s+s1}{1.200}\PYG{l+s+s1}{\PYGZsq{}}\PYG{p}{)}
\end{sphinxVerbatim}

\begin{sphinxVerbatim}[commandchars=\\\{\}]
\PYG{g+go}{Decimal(\PYGZsq{}1.56\PYGZsq{})}
\end{sphinxVerbatim}

\begin{sphinxVerbatim}[commandchars=\\\{\}]
\PYG{c+c1}{\PYGZsh{} Ajustando a precisão para 10 (dez)}
\PYG{n}{getcontext}\PYG{p}{(}\PYG{p}{)}\PYG{o}{.}\PYG{n}{prec} \PYG{o}{=} \PYG{l+m+mi}{10}

\PYG{c+c1}{\PYGZsh{} Multiplicação:}
\PYG{n}{Decimal}\PYG{p}{(}\PYG{l+s+s1}{\PYGZsq{}}\PYG{l+s+s1}{1.3897}\PYG{l+s+s1}{\PYGZsq{}}\PYG{p}{)} \PYG{o}{*} \PYG{l+m+mi}{2}
\end{sphinxVerbatim}

\begin{sphinxVerbatim}[commandchars=\\\{\}]
\PYG{g+go}{Decimal(\PYGZsq{}2.7794\PYGZsq{})}
\end{sphinxVerbatim}

\begin{sphinxVerbatim}[commandchars=\\\{\}]
\PYG{c+c1}{\PYGZsh{} Ajustando a precisão para 3 (três):}
\PYG{n}{getcontext}\PYG{p}{(}\PYG{p}{)}\PYG{o}{.}\PYG{n}{prec} \PYG{o}{=} \PYG{l+m+mi}{3}

\PYG{c+c1}{\PYGZsh{} Multiplicação:}
\PYG{n}{Decimal}\PYG{p}{(}\PYG{l+s+s1}{\PYGZsq{}}\PYG{l+s+s1}{1.3897}\PYG{l+s+s1}{\PYGZsq{}}\PYG{p}{)} \PYG{o}{*} \PYG{l+m+mi}{2}
\end{sphinxVerbatim}

\begin{sphinxVerbatim}[commandchars=\\\{\}]
\PYG{g+go}{Decimal(\PYGZsq{}2.78\PYGZsq{})}
\end{sphinxVerbatim}

Nota\sphinxhyphen{}se que foi feito um arredondamento do número.


\chapter{Data e Hora}
\label{\detokenize{content/date_time:data-e-hora}}\label{\detokenize{content/date_time::doc}}

\section{O Módulo datetime}
\label{\detokenize{content/date_time:o-modulo-datetime}}
Módulo que fornece classes para manipular datas e horas de maneiras simples e
complexas.
Enquanto data e hora são suportados aritmeticamente, o foco da implementação
estã em uma extração eficiente de atributo para saída formatada e manipulação.

\begin{sphinxVerbatim}[commandchars=\\\{\}]
\PYG{c+c1}{\PYGZsh{} Imports:}
\PYG{k+kn}{from} \PYG{n+nn}{datetime} \PYG{k+kn}{import} \PYG{n}{date}
\PYG{k+kn}{from} \PYG{n+nn}{datetime} \PYG{k+kn}{import} \PYG{n}{datetime}
\PYG{k+kn}{from} \PYG{n+nn}{sys} \PYG{k+kn}{import} \PYG{n}{getsizeof}
\end{sphinxVerbatim}

\begin{sphinxVerbatim}[commandchars=\\\{\}]
\PYG{c+c1}{\PYGZsh{} Inserir dados via teclado conforme sugere o modelo na mensagem:}
\PYG{n}{dt\PYGZus{}evento} \PYG{o}{=} \PYG{n+nb}{input}\PYG{p}{(}\PYG{l+s+s1}{\PYGZsq{}}\PYG{l+s+s1}{Digite a data e hora do evento (AAAA\PYGZhy{}MM\PYGZhy{}DD HH:MM): }\PYG{l+s+s1}{\PYGZsq{}}\PYG{p}{)}
\end{sphinxVerbatim}

\begin{sphinxVerbatim}[commandchars=\\\{\}]
\PYG{g+go}{Digite a data e hora do evento (AAAA\PYGZhy{}MM\PYGZhy{}DD HH:MM):}
\end{sphinxVerbatim}

\begin{sphinxVerbatim}[commandchars=\\\{\}]
\PYG{c+c1}{\PYGZsh{} Verificando o tipo da variável:}
\PYG{n+nb}{type}\PYG{p}{(}\PYG{n}{dt\PYGZus{}evento}\PYG{p}{)}
\end{sphinxVerbatim}

\begin{sphinxVerbatim}[commandchars=\\\{\}]
\PYG{g+go}{str}
\end{sphinxVerbatim}

\begin{sphinxVerbatim}[commandchars=\\\{\}]
\PYG{c+c1}{\PYGZsh{} Quanto custa essa variável em bytes?:}
\PYG{n}{getsizeof}\PYG{p}{(}\PYG{n}{dt\PYGZus{}evento}\PYG{p}{)}
\end{sphinxVerbatim}

\begin{sphinxVerbatim}[commandchars=\\\{\}]
\PYG{g+go}{65}
\end{sphinxVerbatim}

\begin{DUlineblock}{0em}
\item[] Strings não são adequadas para armazenar data e hora.
\item[] strptime transforma uma string para datetime conforme uma dada máscara:
\end{DUlineblock}

\begin{sphinxVerbatim}[commandchars=\\\{\}]
\PYG{c+c1}{\PYGZsh{} Converter a string para datetime:}
\PYG{n}{datetime}\PYG{o}{.}\PYG{n}{strptime}\PYG{p}{(}\PYG{n}{dt\PYGZus{}evento}\PYG{p}{,} \PYG{l+s+s1}{\PYGZsq{}}\PYG{l+s+s1}{\PYGZpc{}}\PYG{l+s+s1}{Y\PYGZhy{}}\PYG{l+s+s1}{\PYGZpc{}}\PYG{l+s+s1}{m\PYGZhy{}}\PYG{l+s+si}{\PYGZpc{}d}\PYG{l+s+s1}{ }\PYG{l+s+s1}{\PYGZpc{}}\PYG{l+s+s1}{H:}\PYG{l+s+s1}{\PYGZpc{}}\PYG{l+s+s1}{M}\PYG{l+s+s1}{\PYGZsq{}}\PYG{p}{)}
\end{sphinxVerbatim}

\begin{sphinxVerbatim}[commandchars=\\\{\}]
\PYG{g+go}{datetime.datetime(2019, 12, 27, 19, 0)}
\end{sphinxVerbatim}

\sphinxstylestrong{strptime: str \sphinxhyphen{}\textgreater{} datetime}

\begin{sphinxVerbatim}[commandchars=\\\{\}]
\PYG{c+c1}{\PYGZsh{} O tamanho em bytes do dado em datetime:}
\PYG{n}{getsizeof}\PYG{p}{(}\PYG{n}{datetime}\PYG{o}{.}\PYG{n}{strptime}\PYG{p}{(}\PYG{n}{dt\PYGZus{}evento}\PYG{p}{,} \PYG{l+s+s1}{\PYGZsq{}}\PYG{l+s+s1}{\PYGZpc{}}\PYG{l+s+s1}{Y\PYGZhy{}}\PYG{l+s+s1}{\PYGZpc{}}\PYG{l+s+s1}{m\PYGZhy{}}\PYG{l+s+si}{\PYGZpc{}d}\PYG{l+s+s1}{ }\PYG{l+s+s1}{\PYGZpc{}}\PYG{l+s+s1}{H:}\PYG{l+s+s1}{\PYGZpc{}}\PYG{l+s+s1}{M}\PYG{l+s+s1}{\PYGZsq{}}\PYG{p}{)}\PYG{p}{)}
\end{sphinxVerbatim}

\begin{sphinxVerbatim}[commandchars=\\\{\}]
\PYG{g+go}{48}
\end{sphinxVerbatim}

A mesma informação armazenada como datetime ocupa menos espaço que string.

\begin{sphinxVerbatim}[commandchars=\\\{\}]
\PYG{c+c1}{\PYGZsh{} Recriar a variável como datetime utilizando seu valor antigo de string:}
\PYG{n}{dt\PYGZus{}evento} \PYG{o}{=} \PYG{n}{datetime}\PYG{o}{.}\PYG{n}{strptime}\PYG{p}{(}\PYG{n}{dt\PYGZus{}evento}\PYG{p}{,} \PYG{l+s+s1}{\PYGZsq{}}\PYG{l+s+s1}{\PYGZpc{}}\PYG{l+s+s1}{Y\PYGZhy{}}\PYG{l+s+s1}{\PYGZpc{}}\PYG{l+s+s1}{m\PYGZhy{}}\PYG{l+s+si}{\PYGZpc{}d}\PYG{l+s+s1}{ }\PYG{l+s+s1}{\PYGZpc{}}\PYG{l+s+s1}{H:}\PYG{l+s+s1}{\PYGZpc{}}\PYG{l+s+s1}{M}\PYG{l+s+s1}{\PYGZsq{}}\PYG{p}{)}

\PYG{c+c1}{\PYGZsh{} Verificando o tipo:}
\PYG{n+nb}{type}\PYG{p}{(}\PYG{n}{dt\PYGZus{}evento}\PYG{p}{)}
\end{sphinxVerbatim}

\begin{sphinxVerbatim}[commandchars=\\\{\}]
\PYG{g+go}{datetime.datetime}
\end{sphinxVerbatim}

Pode ser necessário também fazer o caminho inverso, para transformar um dado datetime para string.
\begin{quote}

Para isso pode\sphinxhyphen{}se usar strftime:
\end{quote}

\begin{sphinxVerbatim}[commandchars=\\\{\}]
\PYG{c+c1}{\PYGZsh{} Extrair como string de um dado datetime:}
\PYG{n}{datetime}\PYG{o}{.}\PYG{n}{strftime}\PYG{p}{(}\PYG{n}{dt\PYGZus{}evento}\PYG{p}{,} \PYG{l+s+s1}{\PYGZsq{}}\PYG{l+s+s1}{\PYGZpc{}}\PYG{l+s+s1}{Y\PYGZhy{}}\PYG{l+s+s1}{\PYGZpc{}}\PYG{l+s+s1}{m\PYGZhy{}}\PYG{l+s+si}{\PYGZpc{}d}\PYG{l+s+s1}{ }\PYG{l+s+s1}{\PYGZpc{}}\PYG{l+s+s1}{H:}\PYG{l+s+s1}{\PYGZpc{}}\PYG{l+s+s1}{M}\PYG{l+s+s1}{\PYGZsq{}}\PYG{p}{)}
\end{sphinxVerbatim}

\begin{sphinxVerbatim}[commandchars=\\\{\}]
\PYG{g+go}{\PYGZsq{}2019\PYGZhy{}12\PYGZhy{}27 19:00\PYGZsq{}}
\end{sphinxVerbatim}

\sphinxstylestrong{strftime: datetime \sphinxhyphen{}\textgreater{} str}

\begin{sphinxVerbatim}[commandchars=\\\{\}]
\PYG{c+c1}{\PYGZsh{} Variável que contém apenas a data atual:}
\PYG{n}{hoje} \PYG{o}{=} \PYG{n}{date}\PYG{o}{.}\PYG{n}{today}\PYG{p}{(}\PYG{p}{)}

\PYG{c+c1}{\PYGZsh{} Exibindo o valor da variável:}
\PYG{n+nb}{print}\PYG{p}{(}\PYG{n}{hoje}\PYG{p}{)}
\end{sphinxVerbatim}

\begin{sphinxVerbatim}[commandchars=\\\{\}]
\PYG{g+go}{2019\PYGZhy{}12\PYGZhy{}26}
\end{sphinxVerbatim}

Exibindo apenas partes da data:

\begin{sphinxVerbatim}[commandchars=\\\{\}]
\PYG{c+c1}{\PYGZsh{} dia:}
\PYG{n+nb}{print}\PYG{p}{(}\PYG{n}{hoje}\PYG{o}{.}\PYG{n}{day}\PYG{p}{)}
\end{sphinxVerbatim}

\begin{sphinxVerbatim}[commandchars=\\\{\}]
\PYG{g+go}{26}
\end{sphinxVerbatim}

\begin{sphinxVerbatim}[commandchars=\\\{\}]
\PYG{c+c1}{\PYGZsh{} mês:}
\PYG{n+nb}{print}\PYG{p}{(}\PYG{n}{hoje}\PYG{o}{.}\PYG{n}{month}\PYG{p}{)}
\end{sphinxVerbatim}

\begin{sphinxVerbatim}[commandchars=\\\{\}]
\PYG{g+go}{12}
\end{sphinxVerbatim}

\begin{sphinxVerbatim}[commandchars=\\\{\}]
\PYG{c+c1}{\PYGZsh{} ano:}
\PYG{n+nb}{print}\PYG{p}{(}\PYG{n}{hoje}\PYG{o}{.}\PYG{n}{year}\PYG{p}{)}
\end{sphinxVerbatim}

\begin{sphinxVerbatim}[commandchars=\\\{\}]
\PYG{g+go}{2019}
\end{sphinxVerbatim}

\begin{sphinxVerbatim}[commandchars=\\\{\}]
\PYG{c+c1}{\PYGZsh{} Formato ISO:}
\PYG{n}{hoje}\PYG{o}{.}\PYG{n}{isoformat}\PYG{p}{(}\PYG{p}{)}
\end{sphinxVerbatim}

\begin{sphinxVerbatim}[commandchars=\\\{\}]
\PYG{g+go}{\PYGZsq{}2019\PYGZhy{}12\PYGZhy{}26\PYGZsq{}}
\end{sphinxVerbatim}

\begin{sphinxVerbatim}[commandchars=\\\{\}]
\PYG{c+c1}{\PYGZsh{} Método toordinal; retorna a quantidade de dias}
\PYG{c+c1}{\PYGZsh{} passados desde 01/01/0001:}
\PYG{n}{hoje}\PYG{o}{.}\PYG{n}{toordinal}\PYG{p}{(}\PYG{p}{)}
\end{sphinxVerbatim}

\begin{sphinxVerbatim}[commandchars=\\\{\}]
\PYG{g+go}{737419}
\end{sphinxVerbatim}

\begin{sphinxVerbatim}[commandchars=\\\{\}]
\PYG{c+c1}{\PYGZsh{} Método fromordinal; retorna a data a partir da quantidade}
\PYG{c+c1}{\PYGZsh{} de dias passados desde 01/01/0001:}
\PYG{n}{date}\PYG{o}{.}\PYG{n}{fromordinal}\PYG{p}{(}\PYG{l+m+mi}{737419}\PYG{p}{)}
\end{sphinxVerbatim}

\begin{sphinxVerbatim}[commandchars=\\\{\}]
\PYG{g+go}{datetime.date(2019, 12, 26)}
\end{sphinxVerbatim}

\begin{sphinxVerbatim}[commandchars=\\\{\}]
\PYG{c+c1}{\PYGZsh{} Que dia será daqui a 40 dias?:}
\PYG{n}{date}\PYG{o}{.}\PYG{n}{fromordinal}\PYG{p}{(}\PYG{n}{hoje}\PYG{o}{.}\PYG{n}{toordinal}\PYG{p}{(}\PYG{p}{)} \PYG{o}{+} \PYG{l+m+mi}{40}\PYG{p}{)}   \PYG{c+c1}{\PYGZsh{} formato datetime.date}
\end{sphinxVerbatim}

\begin{sphinxVerbatim}[commandchars=\\\{\}]
\PYG{g+go}{datetime.date(2020, 2, 4)}
\end{sphinxVerbatim}

\begin{sphinxVerbatim}[commandchars=\\\{\}]
\PYG{c+c1}{\PYGZsh{} Formato ISO:}
\PYG{n}{date}\PYG{o}{.}\PYG{n}{fromordinal}\PYG{p}{(}\PYG{n}{hoje}\PYG{o}{.}\PYG{n}{toordinal}\PYG{p}{(}\PYG{p}{)} \PYG{o}{+} \PYG{l+m+mi}{40}\PYG{p}{)}\PYG{o}{.}\PYG{n}{isoformat}\PYG{p}{(}\PYG{p}{)}
\end{sphinxVerbatim}

\begin{sphinxVerbatim}[commandchars=\\\{\}]
\PYG{g+go}{\PYGZsq{}2020\PYGZhy{}02\PYGZhy{}04\PYGZsq{}}
\end{sphinxVerbatim}

\begin{sphinxVerbatim}[commandchars=\\\{\}]
\PYG{c+c1}{\PYGZsh{} Método weekday (dia da semana), em que segunda\PYGZhy{}feira = 0 e domingo = 6:}
\PYG{n}{hoje}\PYG{o}{.}\PYG{n}{weekday}\PYG{p}{(}\PYG{p}{)}
\end{sphinxVerbatim}

\begin{sphinxVerbatim}[commandchars=\\\{\}]
\PYG{g+go}{3}
\end{sphinxVerbatim}

\begin{sphinxVerbatim}[commandchars=\\\{\}]
\PYG{c+c1}{\PYGZsh{} Método isoweekday, em que segunda\PYGZhy{}feira = 1 e domingo = 7}
\PYG{n}{hoje}\PYG{o}{.}\PYG{n}{isoweekday}\PYG{p}{(}\PYG{p}{)}
\end{sphinxVerbatim}

\begin{sphinxVerbatim}[commandchars=\\\{\}]
\PYG{g+go}{4}
\end{sphinxVerbatim}


\section{O Módulo time}
\label{\detokenize{content/date_time:o-modulo-time}}\begin{quote}

\begin{DUlineblock}{0em}
\item[]
\begin{DUlineblock}{\DUlineblockindent}
\item[] Módulo cujos objetos representam uma hora (local) de dia, independente de
\end{DUlineblock}
\item[] qualquer dia em particular, e sujeito a ajustes via um objeto tzinfo.
\item[]
\begin{DUlineblock}{\DUlineblockindent}
\item[] Fornece várias funções para manipular valores de hora. Não confundir com
\end{DUlineblock}
\item[] a classe time do módulo datetime.
\end{DUlineblock}
\end{quote}

\begin{sphinxVerbatim}[commandchars=\\\{\}]
\PYG{c+c1}{\PYGZsh{} Imports:}
\PYG{k+kn}{from} \PYG{n+nn}{time} \PYG{k+kn}{import} \PYG{n}{ctime}
\PYG{k+kn}{from} \PYG{n+nn}{time} \PYG{k+kn}{import} \PYG{n}{sleep}
\PYG{k+kn}{from} \PYG{n+nn}{time} \PYG{k+kn}{import} \PYG{n}{time}
\PYG{k+kn}{from} \PYG{n+nn}{time} \PYG{k+kn}{import} \PYG{n}{tzname}
\end{sphinxVerbatim}

\begin{sphinxVerbatim}[commandchars=\\\{\}]
\PYG{c+c1}{\PYGZsh{} Criação de função que espera n segundos e exibe uma mensagem no final:}
\PYG{k}{def} \PYG{n+nf}{espera}\PYG{p}{(}\PYG{n}{tempo}\PYG{p}{)}\PYG{p}{:}
    \PYG{n}{sleep}\PYG{p}{(}\PYG{n}{tempo}\PYG{p}{)}
    \PYG{n+nb}{print}\PYG{p}{(}\PYG{l+s+sa}{f}\PYG{l+s+s1}{\PYGZsq{}}\PYG{l+s+s1}{Passaram\PYGZhy{}se }\PYG{l+s+si}{\PYGZob{}}\PYG{n}{tempo}\PYG{l+s+si}{\PYGZcb{}}\PYG{l+s+s1}{ segundos}\PYG{l+s+s1}{\PYGZsq{}}\PYG{p}{)}
\end{sphinxVerbatim}

\begin{sphinxVerbatim}[commandchars=\\\{\}]
\PYG{c+c1}{\PYGZsh{} Execução da função:}
\PYG{n}{espera}\PYG{p}{(}\PYG{l+m+mi}{3}\PYG{p}{)}
\end{sphinxVerbatim}

\begin{sphinxVerbatim}[commandchars=\\\{\}]
\PYG{g+go}{Passaram\PYGZhy{}se 3 segundos}
\end{sphinxVerbatim}

\begin{sphinxVerbatim}[commandchars=\\\{\}]
\PYG{c+c1}{\PYGZsh{} time.time retorna o tempo atual em segundos}
\PYG{c+c1}{\PYGZsh{} desde Epoch (01/01/1970 00:00:00):}
\PYG{n}{time}\PYG{p}{(}\PYG{p}{)}
\end{sphinxVerbatim}

\begin{sphinxVerbatim}[commandchars=\\\{\}]
\PYG{g+go}{1577375404.8968937}
\end{sphinxVerbatim}

\begin{sphinxVerbatim}[commandchars=\\\{\}]
\PYG{c+c1}{\PYGZsh{} Converte um tempo em segundos desde Epoch para uma string,}
\PYG{c+c1}{\PYGZsh{} se nenhum parâmetro for passado retorna string do momento atual:}
\PYG{n}{ctime}\PYG{p}{(}\PYG{p}{)}
\end{sphinxVerbatim}

\begin{sphinxVerbatim}[commandchars=\\\{\}]
\PYG{g+go}{\PYGZsq{}Thu Dec 26 12:50:22 2019\PYGZsq{}}
\end{sphinxVerbatim}

\begin{sphinxVerbatim}[commandchars=\\\{\}]
\PYG{n}{ctime}\PYG{p}{(}\PYG{l+m+mi}{1540000000}\PYG{p}{)}
\end{sphinxVerbatim}

\begin{sphinxVerbatim}[commandchars=\\\{\}]
\PYG{g+go}{\PYGZsq{}Fri Oct 19 22:46:40 2018\PYGZsq{}}
\end{sphinxVerbatim}


\chapter{Sequências}
\label{\detokenize{content/sequences:sequencias}}\label{\detokenize{content/sequences::doc}}\begin{quote}

Com sequências podemos fazer iteração, indexação, fatiamento (slice) e operações de list comprehension.
\end{quote}

Mútável:

\sphinxstylestrong{lista \sphinxhyphen{}\textgreater{} list \sphinxhyphen{}\textgreater{} {[}0, ‘string’, 7.0, 1000L{]}}

Imutáveis:

\sphinxstylestrong{string \sphinxhyphen{}\textgreater{} str \sphinxhyphen{}\textgreater{} ‘texto’}

\sphinxstylestrong{tupla \sphinxhyphen{}\textgreater{} tuple \sphinxhyphen{}\textgreater{} (0, ‘string’, 7.0, 1000L)}

Strings e tuplas funcionam de forma idêntica à lista, porém, seus itens são imutáveis, consomem menos recursos.


\section{Operações com Sequências}
\label{\detokenize{content/sequences:operacoes-com-sequencias}}\begin{itemize}
\item {} 
Índices
\begin{quote}

Toda sequência, seguindo a mesma idéia de vetores de outras linguagens, como C, por exemplo, começa com o índice 0 (zero).
Para se obter um índice, fazemos da seguinte forma:

sequencia{[}indice{]}
\end{quote}

\end{itemize}

Segundo elemento da string:

\begin{sphinxVerbatim}[commandchars=\\\{\}]
\PYG{l+s+s1}{\PYGZsq{}}\PYG{l+s+s1}{Python}\PYG{l+s+s1}{\PYGZsq{}}\PYG{p}{[}\PYG{l+m+mi}{1}\PYG{p}{]}
\end{sphinxVerbatim}

\begin{sphinxVerbatim}[commandchars=\\\{\}]
\PYG{g+go}{\PYGZsq{}y\PYGZsq{}}
\end{sphinxVerbatim}

Primeiro elemento da lista:

\begin{sphinxVerbatim}[commandchars=\\\{\}]
\PYG{p}{[}\PYG{l+s+s1}{\PYGZsq{}}\PYG{l+s+s1}{foo}\PYG{l+s+s1}{\PYGZsq{}}\PYG{p}{,} \PYG{l+s+s1}{\PYGZsq{}}\PYG{l+s+s1}{bar}\PYG{l+s+s1}{\PYGZsq{}}\PYG{p}{,} \PYG{l+m+mf}{2.7}\PYG{p}{,} \PYG{l+m+mi}{80}\PYG{p}{,}  \PYG{l+m+mi}{2} \PYG{o}{+} \PYG{l+m+mi}{7}\PYG{n}{j}\PYG{p}{]}\PYG{p}{[}\PYG{l+m+mi}{0}\PYG{p}{]}
\end{sphinxVerbatim}

\begin{sphinxVerbatim}[commandchars=\\\{\}]
\PYG{g+go}{\PYGZsq{}foo\PYGZsq{}}
\end{sphinxVerbatim}

Quarto elemento da tupla:

\begin{sphinxVerbatim}[commandchars=\\\{\}]
\PYG{p}{(}\PYG{l+s+s1}{\PYGZsq{}}\PYG{l+s+s1}{Python}\PYG{l+s+s1}{\PYGZsq{}}\PYG{p}{,} \PYG{l+s+s1}{\PYGZsq{}}\PYG{l+s+s1}{C}\PYG{l+s+s1}{\PYGZsq{}}\PYG{p}{,} \PYG{l+s+s1}{\PYGZsq{}}\PYG{l+s+s1}{C++}\PYG{l+s+s1}{\PYGZsq{}}\PYG{p}{,} \PYG{l+m+mf}{2.7}\PYG{p}{,} \PYG{l+m+mf}{3.7}\PYG{p}{)}\PYG{p}{[}\PYG{l+m+mi}{3}\PYG{p}{]}
\end{sphinxVerbatim}

\begin{sphinxVerbatim}[commandchars=\\\{\}]
\PYG{g+go}{2.7}
\end{sphinxVerbatim}


\section{Iteráveis}
\label{\detokenize{content/sequences:iteraveis}}\begin{quote}

Sequências nos permite também fazer iteração sobre cada elemento.
\end{quote}

Definição de uma tupla:

\begin{sphinxVerbatim}[commandchars=\\\{\}]
\PYG{n}{regiao\PYGZus{}sudeste} \PYG{o}{=} \PYG{p}{(}\PYG{l+s+s1}{\PYGZsq{}}\PYG{l+s+s1}{SP}\PYG{l+s+s1}{\PYGZsq{}}\PYG{p}{,} \PYG{l+s+s1}{\PYGZsq{}}\PYG{l+s+s1}{MG}\PYG{l+s+s1}{\PYGZsq{}}\PYG{p}{,} \PYG{l+s+s1}{\PYGZsq{}}\PYG{l+s+s1}{ES}\PYG{l+s+s1}{\PYGZsq{}}\PYG{p}{,} \PYG{l+s+s1}{\PYGZsq{}}\PYG{l+s+s1}{RJ}\PYG{l+s+s1}{\PYGZsq{}}\PYG{p}{)}
\end{sphinxVerbatim}

Loop sobre a tupla e impressão em tela de cada elemento:

\begin{sphinxVerbatim}[commandchars=\\\{\}]
\PYG{k}{for} \PYG{n}{i} \PYG{o+ow}{in} \PYG{n}{regiao\PYGZus{}sudeste}\PYG{p}{:}
    \PYG{n+nb}{print}\PYG{p}{(}\PYG{n}{i}\PYG{p}{)}
\end{sphinxVerbatim}

\begin{sphinxVerbatim}[commandchars=\\\{\}]
\PYG{g+go}{SP}
\PYG{g+go}{MG}
\PYG{g+go}{ES}
\PYG{g+go}{RJ}
\end{sphinxVerbatim}

Loop sobre a string e impressão em tela de cada caractere:

\begin{sphinxVerbatim}[commandchars=\\\{\}]
\PYG{k}{for} \PYG{n}{i} \PYG{o+ow}{in} \PYG{l+s+s1}{\PYGZsq{}}\PYG{l+s+s1}{Python}\PYG{l+s+s1}{\PYGZsq{}}\PYG{p}{:}
    \PYG{n+nb}{print}\PYG{p}{(}\PYG{n}{i}\PYG{p}{)}
\end{sphinxVerbatim}

\begin{sphinxVerbatim}[commandchars=\\\{\}]
\PYG{g+go}{P}
\PYG{g+go}{y}
\PYG{g+go}{t}
\PYG{g+go}{h}
\PYG{g+go}{o}
\PYG{g+go}{n}
\end{sphinxVerbatim}

Loop sobre um range de 0 (zero) a 20 (vinte) com a condição de exibir somente 0 (zero) e divisíveis por 5 (cinco):

\begin{sphinxVerbatim}[commandchars=\\\{\}]
\PYG{k}{for} \PYG{n}{i} \PYG{o+ow}{in} \PYG{n+nb}{range}\PYG{p}{(}\PYG{l+m+mi}{21}\PYG{p}{)}\PYG{p}{:}
    \PYG{k}{if} \PYG{p}{(}\PYG{n}{i} \PYG{o}{\PYGZpc{}} \PYG{l+m+mi}{5} \PYG{o}{==} \PYG{l+m+mi}{0}\PYG{p}{)}\PYG{p}{:}
        \PYG{n+nb}{print}\PYG{p}{(}\PYG{n}{i}\PYG{p}{)}
\end{sphinxVerbatim}

\begin{sphinxVerbatim}[commandchars=\\\{\}]
\PYG{g+go}{0}
\PYG{g+go}{5}
\PYG{g+go}{10}
\PYG{g+go}{15}
\PYG{g+go}{20}
\end{sphinxVerbatim}


\section{Fatiamento / Slicing}
\label{\detokenize{content/sequences:fatiamento-slicing}}\begin{quote}

É o corte de uma sequência.

\sphinxstylestrong{{[}inicio:fim \sphinxhyphen{} 1:incremento{]}}
\end{quote}

Fatiamento sem qualquer determinação:

\begin{sphinxVerbatim}[commandchars=\\\{\}]
\PYG{l+s+s1}{\PYGZsq{}}\PYG{l+s+s1}{Python Language}\PYG{l+s+s1}{\PYGZsq{}}\PYG{p}{[}\PYG{p}{:}\PYG{p}{:}\PYG{p}{]}
\end{sphinxVerbatim}

\begin{sphinxVerbatim}[commandchars=\\\{\}]
\PYG{g+go}{\PYGZsq{}Python Language\PYGZsq{}}
\end{sphinxVerbatim}

Não foram determinados início, fim e incremento.

Fatiamento determinando apenas o início, que é o último elemento:

\begin{sphinxVerbatim}[commandchars=\\\{\}]
\PYG{l+s+s1}{\PYGZsq{}}\PYG{l+s+s1}{Python Language}\PYG{l+s+s1}{\PYGZsq{}}\PYG{p}{[}\PYG{o}{\PYGZhy{}}\PYG{l+m+mi}{1}\PYG{p}{:}\PYG{p}{:}\PYG{p}{]}
\end{sphinxVerbatim}

\begin{sphinxVerbatim}[commandchars=\\\{\}]
\PYG{g+go}{\PYGZsq{}e\PYGZsq{}}
\end{sphinxVerbatim}

Pelo sinal de subtração, os três últimos caracteres da string:

\begin{sphinxVerbatim}[commandchars=\\\{\}]
\PYG{l+s+s1}{\PYGZsq{}}\PYG{l+s+s1}{Python Language}\PYG{l+s+s1}{\PYGZsq{}}\PYG{p}{[}\PYG{o}{\PYGZhy{}}\PYG{l+m+mi}{3}\PYG{p}{:}\PYG{p}{:}\PYG{p}{]}
\end{sphinxVerbatim}

\begin{sphinxVerbatim}[commandchars=\\\{\}]
\PYG{g+go}{\PYGZsq{}age\PYGZsq{}}
\end{sphinxVerbatim}

Determinando apenas o incremento de 4 (quatro) em 4:

\begin{sphinxVerbatim}[commandchars=\\\{\}]
\PYG{p}{(}\PYG{l+m+mi}{0}\PYG{p}{,} \PYG{l+m+mi}{1}\PYG{p}{,} \PYG{l+m+mi}{2}\PYG{p}{,} \PYG{l+m+mi}{3}\PYG{p}{,} \PYG{l+m+mi}{4}\PYG{p}{,} \PYG{l+m+mi}{5}\PYG{p}{,} \PYG{l+m+mi}{6}\PYG{p}{,} \PYG{l+m+mi}{7}\PYG{p}{,} \PYG{l+m+mi}{8}\PYG{p}{,} \PYG{l+m+mi}{9}\PYG{p}{)}\PYG{p}{[}\PYG{p}{:}\PYG{p}{:}\PYG{l+m+mi}{4}\PYG{p}{]}
\end{sphinxVerbatim}

\begin{sphinxVerbatim}[commandchars=\\\{\}]
\PYG{g+gp+gpVirtualEnv}{(0, 4, 8)}
\end{sphinxVerbatim}

Incremento negativo faz com que a string seja colocada em ordem reversa:

\begin{sphinxVerbatim}[commandchars=\\\{\}]
\PYG{l+s+s1}{\PYGZsq{}}\PYG{l+s+s1}{Python Language}\PYG{l+s+s1}{\PYGZsq{}}\PYG{p}{[}\PYG{p}{:}\PYG{p}{:}\PYG{o}{\PYGZhy{}}\PYG{l+m+mi}{1}\PYG{p}{]}
\end{sphinxVerbatim}

\begin{sphinxVerbatim}[commandchars=\\\{\}]
\PYG{g+go}{\PYGZsq{}egaugnaL nohtyP\PYGZsq{}}
\end{sphinxVerbatim}

A partir do primeiro caractere:

\begin{sphinxVerbatim}[commandchars=\\\{\}]
\PYG{l+s+s1}{\PYGZsq{}}\PYG{l+s+s1}{Python Language}\PYG{l+s+s1}{\PYGZsq{}}\PYG{p}{[}\PYG{l+m+mi}{0}\PYG{p}{:}\PYG{p}{]}
\end{sphinxVerbatim}

\begin{sphinxVerbatim}[commandchars=\\\{\}]
\PYG{g+go}{\PYGZsq{}Python Language\PYGZsq{}}
\end{sphinxVerbatim}

Do primeiro ao primeiro caractere:

\begin{sphinxVerbatim}[commandchars=\\\{\}]
\PYG{l+s+s1}{\PYGZsq{}}\PYG{l+s+s1}{Python Language}\PYG{l+s+s1}{\PYGZsq{}}\PYG{p}{[}\PYG{l+m+mi}{0}\PYG{p}{:}\PYG{l+m+mi}{1}\PYG{p}{]}
\end{sphinxVerbatim}

\begin{sphinxVerbatim}[commandchars=\\\{\}]
\PYG{g+go}{\PYGZsq{}P\PYGZsq{}}
\end{sphinxVerbatim}

Do primeiro ao sexto caractere:

\begin{sphinxVerbatim}[commandchars=\\\{\}]
\PYG{n}{Python} \PYG{n}{Language}\PYG{l+s+s1}{\PYGZsq{}}\PYG{l+s+s1}{[0:6]}
\end{sphinxVerbatim}

\begin{sphinxVerbatim}[commandchars=\\\{\}]
\PYG{g+go}{\PYGZsq{}Python\PYGZsq{}}
\end{sphinxVerbatim}

Do oitavo caractere em diante:

\begin{sphinxVerbatim}[commandchars=\\\{\}]
\PYG{l+s+s1}{\PYGZsq{}}\PYG{l+s+s1}{Python Language}\PYG{l+s+s1}{\PYGZsq{}}\PYG{p}{[}\PYG{l+m+mi}{7}\PYG{p}{:}\PYG{p}{]}
\end{sphinxVerbatim}

\begin{sphinxVerbatim}[commandchars=\\\{\}]
\PYG{g+go}{\PYGZsq{}Language\PYGZsq{}}
\end{sphinxVerbatim}

Criação de uma tupla de exemplo:

\begin{sphinxVerbatim}[commandchars=\\\{\}]
\PYG{n}{linux\PYGZus{}distros} \PYG{o}{=} \PYG{p}{(}
                 \PYG{l+s+s1}{\PYGZsq{}}\PYG{l+s+s1}{Debian}\PYG{l+s+s1}{\PYGZsq{}}\PYG{p}{,}
                 \PYG{l+s+s1}{\PYGZsq{}}\PYG{l+s+s1}{RedHat}\PYG{l+s+s1}{\PYGZsq{}}\PYG{p}{,}
                 \PYG{l+s+s1}{\PYGZsq{}}\PYG{l+s+s1}{Slackware}\PYG{l+s+s1}{\PYGZsq{}}\PYG{p}{,}
                 \PYG{l+s+s1}{\PYGZsq{}}\PYG{l+s+s1}{Ubuntu}\PYG{l+s+s1}{\PYGZsq{}}\PYG{p}{,}
                 \PYG{l+s+s1}{\PYGZsq{}}\PYG{l+s+s1}{CentOS}\PYG{l+s+s1}{\PYGZsq{}}\PYG{p}{,}
                 \PYG{l+s+s1}{\PYGZsq{}}\PYG{l+s+s1}{SuSE}\PYG{l+s+s1}{\PYGZsq{}}\PYG{p}{,}
                \PYG{p}{)}
\end{sphinxVerbatim}

Do primeiro ao terceiro elemento:

\begin{sphinxVerbatim}[commandchars=\\\{\}]
\PYG{n}{linux\PYGZus{}distros}\PYG{p}{[}\PYG{l+m+mi}{0}\PYG{p}{:}\PYG{l+m+mi}{3}\PYG{p}{]}
\end{sphinxVerbatim}

\begin{sphinxVerbatim}[commandchars=\\\{\}]
\PYG{g+gp+gpVirtualEnv}{(\PYGZsq{}Debian\PYGZsq{}, \PYGZsq{}RedHat\PYGZsq{}, \PYGZsq{}Slackware\PYGZsq{})}
\end{sphinxVerbatim}


\section{List Comprehension}
\label{\detokenize{content/sequences:list-comprehension}}\begin{quote}

Ou em português, “Compreensão de Lista”, fornece uma maneira concisa para criar listas.
Usos comuns são para fazer novas listas onde cada elemento é o resultado de algumas operações aplicadas para cada membro de outra sequência ou iterável, criar uma subsequência desses elementos que satisfaçam uma certa condição.
Sempre retornará uma lista.
\end{quote}

Lista a partir de uma list comprehension do range:

\begin{sphinxVerbatim}[commandchars=\\\{\}]
\PYG{p}{[}\PYG{n}{i} \PYG{k}{for} \PYG{n}{i} \PYG{o+ow}{in} \PYG{n+nb}{range}\PYG{p}{(}\PYG{l+m+mi}{21}\PYG{p}{)}\PYG{p}{]}
\end{sphinxVerbatim}

\begin{sphinxVerbatim}[commandchars=\\\{\}]
\PYG{g+go}{[0, 1, 2, 3, 4, 5, 6, 7, 8, 9, 10, 11, 12, 13, 14, 15, 16, 17, 18, 19, 20]}
\end{sphinxVerbatim}

Lista cujos elementos são a metade de cada elemento do range:

\begin{sphinxVerbatim}[commandchars=\\\{\}]
\PYG{p}{[}\PYG{n}{i} \PYG{o}{/} \PYG{l+m+mf}{2.0} \PYG{k}{for} \PYG{n}{i} \PYG{o+ow}{in} \PYG{n+nb}{range}\PYG{p}{(}\PYG{l+m+mi}{10}\PYG{p}{)}\PYG{p}{]}
\end{sphinxVerbatim}

\begin{sphinxVerbatim}[commandchars=\\\{\}]
\PYG{g+go}{[0.0, 0.5, 1.0, 1.5, 2.0, 2.5, 3.0, 3.5, 4.0, 4.5]}
\end{sphinxVerbatim}

Lista com condição que seja 0 (zero) ou divisível por 5 (cinco):

\begin{sphinxVerbatim}[commandchars=\\\{\}]
\PYG{p}{[}\PYG{n}{i} \PYG{k}{for} \PYG{n}{i} \PYG{o+ow}{in} \PYG{n+nb}{range}\PYG{p}{(}\PYG{l+m+mi}{21}\PYG{p}{)} \PYG{k}{if} \PYG{p}{(}\PYG{n}{i} \PYG{o}{\PYGZpc{}} \PYG{l+m+mi}{5} \PYG{o}{==} \PYG{l+m+mi}{0}\PYG{p}{)}\PYG{p}{]}
\end{sphinxVerbatim}

\begin{sphinxVerbatim}[commandchars=\\\{\}]
\PYG{g+go}{[0, 5, 10, 15, 20]}
\end{sphinxVerbatim}


\section{Tuple Comprehension}
\label{\detokenize{content/sequences:tuple-comprehension}}\begin{quote}

Ou em português “Compreenção de Tupla” é similar a uma list comprehension, no entanto resulta em um generator.
\end{quote}

Criação de um generator a partir de uma tuple comprehension:

\begin{sphinxVerbatim}[commandchars=\\\{\}]
\PYG{n}{x} \PYG{o}{=} \PYG{p}{(}\PYG{n}{i} \PYG{k}{for} \PYG{n}{i} \PYG{o+ow}{in} \PYG{n+nb}{range}\PYG{p}{(}\PYG{l+m+mi}{21}\PYG{p}{)}\PYG{p}{)}
\end{sphinxVerbatim}

Verificando o tipo do objeto:

\begin{sphinxVerbatim}[commandchars=\\\{\}]
\PYG{n+nb}{type}\PYG{p}{(}\PYG{n}{x}\PYG{p}{)}
\end{sphinxVerbatim}

\begin{sphinxVerbatim}[commandchars=\\\{\}]
\PYG{g+go}{generator}
\end{sphinxVerbatim}


\section{Dict Comprehension}
\label{\detokenize{content/sequences:dict-comprehension}}\begin{quote}

Ou também conhecido em português como “Compreenção de Dicionário”
\end{quote}

Objeto dicionário a ser criado:

\begin{sphinxVerbatim}[commandchars=\\\{\}]
\PYG{n}{d1} \PYG{o}{=} \PYG{p}{\PYGZob{}}\PYG{l+s+s1}{\PYGZsq{}}\PYG{l+s+s1}{a}\PYG{l+s+s1}{\PYGZsq{}}\PYG{p}{:} \PYG{l+m+mi}{1}\PYG{p}{,} \PYG{l+s+s1}{\PYGZsq{}}\PYG{l+s+s1}{b}\PYG{l+s+s1}{\PYGZsq{}}\PYG{p}{:}\PYG{l+m+mi}{2}\PYG{p}{,} \PYG{l+s+s1}{\PYGZsq{}}\PYG{l+s+s1}{c}\PYG{l+s+s1}{\PYGZsq{}}\PYG{p}{:} \PYG{l+m+mi}{3}\PYG{p}{\PYGZcb{}}
\end{sphinxVerbatim}

Novo dicionário criado a partir de dict comprehension:

\begin{sphinxVerbatim}[commandchars=\\\{\}]
\PYG{n}{d2} \PYG{o}{=} \PYG{p}{\PYGZob{}}\PYG{n}{k}\PYG{o}{.}\PYG{n}{upper}\PYG{p}{(}\PYG{p}{)}\PYG{p}{:} \PYG{n}{v} \PYG{o}{*} \PYG{l+m+mi}{10} \PYG{k}{for} \PYG{n}{k}\PYG{p}{,} \PYG{n}{v} \PYG{o+ow}{in} \PYG{n}{d1}\PYG{o}{.}\PYG{n}{items}\PYG{p}{(}\PYG{p}{)}\PYG{p}{\PYGZcb{}}
\end{sphinxVerbatim}

Cada chave é o caractere maiúsculo das chave correspondente ao dicionário
original e seus valores são multiplicados por 10 (dez).

Exibindo o dicionário gerado a partir da dict comprehension:

\begin{sphinxVerbatim}[commandchars=\\\{\}]
\PYG{n+nb}{print}\PYG{p}{(}\PYG{n}{d2}\PYG{p}{)}
\end{sphinxVerbatim}

\begin{sphinxVerbatim}[commandchars=\\\{\}]
\PYG{g+go}{\PYGZob{}\PYGZsq{}A\PYGZsq{}: 10, \PYGZsq{}B\PYGZsq{}: 20, \PYGZsq{}C\PYGZsq{}: 30\PYGZcb{}}
\end{sphinxVerbatim}


\chapter{Lista}
\label{\detokenize{content/list:lista}}\label{\detokenize{content/list::doc}}\begin{quote}

Uma lista é uma estrutura de dados similar ao tradicional array que tem em
outras linguagens, porém seus elementos podem ser de tipos diferentes.
Uma lista é também uma coleção de dados mutável, pois sua estrutura pode ser mudada.
\end{quote}

Formas de criar uma lista vazia:

\begin{sphinxVerbatim}[commandchars=\\\{\}]
\PYG{n}{l} \PYG{o}{=} \PYG{p}{[}\PYG{p}{]}
\end{sphinxVerbatim}

ou

\begin{sphinxVerbatim}[commandchars=\\\{\}]
\PYG{n}{l} \PYG{o}{=} \PYG{n+nb}{list}\PYG{p}{(}\PYG{p}{)}
\end{sphinxVerbatim}

Formas de criar uma lista com elementos:

\begin{sphinxVerbatim}[commandchars=\\\{\}]
\PYG{n}{l} \PYG{o}{=} \PYG{n+nb}{list}\PYG{p}{(}\PYG{p}{[}\PYG{l+m+mi}{1}\PYG{p}{,} \PYG{l+m+mi}{2}\PYG{p}{,} \PYG{l+m+mi}{3}\PYG{p}{]}\PYG{p}{)}
\end{sphinxVerbatim}

ou

\begin{sphinxVerbatim}[commandchars=\\\{\}]
\PYG{n}{l} \PYG{o}{=} \PYG{p}{[}\PYG{l+m+mi}{1}\PYG{p}{,} \PYG{l+m+mi}{2}\PYG{p}{,} \PYG{l+m+mi}{3}\PYG{p}{]}
\end{sphinxVerbatim}

Declaração de uma nova lista:

\begin{sphinxVerbatim}[commandchars=\\\{\}]
\PYG{n}{x} \PYG{o}{=} \PYG{p}{[}\PYG{l+s+s1}{\PYGZsq{}}\PYG{l+s+s1}{a}\PYG{l+s+s1}{\PYGZsq{}}\PYG{p}{,} \PYG{l+s+s1}{\PYGZsq{}}\PYG{l+s+s1}{b}\PYG{l+s+s1}{\PYGZsq{}}\PYG{p}{,} \PYG{l+m+mi}{3}\PYG{p}{,} \PYG{l+m+mf}{4.0}\PYG{p}{]}
\end{sphinxVerbatim}

Exibindo o terceiro elemento da lista:

\begin{sphinxVerbatim}[commandchars=\\\{\}]
\PYG{n+nb}{print}\PYG{p}{(}\PYG{n}{x}\PYG{p}{[}\PYG{l+m+mi}{2}\PYG{p}{]}\PYG{p}{)}
\end{sphinxVerbatim}

\begin{sphinxVerbatim}[commandchars=\\\{\}]
\PYG{g+go}{3}
\end{sphinxVerbatim}

Alterando o terceiro elemento da lista:

\begin{sphinxVerbatim}[commandchars=\\\{\}]
\PYG{n}{x}\PYG{p}{[}\PYG{l+m+mi}{2}\PYG{p}{]} \PYG{o}{=} \PYG{l+m+mi}{30}
\end{sphinxVerbatim}

Exibindo o terceiro elemento da lista:

\begin{sphinxVerbatim}[commandchars=\\\{\}]
\PYG{n+nb}{print}\PYG{p}{(}\PYG{n}{x}\PYG{p}{[}\PYG{l+m+mi}{2}\PYG{p}{]}\PYG{p}{)}
\end{sphinxVerbatim}

\begin{sphinxVerbatim}[commandchars=\\\{\}]
\PYG{g+go}{30}
\end{sphinxVerbatim}

Exibindo os elementos da lista:

\begin{sphinxVerbatim}[commandchars=\\\{\}]
\PYG{n+nb}{print}\PYG{p}{(}\PYG{n}{x}\PYG{p}{)}
\end{sphinxVerbatim}

\begin{sphinxVerbatim}[commandchars=\\\{\}]
\PYG{g+go}{[\PYGZsq{}a\PYGZsq{}, \PYGZsq{}b\PYGZsq{}, 30, 4.0]}
\end{sphinxVerbatim}

Existe “7” na lista?:

\begin{sphinxVerbatim}[commandchars=\\\{\}]
\PYG{l+m+mi}{7} \PYG{o+ow}{in} \PYG{n}{x}
\end{sphinxVerbatim}

\begin{sphinxVerbatim}[commandchars=\\\{\}]
\PYG{g+go}{False}
\end{sphinxVerbatim}

Existe “3” na lista?:

\begin{sphinxVerbatim}[commandchars=\\\{\}]
\PYG{l+m+mi}{3} \PYG{o+ow}{in} \PYG{n}{x}
\end{sphinxVerbatim}

\begin{sphinxVerbatim}[commandchars=\\\{\}]
\PYG{g+go}{True}
\end{sphinxVerbatim}

O método append adiciona um elemento ao final da lista:

\begin{sphinxVerbatim}[commandchars=\\\{\}]
\PYG{n}{x}\PYG{o}{.}\PYG{n}{append}\PYG{p}{(}\PYG{l+s+s1}{\PYGZsq{}}\PYG{l+s+s1}{Uma string qualquer...}\PYG{l+s+s1}{\PYGZsq{}}\PYG{p}{)}
\end{sphinxVerbatim}

Exibindo os elementos da lista:

\begin{sphinxVerbatim}[commandchars=\\\{\}]
\PYG{n+nb}{print}\PYG{p}{(}\PYG{n}{x}\PYG{p}{)}
\end{sphinxVerbatim}

\begin{sphinxVerbatim}[commandchars=\\\{\}]
\PYG{g+go}{[\PYGZsq{}a\PYGZsq{}, \PYGZsq{}b\PYGZsq{}, 30, 4.0, \PYGZsq{}Uma string qualquer...\PYGZsq{}]}
\end{sphinxVerbatim}

O método extend faz com que os elementos de outra lista sejam adicionadas a
uma lista atual:

\begin{sphinxVerbatim}[commandchars=\\\{\}]
\PYG{n}{foo} \PYG{o}{=} \PYG{p}{[}\PYG{l+s+s1}{\PYGZsq{}}\PYG{l+s+s1}{a}\PYG{l+s+s1}{\PYGZsq{}}\PYG{p}{,} \PYG{l+s+s1}{\PYGZsq{}}\PYG{l+s+s1}{b}\PYG{l+s+s1}{\PYGZsq{}}\PYG{p}{,} \PYG{l+s+s1}{\PYGZsq{}}\PYG{l+s+s1}{c}\PYG{l+s+s1}{\PYGZsq{}}\PYG{p}{]}  \PYG{c+c1}{\PYGZsh{} Definição da primeira lista}
\PYG{n}{bar} \PYG{o}{=} \PYG{p}{[}\PYG{l+m+mi}{1}\PYG{p}{,} \PYG{l+m+mi}{2}\PYG{p}{]}  \PYG{c+c1}{\PYGZsh{} Definição da segunda lista}
\PYG{n}{foo}\PYG{o}{.}\PYG{n}{extend}\PYG{p}{(}\PYG{n}{bar}\PYG{p}{)}  \PYG{c+c1}{\PYGZsh{} Extendendo a primeira lista com os elementos da segunda}
\PYG{n+nb}{print}\PYG{p}{(}\PYG{n}{foo}\PYG{p}{)}  \PYG{c+c1}{\PYGZsh{} Exibindo a lista extendida}
\end{sphinxVerbatim}

\begin{sphinxVerbatim}[commandchars=\\\{\}]
\PYG{g+go}{[\PYGZsq{}a\PYGZsq{}, \PYGZsq{}b\PYGZsq{}, \PYGZsq{}c\PYGZsq{}, 1, 2]}
\end{sphinxVerbatim}

O operador “+” sendo utilizado para criar uma nova lista a partir de outra somada à outra:

\begin{sphinxVerbatim}[commandchars=\\\{\}]
\PYG{n}{y} \PYG{o}{=} \PYG{n}{x} \PYG{o}{+} \PYG{p}{[}\PYG{l+s+s1}{\PYGZsq{}}\PYG{l+s+s1}{abobrinha}\PYG{l+s+s1}{\PYGZsq{}}\PYG{p}{,} \PYG{l+s+s1}{\PYGZsq{}}\PYG{l+s+s1}{7}\PYG{l+s+s1}{\PYGZsq{}}\PYG{p}{,} \PYG{p}{\PYGZob{}}\PYG{l+s+s1}{\PYGZsq{}}\PYG{l+s+s1}{chave}\PYG{l+s+s1}{\PYGZsq{}}\PYG{p}{:} \PYG{l+s+s1}{\PYGZsq{}}\PYG{l+s+s1}{valor}\PYG{l+s+s1}{\PYGZsq{}}\PYG{p}{\PYGZcb{}}\PYG{p}{,} \PYG{p}{(}\PYG{l+s+s1}{\PYGZsq{}}\PYG{l+s+s1}{SP}\PYG{l+s+s1}{\PYGZsq{}}\PYG{p}{,} \PYG{l+s+s1}{\PYGZsq{}}\PYG{l+s+s1}{MG}\PYG{l+s+s1}{\PYGZsq{}}\PYG{p}{,} \PYG{l+s+s1}{\PYGZsq{}}\PYG{l+s+s1}{PR}\PYG{l+s+s1}{\PYGZsq{}}\PYG{p}{,} \PYG{l+s+s1}{\PYGZsq{}}\PYG{l+s+s1}{RO}\PYG{l+s+s1}{\PYGZsq{}}\PYG{p}{)}\PYG{p}{]}
\end{sphinxVerbatim}

Exibindo a nova lista:

\begin{sphinxVerbatim}[commandchars=\\\{\}]
\PYG{n}{y}
\end{sphinxVerbatim}

\begin{sphinxVerbatim}[commandchars=\\\{\}]
\PYG{p}{[}\PYG{l+s+s1}{\PYGZsq{}}\PYG{l+s+s1}{a}\PYG{l+s+s1}{\PYGZsq{}}\PYG{p}{,}
 \PYG{l+s+s1}{\PYGZsq{}}\PYG{l+s+s1}{b}\PYG{l+s+s1}{\PYGZsq{}}\PYG{p}{,}
 \PYG{l+m+mi}{30}\PYG{p}{,}
 \PYG{l+m+mf}{4.0}\PYG{p}{,}
 \PYG{l+s+s1}{\PYGZsq{}}\PYG{l+s+s1}{Uma string qualquer...}\PYG{l+s+s1}{\PYGZsq{}}\PYG{p}{,}
 \PYG{l+s+s1}{\PYGZsq{}}\PYG{l+s+s1}{abobrinha}\PYG{l+s+s1}{\PYGZsq{}}\PYG{p}{,}
 \PYG{l+s+s1}{\PYGZsq{}}\PYG{l+s+s1}{7}\PYG{l+s+s1}{\PYGZsq{}}\PYG{p}{,}
 \PYG{p}{\PYGZob{}}\PYG{l+s+s1}{\PYGZsq{}}\PYG{l+s+s1}{chave}\PYG{l+s+s1}{\PYGZsq{}}\PYG{p}{:} \PYG{l+s+s1}{\PYGZsq{}}\PYG{l+s+s1}{valor}\PYG{l+s+s1}{\PYGZsq{}}\PYG{p}{\PYGZcb{}}\PYG{p}{,}
 \PYG{p}{(}\PYG{l+s+s1}{\PYGZsq{}}\PYG{l+s+s1}{SP}\PYG{l+s+s1}{\PYGZsq{}}\PYG{p}{,} \PYG{l+s+s1}{\PYGZsq{}}\PYG{l+s+s1}{MG}\PYG{l+s+s1}{\PYGZsq{}}\PYG{p}{,} \PYG{l+s+s1}{\PYGZsq{}}\PYG{l+s+s1}{PR}\PYG{l+s+s1}{\PYGZsq{}}\PYG{p}{,} \PYG{l+s+s1}{\PYGZsq{}}\PYG{l+s+s1}{RO}\PYG{l+s+s1}{\PYGZsq{}}\PYG{p}{)}\PYG{p}{]}
\end{sphinxVerbatim}

Utilizando “+=” como um atalho para o método extend e exibindo seu novo conteúdo:

\begin{sphinxVerbatim}[commandchars=\\\{\}]
\PYG{n}{x} \PYG{o}{+}\PYG{o}{=} \PYG{p}{[}\PYG{l+s+s1}{\PYGZsq{}}\PYG{l+s+s1}{xyz}\PYG{l+s+s1}{\PYGZsq{}}\PYG{p}{]}
\PYG{n}{x}
\end{sphinxVerbatim}

\begin{sphinxVerbatim}[commandchars=\\\{\}]
\PYG{g+go}{[\PYGZsq{}a\PYGZsq{}, \PYGZsq{}b\PYGZsq{}, 30, 4.0, \PYGZsq{}Uma string qualquer...\PYGZsq{}, \PYGZsq{}xyz\PYGZsq{}]}
\end{sphinxVerbatim}

Removendo um elemento da lista e exibindo seu novo conteúdo:

\begin{sphinxVerbatim}[commandchars=\\\{\}]
\PYG{n}{x}\PYG{o}{.}\PYG{n}{remove}\PYG{p}{(}\PYG{l+s+s1}{\PYGZsq{}}\PYG{l+s+s1}{xyz}\PYG{l+s+s1}{\PYGZsq{}}\PYG{p}{)}
\PYG{n}{x}
\end{sphinxVerbatim}

\begin{sphinxVerbatim}[commandchars=\\\{\}]
\PYG{g+go}{[\PYGZsq{}a\PYGZsq{}, \PYGZsq{}b\PYGZsq{}, 30, 4.0, \PYGZsq{}Uma string qualquer...\PYGZsq{}]}
\end{sphinxVerbatim}

Método extend com uma lista como parâmetro:

\begin{sphinxVerbatim}[commandchars=\\\{\}]
\PYG{n}{x}\PYG{o}{.}\PYG{n}{extend}\PYG{p}{(}\PYG{p}{[}\PYG{l+s+s1}{\PYGZsq{}}\PYG{l+s+s1}{xyz}\PYG{l+s+s1}{\PYGZsq{}}\PYG{p}{]}\PYG{p}{)}
\PYG{n}{x}
\end{sphinxVerbatim}

\begin{sphinxVerbatim}[commandchars=\\\{\}]
\PYG{g+go}{[\PYGZsq{}a\PYGZsq{}, \PYGZsq{}b\PYGZsq{}, 30, 4.0, \PYGZsq{}Uma string qualquer...\PYGZsq{}, \PYGZsq{}xyz\PYGZsq{}]}
\end{sphinxVerbatim}

Método extend com uma string como parâmetro:

\begin{sphinxVerbatim}[commandchars=\\\{\}]
\PYG{n}{x}\PYG{o}{.}\PYG{n}{extend}\PYG{p}{(}\PYG{l+s+s1}{\PYGZsq{}}\PYG{l+s+s1}{xyz}\PYG{l+s+s1}{\PYGZsq{}}\PYG{p}{)}
\PYG{n}{x}
\end{sphinxVerbatim}

\begin{sphinxVerbatim}[commandchars=\\\{\}]
\PYG{g+go}{[\PYGZsq{}a\PYGZsq{}, \PYGZsq{}b\PYGZsq{}, 30, 4.0, \PYGZsq{}Uma string qualquer...\PYGZsq{}, \PYGZsq{}xyz\PYGZsq{}, \PYGZsq{}x\PYGZsq{}, \PYGZsq{}y\PYGZsq{}, \PYGZsq{}z\PYGZsq{}]}
\end{sphinxVerbatim}

Ao utilizarmos uma string como parâmetro do método extend, a string foi
convertida em lista de forma a transformar cada caractere em elemento
de uma lista.

Método append:

\begin{sphinxVerbatim}[commandchars=\\\{\}]
\PYG{n}{x}\PYG{o}{.}\PYG{n}{append}\PYG{p}{(}\PYG{l+s+s1}{\PYGZsq{}}\PYG{l+s+s1}{xyz}\PYG{l+s+s1}{\PYGZsq{}}\PYG{p}{)}
\PYG{n}{x}
\end{sphinxVerbatim}

\begin{sphinxVerbatim}[commandchars=\\\{\}]
\PYG{g+go}{[\PYGZsq{}a\PYGZsq{}, \PYGZsq{}b\PYGZsq{}, 30, 4.0, \PYGZsq{}Uma string qualquer...\PYGZsq{}, \PYGZsq{}xyz\PYGZsq{}, \PYGZsq{}x\PYGZsq{}, \PYGZsq{}y\PYGZsq{}, \PYGZsq{}z\PYGZsq{}, \PYGZsq{}xyz\PYGZsq{}]}
\end{sphinxVerbatim}

Aqui podemos ver claramente a diferença entre os métodos extend e append, que
pode causar uma certa confusão inicial para quem está iniciando em Python.
Nota\sphinxhyphen{}se que o método append adicionou a string inteira como um novo elemento
da lista.

Extendendo a lista com o operador “+=”:

\begin{sphinxVerbatim}[commandchars=\\\{\}]
\PYG{n}{x} \PYG{o}{+}\PYG{o}{=} \PYG{l+s+s1}{\PYGZsq{}}\PYG{l+s+s1}{String}\PYG{l+s+s1}{\PYGZsq{}}
\PYG{n}{x}
\end{sphinxVerbatim}

\begin{sphinxVerbatim}[commandchars=\\\{\}]
\PYG{g+go}{[\PYGZsq{}a\PYGZsq{},}
\PYG{g+go}{\PYGZsq{}b\PYGZsq{},}
\PYG{g+go}{30,}
\PYG{g+go}{4.0,}
\PYG{g+go}{\PYGZsq{}Uma string qualquer...\PYGZsq{},}
\PYG{g+go}{\PYGZsq{}xyz\PYGZsq{},}
\PYG{g+go}{\PYGZsq{}x\PYGZsq{},}
\PYG{g+go}{\PYGZsq{}y\PYGZsq{},}
\PYG{g+go}{\PYGZsq{}z\PYGZsq{},}
\PYG{g+go}{\PYGZsq{}xyz\PYGZsq{},}
\PYG{g+go}{\PYGZsq{}S\PYGZsq{},}
\PYG{g+go}{\PYGZsq{}t\PYGZsq{},}
\PYG{g+go}{\PYGZsq{}r\PYGZsq{},}
\PYG{g+go}{\PYGZsq{}i\PYGZsq{},}
\PYG{g+go}{\PYGZsq{}n\PYGZsq{},}
\PYG{g+go}{\PYGZsq{}g\PYGZsq{}]}
\end{sphinxVerbatim}

Como o operador “+=” ser um atalho para o método extend ele transformou a
string numa lista, uma lista cujos elementos são os caracteres da string.

Tentativa de utilizar o operador “+” em uma lista:

\begin{sphinxVerbatim}[commandchars=\\\{\}]
\PYG{n}{x} \PYG{o}{+} \PYG{l+s+s1}{\PYGZsq{}}\PYG{l+s+s1}{bla bla bla}\PYG{l+s+s1}{\PYGZsq{}}
\end{sphinxVerbatim}

\begin{sphinxVerbatim}[commandchars=\\\{\}]
\PYG{g+go}{TypeError: can only concatenate list (not \PYGZdq{}str\PYGZdq{}) to list}
\end{sphinxVerbatim}

Não se pode concatenar uma lista com uma string.

Criação de uma nova lista a partir de uma string:

\begin{sphinxVerbatim}[commandchars=\\\{\}]
\PYG{n}{z} \PYG{o}{=} \PYG{n+nb}{list}\PYG{p}{(}\PYG{l+s+s1}{\PYGZsq{}}\PYG{l+s+s1}{Hobbit}\PYG{l+s+s1}{\PYGZsq{}}\PYG{p}{)}
\PYG{n}{z}
\end{sphinxVerbatim}

\begin{sphinxVerbatim}[commandchars=\\\{\}]
\PYG{g+go}{[\PYGZsq{}H\PYGZsq{}, \PYGZsq{}o\PYGZsq{}, \PYGZsq{}b\PYGZsq{}, \PYGZsq{}b\PYGZsq{}, \PYGZsq{}i\PYGZsq{}, \PYGZsq{}t\PYGZsq{}]}
\end{sphinxVerbatim}

Novamente o método append:

\begin{sphinxVerbatim}[commandchars=\\\{\}]
\PYG{n}{z}\PYG{o}{.}\PYG{n}{append}\PYG{p}{(}\PYG{l+s+s1}{\PYGZsq{}}\PYG{l+s+s1}{Hobbit}\PYG{l+s+s1}{\PYGZsq{}}\PYG{p}{)}
\PYG{n}{z}
\end{sphinxVerbatim}

\begin{sphinxVerbatim}[commandchars=\\\{\}]
\PYG{g+go}{[\PYGZsq{}H\PYGZsq{}, \PYGZsq{}o\PYGZsq{}, \PYGZsq{}b\PYGZsq{}, \PYGZsq{}b\PYGZsq{}, \PYGZsq{}i\PYGZsq{}, \PYGZsq{}t\PYGZsq{}, \PYGZsq{}Hobbit\PYGZsq{}]}
\end{sphinxVerbatim}

Qual o tamanho (quantos elementos) da lista?

\begin{sphinxVerbatim}[commandchars=\\\{\}]
\PYG{n+nb}{len}\PYG{p}{(}\PYG{n}{z}\PYG{p}{)}
\end{sphinxVerbatim}

\begin{sphinxVerbatim}[commandchars=\\\{\}]
\PYG{g+go}{7}
\end{sphinxVerbatim}

Tamanho 7, posições variam de 0 a 6.

Método insert utilizando a priemira posição (0):

\begin{sphinxVerbatim}[commandchars=\\\{\}]
\PYG{n}{z}\PYG{o}{.}\PYG{n}{insert}\PYG{p}{(}\PYG{l+m+mi}{0}\PYG{p}{,} \PYG{l+s+s1}{\PYGZsq{}}\PYG{l+s+s1}{Gandalf}\PYG{l+s+s1}{\PYGZsq{}}\PYG{p}{)}
\PYG{n}{z}
\end{sphinxVerbatim}

\begin{sphinxVerbatim}[commandchars=\\\{\}]
\PYG{g+go}{[\PYGZsq{}Gandalf\PYGZsq{}, \PYGZsq{}H\PYGZsq{}, \PYGZsq{}o\PYGZsq{}, \PYGZsq{}b\PYGZsq{}, \PYGZsq{}b\PYGZsq{}, \PYGZsq{}i\PYGZsq{}, \PYGZsq{}t\PYGZsq{}, \PYGZsq{}Hobbit\PYGZsq{}]}
\end{sphinxVerbatim}

Pode\sphinxhyphen{}se também verificar o tamanho de uma lista com o dunder len:

\begin{sphinxVerbatim}[commandchars=\\\{\}]
\PYG{n}{z}\PYG{o}{.}\PYG{n+nf+fm}{\PYGZus{}\PYGZus{}len\PYGZus{}\PYGZus{}}\PYG{p}{(}\PYG{p}{)}
\end{sphinxVerbatim}

\begin{sphinxVerbatim}[commandchars=\\\{\}]
\PYG{g+go}{8}
\end{sphinxVerbatim}

Dado que a lista tem 8 (oito) elementos, inserir num novo elemento na nona (8) posição:

\begin{sphinxVerbatim}[commandchars=\\\{\}]
\PYG{n}{z}\PYG{o}{.}\PYG{n}{insert}\PYG{p}{(}\PYG{l+m+mi}{8}\PYG{p}{,} \PYG{l+s+s1}{\PYGZsq{}}\PYG{l+s+s1}{Bilbo}\PYG{l+s+s1}{\PYGZsq{}}\PYG{p}{)}
\PYG{n}{z}
\end{sphinxVerbatim}

\begin{sphinxVerbatim}[commandchars=\\\{\}]
\PYG{g+go}{[\PYGZsq{}Gandalf\PYGZsq{}, \PYGZsq{}H\PYGZsq{}, \PYGZsq{}o\PYGZsq{}, \PYGZsq{}b\PYGZsq{}, \PYGZsq{}b\PYGZsq{}, \PYGZsq{}i\PYGZsq{}, \PYGZsq{}t\PYGZsq{}, \PYGZsq{}Hobbit\PYGZsq{}, \PYGZsq{}Bilbo\PYGZsq{}]}
\end{sphinxVerbatim}

O efeito foi o mesmo que utilizar o método append.

O método pop sem parâmetros retorna o último elemento e o apaga da lista:

\begin{sphinxVerbatim}[commandchars=\\\{\}]
\PYG{n}{z}\PYG{o}{.}\PYG{n}{pop}\PYG{p}{(}\PYG{p}{)}
\end{sphinxVerbatim}

\begin{sphinxVerbatim}[commandchars=\\\{\}]
\PYG{g+go}{\PYGZsq{}Bilbo\PYGZsq{}}
\end{sphinxVerbatim}

\begin{sphinxVerbatim}[commandchars=\\\{\}]
\PYG{n}{z}
\end{sphinxVerbatim}

\begin{sphinxVerbatim}[commandchars=\\\{\}]
\PYG{g+go}{[\PYGZsq{}Gandalf\PYGZsq{}, \PYGZsq{}H\PYGZsq{}, \PYGZsq{}o\PYGZsq{}, \PYGZsq{}b\PYGZsq{}, \PYGZsq{}b\PYGZsq{}, \PYGZsq{}i\PYGZsq{}, \PYGZsq{}t\PYGZsq{}, \PYGZsq{}Hobbit\PYGZsq{}]}
\end{sphinxVerbatim}

Como o método pop retorna o último elemento, o mesmo pode ser utilizado para atribuição:

\begin{sphinxVerbatim}[commandchars=\\\{\}]
\PYG{n}{livro} \PYG{o}{=} \PYG{p}{(}\PYG{l+s+s1}{\PYGZsq{}}\PYG{l+s+s1}{O }\PYG{l+s+si}{\PYGZob{}\PYGZcb{}}\PYG{l+s+s1}{\PYGZsq{}}\PYG{o}{.}\PYG{n}{format}\PYG{p}{(}\PYG{n}{z}\PYG{o}{.}\PYG{n}{pop}\PYG{p}{(}\PYG{p}{)}\PYG{p}{)}\PYG{p}{)}
\PYG{n}{livro}
\end{sphinxVerbatim}

\begin{sphinxVerbatim}[commandchars=\\\{\}]
\PYG{g+go}{\PYGZsq{}O Hobbit\PYGZsq{}}
\end{sphinxVerbatim}

Retornando e apagando o sexto elemento:

\begin{sphinxVerbatim}[commandchars=\\\{\}]
\PYG{n}{z}\PYG{o}{.}\PYG{n}{pop}\PYG{p}{(}\PYG{l+m+mi}{5}\PYG{p}{)}
\end{sphinxVerbatim}

\begin{sphinxVerbatim}[commandchars=\\\{\}]
\PYG{g+go}{\PYGZsq{}i\PYGZsq{}}
\end{sphinxVerbatim}

Método sort, organiza a lista com seus elementos pela ordem alfabética:

\begin{sphinxVerbatim}[commandchars=\\\{\}]
\PYG{n}{z}\PYG{o}{.}\PYG{n}{sort}\PYG{p}{(}\PYG{p}{)}
\PYG{n}{z}
\end{sphinxVerbatim}

\begin{sphinxVerbatim}[commandchars=\\\{\}]
\PYG{g+go}{[\PYGZsq{}Gandalf\PYGZsq{}, \PYGZsq{}H\PYGZsq{}, \PYGZsq{}b\PYGZsq{}, \PYGZsq{}b\PYGZsq{}, \PYGZsq{}i\PYGZsq{}, \PYGZsq{}o\PYGZsq{}, \PYGZsq{}t\PYGZsq{}]}
\end{sphinxVerbatim}

O método reverse pega a atual posição dos elementos e reconstrói a lista
na ordem reversa:

\begin{sphinxVerbatim}[commandchars=\\\{\}]
\PYG{n}{z}\PYG{o}{.}\PYG{n}{reverse}\PYG{p}{(}\PYG{p}{)}
\PYG{n}{z}
\end{sphinxVerbatim}

\begin{sphinxVerbatim}[commandchars=\\\{\}]
\PYG{p}{[}\PYG{l+s+s1}{\PYGZsq{}}\PYG{l+s+s1}{t}\PYG{l+s+s1}{\PYGZsq{}}\PYG{p}{,} \PYG{l+s+s1}{\PYGZsq{}}\PYG{l+s+s1}{o}\PYG{l+s+s1}{\PYGZsq{}}\PYG{p}{,} \PYG{l+s+s1}{\PYGZsq{}}\PYG{l+s+s1}{b}\PYG{l+s+s1}{\PYGZsq{}}\PYG{p}{,} \PYG{l+s+s1}{\PYGZsq{}}\PYG{l+s+s1}{b}\PYG{l+s+s1}{\PYGZsq{}}\PYG{p}{,} \PYG{l+s+s1}{\PYGZsq{}}\PYG{l+s+s1}{H}\PYG{l+s+s1}{\PYGZsq{}}\PYG{p}{,} \PYG{l+s+s1}{\PYGZsq{}}\PYG{l+s+s1}{Gandalf}\PYG{l+s+s1}{\PYGZsq{}}\PYG{p}{]}
\end{sphinxVerbatim}

A função sorted não altera a lista, apenas retorna o conteúdo pela ordem
alfabética:

\begin{sphinxVerbatim}[commandchars=\\\{\}]
\PYG{n+nb}{sorted}\PYG{p}{(}\PYG{n}{z}\PYG{p}{)}
\end{sphinxVerbatim}

\begin{sphinxVerbatim}[commandchars=\\\{\}]
\PYG{g+go}{[\PYGZsq{}Gandalf\PYGZsq{}, \PYGZsq{}H\PYGZsq{}, \PYGZsq{}b\PYGZsq{}, \PYGZsq{}b\PYGZsq{}, \PYGZsq{}o\PYGZsq{}, \PYGZsq{}t\PYGZsq{}]}
\end{sphinxVerbatim}

\begin{sphinxVerbatim}[commandchars=\\\{\}]
\PYG{n}{z}
\end{sphinxVerbatim}

\begin{sphinxVerbatim}[commandchars=\\\{\}]
\PYG{g+go}{[\PYGZsq{}t\PYGZsq{}, \PYGZsq{}o\PYGZsq{}, \PYGZsq{}i\PYGZsq{}, \PYGZsq{}b\PYGZsq{}, \PYGZsq{}b\PYGZsq{}, \PYGZsq{}H\PYGZsq{}, \PYGZsq{}Gandalf\PYGZsq{}]}
\end{sphinxVerbatim}

A função sorted também pode somente retornar o reverso de uma lista:

\begin{sphinxVerbatim}[commandchars=\\\{\}]
\PYG{n+nb}{sorted}\PYG{p}{(}\PYG{p}{[}\PYG{l+m+mi}{1}\PYG{p}{,} \PYG{l+m+mi}{2}\PYG{p}{,} \PYG{l+m+mi}{3}\PYG{p}{]}\PYG{p}{,} \PYG{n}{reverse}\PYG{o}{=}\PYG{k+kc}{True}\PYG{p}{)}
\end{sphinxVerbatim}

\begin{sphinxVerbatim}[commandchars=\\\{\}]
\PYG{g+go}{[3, 2, 1]}
\end{sphinxVerbatim}

Criação de uma lista a partir do retorno de sorted:

\begin{sphinxVerbatim}[commandchars=\\\{\}]
\PYG{n}{w} \PYG{o}{=} \PYG{n+nb}{sorted}\PYG{p}{(}\PYG{n}{z}\PYG{p}{)}
\PYG{n}{w}
\end{sphinxVerbatim}

\begin{sphinxVerbatim}[commandchars=\\\{\}]
\PYG{g+go}{[\PYGZsq{}Gandalf\PYGZsq{}, \PYGZsq{}H\PYGZsq{}, \PYGZsq{}b\PYGZsq{}, \PYGZsq{}b\PYGZsq{}, \PYGZsq{}o\PYGZsq{}, \PYGZsq{}t\PYGZsq{}]}
\end{sphinxVerbatim}

\begin{sphinxVerbatim}[commandchars=\\\{\}]
\PYG{n}{z}
\end{sphinxVerbatim}

\begin{sphinxVerbatim}[commandchars=\\\{\}]
\PYG{g+go}{[\PYGZsq{}t\PYGZsq{}, \PYGZsq{}o\PYGZsq{}, \PYGZsq{}b\PYGZsq{}, \PYGZsq{}b\PYGZsq{}, \PYGZsq{}H\PYGZsq{}, \PYGZsq{}Gandalf\PYGZsq{}]}
\end{sphinxVerbatim}

A função reversed() sempre retorna um iterador:

\textgreater{} reversed(z)

\textless{}listreverseiterator object at 0x1653c10\textgreater{}

Convertendo para lista o iterador gerado pela função reversed:

\begin{sphinxVerbatim}[commandchars=\\\{\}]
\PYG{n}{z} \PYG{o}{=} \PYG{n+nb}{list}\PYG{p}{(}\PYG{n+nb}{reversed}\PYG{p}{(}\PYG{n}{z}\PYG{p}{)}\PYG{p}{)}
\PYG{n}{z}
\end{sphinxVerbatim}

\begin{sphinxVerbatim}[commandchars=\\\{\}]
\PYG{g+go}{[\PYGZsq{}Gandalf\PYGZsq{}, \PYGZsq{}H\PYGZsq{}, \PYGZsq{}b\PYGZsq{}, \PYGZsq{}b\PYGZsq{}, \PYGZsq{}o\PYGZsq{}, \PYGZsq{}t\PYGZsq{}]}
\end{sphinxVerbatim}

Função sorted transforma a string numa lista e organiza por ordem
alfabética:

\begin{sphinxVerbatim}[commandchars=\\\{\}]
\PYG{n+nb}{sorted}\PYG{p}{(}\PYG{l+s+s1}{\PYGZsq{}}\PYG{l+s+s1}{aAcb}\PYG{l+s+s1}{\PYGZsq{}}\PYG{p}{)}
\end{sphinxVerbatim}

\begin{sphinxVerbatim}[commandchars=\\\{\}]
\PYG{g+go}{[\PYGZsq{}A\PYGZsq{}, \PYGZsq{}a\PYGZsq{}, \PYGZsq{}b\PYGZsq{}, \PYGZsq{}c\PYGZsq{}]}
\end{sphinxVerbatim}

Função sorted transforma a string numa lista e organiza por ordem alfabética:

\begin{sphinxVerbatim}[commandchars=\\\{\}]
\PYG{n+nb}{sorted}\PYG{p}{(}\PYG{l+s+s1}{\PYGZsq{}}\PYG{l+s+s1}{aAcb}\PYG{l+s+s1}{\PYGZsq{}}\PYG{p}{,} \PYG{n}{reverse}\PYG{o}{=}\PYG{k+kc}{True}\PYG{p}{)}
\end{sphinxVerbatim}

\begin{sphinxVerbatim}[commandchars=\\\{\}]
\PYG{g+go}{[\PYGZsq{}c\PYGZsq{}, \PYGZsq{}b\PYGZsq{}, \PYGZsq{}a\PYGZsq{}, \PYGZsq{}A\PYGZsq{}]}
\end{sphinxVerbatim}

Função reversed transforma a string em uma lista com seus caracteres em ordem
reversa:

\begin{sphinxVerbatim}[commandchars=\\\{\}]
\PYG{n+nb}{list}\PYG{p}{(}\PYG{n+nb}{reversed}\PYG{p}{(}\PYG{l+s+s1}{\PYGZsq{}}\PYG{l+s+s1}{aAcb}\PYG{l+s+s1}{\PYGZsq{}}\PYG{p}{)}\PYG{p}{)}
\end{sphinxVerbatim}

\begin{sphinxVerbatim}[commandchars=\\\{\}]
\PYG{g+go}{[\PYGZsq{}b\PYGZsq{}, \PYGZsq{}c\PYGZsq{}, \PYGZsq{}A\PYGZsq{}, \PYGZsq{}a\PYGZsq{}]}
\end{sphinxVerbatim}

Definição de lista com 3 (três) elementos:

\begin{sphinxVerbatim}[commandchars=\\\{\}]
\PYG{n}{x} \PYG{o}{=} \PYG{p}{[}\PYG{l+m+mi}{1}\PYG{p}{,} \PYG{l+m+mi}{2}\PYG{p}{,} \PYG{l+m+mi}{3}\PYG{p}{]}
\end{sphinxVerbatim}

A partir da lista “x”, atribuir respectivamente seus elementos como valores
para as variáveis à esquerdae exibir seus valores:

\begin{sphinxVerbatim}[commandchars=\\\{\}]
\PYG{n}{a}\PYG{p}{,} \PYG{n}{b}\PYG{p}{,} \PYG{n}{c} \PYG{o}{=} \PYG{n}{x}
\PYG{n+nb}{print}\PYG{p}{(}\PYG{n}{a}\PYG{p}{)}
\end{sphinxVerbatim}

\begin{sphinxVerbatim}[commandchars=\\\{\}]
\PYG{g+go}{1}
\end{sphinxVerbatim}

\begin{sphinxVerbatim}[commandchars=\\\{\}]
\PYG{n+nb}{print}\PYG{p}{(}\PYG{n}{b}\PYG{p}{)}
\end{sphinxVerbatim}

\begin{sphinxVerbatim}[commandchars=\\\{\}]
\PYG{g+go}{2}
\end{sphinxVerbatim}

\begin{sphinxVerbatim}[commandchars=\\\{\}]
\PYG{n+nb}{print}\PYG{p}{(}\PYG{n}{c}\PYG{p}{)}
\end{sphinxVerbatim}

\begin{sphinxVerbatim}[commandchars=\\\{\}]
\PYG{g+go}{3}
\end{sphinxVerbatim}

E se a quantidade de variáveis que receberão os valores forem em menor
número que a quantidade de elementos da lista?:

\begin{sphinxVerbatim}[commandchars=\\\{\}]
\PYG{n}{y}\PYG{p}{,} \PYG{n}{z}\PYG{p}{,} \PYG{o}{=} \PYG{n}{x}
\end{sphinxVerbatim}

\begin{sphinxVerbatim}[commandchars=\\\{\}]
\PYG{g+go}{ValueError: too many values to unpack (expected 2)}
\end{sphinxVerbatim}

Lista com três elementos não pode fazer atribuição respectiva a apenas
duas variáveis.

Utilização do caractere underscore como solução:

\begin{sphinxVerbatim}[commandchars=\\\{\}]
\PYG{n}{y}\PYG{p}{,} \PYG{n}{z}\PYG{p}{,} \PYG{n}{\PYGZus{}} \PYG{o}{=} \PYG{n}{x}
\PYG{n+nb}{print}\PYG{p}{(}\PYG{n}{y}\PYG{p}{)}
\end{sphinxVerbatim}

\begin{sphinxVerbatim}[commandchars=\\\{\}]
\PYG{g+go}{1}
\end{sphinxVerbatim}

\begin{sphinxVerbatim}[commandchars=\\\{\}]
\PYG{n+nb}{print}\PYG{p}{(}\PYG{n}{z}\PYG{p}{)}
\end{sphinxVerbatim}

\begin{sphinxVerbatim}[commandchars=\\\{\}]
\PYG{g+go}{2}
\end{sphinxVerbatim}


\chapter{Tupla}
\label{\detokenize{content/tuple:tupla}}\label{\detokenize{content/tuple::doc}}
\begin{DUlineblock}{0em}
\item[]
\begin{DUlineblock}{\DUlineblockindent}
\item[] Tupla têm sua estrutura muito similar à da lista, no entanto, ela é imutável.
\item[] Ela é mais recomendada para uso de dados estáticos, pois comparada à lista, tem
\end{DUlineblock}
\item[] um desempenho melhor devido à sua simplicidade e consome menos recursos.
\end{DUlineblock}

\begin{sphinxVerbatim}[commandchars=\\\{\}]
\PYG{c+c1}{\PYGZsh{} Declaração de uma tupla vazia:}
    \PYG{n}{t} \PYG{o}{=} \PYG{p}{(}\PYG{p}{)}
\end{sphinxVerbatim}

ou

\begin{sphinxVerbatim}[commandchars=\\\{\}]
\PYG{n}{t} \PYG{o}{=} \PYG{n+nb}{tuple}\PYG{p}{(}\PYG{p}{)}
\end{sphinxVerbatim}

\begin{sphinxVerbatim}[commandchars=\\\{\}]
\PYG{c+c1}{\PYGZsh{} Declaração de uma tupla com três elementos:}
    \PYG{n}{t} \PYG{o}{=} \PYG{n+nb}{tuple}\PYG{p}{(}\PYG{p}{[}\PYG{l+m+mi}{1}\PYG{p}{,} \PYG{l+m+mi}{2}\PYG{p}{,} \PYG{l+m+mi}{3}\PYG{p}{]}\PYG{p}{)}
\end{sphinxVerbatim}

ou

\begin{sphinxVerbatim}[commandchars=\\\{\}]
\PYG{n}{t} \PYG{o}{=} \PYG{p}{(}\PYG{l+m+mi}{1}\PYG{p}{,} \PYG{l+m+mi}{2}\PYG{p}{,} \PYG{l+m+mi}{3}\PYG{p}{)}
\end{sphinxVerbatim}

ou

\begin{sphinxVerbatim}[commandchars=\\\{\}]
\PYG{n}{t} \PYG{o}{=} \PYG{l+m+mi}{1}\PYG{p}{,} \PYG{l+m+mi}{2}\PYG{p}{,} \PYG{l+m+mi}{3}
\end{sphinxVerbatim}

\begin{sphinxVerbatim}[commandchars=\\\{\}]
\PYG{c+c1}{\PYGZsh{} Exibindo o conteúdo da tupla:}
    \PYG{n}{t}
\end{sphinxVerbatim}

\begin{sphinxVerbatim}[commandchars=\\\{\}]
\PYG{g+gp+gpVirtualEnv}{(1, 2, 3)}
\end{sphinxVerbatim}

\begin{sphinxVerbatim}[commandchars=\\\{\}]
\PYG{c+c1}{\PYGZsh{} Criação de um conjunto (set) contendo todos atributos e}
\PYG{c+c1}{\PYGZsh{} métodos de uma tupla:}
\PYG{n}{tupla} \PYG{o}{=} \PYG{n+nb}{set}\PYG{p}{(}\PYG{n+nb}{dir}\PYG{p}{(}\PYG{n+nb}{tuple}\PYG{p}{)}\PYG{p}{)}
\end{sphinxVerbatim}

\begin{sphinxVerbatim}[commandchars=\\\{\}]
\PYG{c+c1}{\PYGZsh{} Criação de um conjunto (set) contendo todos atributos e}
    \PYG{c+c1}{\PYGZsh{} métodos de uma lista:}
    \PYG{n}{lista} \PYG{o}{=} \PYG{n+nb}{set}\PYG{p}{(}\PYG{n+nb}{dir}\PYG{p}{(}\PYG{n+nb}{list}\PYG{p}{)}\PYG{p}{)}
\end{sphinxVerbatim}

\begin{sphinxVerbatim}[commandchars=\\\{\}]
\PYG{c+c1}{\PYGZsh{} Via intersecção, o que há em comum entre lista e tupla?:}
    \PYG{n}{tupla}\PYG{o}{.}\PYG{n}{intersection}\PYG{p}{(}\PYG{n}{lista}\PYG{p}{)}
\end{sphinxVerbatim}

…

Tuplas tem apenas os métodos count e index.

\begin{sphinxVerbatim}[commandchars=\\\{\}]
\PYG{c+c1}{\PYGZsh{} Tupla de um único elemento:}
    \PYG{n}{t} \PYG{o}{=} \PYG{p}{(}\PYG{l+m+mi}{1}\PYG{p}{,} \PYG{p}{)}
\end{sphinxVerbatim}

\begin{sphinxVerbatim}[commandchars=\\\{\}]
\PYG{c+c1}{\PYGZsh{} Exibir o conteúdo da tupla:}
    \PYG{n}{t}
\end{sphinxVerbatim}

\begin{sphinxVerbatim}[commandchars=\\\{\}]
\PYG{g+gp+gpVirtualEnv}{(1,)}
\end{sphinxVerbatim}

\begin{sphinxVerbatim}[commandchars=\\\{\}]
\PYG{c+c1}{\PYGZsh{} Função type para verificar o tipo do objeto:}
    \PYG{n+nb}{type}\PYG{p}{(}\PYG{n}{t}\PYG{p}{)}
\end{sphinxVerbatim}

\begin{sphinxVerbatim}[commandchars=\\\{\}]
\PYG{g+go}{tuple}
\end{sphinxVerbatim}

\begin{sphinxVerbatim}[commandchars=\\\{\}]
\PYG{c+c1}{\PYGZsh{} Declaração de duas variáveis e trocando o valor entre elas:}
    \PYG{n}{x} \PYG{o}{=} \PYG{l+m+mi}{0}
    \PYG{n}{y} \PYG{o}{=} \PYG{l+m+mi}{1}
    \PYG{n}{x}\PYG{p}{,} \PYG{n}{y} \PYG{o}{=} \PYG{n}{y}\PYG{p}{,} \PYG{n}{x}  \PYG{c+c1}{\PYGZsh{} A troca se dá pela atribuição respectiva}
\end{sphinxVerbatim}

\begin{sphinxVerbatim}[commandchars=\\\{\}]
\PYG{c+c1}{\PYGZsh{} Verificando os valores das variáveis:}
\PYG{n}{x}
\end{sphinxVerbatim}

\begin{sphinxVerbatim}[commandchars=\\\{\}]
\PYG{g+go}{1}
\end{sphinxVerbatim}

\begin{sphinxVerbatim}[commandchars=\\\{\}]
\PYG{n}{y}
\end{sphinxVerbatim}

\begin{sphinxVerbatim}[commandchars=\\\{\}]
\PYG{g+go}{0}
\end{sphinxVerbatim}

\begin{sphinxVerbatim}[commandchars=\\\{\}]
\PYG{c+c1}{\PYGZsh{} Criação de uma variável que retorna uma tupla com três elementos:}
    \PYG{k}{def} \PYG{n+nf}{retorna\PYGZus{}tupla}\PYG{p}{(}\PYG{p}{)}\PYG{p}{:}
            \PYG{k}{return} \PYG{l+m+mi}{1}\PYG{p}{,} \PYG{l+m+mi}{2}\PYG{p}{,} \PYG{l+m+mi}{3}
\end{sphinxVerbatim}

\begin{sphinxVerbatim}[commandchars=\\\{\}]
\PYG{c+c1}{\PYGZsh{} Atribuição respectiva:}
    \PYG{n}{x}\PYG{p}{,} \PYG{n}{y}\PYG{p}{,} \PYG{n}{z} \PYG{o}{=} \PYG{n}{retorna\PYGZus{}tupla}\PYG{p}{(}\PYG{p}{)}
\end{sphinxVerbatim}

\begin{sphinxVerbatim}[commandchars=\\\{\}]
\PYG{c+c1}{\PYGZsh{} Verificando os valores das variáveis:}
    \PYG{n+nb}{print}\PYG{p}{(}\PYG{n}{x}\PYG{p}{)}
\end{sphinxVerbatim}

\begin{sphinxVerbatim}[commandchars=\\\{\}]
\PYG{g+go}{1}
\end{sphinxVerbatim}

\begin{sphinxVerbatim}[commandchars=\\\{\}]
\PYG{n+nb}{print}\PYG{p}{(}\PYG{n}{y}\PYG{p}{)}
\end{sphinxVerbatim}

\begin{sphinxVerbatim}[commandchars=\\\{\}]
\PYG{g+go}{2}
\end{sphinxVerbatim}

\begin{sphinxVerbatim}[commandchars=\\\{\}]
\PYG{n+nb}{print}\PYG{p}{(}\PYG{n}{z}\PYG{p}{)}
\end{sphinxVerbatim}

\begin{sphinxVerbatim}[commandchars=\\\{\}]
\PYG{g+go}{3}
\end{sphinxVerbatim}

Tuplas são imutáveis, mas seus elementos não necessariamente:

\begin{sphinxVerbatim}[commandchars=\\\{\}]
\PYG{n}{t} \PYG{o}{=} \PYG{p}{(}\PYG{p}{\PYGZob{}}\PYG{p}{\PYGZcb{}}\PYG{p}{,} \PYG{p}{[}\PYG{p}{]}\PYG{p}{)}   \PYG{c+c1}{\PYGZsh{} Tupla com dois elementos; um dicionário e uma lista}
    \PYG{n}{t}\PYG{p}{[}\PYG{l+m+mi}{0}\PYG{p}{]}\PYG{o}{.}\PYG{n}{update}\PYG{p}{(}\PYG{p}{\PYGZob{}}\PYG{l+s+s1}{\PYGZsq{}}\PYG{l+s+s1}{chave}\PYG{l+s+s1}{\PYGZsq{}}\PYG{p}{:} \PYG{l+s+s1}{\PYGZsq{}}\PYG{l+s+s1}{valor}\PYG{l+s+s1}{\PYGZsq{}}\PYG{p}{\PYGZcb{}}\PYG{p}{)}  \PYG{c+c1}{\PYGZsh{} Alterando o primeiro elemento}
    \PYG{n}{t}\PYG{p}{[}\PYG{l+m+mi}{1}\PYG{p}{]}\PYG{o}{.}\PYG{n}{append}\PYG{p}{(}\PYG{l+m+mi}{7}\PYG{p}{)}  \PYG{c+c1}{\PYGZsh{} Alterando o segundo elemento}
    \PYG{n}{t}  \PYG{c+c1}{\PYGZsh{} Exibindo a tupla}
\end{sphinxVerbatim}

\begin{sphinxVerbatim}[commandchars=\\\{\}]
\PYG{g+gp+gpVirtualEnv}{(\PYGZob{}\PYGZsq{}chave\PYGZsq{}: \PYGZsq{}valor\PYGZsq{}\PYGZcb{}, [7])}
\end{sphinxVerbatim}


\chapter{set e frozenset}
\label{\detokenize{content/set_frozenset:set-e-frozenset}}\label{\detokenize{content/set_frozenset::doc}}
\begin{DUlineblock}{0em}
\item[] Similarmente a list e tuple, a diferença entre eles é que frozenset é um tipo imutável.
\item[] Seus elementos são únicos.
\item[] Implementam operações matemáticas de conjuntos.
\end{DUlineblock}

\begin{sphinxVerbatim}[commandchars=\\\{\}]
\PYG{c+c1}{\PYGZsh{} Set vazio:}
\PYG{n}{s} \PYG{o}{=} \PYG{p}{\PYGZob{}}\PYG{o}{*}\PYG{p}{(}\PYG{p}{)}\PYG{p}{\PYGZcb{}}
\end{sphinxVerbatim}

ou

\begin{sphinxVerbatim}[commandchars=\\\{\}]
\PYG{n}{s} \PYG{o}{=} \PYG{n+nb}{set}\PYG{p}{(}\PYG{p}{)}
\end{sphinxVerbatim}

\begin{sphinxVerbatim}[commandchars=\\\{\}]
\PYG{c+c1}{\PYGZsh{} Criação de um conjunto (set) não vazio:}
\PYG{n}{s} \PYG{o}{=} \PYG{n+nb}{set}\PYG{p}{(}\PYG{p}{[}\PYG{l+m+mi}{1}\PYG{p}{,} \PYG{l+m+mi}{1}\PYG{p}{,} \PYG{l+m+mi}{2}\PYG{p}{,} \PYG{l+m+mi}{3}\PYG{p}{]}\PYG{p}{)}
\end{sphinxVerbatim}

ou

\begin{sphinxVerbatim}[commandchars=\\\{\}]
\PYG{n}{s} \PYG{o}{=} \PYG{n+nb}{set}\PYG{p}{(}\PYG{p}{(}\PYG{l+m+mi}{1}\PYG{p}{,} \PYG{l+m+mi}{1}\PYG{p}{,} \PYG{l+m+mi}{2}\PYG{p}{,} \PYG{l+m+mi}{3}\PYG{p}{)}\PYG{p}{)}
\end{sphinxVerbatim}

ou

\begin{sphinxVerbatim}[commandchars=\\\{\}]
\PYG{n}{s} \PYG{o}{=} \PYG{n+nb}{set}\PYG{p}{(}\PYG{p}{[}\PYG{l+m+mi}{1}\PYG{p}{,} \PYG{l+m+mi}{1}\PYG{p}{,} \PYG{l+m+mi}{2}\PYG{p}{,} \PYG{l+m+mi}{3}\PYG{p}{]}\PYG{p}{)}
\end{sphinxVerbatim}

ou

\begin{sphinxVerbatim}[commandchars=\\\{\}]
\PYG{n}{s} \PYG{o}{=} \PYG{n+nb}{set}\PYG{p}{(}\PYG{p}{\PYGZob{}}\PYG{l+m+mi}{1}\PYG{p}{,} \PYG{l+m+mi}{1}\PYG{p}{,} \PYG{l+m+mi}{2}\PYG{p}{,} \PYG{l+m+mi}{3}\PYG{p}{\PYGZcb{}}\PYG{p}{)}
\end{sphinxVerbatim}

ou

\begin{sphinxVerbatim}[commandchars=\\\{\}]
\PYG{n}{s} \PYG{o}{=} \PYG{p}{\PYGZob{}}\PYG{l+m+mi}{1}\PYG{p}{,} \PYG{l+m+mi}{2}\PYG{p}{,} \PYG{l+m+mi}{3}\PYG{p}{\PYGZcb{}}
\end{sphinxVerbatim}

\begin{sphinxVerbatim}[commandchars=\\\{\}]
\PYG{c+c1}{\PYGZsh{} Exibindo o conteúdo do set:}
\PYG{n}{s}
\end{sphinxVerbatim}

\begin{sphinxVerbatim}[commandchars=\\\{\}]
\PYG{g+go}{\PYGZob{}1, 2, 3\PYGZcb{}}
\end{sphinxVerbatim}

\begin{sphinxVerbatim}[commandchars=\\\{\}]
\PYG{c+c1}{\PYGZsh{} Definição de dois conjuntos:}
\PYG{n}{a} \PYG{o}{=} \PYG{n+nb}{set}\PYG{p}{(}\PYG{p}{[}\PYG{l+m+mi}{1}\PYG{p}{,}\PYG{l+m+mi}{2}\PYG{p}{,}\PYG{l+m+mi}{3}\PYG{p}{]}\PYG{p}{)}
\PYG{n}{b} \PYG{o}{=} \PYG{n+nb}{set}\PYG{p}{(}\PYG{p}{[}\PYG{l+m+mi}{2}\PYG{p}{,}\PYG{l+m+mi}{3}\PYG{p}{,}\PYG{l+m+mi}{4}\PYG{p}{]}\PYG{p}{)}
\end{sphinxVerbatim}

\begin{sphinxVerbatim}[commandchars=\\\{\}]
\PYG{c+c1}{\PYGZsh{} Operação de união entre os conjuntos:}
\PYG{n}{a} \PYG{o}{|} \PYG{n}{b}
\end{sphinxVerbatim}

ou

\begin{sphinxVerbatim}[commandchars=\\\{\}]
\PYG{n}{a}\PYG{o}{.}\PYG{n}{union}\PYG{p}{(}\PYG{n}{b}\PYG{p}{)}
\end{sphinxVerbatim}

\begin{sphinxVerbatim}[commandchars=\\\{\}]
\PYG{g+go}{\PYGZob{}1, 2, 3, 4\PYGZcb{}}
\end{sphinxVerbatim}

\begin{sphinxVerbatim}[commandchars=\\\{\}]
\PYG{c+c1}{\PYGZsh{} Operação de intersecção entre os conjuntos:}
\PYG{n}{a} \PYG{o}{\PYGZam{}} \PYG{n}{b}
\end{sphinxVerbatim}

ou

\begin{sphinxVerbatim}[commandchars=\\\{\}]
\PYG{n}{a}\PYG{o}{.}\PYG{n}{intersection}\PYG{p}{(}\PYG{n}{b}\PYG{p}{)}
\end{sphinxVerbatim}

\begin{sphinxVerbatim}[commandchars=\\\{\}]
\PYG{g+go}{\PYGZob{}2, 3\PYGZcb{}}
\end{sphinxVerbatim}

\begin{sphinxVerbatim}[commandchars=\\\{\}]
\PYG{c+c1}{\PYGZsh{} Frozenset vazio:}
\PYG{n}{f} \PYG{o}{=} \PYG{n+nb}{frozenset}\PYG{p}{(}\PYG{p}{)}
\end{sphinxVerbatim}

\begin{sphinxVerbatim}[commandchars=\\\{\}]
\PYG{c+c1}{\PYGZsh{} Frozenset não vazio:}
\PYG{n}{f} \PYG{o}{=} \PYG{n+nb}{frozenset}\PYG{p}{(}\PYG{p}{(}\PYG{l+m+mi}{1}\PYG{p}{,} \PYG{l+m+mi}{2}\PYG{p}{,} \PYG{l+m+mi}{3}\PYG{p}{)}\PYG{p}{)}
\end{sphinxVerbatim}

ou

\begin{sphinxVerbatim}[commandchars=\\\{\}]
\PYG{n}{f} \PYG{o}{=} \PYG{n+nb}{frozenset}\PYG{p}{(}\PYG{p}{\PYGZob{}}\PYG{l+m+mi}{1}\PYG{p}{,} \PYG{l+m+mi}{2}\PYG{p}{,} \PYG{l+m+mi}{3}\PYG{p}{\PYGZcb{}}\PYG{p}{)}
\end{sphinxVerbatim}

ou

\begin{sphinxVerbatim}[commandchars=\\\{\}]
\PYG{n}{f} \PYG{o}{=} \PYG{n+nb}{frozenset}\PYG{p}{(}\PYG{p}{[}\PYG{l+m+mi}{1}\PYG{p}{,} \PYG{l+m+mi}{2}\PYG{p}{,} \PYG{l+m+mi}{3}\PYG{p}{]}\PYG{p}{)}
\end{sphinxVerbatim}

ou

\begin{sphinxVerbatim}[commandchars=\\\{\}]
\PYG{n}{f} \PYG{o}{=} \PYG{n+nb}{frozenset}\PYG{p}{(}\PYG{p}{\PYGZob{}}\PYG{l+m+mi}{1}\PYG{p}{,} \PYG{l+m+mi}{2}\PYG{p}{,} \PYG{l+m+mi}{3}\PYG{p}{\PYGZcb{}}\PYG{p}{)}
\end{sphinxVerbatim}

\begin{sphinxVerbatim}[commandchars=\\\{\}]
\PYG{c+c1}{\PYGZsh{} Exibindo o conteúdo do frozenset:}
\PYG{n}{f}
\end{sphinxVerbatim}

\begin{sphinxVerbatim}[commandchars=\\\{\}]
\PYG{g+go}{frozenset(\PYGZob{}1, 2, 3\PYGZcb{})}
\end{sphinxVerbatim}


\chapter{Dicionários}
\label{\detokenize{content/dict:dicionarios}}\label{\detokenize{content/dict::doc}}\begin{quote}

Em Python um dicionário é uma estrutura de dados envolta por chaves cujos elementos são compostos por chave:valor e separados por vírgula.
\end{quote}

Dicionário vazio:

\textgreater{} d = \{\}

Criação de um dicionário utilizando a função dict:

\textgreater{} d = dict(chave1 = ‘valor2’, chave2 = 1980)

Criação de um dicionário utilizando chaves:
\begin{description}
\item[{\textgreater{} d = \{}] \leavevmode
‘nome’: ‘Chiquinho’,
‘sobrenome’: ‘da Silva’,
‘email’: \sphinxhref{mailto:\textquotesingle{}chiquinho.silva@zeninguem.com}{‘chiquinho.silva@zeninguem.com}’,

\end{description}

\}

Exibir o valor de uma chave do dicionário:

\textgreater{} d{[}‘nome’{]}

‘Chiquinho’

Loop para exibir cada par de chave e valor do dicionário:
\begin{description}
\item[{\textgreater{} for k in d:}] \leavevmode
print(f’\{k\} = \{d{[}k{]}\}’)

\end{description}

nome = Chiquinho
sobrenome = da Silva
email = \sphinxhref{mailto:chiquinho.silva@zeninguem.com}{chiquinho.silva@zeninguem.com}

Loop para exibir cada par de chave e valor do dicionário ordenado por chave:
\begin{description}
\item[{\textgreater{} for k in sorted(d):}] \leavevmode
print(f’\{k\} = \{d{[}k{]}\}’)

\end{description}

email = \sphinxhref{mailto:chiquinho.silva@zeninguem.com}{chiquinho.silva@zeninguem.com}
nome = Chiquinho
sobrenome = da Silva

Itens do dicionário:

\textgreater{} d.items()

dict\_items({[}(‘nome’, ‘Chiquinho’), (‘sobrenome’, ‘da Silva’), (‘email’, \sphinxhref{mailto:\textquotesingle{}chiquinho.silva@zeninguem.com}{‘chiquinho.silva@zeninguem.com}’){]})

Loop sobre os itens do dicionário:
\begin{description}
\item[{\textgreater{} for k, v in d.items():}] \leavevmode
print(f’\{k\} = \{v\}’)

\end{description}

nome = Chiquinho
sobrenome = da Silva
email = \sphinxhref{mailto:chiquinho.silva@zeninguem.com}{chiquinho.silva@zeninguem.com}

Método para exibir as chaves do dicionário:

\textgreater{} d.keys()

dict\_keys({[}‘nome’, ‘sobrenome’, ‘email’{]})

Método para exibir as chaves do dicionário:

\textgreater{} d.values()

dict\_values({[}‘Chiquinho’, ‘da Silva’, \sphinxhref{mailto:\textquotesingle{}chiquinho.silva@zeninguem.com}{‘chiquinho.silva@zeninguem.com}’{]})

Método para limpar o dicionário:

\textgreater{} d.clear()

Dicionário após a execução do método clear:

\textgreater{} d

\{\}

Método update para adicionar chaves e valores:
\begin{description}
\item[{\textgreater{} d.update(\{}] \leavevmode
‘nome’: ‘Chiquinho’,
‘sobrenome’: ‘da Silva’,
‘email’: \sphinxhref{mailto:\textquotesingle{}chiquinho.silva@zeninguem.com}{‘chiquinho.silva@zeninguem.com}’,

\end{description}

\})

Exibindo o conteúdo do dicionário:

\textgreater{} d
\begin{description}
\item[{\{‘email’: \sphinxhref{mailto:\textquotesingle{}chiquinho.silva@zeninguem.com}{‘chiquinho.silva@zeninguem.com}’,}] \leavevmode
‘nome’: ‘Chiquinho’,
‘sobrenome’: ‘da Silva’\}

\end{description}

Adicionando uma chave e valor ao dicionário:

\textgreater{} d.update(\{‘data\_nascimento’: ‘28/02/1936’\})

Exibindo o conteúdo do dicionário:

\textgreater{} d
\begin{description}
\item[{\{‘nome’: ‘Chiquinho’,}] \leavevmode
‘sobrenome’: ‘da Silva’,
‘email’: \sphinxhref{mailto:\textquotesingle{}chiquinho.silva@zeninguem.com}{‘chiquinho.silva@zeninguem.com}’,
‘data\_nascimento’: ‘28/02/1936’\}

\end{description}

Alterando uma chave do dicionário:

\textgreater{} d{[}‘data\_nascimento’{]} = ‘08/02/1936’

Exibindo o conteúdo do dicionário:

\textgreater{} d
\begin{description}
\item[{\{‘nome’: ‘Chiquinho’,}] \leavevmode
‘sobrenome’: ‘da Silva’,
‘email’: \sphinxhref{mailto:\textquotesingle{}chiquinho.silva@zeninguem.com}{‘chiquinho.silva@zeninguem.com}’,
‘data\_nascimento’: ‘08/02/1936’\}

\end{description}

Adicionando uma nova chave e valor:

\textgreater{} d{[}‘cidade\_origem’{]} = ‘Cascatinha’

Exibindo o conteúdo do dicionário:

\textgreater{} d
\begin{description}
\item[{\{‘nome’: ‘Chiquinho’,}] \leavevmode
‘sobrenome’: ‘da Silva’,
‘email’: \sphinxhref{mailto:\textquotesingle{}chiquinho.silva@zeninguem.com}{‘chiquinho.silva@zeninguem.com}’,
‘data\_nascimento’: ‘08/02/1936’,
‘cidade\_origem’: ‘Cascatinha’\}
\begin{quote}

Como chave só são aceitos tipos que utilizam hash, listas não são aceitas, mas instâncias de classes e tuplas são aceitas.
\end{quote}

\end{description}

Criação de uma lista e uma tupla respectivamente:

\textgreater{} l = {[}1, 2, 3{]}

\textgreater{} t = (5, 7)

Criação de uma classe para teste:
\begin{description}
\item[{\textgreater{} class Funcionario(object):}] \leavevmode
pass

\end{description}

Objeto da classe Funcionário:

\textgreater{} f1 = Funcionario()

Tentativa de fazer uma lista ser chave do dicionário e atribuir um valor:

\textgreater{} d{[}l{]} = 0

TypeError: unhashable type: ‘list’
\begin{quote}

Listas são “unhashable”, portanto não podem ser chaves de um dicionário.
\end{quote}

Objeto da classe Funcionário como chave do dicionário:

\textgreater{} d{[}f1{]} = ‘Funcionário 1’

Exibir o conteúdo do dicionário:

\textgreater{} d
\begin{description}
\item[{\{‘nome’: ‘Chiquinho’,}] \leavevmode
‘sobrenome’: ‘da Silva’,
‘email’: \sphinxhref{mailto:\textquotesingle{}chiquinho.silva@zeninguem.com}{‘chiquinho.silva@zeninguem.com}’,
‘data\_nascimento’: ‘08/02/1936’,
‘cidade\_origem’: ‘Cascatinha’,
\textless{}\_\_main\_\_.Funcionario at 0x7f3769ca9c88\textgreater{}: ‘Funcionário 1’\}

\end{description}

Tupla como chave do dicionário e atribuir um valor:

\textgreater{} d{[}t{]} = 0

Exibir o conteúdo do dicionário:

\textgreater{} d
\begin{description}
\item[{\{‘nome’: ‘Chiquinho’,}] \leavevmode
‘sobrenome’: ‘da Silva’,
‘email’: \sphinxhref{mailto:\textquotesingle{}chiquinho.silva@zeninguem.com}{‘chiquinho.silva@zeninguem.com}’,
‘data\_nascimento’: ‘08/02/1936’,
‘cidade\_origem’: ‘Cascatinha’,
\textless{}\_\_main\_\_.Funcionario at 0x7f3769ca9c88\textgreater{}: ‘Funcionário 1’,
(5, 7): 0\}

\end{description}

Tentativa de acessar uma chave inexistente:

\textgreater{} d{[}endereco{]}

NameError: name ‘endereco’ is not defined

Método get:

\textgreater{} d.get(‘nome’)

‘Chiquinho’
\begin{quote}

Existe a chave “nome”, então seu valor foi retornado.
\end{quote}

\textgreater{} d.get(‘endereco’)
\begin{quote}

Não existe a chave “endereço”, por isso nada foi retornado, no entanto, não foi lançada exceção.
\end{quote}

\textgreater{} d.get(‘endereco’, ‘R. do Cafezal, 30’)

‘R. do Cafezal, 30’
\begin{quote}

Não existe a chave “endereço”, mas no método get foi passado um segundo parâmetro, o qual foi retornado, porém não houve modificação no dicionário.
\end{quote}

\textgreater{} d.get(‘nome’, ‘Zezinho’)

‘Chiquinho’
\begin{quote}

A chave “nome” já existia, sendo um valor diferente do valor da mesma, foi retornado o valor que pertence à chave.
\end{quote}

Criação de um dicionário:
\begin{description}
\item[{\textgreater{} carro = \{}] \leavevmode
‘marca’: ‘Fiat’,
‘modelo’: ‘147’

\end{description}

\}

Tentando acessar uma chave inexistente:

\textgreater{} carro{[}‘cor’{]}

KeyError: ‘cor’

Método get apenas para retorno:

\textgreater{} carro.get(‘cor’, ‘amarelo’)

‘amarelo’

Verificando o conteúdo do dicionário:

\textgreater{} carro

\{‘marca’: ‘Fiat’, ‘modelo’: ‘147’\}

Método setdefault:

\textgreater{} carro.setdefault(‘modelo’, ‘Topazio’)

‘147’
\begin{quote}

Já havia uma chave “modelo”, então foi retornado seu valor e não o segundo parâmetro fornecido.
\end{quote}

\textgreater{} carro.setdefault(‘cor’, ‘verde’)

‘verde’
\begin{quote}

Não havia uma chave “cor”, agora ela e seu valor fazem parte do dicionário.
\end{quote}

Verificando o conteúdo do dicionário:

\textgreater{} carro

\{‘cor’: ‘verde’, ‘marca’: ‘Fiat’, ‘modelo’: ‘147’\}

Método update para alterar valores de chaves pré existentes ou mesmo para adicionar novos pares de chave\sphinxhyphen{}valor:

\textgreater{} carro.update(modelo = ‘Topazio’, cor = ‘cinza’)

Verificando o conteúdo do dicionário:

\textgreater{} carro

\{‘cor’: ‘cinza’, ‘marca’: ‘Fiat’, ‘modelo’: ‘Topazio’\}

Método pop; retira uma chave do dicionário e retorna seu valor:

\textgreater{} carro.pop(‘cor’)

‘cinza’

Verificando o conteúdo do dicionário:

\textgreater{} carro

\{‘marca’: ‘Fiat’, ‘modelo’: ‘Topazio’\}

Método pop para uma chave que não existe:

\textgreater{} carro.pop(‘ano’)

KeyError: ‘ano’

Método pop para uma chave que não existe, mas fornecendo um valor:

\textgreater{} carro.pop(‘ano’, 1981)

1981
\begin{quote}

O dicionário continua sem a chave, mas não lançou uma exceção.
\end{quote}

Existe a chave “marca” no dicionário?:

\textgreater{} ‘marca’ in carro

True

Existe a chave “cor” no dicionário?:

\textgreater{} ‘cor’ in carro

False

Adicionando novas chaves e seus respectivos valores:

\textgreater{} carro{[}‘cor’{]} = ‘cinza’

\textgreater{} carro{[}‘ano’{]} = 1981

Verificando o conteúdo do dicionário:

\textgreater{} carro

\{‘marca’: ‘Fiat’, ‘modelo’: ‘Topazio’, ‘cor’: ‘cinza’, ‘ano’: 1981\}

\# \sphinxurl{https://data-flair.training/blogs/python-operator/}
\# \sphinxurl{https://www.tutorialspoint.com/python/bitwise\_operators\_example}


\chapter{Aritméticos}
\label{\detokenize{content/operators:aritmeticos}}\label{\detokenize{content/operators::doc}}
\begin{DUlineblock}{0em}
\item[] Descrição       | Operador | Exemplo | Saída |
\end{DUlineblock}

{\color{red}\bfseries{}|\sphinxhyphen{}\sphinxhyphen{}\sphinxhyphen{}\sphinxhyphen{}\sphinxhyphen{}\sphinxhyphen{}\sphinxhyphen{}\sphinxhyphen{}\sphinxhyphen{}\sphinxhyphen{}\sphinxhyphen{}\sphinxhyphen{}\sphinxhyphen{}\sphinxhyphen{}\sphinxhyphen{}\sphinxhyphen{}\sphinxhyphen{}|}———\sphinxhyphen{}{\color{red}\bfseries{}|\sphinxhyphen{}\sphinxhyphen{}\sphinxhyphen{}\sphinxhyphen{}\sphinxhyphen{}\sphinxhyphen{}\sphinxhyphen{}\sphinxhyphen{}\sphinxhyphen{}|}——\sphinxhyphen{}|
| Soma            | +        |  5 + 2  |  7    |
|  Subtração      | \sphinxhyphen{}        | 13 \sphinxhyphen{} 3  | 10    |
| Multiplicação   | *        | 5 * 3   | 15    |
| Divisão         | /        | 7 / 2   | 3.5   |
| Divisão Inteira | //       | 7 // 2  | 3     |
| Potenciação     | **       | 2 ** 9  | 512   |

\sphinxstylestrong{Soma}
\begin{itemize}
\item {} 
\end{itemize}

Menor
\textless{}

Deslocamento para esquerda
\textless{}\textless{}
Subtração
\sphinxhyphen{}

Maior
\textgreater{}

Deslocamento para direita
\textgreater{}\textgreater{}
Multiplicação
*

Menor ou igual
\textless{}=

E bit\sphinxhyphen{}a\sphinxhyphen{}bit (AND)
\&

Divisão
/

Divisão Inteira
//

\begin{sphinxVerbatim}[commandchars=\\\{\}]
\PYG{g+gp}{\PYGZgt{}\PYGZgt{}\PYGZgt{} }\PYG{l+m+mi}{7} \PYG{o}{/} \PYG{l+m+mi}{2}
\PYG{g+go}{3.5}
\PYG{g+gp}{\PYGZgt{}\PYGZgt{}\PYGZgt{} }\PYG{l+m+mi}{7} \PYG{o}{/} \PYG{l+m+mf}{2.0}
\PYG{g+go}{3.5}
\PYG{g+gp}{\PYGZgt{}\PYGZgt{}\PYGZgt{} }\PYG{l+m+mi}{7} \PYG{o}{/}\PYG{o}{/} \PYG{l+m+mf}{2.0}
\PYG{g+go}{3.0}
\PYG{g+gp}{\PYGZgt{}\PYGZgt{}\PYGZgt{} }\PYG{l+m+mi}{7} \PYG{o}{/}\PYG{o}{/} \PYG{l+m+mi}{2}
\PYG{g+go}{3}
\end{sphinxVerbatim}

Maior ou igual
\textgreater{}=

Ou bit\sphinxhyphen{}a\sphinxhyphen{}bit (OR)
|

Igual
==

Ou exclusivo bit\sphinxhyphen{}a\sphinxhyphen{}bit (XOR)
\textasciicircum{}
Módulo
\%

Diferente
!=

Inversão (NOT)
\textasciitilde{}
Potência
**

\# Operadores bit a bit / Python Bitwise

\# Operadores Lógicos OR (|) e AND (\&)
\begin{description}
\item[{‘’’}] \leavevmode
Pipe (|) e ampersand (\&) são operadores lógicos, utilizados respectivamente para as lógicas or e and.

\end{description}

‘’’

\# Operador pipe “|”: Ou Binário / Binary Or
\begin{description}
\item[{‘’’}] \leavevmode
O operador pipe faz a lógica “or” binária, em português “ou”.

Dada a seguinte tabela da verdade em que;

0 = False
1 = True

Observe os resultados e em seguida via statements Python:

\end{description}


\begin{savenotes}\sphinxattablestart
\centering
\begin{tabulary}{\linewidth}[t]{|T|T|T|}
\hline

OR
&
0
&
1
\\
\hline
0
&
0
&
1
\\
\hline
1
&
1
&
1
\\
\hline
\end{tabulary}
\par
\sphinxattableend\end{savenotes}

‘’’

False | False
False

False | True
True

True | False
True

True | True
True

\# 0010 | 0001 = 0011 \sphinxhyphen{}\textgreater{} 2 | 1 = 3

2 | 1
3

‘’’
0010
ou
0001
——\sphinxhyphen{}
0011
‘’’

\# 1010 | 0011 = 1011 \sphinxhyphen{}\textgreater{} 10 | 3 = 11

10 | 3
11

\# Operador ampersand “\&”: E Binário / Binary And
\begin{description}
\item[{‘’’}] \leavevmode
O operador ampersand faz a lógica “and” binária, em português “e”.

Como vimos nos exemplos anteriores, mas agora com a lógica and, observe os resultados e em seguida via statements Python:

\end{description}


\begin{savenotes}\sphinxattablestart
\centering
\begin{tabulary}{\linewidth}[t]{|T|T|T|}
\hline

AND
&
0
&
1
\\
\hline
0
&
0
&
0
\\
\hline
1
&
0
&
1
\\
\hline
\end{tabulary}
\par
\sphinxattableend\end{savenotes}

‘’’

False \& False
False

False \& True
False

True \& False
False

True \& True
True

\# 0010 \& 0001 = 0000 \sphinxhyphen{}\textgreater{} 2 \& 1 = 0

2 \& 1
0

\# 1010 \& 0011 = 0010 \sphinxhyphen{}\textgreater{} 10 \& 3 = 2

10 \& 3
2

\#\# Operador ampersand “\textasciicircum{}”: Ou Exclusivo Binário / Binary XOr

\# Deslocamento de Bits / Bit Shift

False \textgreater{}\textgreater{} False
0

False \textgreater{}\textgreater{} True
0

True \textgreater{}\textgreater{} False
1

True \textgreater{}\textgreater{} True
0


\section{Atribuição de Valores}
\label{\detokenize{content/operators:atribuicao-de-valores}}
\begin{sphinxVerbatim}[commandchars=\\\{\}]
\PYG{c+c1}{\PYGZsh{} Atribuição Simples}

\PYG{n}{foo} \PYG{o}{=} \PYG{l+m+mi}{0}
\PYG{n}{bar} \PYG{o}{=} \PYG{l+s+s1}{\PYGZsq{}}\PYG{l+s+s1}{bla bla bla}\PYG{l+s+s1}{\PYGZsq{}}
\PYG{n+nb}{print}\PYG{p}{(}\PYG{n}{foo}\PYG{p}{)}
\end{sphinxVerbatim}

\begin{sphinxVerbatim}[commandchars=\\\{\}]
\PYG{g+go}{0}
\end{sphinxVerbatim}

\begin{sphinxVerbatim}[commandchars=\\\{\}]
\PYG{n+nb}{print}\PYG{p}{(}\PYG{n}{bar}\PYG{p}{)}
\end{sphinxVerbatim}

\begin{sphinxVerbatim}[commandchars=\\\{\}]
\PYG{g+go}{bla bla bla}
\end{sphinxVerbatim}

\begin{sphinxVerbatim}[commandchars=\\\{\}]
\PYG{c+c1}{\PYGZsh{} Atribuição Composta ou Atribuição por Tupla}
\PYG{n}{x}\PYG{p}{,} \PYG{n}{y}\PYG{p}{,} \PYG{n}{z} \PYG{o}{=} \PYG{p}{(}\PYG{l+m+mi}{1}\PYG{p}{,} \PYG{l+m+mi}{2}\PYG{p}{,} \PYG{l+m+mi}{3}\PYG{p}{)}

\PYG{n+nb}{print}\PYG{p}{(}\PYG{n}{x}\PYG{p}{)}
\end{sphinxVerbatim}

\begin{sphinxVerbatim}[commandchars=\\\{\}]
\PYG{g+go}{1}
\end{sphinxVerbatim}

\begin{sphinxVerbatim}[commandchars=\\\{\}]
\PYG{n+nb}{print}\PYG{p}{(}\PYG{n}{y}\PYG{p}{)}
\end{sphinxVerbatim}

\begin{sphinxVerbatim}[commandchars=\\\{\}]
\PYG{g+go}{2}
\end{sphinxVerbatim}

\begin{sphinxVerbatim}[commandchars=\\\{\}]
\PYG{n+nb}{print}\PYG{p}{(}\PYG{n}{z}\PYG{p}{)}
\end{sphinxVerbatim}

\begin{sphinxVerbatim}[commandchars=\\\{\}]
\PYG{g+go}{3}
\end{sphinxVerbatim}

Invertendo valores:

\begin{sphinxVerbatim}[commandchars=\\\{\}]
\PYG{n}{x} \PYG{o}{=} \PYG{l+m+mi}{10}
\PYG{n}{y} \PYG{o}{=} \PYG{l+m+mi}{20}
\PYG{n}{x}\PYG{p}{,} \PYG{n}{y} \PYG{o}{=} \PYG{n}{y}\PYG{p}{,} \PYG{n}{x}
\PYG{n+nb}{print}\PYG{p}{(}\PYG{n}{x}\PYG{p}{)}
\end{sphinxVerbatim}

\begin{sphinxVerbatim}[commandchars=\\\{\}]
\PYG{g+go}{20}
\end{sphinxVerbatim}

\begin{sphinxVerbatim}[commandchars=\\\{\}]
\PYG{n+nb}{print}\PYG{p}{(}\PYG{n}{y}\PYG{p}{)}
\end{sphinxVerbatim}

\begin{sphinxVerbatim}[commandchars=\\\{\}]
\PYG{g+go}{10}
\end{sphinxVerbatim}


\chapter{Iteratoradores e Geradores}
\label{\detokenize{content/iter_gen:iteratoradores-e-geradores}}\label{\detokenize{content/iter_gen::doc}}

\section{Iterator}
\label{\detokenize{content/iter_gen:iterator}}\begin{quote}

Um iterator (iterador) permite acessar elementos de uma coleção / sequência, retornando cada elemento sequencialmente.
Todo objeto que tiver o método \_\_next\_\_() é um iterator.
\end{quote}

Criação de um iterator com uma string:

\textgreater{} it = iter(‘bar’)

Representação:

\textgreater{} repr(it)

‘\textless{}iterator object at 0x7f123cd62c50\textgreater{}’

Executando o método \_\_next\_\_ até seu fim:

\textgreater{} it.\_\_next\_\_()

‘b’

\textgreater{} it.\_\_next\_\_()

‘a’

\textgreater{} it.\_\_next\_\_()

‘r’

\textgreater{} it.\_\_next\_\_()

StopIteration:
\begin{quote}

Bla bla bla

Nota\sphinxhyphen{}se que a iteração foi feita sobre a string declarada, de forma a retornar caractere por caractere e após o último foi lançada uma exceção indicando que não há mais elementos a serem retornados.
\end{quote}

Classe Iterator
\begin{quote}

É possível também implementar um iterador como um objeto de uma classe personalizada.
É necessário implementar os métodos \_\_iter\_\_ e \_\_next\_\_.
\_\_iter\_\_ retorna o objeto iterador por si.
\_\_next\_\_ retorna o próximo item da coleção e ao alcançar o fim e se houver uma chamada sequente uma exceção é lançada (StopIteration).
\end{quote}

Criação da classe de iterador:

\textgreater{} class FirstNumbers(object):
\begin{quote}
\begin{description}
\item[{def \_\_init\_\_(self, n):}] \leavevmode
self.n = n
self.i = 0

\item[{def \_\_iter\_\_(self):}] \leavevmode
return self

\item[{def \_\_next\_\_(self):}] \leavevmode\begin{description}
\item[{if self.i \textless{}= self.n:}] \leavevmode
cur = self.i
self.i += 1
return cur

\item[{else:}] \leavevmode
raise StopIteration()

\end{description}

\end{description}
\end{quote}

Somatória dos 10 primeiros números:

\textgreater{} print(sum(FirstNumbers(10)))

45

Generator
\begin{quote}

Um generator é um objeto iterável assim como um iterator, mas nem todo iterator é um generator.
Funções de generator permite declarar uma função que se comporta como um iterador, podendo ser usadas em loops.
Um generator implementa o conceito de lazy evaluation, o que faz com que em determinadas situações economize\sphinxhyphen{}se recursos de processamento, pois cada elemento é processado conforme a demanda.
\end{quote}

Criando um objeto range que vai de 0  a 9:

\textgreater{} numeros = range(0, 10)

Se for utilizado list comrprehension será gerada uma lista:

\textgreater{} rq = {[}x ** 2 for x in numeros{]}

Verificando os elementos:

\textgreater{} rq

{[}0, 1, 4, 9, 16, 25, 36, 49, 64, 81{]}

Verificando o tipo:

\textgreater{} type(rq)

list

Tuple comprehension é uma maneira de se criar um generator:

\textgreater{} rq = (x ** 2 for x in numeros)

Verificando o tipo do objeto:

\textgreater{} type(rq)

generator

Executando o método dunder next até o fim dos elementos:

\textgreater{} rq.\_\_next\_\_()

0

\textgreater{} rq.\_\_next\_\_()

1

…

81

\textgreater{} rq.\_\_next\_\_()

StopIteration:
\begin{quote}

Bla bla bla
\end{quote}

Funções Generator
\begin{quote}

Uma função generator utiliza o comando yield em vez de return, o que faz com que retorne o próximo elemento da sequência.
\end{quote}

Criação de uma função generator:

\textgreater{} def gen():
\begin{quote}

i = 0
\begin{description}
\item[{while i \textless{} 10:}] \leavevmode
yield i
i += 1

\end{description}
\end{quote}

Criação do gerador via execução da função:

\textgreater{} x = gen()

Verificando os tipos:

\textgreater{} type(gen)

function

\textgreater{} type(x)

generator

Execução do método \_\_next\_\_ até o fim:

\textgreater{} x.\_\_next\_\_()

0

…

\textgreater{} x.\_\_next\_\_()

9

\textgreater{} x.\_\_next\_\_()

StopIteration:

Iterator vs Generator
\begin{quote}
\begin{itemize}
\item {} 
Para criar um generator utilizamos ou uma função com yield no lugar de return ou tuple comprehension.

\end{itemize}

Para criar um iterador utilizamos a função iter();
\begin{itemize}
\item {} 
Generator utiliza yield, iterator não;

\item {} 
Gerador salva o estado de variáveis locais a cada vez que o yield pausa o loop. Um iterador não faz uso de variáveis locais, tudo o que ele precisa é faz a iteração.

\item {} 
Iteradores fazem uso mais eficiente de memória.

\end{itemize}
\end{quote}

Do módulo timeit importar a função de mesmo nome:

\textgreater{} from timeit import timeit

Verificação de tipos:

\textgreater{} type(iter({[}x for x in range(1, 1001){]}))

list\_iterator

\textgreater{} type((x for x in range(1, 1001)))

generator

Strings com código em loop sobre iterador e gerador, respectivamente:

\textgreater{} code\_it = ‘’’
for i in (iter({[}x for x in range(1, 1001){]})):
\begin{quote}

pass
\end{quote}

‘’’

\textgreater{} code\_gen = ‘’’
for i in ((x for x in range(1, 1001))):
\begin{quote}

pass
\end{quote}

‘’’

Cronometrando os códigos de iterador e gerador, respectivamente:

\textgreater{} timeit(code\_it)

42.666774257901125

\textgreater{} timeit(code\_gen)

53.58039242995437


\chapter{Funções}
\label{\detokenize{content/functions:funcoes}}\label{\detokenize{content/functions::doc}}\begin{quote}

Uma funçao é um recurso de linguagens de programação, que armazena instruções contidas em um bloco de forma a evitar escrever novamente essas mesmas instruções reaproveitando o código ali escrito.
No âmbito (escopo) da função podem ser definidas variáveis que só terão visibilidade dentro da função.
Após definida a função, a mesma é invocada pelo seu nome e seus argumentos (se ela requerir).
Funções em Python são definidas a partir do comando def.
Funções ajudam o código a dividir, agrupar, reusar, reduzir, deixar mais legível além de ser uma boa prática.
\end{quote}


\section{Funções sem Argumentos}
\label{\detokenize{content/functions:funcoes-sem-argumentos}}
Definição da função sem argumentos
\begin{description}
\item[{def funcao():}] \leavevmode
numero = 7 ** 2
msg = ‘O quardrado de 7 é \%d’ \% numero
print(msg)

\end{description}

funcao()
O quardrado de 7 é 49


\section{O Comando return}
\label{\detokenize{content/functions:o-comando-return}}\begin{quote}

O comando return dá um retorno para a função.
Uma função pode retornar um ou mais valores (coleções).
Deve ser o último comando da função, pois se tiver algum código depois será ignorado.
Vale lembrar que return não é a mesma coisa que imprimir em tela. Quando se digita algo que retorne um valor dentro de um shell interativo, esse valor por conveniência é impresso em tela.
Mas em um aplicativo ou script isso não acontecerá, portanto se deseja imprimir um valor em tela, isso deve ser explicitado, o que é mais comumente feito por print().
Quando uma função não tem um comando return, seu retorno é None implícito, o que para outras linguagens como C e Java é a mesma coisa que void.
\end{quote}

Definição da função
\begin{description}
\item[{def funcao():}] \leavevmode
return 7

\end{description}

Utilizando print para imprimir em tela o valor retornado pela função multiplicado por 3

print(funcao() * 3)
21

Definição de uma função que retorna mais de um valor
\begin{description}
\item[{def funcao():}] \leavevmode
return 3, 9

\end{description}

Atribuindo à variável o valor de retorno da função

x = funcao()

Imprimindo o valor da variável

print(x)
(3, 9)

Verificando o tipo da variável

type(x)
tuple

Definição de uma função com código após return
\begin{description}
\item[{def funcao():}] \leavevmode
return 7
print(‘Teste’)

\end{description}

Execução da função

funcao()
7

Como pode\sphinxhyphen{}se notar, o código inserido após return foi completamente ignorado.
Devido ao fato de os comandos serem digitados no shell interativo foi impresso em tela
o valor de retorno da função.


\bigskip\hrule\bigskip


Argumentos Simples (Argumentos Não Nomeados)
\begin{quote}

Uma função pode ter um ou mais argumentos a serem pasados.
Esses argumentos podem ser ou não obrigatórios. Sendo que os argumentos não obrigatórios têm um valor inicial.
\end{quote}

Definição de uma função
\begin{description}
\item[{def funcao(x):}] \leavevmode
return x

\end{description}

Execução da função sem passar argumentos

funcao()

—\sphinxhyphen{}\textgreater{} 1 funcao()

TypeError: funcao() takes exactly 1 argument (0 given)

Devido ao fato de a função exigir que seja passado um argumento

Execução da função passando um argumento

funcao(7)
7


\bigskip\hrule\bigskip


Argumentos Nomeados
\begin{quote}

Podemos definir uma função em que um ou mais argumentos tenham valores padrões de forma que ao invocar a função podemos omitir a declaração, pois será considerado o padrão ou explicitando um valor.
Quando houver mais de um argumento, os argumentos obrigatórios devem vira primeiro.
\end{quote}

Definição de função com um argumento
\begin{description}
\item[{def funcao(x = 7):}] \leavevmode
return x

\end{description}

Chamando a função sem declarar valor de argumento

funcao()
7

Chamando a função explicitando um valor de argumento
funcao(9)
9

funcao(x = 9)
9

Definindo uma função mesclando argumentos padrão e obrigatórios
\begin{description}
\item[{def funcao(x = 7, y):}] \leavevmode
return x + y

\end{description}

SyntaxError: non\sphinxhyphen{}default argument follows default argument

Houve um erro, pois primeiro são os argumentos não padrões.
\begin{description}
\item[{def funcao(x, y = 7):}] \leavevmode
return x + y

\end{description}

funcao(3)
10

funcao(2, 3)
5

funcao()

TypeError: funcao() takes at least 1 argument (0 given)
\begin{description}
\item[{def funcao(x, y = 1, z = 2):}] \leavevmode
return x + y + z

\end{description}

funcao(0)
3

funcao(1, 2)
5

funcao(1, 2, 90)
93

funcao(10, z = 30, y = 50)
90


\bigskip\hrule\bigskip


Argumentos em Lista Não Nomeados
\begin{quote}

É possível passar uma lista de argumentos sem nomear cada um deles, ou seja, atribuir uma variável.
Essa lista, internamente é interpretada como uma tupla (tuple).
Tal recurso nos possibilita passar uma quantidade indeterminada de argumentos.
O identificador da variável que representa esse tipo de argumento vem logo depois do caractere asterisco (*).
\end{quote}
\begin{description}
\item[{def funcao({\color{red}\bfseries{}*}args):}] \leavevmode
qtd = len(args)
primeiro = args{[}0{]}
ultimo = args{[}\sphinxhyphen{}1{]}
print(‘Foram passados \%d argumentos’ \% qtd)
print(‘O primeiro é “\%s”’ \% primeiro)
print(‘O último é “\%s”’ \% ultimo)
print(‘Os argumentos passados foram: \%s’ \% str(args))

\end{description}

funcao(‘abacaxi’, 3, ‘p’, 8.3, 5 + 9j)
Foram passados 5 argumentos
O primeiro é “abacaxi”
O último é “(5+9j)”
Os argumentos passados foram: (‘abacaxi’, 3, ‘p’, 8.3, (5+9j))
\begin{description}
\item[{def funcao({\color{red}\bfseries{}*}args):}] \leavevmode\begin{description}
\item[{for arg in args:}] \leavevmode
print(‘Argumento \%d = \%s’ \% (args.index(arg), arg))

\end{description}

\end{description}

ou
\begin{description}
\item[{def funcao({\color{red}\bfseries{}*}args):}] \leavevmode\begin{description}
\item[{for i, arg in enumerate(args):}] \leavevmode
print(‘Argumento \%d = \%s’ \% (i, arg))

\end{description}

\end{description}

funcao(‘a’, 1.5, 7, 99)

Argumento 0 = a
Argumento 1 = 1.5
Argumento 2 = 7
Argumento 3 = 99
\begin{description}
\item[{def funcao(x, {\color{red}\bfseries{}*}args):}] \leavevmode
return args

\end{description}

spam = (1, 2, 3, 4)

funcao(spam)
()

funcao({\color{red}\bfseries{}*}spam)
(2, 3, 4)
\begin{description}
\item[{def funcao({\color{red}\bfseries{}*}args):}] \leavevmode
return args

\end{description}

funcao(spam)
((1, 2, 3, 4),)

funcao({\color{red}\bfseries{}*}spam)
(1, 2, 3, 4)

Quando o caractere asterisco é posicionado antes de uma variável faz com que considere que aquela variável (coleções) seja
“desempacotada”. Seus elementos são passados como se fossem uma tupla, ou seja, uma sequência de valores estraídos separados
por vírgulas.


\bigskip\hrule\bigskip


Argumentos em Lista Nomeados
\begin{quote}

O identificador da variável desse tipo de argumento é precedido por dois asteriscos (**).
É uma lista com quantidade indeterminada e cada elemento da lista tem um identificador próprio.
\end{quote}
\begin{description}
\item[{def funcao({\color{red}\bfseries{}**}kargs):}] \leavevmode
return kargs

\end{description}

funcao(a = 1, b = 2)
\{‘a’: 1, ‘b’: 2\}
\begin{description}
\item[{def funcao({\color{red}\bfseries{}**}kargs):}] \leavevmode\begin{description}
\item[{for k, v in kargs.items():}] \leavevmode
print(‘\%s = \%s’ \% (k.capitalize(), v))

\end{description}

\item[{funcao(}] \leavevmode\begin{quote}

nome = ‘Chiquinho’,
sobrenome = ‘da Silva’,
idade = 30,
telefone = ‘(11) 99999\sphinxhyphen{}9999’,
\end{quote}

)

\end{description}

Idade = 30
Sobrenome = da Silva
Telefone = (11) 99999\sphinxhyphen{}9999
Nome = Chiquinho
\begin{description}
\item[{def funcao({\color{red}\bfseries{}**}kargs):}] \leavevmode
return kargs

\end{description}

eggs = \{‘a’: 3, ‘b’: 5, ‘c’: ‘x’\}

funcao(eggs)

TypeError: funcao() takes exactly 0 arguments (1 given)

funcao({\color{red}\bfseries{}**}eggs)
\{‘a’: 3, ‘b’: 5, ‘c’: ‘x’\}


\bigskip\hrule\bigskip


Funções com Argumentos Variados
\begin{quote}

E se precisarmos fazer uma função que utilize tipos diferentes conforme visto anteriormente?
A ordem dos tipos de argumentos é a seguinte:

Simples, Nomeados, Lista de Não Nomeados e Lista de Nomeados
\end{quote}
\begin{description}
\item[{def foo(a, b = 3, {\color{red}\bfseries{}*}c, {\color{red}\bfseries{}**}d):}] \leavevmode
print(a + b)
print(c)
print(d)

\end{description}

foo(4, 5, ‘Alemanha’, ‘Holanda’, ‘Inglaterra’, continente = ‘Europa’, hemisferio = ‘Norte’)
9
(‘Alemanha’, ‘Holanda’, ‘Inglaterra’)
\{‘continente’: ‘Europa’, ‘hemisferio’: ‘Norte’\}


\bigskip\hrule\bigskip


Estruturas de Dados como Parâmetro para Funções
\begin{quote}

Em algumas situações pode ser útil utilizar uma estrutura de dados como tupla (tuple), lista (list), dicionário (dict) ou mesmo um conjunto (set / frozenset).
\end{quote}

Criação da função de teste:
\begin{description}
\item[{\textgreater{} def param\_test(x, y):}] \leavevmode
return x + y

\end{description}

Declaração das variáveis de estrutura de dados que serão utilizadas como parâmetro para a função:

\textgreater{} tupla = (5, 2)

\textgreater{} lista = {[}5, 2{]}

\textgreater{} dicio = \{‘x’: 5, ‘y’: 2\}

\textgreater{} conjunto = \{2, 5, 2\}

Testes utilizando as estruturas de dados criadas:

\textgreater{} param\_test({\color{red}\bfseries{}**}dicio)  \# Dicionário (dict) como parâmetro

\textgreater{} param\_test({\color{red}\bfseries{}*}tupla)  \# Tupla (tuple) como parâmetro

\textgreater{} param\_test({\color{red}\bfseries{}*}lista)  \# Lista (list) como parâmetro

\textgreater{} param\_test({\color{red}\bfseries{}*}conjunto)  \# Conjunto (set) como parâmetro

7


\bigskip\hrule\bigskip


Boas Práticas: Função Main
\begin{quote}

Evite execuções globais, quebre seu código em funções o que facilita o reúso e teste de código.
Crie uma função principal (main). Crie primeiro as outras funções e por último a ser definida a função principal.
Na função principal serão feitas as chamadas às outras funções.
Outra coisa interessante a ser feita é colocar a função principal dentro de um if. Sendo que se for executado, terá a variável especial “\_\_name\_\_” como valor “\_\_main\_\_”.
\end{quote}

vim hello.py

\#!/usr/bin/env python
\#\_*\_ coding: utf\sphinxhyphen{}8 \_*\_
\begin{description}
\item[{def funcao():}] \leavevmode
print(‘Função executada’)

\item[{def Main():}] \leavevmode
print(‘==== Início ====’)
funcao()
print(‘==== Fim ====’)

\item[{if \_\_name\_\_ == ‘\_\_main\_\_’:}] \leavevmode
Main()

\end{description}

python hello.py

==== Início ====
Função executada
==== Fim ====


\bigskip\hrule\bigskip


Funções Geradoras
\begin{quote}

Uma função geradora ao invés de utilizar o comando return, utiliza o comando yield, que retorna um objeto generator.
\end{quote}
\begin{description}
\item[{def f\_gen(var):}] \leavevmode
print(‘INÍCIO’)
\begin{description}
\item[{for i in var:}] \leavevmode
yield i

\end{description}

print(‘FIM’)

\end{description}

g = f\_gen(‘Python’)

type(g)
generator

g.next()
INÍCIO
‘P’

g.next()
‘y’

g.next()
‘t’

g.next()
‘h’

g.next()
‘o’

g.next()
‘n’

g.next()
FIM

StopIteration:


\bigskip\hrule\bigskip


Funções Lambda
\begin{quote}

São funções anônimas, ou seja, que não são associadas a um nome. Um recurso similar às funções anônimas em PL/pgSQL (PostgreSQL) e PL/SQL (Oracle).
Sua estrutura é composta apenas por expressões, o que a torna muito limitada, no entanto consome menos recursos do que uma função convencional.
Por só aceitar expressões, o comando return não é permitido em sua estrutura.
\end{quote}

(lambda x, y: x + y)(5, 2)
7

foo = lambda x, y: x ** y

print(foo(2, 5))
32


\bigskip\hrule\bigskip



\chapter{Orientação a Objetos}
\label{\detokenize{content/oo:orientacao-a-objetos}}\label{\detokenize{content/oo::doc}}\begin{quote}

Python suporta herança múltipla;
Tudo em Python é objeto (inclusive classes);
O método construtor é o método inicializador, o método mágico \_\_init\_\_;
\end{quote}
\begin{description}
\item[{class NomeClasse(ClasseMae):}] \leavevmode
\# Método construtor ou método inicializador (sem parâmetro)
def \_\_init\_\_(self):
\begin{quote}

pass
\end{quote}

\# Método comum com 1 (um) parâmetro
def metodo(self, foo):
\begin{quote}

return foo

Em Python a definição de uma classe é bem simples. Usamos o comando class seguindo do nome da classe a ser criada, logo em seguida vindo entre parênteses os nomes das classes mãe separados por vírgulas, pois em Python é permitida herança múltipla.
O método construtor, também chamado de método inicializador é o \_\_init\_\_.
\end{quote}

\item[{class Foo(object):}] \leavevmode
nome = ‘’

\item[{class Bar(object):}] \leavevmode
idade = 0
\begin{description}
\item[{def metodo\_teste(self):}] \leavevmode
print(‘Teste’)

\end{description}

\end{description}

class Baz(Foo, Bar):
\begin{quote}
\begin{description}
\item[{def \_\_init\_\_(self, nome):}] \leavevmode
self.nome = nome

\end{description}

altura = 0.0
\end{quote}

o = Baz(‘Chiquinho da Silva’)

o.idade = 51

o.altura = 1.60

print(‘\%s tem \%s anos e \%.2fm de altura’ \% (o.nome, o.idade, o.altura))

Chiquinho da Silva tem 51 anos e 1.60m de altura

Obs.: Em Python todos métodos e atributos são públicos.
Há uma convenção que colocando um unerline como primeiro caractere no nome de um atributo ou um método o sinaliza como privado.
Porém, isso é apenas uma convenção. Nada impede que sejam acessados externamente.
Para que haja um bloqueio efetivo há um recurso na linguagem chamado “property”.
\begin{description}
\item[{class Carro(object):}] \leavevmode
motor\_ligado = False
\begin{description}
\item[{def \_\_init\_\_(self, marca, modelo):}] \leavevmode
self.marca = marca
self.modelo = modelo

\item[{def ignicao(self):}] \leavevmode\begin{description}
\item[{if (self.motor\_ligado):}] \leavevmode
self.motor\_ligado = False
print(‘Motor desligado!’)

\item[{else:}] \leavevmode
self.motor\_ligado = True
print(‘Motor ligado!’)

\end{description}

\end{description}

\end{description}

c1 = Carro()

TypeError                                 Traceback (most recent call last)
\textless{}ipython\sphinxhyphen{}input\sphinxhyphen{}13\sphinxhyphen{}e2526cbd1648\textgreater{} in \textless{}module\textgreater{}()
\begin{quote}

15
16
\end{quote}

—\textgreater{} 17 c1 = Carro()

TypeError: \_\_init\_\_() takes exactly 3 arguments (1 given)

c1 = Carro(‘Fiat’, ‘147’)

c1.ignicao()

Motor ligado!

c1.ignicao()

Motor desligado!

print(‘Marca: \%snModelo: \%s’ \% (c1.marca, c1.modelo))

Marca: Fiat
Modelo: 147

Método \_\_str\_\_

print(c1)

\textless{}\_\_main\_\_.Carro object at 0x7f1f6313eed0\textgreater{}

repr(c1)

‘\textless{}\_\_main\_\_.Carro object at 0x7f1f6313eed0\textgreater{}’
\begin{description}
\item[{class Carro(object):}] \leavevmode
motor\_ligado = False
\begin{description}
\item[{def \_\_init\_\_(self, marca, modelo):}] \leavevmode
self.marca = marca
self.modelo = modelo

\item[{def \_\_str\_\_(self):}] \leavevmode
return ‘\%s \sphinxhyphen{} \%s’ \% (self.marca, self.modelo)

\item[{def ignicao(self):}] \leavevmode\begin{description}
\item[{if (self.motor\_ligado):}] \leavevmode
self.motor\_ligado = False
print(‘Motor desligado!’)

\item[{else:}] \leavevmode
self.motor\_ligado = True
print(‘Motor ligado!’)

\end{description}

\end{description}

\end{description}

c1 = Carro(‘Fiat’, ‘147’)

print(c1)

Fiat \sphinxhyphen{} 147

repr(c1)
Out{[}32{]}: ‘\textless{}\_\_main\_\_.Carro object at 0x7f1f631273d0\textgreater{}’

Método Definido Externamente à Classe
\begin{description}
\item[{def metodo\_externo(self, frase, numero):}] \leavevmode
self.numero = numero
print(frase)

\item[{class MinhaClasse(object):}] \leavevmode
pass

\end{description}

o = MinhaClasse()

MinhaClasse.metodo = metodo\_externo

o.metodo(‘Bla bla bla’, 800)
Bla bla bla

print(o.numero)
800

Método Definido Externamente ao Objeto
\begin{description}
\item[{def metodo\_objeto(self):}] \leavevmode
return ‘X’

\end{description}

o.metodo\_x = metodo\_objeto


\section{o.metodo\_x()}
\label{\detokenize{content/oo:o-metodo-x}}
TypeError                                 Traceback (most recent call last)
\textless{}ipython\sphinxhyphen{}input\sphinxhyphen{}41\sphinxhyphen{}2f98daa957c2\textgreater{} in \textless{}module\textgreater{}()
—\sphinxhyphen{}\textgreater{} 1 o.metodo\_x()

TypeError: metodo\_objeto() takes exactly 1 argument (0 given)

o.metodo\_x(o)
‘X’


\section{Objetos com Atributos Dinâmicos}
\label{\detokenize{content/oo:objetos-com-atributos-dinamicos}}
Criação da classe Carro:
\begin{description}
\item[{class Carro(object):}] \leavevmode
marca = ‘’
modelo = ‘’

\end{description}

Criação de um objeto da classe Carro:

c1 = Carro()

Vejamos agora o dicionário de atributos com seus respectivos valores:

print(c1.\_\_dict\_\_)

\{\}

O atributo especial \_\_dict\_\_, em um objeto, é um dicionário que é usado para guardar atributos e seus respectivos valores.
O dicionário em questão apresentou um conjunto vazio.

Agora vamos preencher os atributos:

c1.marca = ‘Porsche’
c1.modelo = ‘911’

Consulta ao dicionário do objeto novamente:

print(c1.\_\_dict\_\_)

\{‘modelo’: ‘911’, ‘marca’: ‘Porsche’\}

Com os atributos preenchidos com valores agora o dicionário não está mais vazio.
Python é tão flexível que nos permite até criar um atributo “on the fly”:

c1.ano = 1993

print(c1.\_\_dict\_\_)

\{‘ano’: 1993, ‘modelo’: ‘911’, ‘marca’: ‘Porsche’\}

E que tal se pudermos no momento da criação do objeto, além de poder atribuir valores
aos atributos existentes, também criar atributos que não existem na classe?

Criação da classe Carro agora utilizando o método construtor (\_\_init\_\_()), o qual fará
o trabalho de associar ao objeto instanciado cada par chave / valor declarado:
\begin{description}
\item[{class Carro(object):}] \leavevmode
marca = ‘’
modelo = ‘’

\# Metodo construtor
def \_\_init\_\_(self, {\color{red}\bfseries{}**}kargs):
\begin{quote}
\begin{description}
\item[{for chave,valor in kargs.items():}] \leavevmode
self.\_\_dict\_\_{[}chave{]} = valor

\end{description}
\end{quote}

\end{description}

Criação do objeto com atributos dinâmicos;

c1 = Carro(marca = ‘Porsche’, modelo = ‘911’, cor = ‘verde’, ano = 1991)

Verificando o dicionário do objeto:

print(c1.\_\_dict\_\_)

\{‘ano’: 1991, ‘modelo’: ‘911’, ‘marca’: ‘Porsche’, ‘cor’: ‘verde’\}

O Método super()
\begin{description}
\item[{class Mae(object):}] \leavevmode\begin{description}
\item[{def metodo(self):}] \leavevmode
print(‘Método da classe Mae’)

\end{description}

\item[{class Filha(Mae):}] \leavevmode\begin{description}
\item[{def metodo(self):}] \leavevmode
super().metodo() \# Chamando o método da classe mãe
print(‘Método da classe Filha’)

\end{description}

\end{description}

o = Filha()

o.metodo()


\chapter{Método de Classe e Método Estático}
\label{\detokenize{content/staticmethod_classmethod:metodo-de-classe-e-metodo-estatico}}\label{\detokenize{content/staticmethod_classmethod::doc}}\begin{quote}

Existem dois tipos de métodos que podem ser executados diretamente de uma classe, ou seja, sem precisar criar um objeto dela. Justamente por serem independentes de objetos não esperam um “self” como primeiro parâmetro.
A utilidade desses métodos é para processamento de dados de classes em vez de instâncias.
Mesmo que isso possa ser feito por uma função escrita externamente à classe, não terá uma associação a essa classe e não poderão ser herdados por classes filhas.
Métodos estáticos são usados para agrupar funções que têm conexão lógica com a classe.
Métodos de classe são usados para o mesmo fim que os métodos estáticos, mas que também podem processar dados da classe diretamente.
\end{quote}

Prática:
\begin{quote}

Observe no exemplo a seguir as diferentes assinaturas para respectivamente os métodos comum, de classe e estático:
\end{quote}

class Foo(object):
\begin{quote}
\begin{description}
\item[{def metodo\_comum(self):}] \leavevmode
print(‘Método comum \{\}’.format(self))

\end{description}

@classmethod
def metodo\_de\_classe(cls):
\begin{quote}

print(‘Método de classe \{\}’.format(cls))
\end{quote}

@staticmethod
def metodo\_estatico():
\begin{quote}

print(‘Método estático’)

Na classe foram declarados três métodos, sendo que o primeiro que é um método comum, cujo primeiro argumento é o tradicional self que representa a instância (objeto) criada a partir da classe.
O segundo método é decorado com @classmethod, faz desse método um método de classe, o qual não faz necessária a criação de um objeto para ser utilizado. Esse tipo de método é invocado da seguinte forma: Classe.metodo\_de\_classe(). Ainda sobre o método de classe um detalhe interessante é o “cls” ao invés de self. A variável “cls” é utilizada com um propósito similar ao de self, mas ao invés de representar uma instância criada, representa a classe à qual o método pertence. Assim como self, cls é apenas uma convenção, deixando ao gosto do usuário, se o mesmo desejar utilizar outro nome.
E por fim o método estático, decorado com @staticmethod é como uma função definida externamente à classe. Não recebe um “self” ou um “cls”.
\end{quote}
\end{quote}

Chamada do método comum pela classe

Foo.metodo\_comum()

TypeError: unbound method metodo\_comum() must be called with Foo instance as first argument (got nothing instead)

Aqui podemos ver que se for feita uma tentativa de invocar a partir da classe um método que não seja nem de classe e nem estático será retornado um erro.

Invocando um método de classe a partir da classe

Foo.metodo\_de\_classe()
Método de classe \textless{}class ‘\_\_main\_\_.Foo’\textgreater{}

Invocando um método estático a partir da classe

Foo.metodo\_estatico()
Método estático

Criação de objeto:

o = Foo()

A instância “o” é implicitamente passada como argumento para o método construtor que não foi declarado.

Chamada do método comum pela instância

o.metodo\_comum()
Método comum \textless{}\_\_main\_\_.Foo object at 0x7f40d812d410\textgreater{}

Chamada do método de classe pela instância

o.metodo\_de\_classe()
Método de classe \textless{}class ‘\_\_main\_\_.Foo’\textgreater{}

Chamada do método estático pela instância

o.metodo\_estatico()
Método estático


\chapter{Encapsulamento}
\label{\detokenize{content/property:encapsulamento}}\label{\detokenize{content/property::doc}}

\section{Modificador Private (\_\_)}
\label{\detokenize{content/property:modificador-private}}
\begin{DUlineblock}{0em}
\item[]
\begin{DUlineblock}{\DUlineblockindent}
\item[] Colocando 2 (dois) caracteres underscore antecedendo o atributo, ele fica
\end{DUlineblock}
\item[] privado, ou seja, não é acessível fora da classe.
\end{DUlineblock}

\begin{sphinxVerbatim}[commandchars=\\\{\}]
\PYG{c+c1}{\PYGZsh{} Criação de classe de teste:}
\PYG{k}{class} \PYG{n+nc}{Foo}\PYG{p}{(}\PYG{n+nb}{object}\PYG{p}{)}\PYG{p}{:}
    \PYG{n}{\PYGZus{}\PYGZus{}atributo} \PYG{o}{=} \PYG{l+m+mi}{0}

\PYG{c+c1}{\PYGZsh{} Instância da classe:}
\PYG{n}{f} \PYG{o}{=} \PYG{n}{Foo}\PYG{p}{(}\PYG{p}{)}

\PYG{c+c1}{\PYGZsh{} Tentativa de acesso a atributo privado:}
\PYG{n}{f}\PYG{o}{.}\PYG{n}{\PYGZus{}\PYGZus{}atributo}
\end{sphinxVerbatim}

\begin{sphinxVerbatim}[commandchars=\\\{\}]
\PYG{g+go}{. . .}
\PYG{g+go}{AttributeError: \PYGZsq{}Foo\PYGZsq{} object has no attribute \PYGZsq{}\PYGZus{}\PYGZus{}atributo\PYGZsq{}}
\end{sphinxVerbatim}


\section{Property}
\label{\detokenize{content/property:property}}
\begin{DUlineblock}{0em}
\item[]
\begin{DUlineblock}{\DUlineblockindent}
\item[] Property é a solução pythônica para implementar getters e setters de forma
\end{DUlineblock}
\item[] inteligente e podendo inclusive impor restrições.
\end{DUlineblock}

\begin{sphinxVerbatim}[commandchars=\\\{\}]
\PYG{c+c1}{\PYGZsh{} Classe com property:}
\PYG{k}{class} \PYG{n+nc}{Carro}\PYG{p}{(}\PYG{n+nb}{object}\PYG{p}{)}\PYG{p}{:}
    \PYG{k}{def} \PYG{n+nf+fm}{\PYGZus{}\PYGZus{}init\PYGZus{}\PYGZus{}}\PYG{p}{(}\PYG{n+nb+bp}{self}\PYG{p}{)}\PYG{p}{:}
        \PYG{n+nb+bp}{self}\PYG{o}{.}\PYG{n}{\PYGZus{}\PYGZus{}velocidade} \PYG{o}{=} \PYG{l+m+mi}{0}

    \PYG{k}{def} \PYG{n+nf}{\PYGZus{}get\PYGZus{}\PYGZus{}velocidade}\PYG{p}{(}\PYG{n+nb+bp}{self}\PYG{p}{)}\PYG{p}{:}
        \PYG{n+nb}{print}\PYG{p}{(}\PYG{l+s+s1}{\PYGZsq{}}\PYG{l+s+s1}{Velocidade: }\PYG{l+s+si}{\PYGZpc{}d}\PYG{l+s+s1}{ km/h}\PYG{l+s+s1}{\PYGZsq{}} \PYG{o}{\PYGZpc{}} \PYG{n+nb+bp}{self}\PYG{o}{.}\PYG{n}{\PYGZus{}\PYGZus{}velocidade}\PYG{p}{)}
        \PYG{k}{return} \PYG{n+nb+bp}{self}\PYG{o}{.}\PYG{n}{\PYGZus{}\PYGZus{}velocidade}

    \PYG{k}{def} \PYG{n+nf}{\PYGZus{}set\PYGZus{}\PYGZus{}velocidade}\PYG{p}{(}\PYG{n+nb+bp}{self}\PYG{p}{,} \PYG{n}{velocidade}\PYG{p}{)}\PYG{p}{:}
        \PYG{k}{if} \PYG{n}{velocidade} \PYG{o}{\PYGZgt{}} \PYG{l+m+mi}{300}\PYG{p}{:}
            \PYG{k}{raise} \PYG{n+ne}{ValueError}\PYG{p}{(}\PYG{l+s+s1}{\PYGZsq{}}\PYG{l+s+s1}{A velocidade máxima permitida é de 300 km/h}\PYG{l+s+s1}{\PYGZsq{}}\PYG{p}{)}
        \PYG{n+nb+bp}{self}\PYG{o}{.}\PYG{n}{\PYGZus{}\PYGZus{}velocidade} \PYG{o}{=} \PYG{n}{velocidade}
        \PYG{n+nb}{print}\PYG{p}{(}\PYG{l+s+s1}{\PYGZsq{}}\PYG{l+s+s1}{Velocidade = }\PYG{l+s+si}{\PYGZpc{}d}\PYG{l+s+s1}{ km/h}\PYG{l+s+s1}{\PYGZsq{}} \PYG{o}{\PYGZpc{}} \PYG{n+nb+bp}{self}\PYG{o}{.}\PYG{n}{\PYGZus{}\PYGZus{}velocidade}\PYG{p}{)}

    \PYG{k}{def} \PYG{n+nf}{\PYGZus{}del\PYGZus{}\PYGZus{}velocidade}\PYG{p}{(}\PYG{n+nb+bp}{self}\PYG{p}{)}\PYG{p}{:}
        \PYG{n+nb}{print}\PYG{p}{(}\PYG{l+s+s1}{\PYGZsq{}}\PYG{l+s+s1}{Removendo a propriedade de velocidade}\PYG{l+s+s1}{\PYGZsq{}}\PYG{p}{)}
        \PYG{k}{del} \PYG{n+nb+bp}{self}\PYG{o}{.}\PYG{n}{\PYGZus{}\PYGZus{}velocidade}

    \PYG{c+c1}{\PYGZsh{} Definição da property velocidade}
    \PYG{n}{velocidade} \PYG{o}{=} \PYG{n+nb}{property}\PYG{p}{(}\PYG{n}{\PYGZus{}get\PYGZus{}\PYGZus{}velocidade}\PYG{p}{,} \PYG{n}{\PYGZus{}set\PYGZus{}\PYGZus{}velocidade}\PYG{p}{,} \PYG{n}{\PYGZus{}del\PYGZus{}\PYGZus{}velocidade}\PYG{p}{,}
                          \PYG{l+s+s1}{\PYGZsq{}}\PYG{l+s+s1}{Velocidade máxima do carro}\PYG{l+s+s1}{\PYGZsq{}}\PYG{p}{)}


\PYG{c+c1}{\PYGZsh{} Instância da classe:}
\PYG{n}{c} \PYG{o}{=} \PYG{n}{Carro}\PYG{p}{(}\PYG{p}{)}

\PYG{c+c1}{\PYGZsh{} Tentativa de acesso ao atributo privado:}
\PYG{n}{c}\PYG{o}{.}\PYG{n}{\PYGZus{}\PYGZus{}velocidade}
\end{sphinxVerbatim}

\begin{sphinxVerbatim}[commandchars=\\\{\}]
\PYG{g+go}{.  .  .}
\PYG{g+go}{AttributeError: \PYGZsq{}Carro\PYGZsq{} object has no attribute \PYGZsq{}\PYGZus{}\PYGZus{}velocidade\PYGZsq{}}
\end{sphinxVerbatim}

\begin{sphinxVerbatim}[commandchars=\\\{\}]
\PYG{c+c1}{\PYGZsh{} Acessando a property \PYGZdq{}velocidade\PYGZdq{}:}
\PYG{n}{c}\PYG{o}{.}\PYG{n}{velocidade}
\end{sphinxVerbatim}

\begin{sphinxVerbatim}[commandchars=\\\{\}]
\PYG{g+go}{Velocidade: 0 km/h}
\PYG{g+go}{0}
\end{sphinxVerbatim}

\begin{sphinxVerbatim}[commandchars=\\\{\}]
\PYG{c+c1}{\PYGZsh{} Atribuindo um valor para a property:}
\PYG{n}{c}\PYG{o}{.}\PYG{n}{velocidade} \PYG{o}{=} \PYG{l+m+mi}{200}
\end{sphinxVerbatim}

\begin{sphinxVerbatim}[commandchars=\\\{\}]
\PYG{g+go}{Velocidade = 200 km/h}
\end{sphinxVerbatim}

\begin{sphinxVerbatim}[commandchars=\\\{\}]
\PYG{c+c1}{\PYGZsh{} Tentativa de atribuir um valor não permitido;}
\PYG{n}{c}\PYG{o}{.}\PYG{n}{velocidade} \PYG{o}{=} \PYG{l+m+mi}{301}
\end{sphinxVerbatim}

\begin{sphinxVerbatim}[commandchars=\\\{\}]
\PYG{o}{.} \PYG{o}{.} \PYG{o}{.}
\PYG{n+ne}{ValueError}\PYG{p}{:} \PYG{n}{A} \PYG{n}{velocidade} \PYG{n}{máxima} \PYG{n}{permitida} \PYG{n}{é} \PYG{n}{de} \PYG{l+m+mi}{300} \PYG{n}{km}\PYG{o}{/}\PYG{n}{h}
\end{sphinxVerbatim}

\begin{sphinxVerbatim}[commandchars=\\\{\}]
\PYG{c+c1}{\PYGZsh{} Remover a property:}
\PYG{k}{del} \PYG{n}{c}\PYG{o}{.}\PYG{n}{velocidade}
\end{sphinxVerbatim}

\begin{sphinxVerbatim}[commandchars=\\\{\}]
\PYG{g+go}{Removendo a propriedade de velocidade}
\end{sphinxVerbatim}

\begin{sphinxVerbatim}[commandchars=\\\{\}]
\PYG{c+c1}{\PYGZsh{} Tentativa de acesso à property apagada:}
\PYG{n}{c}\PYG{o}{.}\PYG{n}{velocidade}
\end{sphinxVerbatim}

\begin{sphinxVerbatim}[commandchars=\\\{\}]
\PYG{g+go}{. . .}
\PYG{g+go}{AttributeError: \PYGZsq{}Carro\PYGZsq{} object has no attribute \PYGZsq{}\PYGZus{}\PYGZus{}velocidade\PYGZsq{}}
\end{sphinxVerbatim}


\section{Property como Decorator}
\label{\detokenize{content/property:property-como-decorator}}
\begin{DUlineblock}{0em}
\item[]
\begin{DUlineblock}{\DUlineblockindent}
\item[] Além da já citada implementação de property, pode\sphinxhyphen{}se também fazer isso
\end{DUlineblock}
\item[] por meio de decorators.
\end{DUlineblock}

\begin{sphinxVerbatim}[commandchars=\\\{\}]
\PYG{c+c1}{\PYGZsh{} Criação de classe com definição de properties via decorators:}
\PYG{k}{class} \PYG{n+nc}{Carro}\PYG{p}{(}\PYG{n+nb}{object}\PYG{p}{)}\PYG{p}{:}
    \PYG{k}{def} \PYG{n+nf+fm}{\PYGZus{}\PYGZus{}init\PYGZus{}\PYGZus{}}\PYG{p}{(}\PYG{n+nb+bp}{self}\PYG{p}{)}\PYG{p}{:}
        \PYG{n+nb+bp}{self}\PYG{o}{.}\PYG{n}{\PYGZus{}\PYGZus{}velocidade} \PYG{o}{=} \PYG{l+m+mi}{0}

    \PYG{n+nd}{@property}
    \PYG{k}{def} \PYG{n+nf}{velocidade}\PYG{p}{(}\PYG{n+nb+bp}{self}\PYG{p}{)}\PYG{p}{:}
        \PYG{l+s+sd}{\PYGZsq{}\PYGZsq{}\PYGZsq{}Velocidade máxima do carro\PYGZsq{}\PYGZsq{}\PYGZsq{}}
        \PYG{n+nb}{print}\PYG{p}{(}\PYG{l+s+s1}{\PYGZsq{}}\PYG{l+s+s1}{Velocidade: }\PYG{l+s+si}{\PYGZob{}\PYGZcb{}}\PYG{l+s+s1}{ km/h}\PYG{l+s+s1}{\PYGZsq{}}\PYG{o}{.}\PYG{n}{format}\PYG{p}{(}\PYG{n+nb+bp}{self}\PYG{o}{.}\PYG{n}{\PYGZus{}\PYGZus{}velocidade}\PYG{p}{)}\PYG{p}{)}
        \PYG{k}{return} \PYG{n+nb+bp}{self}\PYG{o}{.}\PYG{n}{\PYGZus{}\PYGZus{}velocidade}

    \PYG{n+nd}{@velocidade}\PYG{o}{.}\PYG{n}{setter}
    \PYG{k}{def} \PYG{n+nf}{velocidade}\PYG{p}{(}\PYG{n+nb+bp}{self}\PYG{p}{,} \PYG{n}{velocidade}\PYG{p}{)}\PYG{p}{:}
        \PYG{k}{if} \PYG{n}{velocidade} \PYG{o}{\PYGZgt{}} \PYG{l+m+mi}{300}\PYG{p}{:}
            \PYG{k}{raise} \PYG{n+ne}{ValueError}\PYG{p}{(}\PYG{l+s+s1}{\PYGZsq{}}\PYG{l+s+s1}{A velocidade máxima permitida é de 300 km/h}\PYG{l+s+s1}{\PYGZsq{}}\PYG{p}{)}
        \PYG{n+nb+bp}{self}\PYG{o}{.}\PYG{n}{\PYGZus{}\PYGZus{}velocidade} \PYG{o}{=} \PYG{n}{velocidade}
        \PYG{n+nb}{print}\PYG{p}{(}\PYG{l+s+s1}{\PYGZsq{}}\PYG{l+s+s1}{Velocidade = }\PYG{l+s+si}{\PYGZob{}\PYGZcb{}}\PYG{l+s+s1}{ km/h}\PYG{l+s+s1}{\PYGZsq{}}\PYG{o}{.}\PYG{n}{format}\PYG{p}{(}\PYG{n+nb+bp}{self}\PYG{o}{.}\PYG{n}{\PYGZus{}\PYGZus{}velocidade}\PYG{p}{)}\PYG{p}{)}

    \PYG{n+nd}{@velocidade}\PYG{o}{.}\PYG{n}{deleter}
    \PYG{k}{def} \PYG{n+nf}{velocidade}\PYG{p}{(}\PYG{n+nb+bp}{self}\PYG{p}{)}\PYG{p}{:}
        \PYG{n+nb}{print}\PYG{p}{(}\PYG{l+s+s1}{\PYGZsq{}}\PYG{l+s+s1}{Removendo a propriedade de velocidade}\PYG{l+s+s1}{\PYGZsq{}}\PYG{p}{)}
        \PYG{k}{del} \PYG{n+nb+bp}{self}\PYG{o}{.}\PYG{n}{\PYGZus{}\PYGZus{}velocidade}
\end{sphinxVerbatim}


\chapter{Escopo}
\label{\detokenize{content/scope:escopo}}\label{\detokenize{content/scope::doc}}\begin{quote}
\begin{quote}

Por escopo entende\sphinxhyphen{}se o contexto onde um identificador será considerado.
Um identificador pode ser uma variável, um comando, uma função uma clase ou um objeto.
A ordem de busca de um identificador em escopos é a seguinte:
\end{quote}

1º) Local ou dentro de função;
2º) Função externa;
3º) Global ou de módulo;
4º) \_\_builtins\_\_
\end{quote}
\begin{itemize}
\item {} 
Escopo Local ou Dentro de Função
\begin{quote}

É o escopo que tem a maior prioridade.
\end{quote}

\end{itemize}

Prática:

Declaração de uma variável com seu valor

foo = 7

Exibindo na tela o valor da variável

print(foo)
7

A função locals() retorna um dicionário com os identificadores locais e seus respectivos valores.
Então é pedido o valor dada a chave que é o nome da variável local.

locals(){[}‘foo’{]}
7

Dentro de função

Declaração de uma variável externamente ao escopo da função

foo = 7

Definição de uma função
\begin{description}
\item[{def funcao():}] \leavevmode
foo = 9
print(id(foo))
print(foo)

\end{description}

Nota\sphinxhyphen{}se que dentro da função também há uma variável chamada “foo”.
Essa função imprime o ID desse identificador e depois imprime seu valor.

Exibindo o ID de foo

print(id(foo))
162857064

Exibindo o valor de foo

print(foo)
7

Acionando a função

funcao()
162857040
9

Nota\sphinxhyphen{}se também que bem como o ID e o valor retornados pela função,
da variável interna foo são diferentes da variável externa de mesmo nome.
\begin{itemize}
\item {} 
Escopo de Função Externa
\begin{quote}

Ao se criar uma função dentro de outra, a função mais interna pode utilizar um
\end{quote}

\end{itemize}

identificador que esteja no nível mais acima.

Prática:

Definição da funcao
\begin{description}
\item[{def funcao\_principal():}] \leavevmode
x = 1
def funcao\_secundaria():
\begin{quote}

print(x)
\end{quote}

funcao\_secundaria()

A função principal tem uma variável x, cujo valor é impresso em tela

\item[{pela função secundária.}] \leavevmode
A função principal invoca a função secundária.

\end{description}

Chamando a função
funcao\_principal()
1

Uma nova definição da função
\begin{description}
\item[{def funcao\_principal():}] \leavevmode
x = 1
def funcao\_secundaria():
\begin{quote}

x = 2
print(x)
\end{quote}

funcao\_secundaria()

\end{description}

Diferente do exemplo anterior, a função secundária declarou sua própria variável “x”.

Testando a função
funcao\_principal()
2

Nota\sphinxhyphen{}se que o valor considerado foi o de “x”, que é o identificador mais interno.
\begin{itemize}
\item {} 
Escopo Global ou Escopo do Módulo

É também conhecido como escopo de módulo devido ao fato de estar na endentação do mesmo.

\end{itemize}

Prática:

Criação de variável

foo = ‘bar’

Criação de função
\begin{description}
\item[{def funcao():}] \leavevmode
foo = ‘eggs’
print(foo)

\end{description}

A função criada tem uma variável com o mesmo nome que uma variável global, a ela dá um valor e
imprime esse valor em tela.
Será que isso altera o valor da variável global?

Execução da função

funcao()
eggs

Podemos notar que o valor impresso é igual ao da variável “foo” dentro da função.
Pra saber se a variável global foi alterada, vamos testar com a função print.

Imprimindo o valor da variável global

print(foo)
bar

Pode\sphinxhyphen{}se concluir que a função criada não interferiu na variável global.
Para alterar uma variável global em um contexto local precisamos utilizar o comando global.

Criação de função que altera a variável global
\begin{description}
\item[{def funcao():}] \leavevmode
global foo
foo = ‘eggs’
print(foo)

\end{description}

Executar função

funcao()
eggs

OK, a função imprimiu o valor local da função.
Mas será que a variável global também foi alterada?

Imprimir o valor da variável global

print(foo)
eggs

Agora a função pôde alterar a variável global.
Isso se deve ao fato do comando global ter sido empregado.
A variável global a ser alterada deve ser declarada como global antes de sua
atribuição.

Escopo \_\_builtins\_\_
\begin{quote}

O escopo \_\_builtins\_\_ abrange identificadores que já estão definidos antes mesmo do código a ser escrito.
São funções, comandos e variáveis internas de Python.
\end{quote}

Prática:

“str” é está em \_\_builtins\_\_?

‘str’ in dir(\_\_builtins\_\_)
True

Resposta afirmativa (True), ou seja, “str” faz parte desse escopo.
E se subscrevermos esse item localmente?

Criando uma variável cujo identificador pertence ao escopo \_\_builtins\_\_
str = 1

Qual é o tipo?

type(str)
int

“str” que inicialmente era um identificador para o tipo de strings em Python,
aqui agora virou uma variável de inteiro.
Mas e o tipo “str” deixou de existir?

Qual o tipo?
type(\_\_builtins\_\_.str)
type

É do tipo “tipo”

Valor de str?

str
1

\_\_builtins\_\_.str(str)
‘1’

del str

str(7)
‘7’

dir(\_\_builtins\_\_)

global nome\_variavel
nome\_variavel = valor

\# ==============================================================================

Funções globals(), locals() e vars() e Comando global
\begin{quote}

Cada uma das funções retornam dicionários de variáveis e seus respectivos valores.
globals(): Retorna variáveis globais (escopo do módulo);
locals(): Retorna variáveis locais (escopo local);
vars(obj): sem argumentos é equivalente a locals(), com um argumento, equivalente a objeto.\_\_dict\_\_
\end{quote}

bla bla bla:

\textgreater{} foo = ‘escopo global’

bla bla bla:
\begin{description}
\item[{\textgreater{} def f():}] \leavevmode
foo = ‘escopo local’
bar = ‘uma variável qualquer…’
print(globals(){[}‘foo’{]})
print(locals(){[}‘foo’{]})

\end{description}

bla bla bla:

\textgreater{} f()

escopo global
escopo local

bla bla bla:
\begin{description}
\item[{\textgreater{} class Spam(object):}] \leavevmode
foo = ‘’
bar = ‘’

\end{description}

bla bla bla:

\textgreater{} vars(Spam)
\begin{description}
\item[{\textless{}dictproxy \{‘\_\_dict\_\_’: \textless{}attribute ‘\_\_dict\_\_’ of ‘Spam’ objects\textgreater{},}] \leavevmode
‘\_\_doc\_\_’: None,
‘\_\_module\_\_’: ‘\_\_main\_\_’,
‘\_\_weakref\_\_’: \textless{}attribute ‘\_\_weakref\_\_’ of ‘Spam’ objects\textgreater{},
‘bar’: ‘’,
‘foo’: ‘’\}\textgreater{}

\end{description}

bla bla bla:
\begin{description}
\item[{\textgreater{} def f():}] \leavevmode
global x
x = 7

\end{description}

bla bla bla:

\textgreater{} type(x)

NameError: name ‘x’ is not defined

bla bla bla:

\textgreater{} f()

bla bla bla:

\textgreater{} type(x)

int

bla bla bla:

\textgreater{} print(x)

7


\chapter{if / elif / else}
\label{\detokenize{content/if:if-elif-else}}\label{\detokenize{content/if::doc}}\begin{quote}

O comando if, que inglês significa “se” indica uma condição.
\end{quote}

\begin{sphinxVerbatim}[commandchars=\\\{\}]
\PYG{c+c1}{\PYGZsh{} Criação de dois objetos int:}
\PYG{n}{x} \PYG{o}{=} \PYG{l+m+mi}{7}
\PYG{n}{y} \PYG{o}{=} \PYG{l+m+mi}{5}

\PYG{c+c1}{\PYGZsh{} Bloco if:}
\PYG{k}{if} \PYG{n}{x} \PYG{o}{\PYGZgt{}} \PYG{n}{y}\PYG{p}{:}
    \PYG{n+nb}{print}\PYG{p}{(}\PYG{l+s+s1}{\PYGZsq{}}\PYG{l+s+s1}{X é maior}\PYG{l+s+s1}{\PYGZsq{}}\PYG{p}{)}
\end{sphinxVerbatim}

\begin{sphinxVerbatim}[commandchars=\\\{\}]
\PYG{g+go}{X é maior}
\end{sphinxVerbatim}

\begin{sphinxVerbatim}[commandchars=\\\{\}]
\PYG{c+c1}{\PYGZsh{} Bloco if;}
\PYG{k}{if} \PYG{n}{x} \PYG{o}{\PYGZlt{}} \PYG{n}{y}\PYG{p}{:}
    \PYG{n+nb}{print}\PYG{p}{(}\PYG{l+s+s1}{\PYGZsq{}}\PYG{l+s+s1}{X é maior}\PYG{l+s+s1}{\PYGZsq{}}\PYG{p}{)}
\end{sphinxVerbatim}

\begin{sphinxVerbatim}[commandchars=\\\{\}]
\PYG{c+c1}{\PYGZsh{} Objetos booleanos:}
\PYG{n}{foo} \PYG{o}{=} \PYG{k+kc}{True}
\PYG{n}{bar} \PYG{o}{=} \PYG{k+kc}{False}

\PYG{c+c1}{\PYGZsh{} Bloco if:}
\PYG{k}{if} \PYG{n}{foo}\PYG{p}{:}
    \PYG{n+nb}{print}\PYG{p}{(}\PYG{l+s+s1}{\PYGZsq{}}\PYG{l+s+s1}{foo é verdadeiro!}\PYG{l+s+s1}{\PYGZsq{}}\PYG{p}{)}
\end{sphinxVerbatim}

\begin{sphinxVerbatim}[commandchars=\\\{\}]
\PYG{g+go}{foo é verdadeiro!}
\end{sphinxVerbatim}

\begin{sphinxVerbatim}[commandchars=\\\{\}]
\PYG{c+c1}{\PYGZsh{} Bloco if:}
\PYG{k}{if} \PYG{o+ow}{not} \PYG{n}{bar}\PYG{p}{:}
    \PYG{n+nb}{print}\PYG{p}{(}\PYG{l+s+s1}{\PYGZsq{}}\PYG{l+s+s1}{bar é falso!}\PYG{l+s+s1}{\PYGZsq{}}\PYG{p}{)}
\end{sphinxVerbatim}

\begin{sphinxVerbatim}[commandchars=\\\{\}]
\PYG{g+go}{bar é falso!}
\end{sphinxVerbatim}

\begin{sphinxVerbatim}[commandchars=\\\{\}]
\PYG{c+c1}{\PYGZsh{} Objeto string:}
\PYG{n}{texto} \PYG{o}{=} \PYG{l+s+s1}{\PYGZsq{}}\PYG{l+s+s1}{Python e PostgreSQL: Poder!}\PYG{l+s+s1}{\PYGZsq{}}

\PYG{c+c1}{\PYGZsh{} Bloco if:}
\PYG{k}{if} \PYG{n}{texto}\PYG{p}{:}
    \PYG{n+nb}{print}\PYG{p}{(}\PYG{l+s+s1}{\PYGZsq{}}\PYG{l+s+s1}{A string NÃO é vazia!}\PYG{l+s+s1}{\PYGZsq{}}\PYG{p}{)}
\end{sphinxVerbatim}

\begin{sphinxVerbatim}[commandchars=\\\{\}]
\PYG{g+go}{A string NÃO é vazia!}
\end{sphinxVerbatim}

\begin{sphinxVerbatim}[commandchars=\\\{\}]
\PYG{c+c1}{\PYGZsh{} String vazia:}
\PYG{n}{texto} \PYG{o}{=} \PYG{l+s+s1}{\PYGZsq{}}\PYG{l+s+s1}{\PYGZsq{}}

\PYG{c+c1}{\PYGZsh{} Bloco if:}
\PYG{k}{if} \PYG{o+ow}{not} \PYG{n}{texto}\PYG{p}{:}
    \PYG{n+nb}{print}\PYG{p}{(}\PYG{l+s+s1}{\PYGZsq{}}\PYG{l+s+s1}{A string é vazia!}\PYG{l+s+s1}{\PYGZsq{}}\PYG{p}{)}
\end{sphinxVerbatim}

\begin{sphinxVerbatim}[commandchars=\\\{\}]
\PYG{g+go}{A string é vazia!}
\end{sphinxVerbatim}

\begin{sphinxVerbatim}[commandchars=\\\{\}]
\PYG{c+c1}{\PYGZsh{} Objetos int:}
\PYG{n}{x} \PYG{o}{=} \PYG{l+m+mi}{1}
\PYG{n}{y} \PYG{o}{=} \PYG{l+m+mi}{2}

\PYG{c+c1}{\PYGZsh{} Bloco if:}
\PYG{k}{if} \PYG{n}{x} \PYG{o}{\PYGZgt{}} \PYG{n}{y}\PYG{p}{:}
    \PYG{n+nb}{print}\PYG{p}{(}\PYG{l+s+s1}{\PYGZsq{}}\PYG{l+s+s1}{X é maior}\PYG{l+s+s1}{\PYGZsq{}}\PYG{p}{)}
\PYG{k}{else}\PYG{p}{:}
    \PYG{n+nb}{print}\PYG{p}{(}\PYG{l+s+s1}{\PYGZsq{}}\PYG{l+s+s1}{Y é maior}\PYG{l+s+s1}{\PYGZsq{}}\PYG{p}{)}
\end{sphinxVerbatim}

\begin{sphinxVerbatim}[commandchars=\\\{\}]
\PYG{n}{Y} \PYG{n}{é} \PYG{n}{maior}
\end{sphinxVerbatim}

\begin{sphinxVerbatim}[commandchars=\\\{\}]
\PYG{c+c1}{\PYGZsh{} Objetos int:}
\PYG{n}{y} \PYG{o}{=} \PYG{l+m+mi}{1}
\PYG{n}{x} \PYG{o}{=} \PYG{l+m+mi}{1}

\PYG{c+c1}{\PYGZsh{} Bloco if:}
\PYG{k}{if} \PYG{n}{x} \PYG{o}{\PYGZgt{}} \PYG{n}{y}\PYG{p}{:}
    \PYG{n+nb}{print}\PYG{p}{(}\PYG{l+s+s1}{\PYGZsq{}}\PYG{l+s+s1}{X é maior}\PYG{l+s+s1}{\PYGZsq{}}\PYG{p}{)}
\PYG{k}{elif} \PYG{n}{x} \PYG{o}{\PYGZlt{}} \PYG{n}{y}\PYG{p}{:}
    \PYG{n+nb}{print}\PYG{p}{(}\PYG{l+s+s1}{\PYGZsq{}}\PYG{l+s+s1}{Y é maior}\PYG{l+s+s1}{\PYGZsq{}}\PYG{p}{)}
\PYG{k}{else}\PYG{p}{:}
    \PYG{n+nb}{print}\PYG{p}{(}\PYG{l+s+s1}{\PYGZsq{}}\PYG{l+s+s1}{Valores iguais}\PYG{l+s+s1}{\PYGZsq{}}\PYG{p}{)}
\end{sphinxVerbatim}

\begin{sphinxVerbatim}[commandchars=\\\{\}]
\PYG{g+go}{Valores iguais}
\end{sphinxVerbatim}

\begin{sphinxVerbatim}[commandchars=\\\{\}]
\PYG{c+c1}{\PYGZsh{} Objeto int:}
\PYG{n}{x} \PYG{o}{=} \PYG{l+m+mi}{10}

\PYG{c+c1}{\PYGZsh{} Bloco if:}
\PYG{k}{if} \PYG{p}{(}\PYG{n}{x} \PYG{o}{\PYGZgt{}} \PYG{l+m+mi}{5}\PYG{p}{)}\PYG{p}{:}
    \PYG{n}{y} \PYG{o}{=} \PYG{l+m+mi}{3}
\PYG{k}{else}\PYG{p}{:}
    \PYG{n}{y} \PYG{o}{=} \PYG{l+m+mi}{0}
\end{sphinxVerbatim}


\section{if Ternário}
\label{\detokenize{content/if:if-ternario}}
\begin{sphinxVerbatim}[commandchars=\\\{\}]
\PYG{n}{x} \PYG{o}{=} \PYG{l+m+mi}{10}  \PYG{c+c1}{\PYGZsh{} Variável int}

\PYG{c+c1}{\PYGZsh{} Atribuição de valor condicional:}
\PYG{n}{y} \PYG{o}{=} \PYG{p}{(}\PYG{l+m+mi}{50} \PYG{k}{if} \PYG{p}{(}\PYG{n}{x} \PYG{o}{\PYGZgt{}} \PYG{l+m+mi}{5}\PYG{p}{)} \PYG{k}{else} \PYG{l+m+mi}{40}\PYG{p}{)}

\PYG{c+c1}{\PYGZsh{} Exibe o valor de \PYGZdq{}y\PYGZdq{}:}
\PYG{n+nb}{print}\PYG{p}{(}\PYG{n}{y}\PYG{p}{)}
\end{sphinxVerbatim}

\begin{sphinxVerbatim}[commandchars=\\\{\}]
\PYG{g+go}{50}
\end{sphinxVerbatim}

\begin{sphinxVerbatim}[commandchars=\\\{\}]
\PYG{c+c1}{\PYGZsh{} Variável float para receber a nota:}
\PYG{n}{nota} \PYG{o}{=} \PYG{n+nb}{float}\PYG{p}{(}\PYG{n+nb}{input}\PYG{p}{(}\PYG{l+s+s1}{\PYGZsq{}}\PYG{l+s+s1}{Digite a nota do aluno: }\PYG{l+s+s1}{\PYGZsq{}}\PYG{p}{)}\PYG{p}{)}
\end{sphinxVerbatim}

\begin{sphinxVerbatim}[commandchars=\\\{\}]
\PYG{g+go}{Digite a nota do aluno: 8}
\end{sphinxVerbatim}

\begin{sphinxVerbatim}[commandchars=\\\{\}]
\PYG{c+c1}{\PYGZsh{} Atribuição condicional:}
\PYG{n}{estado} \PYG{o}{=} \PYG{l+s+s1}{\PYGZsq{}}\PYG{l+s+s1}{aprovado}\PYG{l+s+s1}{\PYGZsq{}} \PYG{k}{if} \PYG{n}{nota} \PYG{o}{\PYGZgt{}}\PYG{o}{=} \PYG{l+m+mi}{7} \PYG{k}{else} \PYG{l+s+s1}{\PYGZsq{}}\PYG{l+s+s1}{reprovado}\PYG{l+s+s1}{\PYGZsq{}}
\end{sphinxVerbatim}

\begin{sphinxVerbatim}[commandchars=\\\{\}]
\PYG{c+c1}{\PYGZsh{} Exibe a mensagem:}
\PYG{n+nb}{print}\PYG{p}{(}\PYG{l+s+s1}{\PYGZsq{}}\PYG{l+s+s1}{Aluno }\PYG{l+s+si}{\PYGZob{}\PYGZcb{}}\PYG{l+s+s1}{!}\PYG{l+s+s1}{\PYGZsq{}}\PYG{o}{.}\PYG{n}{format}\PYG{p}{(}\PYG{n}{estado}\PYG{p}{)}\PYG{p}{)}
\end{sphinxVerbatim}

\begin{sphinxVerbatim}[commandchars=\\\{\}]
\PYG{g+go}{Aluno aprovado!}
\end{sphinxVerbatim}

\begin{sphinxVerbatim}[commandchars=\\\{\}]
\PYG{c+c1}{\PYGZsh{} Variável int:}
\PYG{n}{num} \PYG{o}{=} \PYG{n+nb}{int}\PYG{p}{(}\PYG{n+nb}{input}\PYG{p}{(}\PYG{l+s+s1}{\PYGZsq{}}\PYG{l+s+s1}{Digite um número: }\PYG{l+s+s1}{\PYGZsq{}}\PYG{p}{)}\PYG{p}{)}
\end{sphinxVerbatim}

\begin{sphinxVerbatim}[commandchars=\\\{\}]
\PYG{g+go}{Digite um número: \PYGZhy{}2}
\end{sphinxVerbatim}

\begin{sphinxVerbatim}[commandchars=\\\{\}]
\PYG{c+c1}{\PYGZsh{} Atribuição condicional:}
\PYG{n}{sinal} \PYG{o}{=} \PYG{l+s+s1}{\PYGZsq{}}\PYG{l+s+s1}{positivo}\PYG{l+s+s1}{\PYGZsq{}} \PYG{k}{if} \PYG{n}{num} \PYG{o}{\PYGZgt{}} \PYG{l+m+mi}{0} \PYG{k}{else} \PYG{l+s+s1}{\PYGZsq{}}\PYG{l+s+s1}{negativo}\PYG{l+s+s1}{\PYGZsq{}} \PYG{k}{if} \PYG{n}{num} \PYG{o}{\PYGZlt{}} \PYG{l+m+mi}{0} \PYG{k}{else} \PYG{l+s+s1}{\PYGZsq{}}\PYG{l+s+s1}{zero}\PYG{l+s+s1}{\PYGZsq{}}

\PYG{c+c1}{\PYGZsh{} Exibe mensagem:}
\PYG{n+nb}{print}\PYG{p}{(}\PYG{l+s+s1}{\PYGZsq{}}\PYG{l+s+s1}{O número é }\PYG{l+s+si}{\PYGZob{}\PYGZcb{}}\PYG{l+s+s1}{\PYGZsq{}}\PYG{o}{.}\PYG{n}{format}\PYG{p}{(}\PYG{n}{sinal}\PYG{p}{)}\PYG{p}{)}
\end{sphinxVerbatim}

\begin{sphinxVerbatim}[commandchars=\\\{\}]
\PYG{g+go}{O número é negativo}
\end{sphinxVerbatim}


\chapter{Loops \sphinxhyphen{} Laços de Repetição}
\label{\detokenize{content/loops:loops-lacos-de-repeticao}}\label{\detokenize{content/loops::doc}}

\section{while}
\label{\detokenize{content/loops:while}}
Executa em laço (loop) enquanto a condição for verdadeira.

\begin{sphinxVerbatim}[commandchars=\\\{\}]
\PYG{c+c1}{\PYGZsh{}}
\PYG{n}{i} \PYG{o}{=} \PYG{l+m+mi}{0}

\PYG{c+c1}{\PYGZsh{}}
\PYG{k}{while} \PYG{n}{i} \PYG{o}{\PYGZlt{}} \PYG{l+m+mi}{5}\PYG{p}{:}
    \PYG{n+nb}{print}\PYG{p}{(}\PYG{n}{i}\PYG{p}{)}
    \PYG{n}{i} \PYG{o}{+}\PYG{o}{=} \PYG{l+m+mi}{1}
\end{sphinxVerbatim}

\begin{sphinxVerbatim}[commandchars=\\\{\}]
\PYG{g+go}{0}
\PYG{g+go}{1}
\PYG{g+go}{2}
\PYG{g+go}{3}
\PYG{g+go}{4}
\end{sphinxVerbatim}


\subsection{O else no loop while}
\label{\detokenize{content/loops:o-else-no-loop-while}}\begin{quote}

Opcionalmente, pode\sphinxhyphen{}se adicionar um else ao while em Python.
A idéia é que se caso o loop seja executado sem interrupção, um break, por exemplo, o que estiver dentro do bloco else será executado.
\end{quote}

\begin{sphinxVerbatim}[commandchars=\\\{\}]
\PYG{c+c1}{\PYGZsh{}}
\PYG{n}{i} \PYG{o}{=} \PYG{l+m+mi}{0}
\PYG{k}{while} \PYG{n}{i} \PYG{o}{\PYGZlt{}} \PYG{l+m+mi}{5}\PYG{p}{:}
    \PYG{n+nb}{print}\PYG{p}{(}\PYG{n}{i}\PYG{p}{)}
    \PYG{n}{i} \PYG{o}{+}\PYG{o}{=} \PYG{l+m+mi}{1}
\PYG{k}{else}\PYG{p}{:}
    \PYG{n+nb}{print}\PYG{p}{(}\PYG{l+s+s1}{\PYGZsq{}}\PYG{l+s+s1}{Fim}\PYG{l+s+s1}{\PYGZsq{}}\PYG{p}{)}
\end{sphinxVerbatim}

\begin{sphinxVerbatim}[commandchars=\\\{\}]
\PYG{g+go}{0}
\PYG{g+go}{1}
\PYG{g+go}{2}
\PYG{g+go}{3}
\PYG{g+go}{4}
\PYG{g+go}{Fim}
\end{sphinxVerbatim}

\begin{sphinxVerbatim}[commandchars=\\\{\}]
\PYG{c+c1}{\PYGZsh{}}
\PYG{n}{i} \PYG{o}{=} \PYG{l+m+mi}{0}

\PYG{c+c1}{\PYGZsh{}}
\PYG{k}{while} \PYG{n}{i} \PYG{o}{\PYGZlt{}}\PYG{o}{=} \PYG{l+m+mi}{10}\PYG{p}{:}
    \PYG{k}{if} \PYG{n}{i} \PYG{o}{==} \PYG{l+m+mi}{5}\PYG{p}{:} \PYG{k}{break}
    \PYG{n+nb}{print}\PYG{p}{(}\PYG{n}{i}\PYG{p}{)}
    \PYG{n}{i} \PYG{o}{+}\PYG{o}{=} \PYG{l+m+mi}{1}
\PYG{k}{else}\PYG{p}{:}
    \PYG{n+nb}{print}\PYG{p}{(}\PYG{l+s+s1}{\PYGZsq{}}\PYG{l+s+s1}{Fim}\PYG{l+s+s1}{\PYGZsq{}}\PYG{p}{)}
\end{sphinxVerbatim}

\begin{sphinxVerbatim}[commandchars=\\\{\}]
\PYG{g+go}{0}
\PYG{g+go}{1}
\PYG{g+go}{2}
\PYG{g+go}{3}
\PYG{g+go}{4}
\end{sphinxVerbatim}

\begin{sphinxVerbatim}[commandchars=\\\{\}]
\PYG{c+c1}{\PYGZsh{}}
\PYG{n}{i} \PYG{o}{=} \PYG{l+m+mi}{0}

\PYG{c+c1}{\PYGZsh{}}
\PYG{k}{while} \PYG{n}{i} \PYG{o}{\PYGZlt{}}\PYG{o}{=} \PYG{l+m+mi}{10}\PYG{p}{:}
    \PYG{k}{if} \PYG{p}{(}\PYG{n}{i} \PYG{o}{\PYGZpc{}} \PYG{l+m+mi}{2} \PYG{o}{==} \PYG{l+m+mi}{0}\PYG{p}{)}\PYG{p}{:}
        \PYG{n}{i} \PYG{o}{+}\PYG{o}{=} \PYG{l+m+mi}{1}
        \PYG{k}{continue}
    \PYG{n+nb}{print}\PYG{p}{(}\PYG{n}{i}\PYG{p}{)}
    \PYG{n}{i} \PYG{o}{+}\PYG{o}{=} \PYG{l+m+mi}{1}
\PYG{k}{else}\PYG{p}{:}
    \PYG{n+nb}{print}\PYG{p}{(}\PYG{l+s+s1}{\PYGZsq{}}\PYG{l+s+s1}{Fim}\PYG{l+s+s1}{\PYGZsq{}}\PYG{p}{)}
\end{sphinxVerbatim}

\begin{sphinxVerbatim}[commandchars=\\\{\}]
\PYG{g+go}{1}
\PYG{g+go}{3}
\PYG{g+go}{5}
\PYG{g+go}{7}
\PYG{g+go}{9}
\PYG{g+go}{Fim}
\end{sphinxVerbatim}


\subsection{Loop Infinito}
\label{\detokenize{content/loops:loop-infinito}}
\begin{sphinxVerbatim}[commandchars=\\\{\}]
\PYG{c+c1}{\PYGZsh{}}
\PYG{k}{while} \PYG{k+kc}{True}\PYG{p}{:}
    \PYG{n+nb}{print}\PYG{p}{(}\PYG{l+s+s1}{\PYGZsq{}}\PYG{l+s+s1}{x}\PYG{l+s+s1}{\PYGZsq{}}\PYG{p}{)}
\end{sphinxVerbatim}

\begin{sphinxVerbatim}[commandchars=\\\{\}]
\PYG{g+go}{x}
\PYG{g+go}{x}
\PYG{g+go}{x}
\PYG{g+go}{. . .}
\end{sphinxVerbatim}


\subsection{for}
\label{\detokenize{content/loops:for}}
\begin{sphinxVerbatim}[commandchars=\\\{\}]
\PYG{c+c1}{\PYGZsh{}}
\PYG{k}{for} \PYG{n}{i} \PYG{o+ow}{in} \PYG{n+nb}{range}\PYG{p}{(}\PYG{l+m+mi}{5}\PYG{p}{)}\PYG{p}{:}
    \PYG{n+nb}{print}\PYG{p}{(}\PYG{n}{i}\PYG{p}{)}
\end{sphinxVerbatim}

\begin{sphinxVerbatim}[commandchars=\\\{\}]
\PYG{g+go}{0}
\PYG{g+go}{1}
\PYG{g+go}{2}
\PYG{g+go}{3}
\PYG{g+go}{4}
\end{sphinxVerbatim}

\begin{sphinxVerbatim}[commandchars=\\\{\}]
\PYG{c+c1}{\PYGZsh{}}
\PYG{n}{lor} \PYG{o}{=} \PYG{p}{(}\PYG{l+s+s1}{\PYGZsq{}}\PYG{l+s+s1}{Gandalf}\PYG{l+s+s1}{\PYGZsq{}}\PYG{p}{,} \PYG{l+s+s1}{\PYGZsq{}}\PYG{l+s+s1}{Bilbo}\PYG{l+s+s1}{\PYGZsq{}}\PYG{p}{,} \PYG{l+s+s1}{\PYGZsq{}}\PYG{l+s+s1}{Frodo}\PYG{l+s+s1}{\PYGZsq{}}\PYG{p}{,} \PYG{l+s+s1}{\PYGZsq{}}\PYG{l+s+s1}{Sauron}\PYG{l+s+s1}{\PYGZsq{}}\PYG{p}{,} \PYG{l+s+s1}{\PYGZsq{}}\PYG{l+s+s1}{Aragorn}\PYG{l+s+s1}{\PYGZsq{}}\PYG{p}{,} \PYG{l+s+s1}{\PYGZsq{}}\PYG{l+s+s1}{Legolas}\PYG{l+s+s1}{\PYGZsq{}}\PYG{p}{)}

\PYG{c+c1}{\PYGZsh{}}
\PYG{k}{for} \PYG{n}{i} \PYG{o+ow}{in} \PYG{n}{lor}\PYG{p}{:}
    \PYG{n+nb}{print}\PYG{p}{(}\PYG{n}{i}\PYG{p}{)}
\end{sphinxVerbatim}

\begin{sphinxVerbatim}[commandchars=\\\{\}]
\PYG{g+go}{Gandalf}
\PYG{g+go}{Bilbo}
\PYG{g+go}{Frodo}
\PYG{g+go}{Sauron}
\PYG{g+go}{Aragorn}
\PYG{g+go}{Legolas}
\end{sphinxVerbatim}

\begin{sphinxVerbatim}[commandchars=\\\{\}]
\PYG{c+c1}{\PYGZsh{}}
\PYG{k}{for} \PYG{n}{i}\PYG{p}{,} \PYG{n}{personagem} \PYG{o+ow}{in} \PYG{n+nb}{enumerate}\PYG{p}{(}\PYG{n}{lor}\PYG{p}{)}\PYG{p}{:}
    \PYG{n+nb}{print}\PYG{p}{(}\PYG{l+s+s1}{\PYGZsq{}}\PYG{l+s+si}{\PYGZpc{}d}\PYG{l+s+s1}{ \PYGZhy{} }\PYG{l+s+si}{\PYGZpc{}s}\PYG{l+s+s1}{\PYGZsq{}} \PYG{o}{\PYGZpc{}} \PYG{p}{(}\PYG{n}{i}\PYG{p}{,} \PYG{n}{personagem}\PYG{p}{)}\PYG{p}{)}
\end{sphinxVerbatim}

\begin{sphinxVerbatim}[commandchars=\\\{\}]
\PYG{g+go}{0 \PYGZhy{} Gandalf}
\PYG{g+go}{1 \PYGZhy{} Bilbo}
\PYG{g+go}{2 \PYGZhy{} Frodo}
\PYG{g+go}{3 \PYGZhy{} Sauron}
\PYG{g+go}{4 \PYGZhy{} Aragorn}
\PYG{g+go}{5 \PYGZhy{} Legolas}
\end{sphinxVerbatim}

\begin{sphinxVerbatim}[commandchars=\\\{\}]
\PYG{c+c1}{\PYGZsh{}}
\PYG{n+nb}{list}\PYG{p}{(}\PYG{n+nb}{enumerate}\PYG{p}{(}\PYG{n}{lor}\PYG{p}{)}\PYG{p}{)}
\end{sphinxVerbatim}

\begin{sphinxVerbatim}[commandchars=\\\{\}]
\PYG{g+go}{[(0, \PYGZsq{}Gandalf\PYGZsq{}), (1, \PYGZsq{}Bilbo\PYGZsq{}), (2, \PYGZsq{}Frodo\PYGZsq{}), (3, \PYGZsq{}Sauron\PYGZsq{}), (4, \PYGZsq{}Aragorn\PYGZsq{}), (5, \PYGZsq{}Legolas\PYGZsq{})]}
\end{sphinxVerbatim}

\begin{sphinxVerbatim}[commandchars=\\\{\}]
\PYG{c+c1}{\PYGZsh{}}
\PYG{n}{dados} \PYG{o}{=}  \PYG{p}{[}\PYG{p}{(}\PYG{l+s+s1}{\PYGZsq{}}\PYG{l+s+s1}{Nome}\PYG{l+s+s1}{\PYGZsq{}}\PYG{p}{,} \PYG{l+s+s1}{\PYGZsq{}}\PYG{l+s+s1}{Chiquinho}\PYG{l+s+s1}{\PYGZsq{}}\PYG{p}{)}\PYG{p}{,} \PYG{p}{(}\PYG{l+s+s1}{\PYGZsq{}}\PYG{l+s+s1}{Sobrenome}\PYG{l+s+s1}{\PYGZsq{}}\PYG{p}{,} \PYG{l+s+s1}{\PYGZsq{}}\PYG{l+s+s1}{da Silva}\PYG{l+s+s1}{\PYGZsq{}}\PYG{p}{)}\PYG{p}{,} \PYG{p}{(}\PYG{l+s+s1}{\PYGZsq{}}\PYG{l+s+s1}{Idade}\PYG{l+s+s1}{\PYGZsq{}}\PYG{p}{,} \PYG{l+m+mi}{50}\PYG{p}{)}\PYG{p}{]}

\PYG{c+c1}{\PYGZsh{}}
\PYG{k}{for} \PYG{n}{k}\PYG{p}{,} \PYG{n}{v} \PYG{o+ow}{in} \PYG{n}{dados}\PYG{p}{:}
    \PYG{n+nb}{print}\PYG{p}{(}\PYG{l+s+s1}{\PYGZsq{}}\PYG{l+s+si}{\PYGZpc{}s}\PYG{l+s+s1}{: }\PYG{l+s+si}{\PYGZpc{}s}\PYG{l+s+s1}{\PYGZsq{}} \PYG{o}{\PYGZpc{}} \PYG{p}{(}\PYG{n}{k}\PYG{p}{,} \PYG{n}{v}\PYG{p}{)}\PYG{p}{)}
\end{sphinxVerbatim}

\begin{sphinxVerbatim}[commandchars=\\\{\}]
\PYG{g+go}{Nome: Chiquinho}
\PYG{g+go}{Sobrenome: da Silva}
\PYG{g+go}{Idade: 50}
\end{sphinxVerbatim}

\begin{sphinxVerbatim}[commandchars=\\\{\}]
\PYG{c+c1}{\PYGZsh{}}
\PYG{n}{dados} \PYG{o}{=} \PYG{p}{\PYGZob{}}
    \PYG{l+s+s1}{\PYGZsq{}}\PYG{l+s+s1}{Nome}\PYG{l+s+s1}{\PYGZsq{}}\PYG{p}{:} \PYG{l+s+s1}{\PYGZsq{}}\PYG{l+s+s1}{Chiquinho}\PYG{l+s+s1}{\PYGZsq{}}\PYG{p}{,}
    \PYG{l+s+s1}{\PYGZsq{}}\PYG{l+s+s1}{Sobrenome}\PYG{l+s+s1}{\PYGZsq{}}\PYG{p}{:} \PYG{l+s+s1}{\PYGZsq{}}\PYG{l+s+s1}{da Silva}\PYG{l+s+s1}{\PYGZsq{}}\PYG{p}{,}
    \PYG{l+s+s1}{\PYGZsq{}}\PYG{l+s+s1}{Idade}\PYG{l+s+s1}{\PYGZsq{}}\PYG{p}{:} \PYG{l+m+mi}{50}
\PYG{p}{\PYGZcb{}}

\PYG{o}{.}\PYG{o}{.} \PYG{n}{code}\PYG{o}{\PYGZhy{}}\PYG{n}{block}\PYG{p}{:}\PYG{p}{:} \PYG{n}{python}

\PYG{c+c1}{\PYGZsh{}}
\PYG{k}{for} \PYG{n}{k}\PYG{p}{,} \PYG{n}{v} \PYG{o+ow}{in} \PYG{n}{dados}\PYG{o}{.}\PYG{n}{items}\PYG{p}{(}\PYG{p}{)}\PYG{p}{:}
    \PYG{n+nb}{print}\PYG{p}{(}\PYG{l+s+s1}{\PYGZsq{}}\PYG{l+s+si}{\PYGZpc{}s}\PYG{l+s+s1}{: }\PYG{l+s+si}{\PYGZpc{}s}\PYG{l+s+s1}{\PYGZsq{}} \PYG{o}{\PYGZpc{}} \PYG{p}{(}\PYG{n}{k}\PYG{p}{,} \PYG{n}{v}\PYG{p}{)}\PYG{p}{)}
\end{sphinxVerbatim}

\begin{sphinxVerbatim}[commandchars=\\\{\}]
\PYG{g+go}{Sobrenome: da Silva}
\PYG{g+go}{Idade: 50}
\PYG{g+go}{Nome: Chiquinho}
\end{sphinxVerbatim}

\begin{sphinxVerbatim}[commandchars=\\\{\}]
\PYG{c+c1}{\PYGZsh{}}
\PYG{k}{for} \PYG{n}{i} \PYG{o+ow}{in} \PYG{n+nb}{range}\PYG{p}{(}\PYG{l+m+mi}{5}\PYG{p}{)}\PYG{p}{:}
    \PYG{n+nb}{print}\PYG{p}{(}\PYG{n}{i}\PYG{p}{)}
\PYG{k}{else}\PYG{p}{:}
    \PYG{n+nb}{print}\PYG{p}{(}\PYG{l+s+s1}{\PYGZsq{}}\PYG{l+s+s1}{Fim}\PYG{l+s+s1}{\PYGZsq{}}\PYG{p}{)}
\end{sphinxVerbatim}

\begin{sphinxVerbatim}[commandchars=\\\{\}]
\PYG{g+go}{0}
\PYG{g+go}{1}
\PYG{g+go}{2}
\PYG{g+go}{3}
\PYG{g+go}{4}
\PYG{g+go}{Fim}
\end{sphinxVerbatim}

\begin{sphinxVerbatim}[commandchars=\\\{\}]
\PYG{c+c1}{\PYGZsh{}}
\PYG{k}{for} \PYG{n}{i} \PYG{o+ow}{in} \PYG{n+nb}{range}\PYG{p}{(}\PYG{l+m+mi}{10}\PYG{p}{)}\PYG{p}{:}
    \PYG{k}{if} \PYG{n}{i} \PYG{o}{==} \PYG{l+m+mi}{6}\PYG{p}{:}
        \PYG{k}{break}
    \PYG{n+nb}{print}\PYG{p}{(}\PYG{n}{i}\PYG{p}{)}
\PYG{k}{else}\PYG{p}{:}
    \PYG{n+nb}{print}\PYG{p}{(}\PYG{l+s+s1}{\PYGZsq{}}\PYG{l+s+s1}{Fim}\PYG{l+s+s1}{\PYGZsq{}}\PYG{p}{)}
\end{sphinxVerbatim}

\begin{sphinxVerbatim}[commandchars=\\\{\}]
\PYG{g+go}{0}
\PYG{g+go}{1}
\PYG{g+go}{2}
\PYG{g+go}{3}
\PYG{g+go}{4}
\PYG{g+go}{5}
\end{sphinxVerbatim}

\begin{sphinxVerbatim}[commandchars=\\\{\}]
\PYG{c+c1}{\PYGZsh{}}
\PYG{k}{for} \PYG{n}{i} \PYG{o+ow}{in} \PYG{n+nb}{range}\PYG{p}{(}\PYG{l+m+mi}{10}\PYG{p}{)}\PYG{p}{:}
    \PYG{k}{if} \PYG{n}{i} \PYG{o}{\PYGZpc{}} \PYG{l+m+mi}{2} \PYG{o}{==} \PYG{l+m+mi}{0}\PYG{p}{:}
        \PYG{k}{continue}
    \PYG{n+nb}{print}\PYG{p}{(}\PYG{n}{i}\PYG{p}{)}
\PYG{k}{else}\PYG{p}{:}
    \PYG{n+nb}{print}\PYG{p}{(}\PYG{l+s+s1}{\PYGZsq{}}\PYG{l+s+s1}{Fim}\PYG{l+s+s1}{\PYGZsq{}}\PYG{p}{)}
\end{sphinxVerbatim}

\begin{sphinxVerbatim}[commandchars=\\\{\}]
\PYG{g+go}{1}
\PYG{g+go}{3}
\PYG{g+go}{5}
\PYG{g+go}{7}
\PYG{g+go}{9}
\PYG{g+go}{Fim}
\end{sphinxVerbatim}


\chapter{open}
\label{\detokenize{content/open:open}}\label{\detokenize{content/open::doc}}\begin{quote}

É a forma nativa de Python para manipular arquivos (leitura e escrita).
A partir da versão 3 de Python use preferencialmente open, pois file foi removido.
Um arquivo é iterável, cujas iterações são por linha.
\end{quote}

open(file, mode=’r’, buffering=\sphinxhyphen{}1, encoding=None, errors=None, newline=None, closefd=True, opener=None)

Modos

——— —————————————————————
‘r’       Leitura (padrão);
‘w’       Escrita (novo arquivo, ou se o mesmo existir será truncado);
‘x’       Cria um novo arquivo e o abre para escrita;
‘a’       Escrita (append; o novo conteúdo é adicionado ao arquivo pré existente);
‘b’       Modo binário;
‘t’       Modo de texto (padrão);
‘+’       Abre um arquivo em disco para a atualização (leitura e escrita).

bla bla bla:

\textgreater{} f = open(‘/tmp/foo.txt’, ‘w+’)

bla bla bla:

\textgreater{} type(f)

\_io.TextIOWrapper

bla bla bla:

\textgreater{} print(‘Teste de escrita em arquivo’, file=f)

bla bla bla:

\textgreater{} print(‘ ‘, file=f)

bla bla bla:

\textgreater{} print(‘Uma linha qualquer’, file=f)

bla bla bla:

\textgreater{} f.close()

bla bla bla:

\textgreater{} f = open(‘/tmp/foo.txt’, ‘r’)

bla bla bla:
\begin{description}
\item[{\textgreater{} for line in f:}] \leavevmode
print(line.strip(‘n’))

\end{description}

Fechamento de arquivo:

\textgreater{} f.close()

bla bla bla:

\$ cat \textless{}\textless{} EOF \textgreater{} /tmp/linhas.txt
linha\_1
linha\_2
linha\_3
EOF

bla bla bla:

\$ cat /tmp/linhas.txt

linha\_1
linha\_2
linha\_3

bla bla bla:

\textgreater{} f = open(‘/tmp/linhas.txt’)

bla bla bla:

\textgreater{} f.readline()

‘linha\_1n’

bla bla bla:

\textgreater{} f.readline().split()

{[}‘linha\_2’{]}

bla bla bla:

\textgreater{} f.readline().split()

{[}‘linha\_3’{]}

bla bla bla:

\textgreater{} f.readline().split()

{[}{]}

bla bla bla:

\textgreater{} f.close()

bla bla bla:

\textgreater{} f = open(‘/tmp/linhas.txt’)

bla bla bla:

\textgreater{} f.readlines()

{[}‘linha\_1n’, ‘linha\_2n’, ‘linha\_3n’{]}

bla bla bla:

\textgreater{} f.close()

bla bla bla:

\$ cat \textless{}\textless{} EOF \textgreater{} /tmp/teste.py
\#!/usr/bin/env python3
\#\_*\_ encoding: utf8 \_*\_

import sys

file\_open = sys.argv{[}1{]}

file\_open = open(file\_open, ‘r’)
\begin{description}
\item[{for i in file\_open:}] \leavevmode
print(i.strip())

\end{description}

file\_open.close()
EOF

bla bla bla:

\$ chmod +x /tmp/teste.py

bla bla bla:

\$ ./teste.py linhas.txt

linha\_1
linha\_2
linha\_3

O Método seek:

\$ cat \textless{}\textless{} EOF \textgreater{} /tmp/cores.txt
1 \sphinxhyphen{} Verde
2 \sphinxhyphen{} Preto
3 \sphinxhyphen{} Branco
EOF

bla bla bla:

\textgreater{} f = open(‘/tmp/cores.txt’, ‘r’)

bla bla bla:
\begin{description}
\item[{\textgreater{} for i in f:}] \leavevmode
print(i.strip())

\end{description}

1 \sphinxhyphen{} Verde
2 \sphinxhyphen{} Preto
3 \sphinxhyphen{} Branco

bla bla bla:
\begin{description}
\item[{\textgreater{} for i in f:}] \leavevmode
print(i.strip())

\end{description}

bla bla bla:

\textgreater{} f.seek(0)

bla bla bla:
\begin{description}
\item[{\textgreater{} for i in f:}] \leavevmode
print(i.strip())

\end{description}

1 \sphinxhyphen{} Verde
2 \sphinxhyphen{} Preto
3 \sphinxhyphen{} Branco

bla bla bla:

\textgreater{} f.seek(1)

bla bla bla:
\begin{description}
\item[{\textgreater{} for i in f:}] \leavevmode
print(i.strip())

\end{description}
\begin{itemize}
\item {} 
Verde

\end{itemize}

2 \sphinxhyphen{} Preto
3 \sphinxhyphen{} Branco

bla bla bla:

\textgreater{} f.seek(0)

0

\textgreater{} f.read(7)

‘1 \sphinxhyphen{} Ver’

\textgreater{} f.read(7)

‘den2 \sphinxhyphen{} ‘

\textgreater{} f.read(7)

‘Preton3’

\textgreater{} f.read(7)

‘ \sphinxhyphen{} Bran’

bla bla bla:

\textgreater{} f.close()

bla bla bla:

\textgreater{} f.closed

True

bla bla bla:

\textgreater{} f = open(‘/tmp/cores.txt’, ‘w’)

bla bla bla:

\textgreater{} f.closed

False

bla bla bla:

\textgreater{} f.close()

bla bla bla:

\$ cat /tmp/cores.txt

bla bla bla:

\textgreater{} f = open(‘/tmp/cores.txt’, ‘w’)

bla bla bla:

\textgreater{} f.write(‘1 \sphinxhyphen{} Verden’)

bla bla bla:

\textgreater{} f.close()

bla bla bla:

\$ cat /tmp/cores.txt

1 \sphinxhyphen{} Verde

bla bla bla:

\textgreater{} print(f.name)

/tmp/cores.txt

bla bla bla:

\textgreater{} f = open(‘/tmp/cores.txt’, ‘a’)

bla bla bla:

\textgreater{} f.close()

bla bla bla:

\$ cat /tmp/cores.txt

1 \sphinxhyphen{} Verde

bla bla bla:

\textgreater{} f = open(‘/tmp/cores.txt’, ‘a’)

bla bla bla:

\textgreater{} f.write(‘2 \sphinxhyphen{} Preton’)

bla bla bla:

\textgreater{} f.write(‘3 \sphinxhyphen{} Brancon’)

bla bla bla:

\textgreater{} f.flush()

bla bla bla:

\$ cat /tmp/cores.txt

1 \sphinxhyphen{} Verde
2 \sphinxhyphen{} Preto
3 \sphinxhyphen{} Branco

bla bla bla:

\textgreater{} f.close()

bla bla bla:

\textgreater{} f = open(‘/tmp/cores.txt’, ‘r’)

bla bla bla:

\textgreater{} f.tell()

0

bla bla bla:

\textgreater{} f.read()

‘1 \sphinxhyphen{} Verden2 \sphinxhyphen{} Preton3 \sphinxhyphen{} Brancon’

bla bla bla:

\textgreater{} f.tell()

31

bla bla bla:

\textgreater{} f.seek(0)

0

bla bla bla:

\textgreater{} f.tell()

0

bla bla bla:

\textgreater{} f.read(7)

‘1 \sphinxhyphen{} Ver’

bla bla bla:

\textgreater{} f.tell()

7

bla bla bla:

\textgreater{} f.close()

bla bla bla:

\textgreater{} f = open(‘/tmp/planetas.txt’, ‘w’)

bla bla bla:

\textgreater{} planetas = (‘Saturnon’, ‘Uranon’, ‘Netunon’)

bla bla bla:

\textgreater{} f.writelines(planetas)

bla bla bla:

\textgreater{} f.flush()

bla bla bla:

\$ cat /tmp/planetas.txt

Saturno
Urano
Netuno

bla bla bla:

\textgreater{} planetas = (‘Marten’, ‘Vênusn’, ‘Plutãon’, ‘Júpitern’)

bla bla bla:

\textgreater{} f.writelines(planetas)

bla bla bla:

\textgreater{} f.close()

bla bla bla:

\$ cat /tmp/planetas.txt

Saturno
Urano
Netuno
Marte
Vênus
Plutão
Júpiter


\chapter{with}
\label{\detokenize{content/with:with}}\label{\detokenize{content/with::doc}}\begin{quote}

É usado para encapsular a execução de um bloco com métodos definidos por um gerenciador de contexto.
\end{quote}

\begin{sphinxVerbatim}[commandchars=\\\{\}]
\PYG{c+c1}{\PYGZsh{}}
\PYGZdl{} cat \PYG{l+s}{\PYGZlt{}\PYGZlt{} EOF \PYGZgt{} /tmp/numbers.txt}
\PYG{l+s}{1}
\PYG{l+s}{2}
\PYG{l+s}{3}
\PYG{l+s}{EOF}


\PYG{c+c1}{\PYGZsh{}}
\PYGZdl{} cat \PYG{l+s}{\PYGZlt{}\PYGZlt{} EOF \PYGZgt{} /tmp/numbers\PYGZus{}str.txt}
\PYG{l+s}{1}
\PYG{l+s}{2}
\PYG{l+s}{three}
\PYG{l+s}{EOF}
\end{sphinxVerbatim}

\begin{sphinxVerbatim}[commandchars=\\\{\}]
\PYG{c+c1}{\PYGZsh{} Abrir o arquivo em modo leitura:}
\PYG{n}{f} \PYG{o}{=} \PYG{n+nb}{open}\PYG{p}{(}\PYG{l+s+s1}{\PYGZsq{}}\PYG{l+s+s1}{/tmp/numbers.txt}\PYG{l+s+s1}{\PYGZsq{}}\PYG{p}{,} \PYG{l+s+s1}{\PYGZsq{}}\PYG{l+s+s1}{r}\PYG{l+s+s1}{\PYGZsq{}}\PYG{p}{)}

\PYG{c+c1}{\PYGZsh{} Loop:}
\PYG{k}{for} \PYG{n}{line} \PYG{o+ow}{in} \PYG{n}{f}\PYG{p}{:}
    \PYG{n+nb}{print}\PYG{p}{(}\PYG{n+nb}{int}\PYG{p}{(}\PYG{n}{line}\PYG{p}{)}\PYG{p}{)}

\PYG{c+c1}{\PYGZsh{} Fecha o arquivo:}
\PYG{n}{f}\PYG{o}{.}\PYG{n}{close}\PYG{p}{(}\PYG{p}{)}
\end{sphinxVerbatim}

\begin{sphinxVerbatim}[commandchars=\\\{\}]
\PYG{g+go}{1}
\PYG{g+go}{2}
\PYG{g+go}{3}
\end{sphinxVerbatim}

\begin{sphinxVerbatim}[commandchars=\\\{\}]
\PYG{c+c1}{\PYGZsh{} O arquivo foi fechado?:}
\PYG{n+nb}{print}\PYG{p}{(}\PYG{n}{f}\PYG{o}{.}\PYG{n}{closed}\PYG{p}{)}
\end{sphinxVerbatim}

\begin{sphinxVerbatim}[commandchars=\\\{\}]
\PYG{g+go}{True}
\end{sphinxVerbatim}

\begin{sphinxVerbatim}[commandchars=\\\{\}]
\PYG{c+c1}{\PYGZsh{}}
\PYG{n}{f} \PYG{o}{=} \PYG{n+nb}{open}\PYG{p}{(}\PYG{l+s+s1}{\PYGZsq{}}\PYG{l+s+s1}{/tmp/numbers\PYGZus{}str.txt}\PYG{l+s+s1}{\PYGZsq{}}\PYG{p}{,} \PYG{l+s+s1}{\PYGZsq{}}\PYG{l+s+s1}{r}\PYG{l+s+s1}{\PYGZsq{}}\PYG{p}{)}
\PYG{k}{for} \PYG{n}{line} \PYG{o+ow}{in} \PYG{n}{f}\PYG{p}{:}
    \PYG{n+nb}{print}\PYG{p}{(}\PYG{n+nb}{int}\PYG{p}{(}\PYG{n}{line}\PYG{p}{)}\PYG{p}{)}
\PYG{n}{f}\PYG{o}{.}\PYG{n}{close}\PYG{p}{(}\PYG{p}{)}
\PYG{n+nb}{print}\PYG{p}{(}\PYG{n}{f}\PYG{o}{.}\PYG{n}{closed}\PYG{p}{)}
\end{sphinxVerbatim}

\begin{sphinxVerbatim}[commandchars=\\\{\}]
\PYG{g+go}{1}
\PYG{g+go}{2}

\PYG{g+go}{ValueError: invalid literal for int() with base 10: \PYGZsq{}three\PYGZbs{}n\PYGZsq{}}
\end{sphinxVerbatim}

\begin{sphinxVerbatim}[commandchars=\\\{\}]
\PYG{c+c1}{\PYGZsh{}}
\PYG{n}{f}\PYG{o}{.}\PYG{n}{closed}
\end{sphinxVerbatim}

\begin{sphinxVerbatim}[commandchars=\\\{\}]
\PYG{g+go}{False}
\end{sphinxVerbatim}

\begin{sphinxVerbatim}[commandchars=\\\{\}]
\PYG{c+c1}{\PYGZsh{}}
\PYG{n}{f}\PYG{o}{.}\PYG{n}{close}\PYG{p}{(}\PYG{p}{)}
\PYG{n}{f}\PYG{o}{.}\PYG{n}{closed}
\end{sphinxVerbatim}

\begin{sphinxVerbatim}[commandchars=\\\{\}]
\PYG{g+go}{True}
\end{sphinxVerbatim}

\begin{sphinxVerbatim}[commandchars=\\\{\}]
\PYG{c+c1}{\PYGZsh{}}
\PYG{k}{try}\PYG{p}{:}
    \PYG{n}{f} \PYG{o}{=} \PYG{n+nb}{open}\PYG{p}{(}\PYG{l+s+s1}{\PYGZsq{}}\PYG{l+s+s1}{/tmp/numbers\PYGZus{}str.txt}\PYG{l+s+s1}{\PYGZsq{}}\PYG{p}{,} \PYG{l+s+s1}{\PYGZsq{}}\PYG{l+s+s1}{r}\PYG{l+s+s1}{\PYGZsq{}}\PYG{p}{)}
    \PYG{k}{for} \PYG{n}{line} \PYG{o+ow}{in} \PYG{n}{f}\PYG{p}{:}
        \PYG{n+nb}{print}\PYG{p}{(}\PYG{n+nb}{int}\PYG{p}{(}\PYG{n}{line}\PYG{p}{)}\PYG{p}{)}
\PYG{k}{except} \PYG{n+ne}{ValueError}\PYG{p}{:}
    \PYG{n+nb}{print}\PYG{p}{(}\PYG{l+s+s1}{\PYGZsq{}}\PYG{l+s+s1}{Ops... Isso não é um número em forma de dígitos...}\PYG{l+s+s1}{\PYGZsq{}}\PYG{p}{)}
\PYG{k}{finally}\PYG{p}{:}
    \PYG{n}{f}\PYG{o}{.}\PYG{n}{close}\PYG{p}{(}\PYG{p}{)}
    \PYG{n+nb}{print}\PYG{p}{(}\PYG{n}{f}\PYG{o}{.}\PYG{n}{closed}\PYG{p}{)}
\end{sphinxVerbatim}

\begin{sphinxVerbatim}[commandchars=\\\{\}]
\PYG{g+go}{1}
\PYG{g+go}{2}
\PYG{g+go}{Ops... Isso não é um número em forma de dígitos...}
\PYG{g+go}{True}
\end{sphinxVerbatim}

\begin{sphinxVerbatim}[commandchars=\\\{\}]
\PYG{c+c1}{\PYGZsh{}}
\PYG{k}{with} \PYG{n+nb}{open}\PYG{p}{(}\PYG{l+s+s1}{\PYGZsq{}}\PYG{l+s+s1}{/tmp/numbers.txt}\PYG{l+s+s1}{\PYGZsq{}}\PYG{p}{,} \PYG{l+s+s1}{\PYGZsq{}}\PYG{l+s+s1}{r}\PYG{l+s+s1}{\PYGZsq{}}\PYG{p}{)} \PYG{k}{as} \PYG{n}{f}\PYG{p}{:}
    \PYG{k}{for} \PYG{n}{line} \PYG{o+ow}{in} \PYG{n}{f}\PYG{p}{:}
        \PYG{n+nb}{print}\PYG{p}{(}\PYG{n+nb}{int}\PYG{p}{(}\PYG{n}{line}\PYG{p}{)}\PYG{p}{)}
\PYG{n+nb}{print}\PYG{p}{(}\PYG{n}{f}\PYG{o}{.}\PYG{n}{closed}\PYG{p}{)}
\end{sphinxVerbatim}

\begin{sphinxVerbatim}[commandchars=\\\{\}]
\PYG{g+go}{1}
\PYG{g+go}{2}
\PYG{g+go}{3}
\PYG{g+go}{True}
\end{sphinxVerbatim}

\begin{sphinxVerbatim}[commandchars=\\\{\}]
\PYG{c+c1}{\PYGZsh{}}
\PYG{k}{try}\PYG{p}{:}
    \PYG{k}{with} \PYG{n+nb}{open}\PYG{p}{(}\PYG{l+s+s1}{\PYGZsq{}}\PYG{l+s+s1}{/tmp/numbers\PYGZus{}str.txt}\PYG{l+s+s1}{\PYGZsq{}}\PYG{p}{,} \PYG{l+s+s1}{\PYGZsq{}}\PYG{l+s+s1}{r}\PYG{l+s+s1}{\PYGZsq{}}\PYG{p}{)} \PYG{k}{as} \PYG{n}{f}\PYG{p}{:}
        \PYG{k}{for} \PYG{n}{line} \PYG{o+ow}{in} \PYG{n}{f}\PYG{p}{:}
            \PYG{n+nb}{print}\PYG{p}{(}\PYG{n+nb}{int}\PYG{p}{(}\PYG{n}{line}\PYG{p}{)}\PYG{p}{)}
\PYG{k}{except} \PYG{n+ne}{ValueError}\PYG{p}{:}
    \PYG{n+nb}{print}\PYG{p}{(}\PYG{l+s+s1}{\PYGZsq{}}\PYG{l+s+s1}{Ops... Isso não é um número em forma de dígitos...}\PYG{l+s+s1}{\PYGZsq{}}\PYG{p}{)}
\PYG{k}{finally}\PYG{p}{:}
    \PYG{n+nb}{print}\PYG{p}{(}\PYG{n}{f}\PYG{o}{.}\PYG{n}{closed}\PYG{p}{)}
\end{sphinxVerbatim}

\begin{sphinxVerbatim}[commandchars=\\\{\}]
\PYG{g+go}{1}
\PYG{g+go}{2}
\PYG{g+go}{Ops... Isso não é um número em forma de dígitos...}
\PYG{g+go}{True}
\end{sphinxVerbatim}

\begin{sphinxVerbatim}[commandchars=\\\{\}]
\PYG{k+kn}{from} \PYG{n+nn}{psycopg2} \PYG{k+kn}{import} \PYG{n}{connect} \PYG{k}{as} \PYG{n}{pg\PYGZus{}conn}

\PYG{c+c1}{\PYGZsh{} Parâmetros de conexão}
\PYG{n}{PGHOST} \PYG{o}{=} \PYG{l+s+s1}{\PYGZsq{}}\PYG{l+s+s1}{localhost}\PYG{l+s+s1}{\PYGZsq{}}
\PYG{n}{PGDB} \PYG{o}{=} \PYG{l+s+s1}{\PYGZsq{}}\PYG{l+s+s1}{postgres}\PYG{l+s+s1}{\PYGZsq{}}
\PYG{n}{PGPORT} \PYG{o}{=} \PYG{l+m+mi}{5432}
\PYG{n}{PGUSER} \PYG{o}{=} \PYG{l+s+s1}{\PYGZsq{}}\PYG{l+s+s1}{postgres}\PYG{l+s+s1}{\PYGZsq{}}
\PYG{n}{PGPASS} \PYG{o}{=} \PYG{l+s+s1}{\PYGZsq{}}\PYG{l+s+s1}{123}\PYG{l+s+s1}{\PYGZsq{}}
\PYG{n}{APPLICATION\PYGZus{}NAME} \PYG{o}{=} \PYG{l+s+s1}{\PYGZsq{}}\PYG{l+s+s1}{python}\PYG{l+s+s1}{\PYGZsq{}}

\PYG{c+c1}{\PYGZsh{} Máscara da string de conexão}
\PYG{n}{str\PYGZus{}con} \PYG{o}{=} \PYG{l+s+sa}{f}\PYG{l+s+s1}{\PYGZsq{}\PYGZsq{}\PYGZsq{}}
\PYG{l+s+s1}{          host=}\PYG{l+s+si}{\PYGZob{}}\PYG{n}{PGHOST}\PYG{l+s+si}{\PYGZcb{}}
\PYG{l+s+s1}{          dbname=}\PYG{l+s+si}{\PYGZob{}}\PYG{n}{PGDB}\PYG{l+s+si}{\PYGZcb{}}
\PYG{l+s+s1}{          port=}\PYG{l+s+si}{\PYGZob{}}\PYG{n}{PGPORT}\PYG{l+s+si}{\PYGZcb{}}
\PYG{l+s+s1}{          user=}\PYG{l+s+si}{\PYGZob{}}\PYG{n}{PGUSER}\PYG{l+s+si}{\PYGZcb{}}
\PYG{l+s+s1}{          password=}\PYG{l+s+si}{\PYGZob{}}\PYG{n}{PGPASS}\PYG{l+s+si}{\PYGZcb{}}
\PYG{l+s+s1}{          application\PYGZus{}name=}\PYG{l+s+si}{\PYGZob{}}\PYG{n}{APPLICATION\PYGZus{}NAME}\PYG{l+s+si}{\PYGZcb{}}
\PYG{l+s+s1}{          }\PYG{l+s+s1}{\PYGZsq{}\PYGZsq{}\PYGZsq{}}

\PYG{n}{str\PYGZus{}sql} \PYG{o}{=} \PYG{l+s+s2}{\PYGZdq{}}\PYG{l+s+s2}{SELECT }\PYG{l+s+s2}{\PYGZsq{}}\PYG{l+s+s2}{Teste...}\PYG{l+s+s2}{\PYGZsq{}}\PYG{l+s+s2}{;}\PYG{l+s+s2}{\PYGZdq{}}


\PYG{k}{class} \PYG{n+nc}{PgConnection}\PYG{p}{(}\PYG{n+nb}{object}\PYG{p}{)}\PYG{p}{:}
    \PYG{k}{def} \PYG{n+nf+fm}{\PYGZus{}\PYGZus{}init\PYGZus{}\PYGZus{}}\PYG{p}{(}\PYG{n+nb+bp}{self}\PYG{p}{,} \PYG{n}{str\PYGZus{}con}\PYG{p}{,} \PYG{n}{str\PYGZus{}sql}\PYG{p}{)}\PYG{p}{:}
        \PYG{n+nb+bp}{self}\PYG{o}{.}\PYG{n}{str\PYGZus{}con} \PYG{o}{=} \PYG{n}{str\PYGZus{}con}
        \PYG{n+nb+bp}{self}\PYG{o}{.}\PYG{n}{str\PYGZus{}sql} \PYG{o}{=} \PYG{n}{str\PYGZus{}sql}


    \PYG{k}{def} \PYG{n+nf+fm}{\PYGZus{}\PYGZus{}enter\PYGZus{}\PYGZus{}}\PYG{p}{(}\PYG{n+nb+bp}{self}\PYG{p}{)}\PYG{p}{:}
        \PYG{n+nb}{print}\PYG{p}{(}\PYG{l+s+s1}{\PYGZsq{}}\PYG{l+s+s1}{===== \PYGZus{}\PYGZus{}enter\PYGZus{}\PYGZus{} =====}\PYG{l+s+se}{\PYGZbs{}n}\PYG{l+s+s1}{\PYGZsq{}}\PYG{p}{)}
        \PYG{n+nb+bp}{self}\PYG{o}{.}\PYG{n}{conn} \PYG{o}{=} \PYG{n}{pg\PYGZus{}conn}\PYG{p}{(}\PYG{n}{str\PYGZus{}con}\PYG{p}{)}
        \PYG{n}{cursor} \PYG{o}{=} \PYG{n+nb+bp}{self}\PYG{o}{.}\PYG{n}{conn}\PYG{o}{.}\PYG{n}{cursor}\PYG{p}{(}\PYG{p}{)}
        \PYG{n}{cursor}\PYG{o}{.}\PYG{n}{execute}\PYG{p}{(}\PYG{n}{str\PYGZus{}sql}\PYG{p}{)}
        \PYG{n+nb+bp}{self}\PYG{o}{.}\PYG{n}{data} \PYG{o}{=} \PYG{n}{cursor}\PYG{o}{.}\PYG{n}{fetchone}\PYG{p}{(}\PYG{p}{)}
        \PYG{k}{return} \PYG{n+nb+bp}{self}\PYG{o}{.}\PYG{n}{data}

    \PYG{k}{def} \PYG{n+nf+fm}{\PYGZus{}\PYGZus{}exit\PYGZus{}\PYGZus{}}\PYG{p}{(}\PYG{n+nb+bp}{self}\PYG{p}{,} \PYG{n+nb}{type}\PYG{p}{,} \PYG{n}{value}\PYG{p}{,} \PYG{n}{traceback}\PYG{p}{)}\PYG{p}{:}
        \PYG{n+nb}{print}\PYG{p}{(}\PYG{l+s+s1}{\PYGZsq{}}\PYG{l+s+se}{\PYGZbs{}n}\PYG{l+s+s1}{===== \PYGZus{}\PYGZus{}exit\PYGZus{}\PYGZus{} =====}\PYG{l+s+s1}{\PYGZsq{}}\PYG{p}{)}
        \PYG{n+nb+bp}{self}\PYG{o}{.}\PYG{n}{conn}\PYG{o}{.}\PYG{n}{close}\PYG{p}{(}\PYG{p}{)}
        \PYG{k}{return} \PYG{l+m+mi}{0}


\PYG{k}{with} \PYG{n}{PgConnection}\PYG{p}{(}\PYG{n}{str\PYGZus{}con}\PYG{p}{,} \PYG{n}{str\PYGZus{}sql}\PYG{p}{)} \PYG{k}{as} \PYG{n}{x}\PYG{p}{:}
    \PYG{n+nb}{print}\PYG{p}{(}\PYG{n}{x}\PYG{p}{[}\PYG{l+m+mi}{0}\PYG{p}{]}\PYG{p}{)}
\end{sphinxVerbatim}

\begin{sphinxVerbatim}[commandchars=\\\{\}]
\PYG{g+go}{===== \PYGZus{}\PYGZus{}enter\PYGZus{}\PYGZus{} =====}

\PYG{g+go}{Teste...}

\PYG{g+go}{===== \PYGZus{}\PYGZus{}exit\PYGZus{}\PYGZus{} =====}
\end{sphinxVerbatim}


\chapter{Módulos}
\label{\detokenize{content/modules:modulos}}\label{\detokenize{content/modules::doc}}\begin{quote}

Módulos são as bibliotecas de Python.
São simples arquivos de código Python (.py) e têm que estar dentro do PYTHONPATH.
\end{quote}

\textgreater{} import sys

\textgreater{} dir(sys)
{[}‘\_\_displayhook\_\_’, ‘\_\_doc\_\_’, ‘\_\_excepthook\_\_’, ‘\_\_interactivehook\_\_’, ‘\_\_loader\_\_’, ‘\_\_name\_\_’, ‘\_\_package\_\_’, ‘\_\_spe
{\color{red}\bfseries{}c\_\_}’, ‘\_\_stderr\_\_’, ‘\_\_stdin\_\_’, ‘\_\_stdout\_\_’, ‘\_clear\_type\_cache’, ‘\_current\_frames’, ‘\_debugmallocstats’, ‘\_getframe’
, ‘\_home’, ‘\_mercurial’, ‘\_xoptions’, ‘abiflags’, ‘api\_version’, ‘argv’, ‘base\_exec\_prefix’, ‘base\_prefix’, ‘builtin\_mo
dule\_names’, ‘byteorder’, ‘call\_tracing’, ‘callstats’, ‘copyright’, ‘displayhook’, ‘dont\_write\_bytecode’, ‘exc\_info’, ‘
excepthook’, ‘exec\_prefix’, ‘executable’, ‘exit’, ‘flags’, ‘float\_info’, ‘float\_repr\_style’, ‘getallocatedblocks’, ‘get
checkinterval’, ‘getdefaultencoding’, ‘getdlopenflags’, ‘getfilesystemencoding’, ‘getprofile’, ‘getrecursionlimit’, ‘ge
trefcount’, ‘getsizeof’, ‘getswitchinterval’, ‘gettrace’, ‘hash\_info’, ‘hexversion’, ‘implementation’, ‘int\_info’, ‘int
ern’, ‘last\_traceback’, ‘last\_type’, ‘last\_value’, ‘maxsize’, ‘maxunicode’, ‘meta\_path’, ‘modules’, ‘path’, ‘path\_hooks
‘, ‘path\_importer\_cache’, ‘platform’, ‘prefix’, ‘setcheckinterval’, ‘setdlopenflags’, ‘setprofile’, ‘setrecursionlimit’
, ‘setswitchinterval’, ‘settrace’, ‘stderr’, ‘stdin’, ‘stdout’, ‘thread\_info’, ‘version’, ‘version\_info’, ‘warnoptions’
{]}

\textgreater{} print(sys.path)

{[}‘’, ‘/usr/bin’, ‘/usr/lib/python3.4’, ‘/usr/lib/python3.4/plat\sphinxhyphen{}x86\_64\sphinxhyphen{}linux\sphinxhyphen{}gnu’, ‘/usr/lib/python3.4/lib\sphinxhyphen{}dynload’, ‘/usr/local/lib/python3.4/dist\sphinxhyphen{}packages’, ‘/usr/lib/python3/dist\sphinxhyphen{}packages’{]}
\begin{quote}

A primeira incidência é uma string vazia (‘’), o que significa ser o diretório corrente.
\end{quote}

\$ vim foo.py

\# \_*\_ encoding: utf\sphinxhyphen{}8 \_*\_

print(‘Este arquivo, \{\} é um módulo!’.format(\_\_file\_\_))
print(‘Seu nome é \{\}.’.format(\_\_name\_\_))

\textgreater{} import foo
Este arquivo, /tmp/foo.py é um módulo!
Seu nome é foo.

\$ python3 foo.py
Este arquivo, foo.py é um módulo!
Seu nome é \_\_main\_\_.

\$ vim foo.py

\# \_*\_ encoding: utf\sphinxhyphen{}8 \_*\_
\begin{description}
\item[{if \_\_name\_\_ == ‘\_\_main\_\_’:}] \leavevmode
print(‘Executado a partir da linha de comando.’)

\item[{else:}] \leavevmode
print(‘Módulo importado.’)

\end{description}

\textgreater{} import foo
Módulo importado.

\textgreater{} type(foo)
\textless{}class ‘module’\textgreater{}

\$ python3 foo.py
Executado a partir da linha de comando.

Recarregando um Módulo

\textgreater{} import importlib
\textgreater{} importlib.reload(foo)
Módulo importado.
\textless{}module ‘foo’ from ‘/tmp/foo.py’\textgreater{}

\$ vim foo.py

\# \_*\_ encoding: utf\sphinxhyphen{}8 \_*\_
\begin{description}
\item[{class Pessoa(object):}] \leavevmode
nome = ‘’
idade = 0
\begin{description}
\item[{def saudacao(self):}] \leavevmode
print(‘Olá, meu nome é \{\}’.format(self.nome))

\end{description}

\item[{class Funcionario(Pessoa):}] \leavevmode
matricula = ‘’

\item[{def cubo(x):}] \leavevmode
return x ** 3

\end{description}

Recarregando um módulo

\textgreater{} importlib.reload(foo)
\textless{}module ‘foo’ from ‘/tmp/foo.py’\textgreater{}

\textgreater{} f1 = foo.Funcionario()
\textgreater{} f1.nome = ‘Chiquinho’
\textgreater{} f1.saudacao()
Olá, meu nome é Chiquinho

\textgreater{} print(foo.cubo(7))
343

import as …

\textgreater{} import foo as meu\_modulo
\textgreater{} f1 = meu\_modulo.Funcionario()
\textgreater{} f1.matricula = ‘xyz12345’
\textgreater{} print(f1.matricula)
xyz12345

from módulo import …

\textgreater{} from foo import Funcionario
\textgreater{} f1 = Funcionario()
\textgreater{} f1.nome = ‘Chiquinho’
\textgreater{} f1.saudacao()
Olá, meu nome é Chiquinho


\chapter{Pacotes}
\label{\detokenize{content/packages:pacotes}}\label{\detokenize{content/packages::doc}}
\$ mkdir pack\_0

\$ mv foo.py pack\_0/

\$ \textgreater{} pack\_0/\_\_init\_\_.py

import pacote.modulo

\textgreater{} import pack\_0.foo
\textgreater{} f1 = pack\_0.foo.Funcionario()
\textgreater{} f1.nome = ‘Zezinho’
\textgreater{} f1.saudacao()
Olá, meu nome é Zezinho

\$ ls pack\_0/
foo.py  \_\_init\_\_.py  \_\_pycache\_\_

\$ ls pack\_0/\_\_pycache\_\_/
foo.cpython\sphinxhyphen{}34.pyc  \_\_init\_\_.cpython\sphinxhyphen{}34.pyc

\$ file pack\_0/\_\_pycache\_\_/foo.cpython\sphinxhyphen{}34.pyc
pack\_0/\_\_pycache\_\_/foo.cpython\sphinxhyphen{}34.pyc: python 3.4 byte\sphinxhyphen{}compiled

\$ vim pack\_0/\_\_init\_\_.py

\# \_*\_ encoding: utf\sphinxhyphen{}8 \_*\_
\begin{description}
\item[{if \_\_name\_\_ != ‘\_\_main\_\_’:}] \leavevmode
print(‘Pacote \{\} importado’.format(\_\_name\_\_))

\end{description}

\$ vim pack\_0/foo.py

\# \_*\_ encoding: utf\sphinxhyphen{}8 \_*\_

print(\_\_name\_\_)
\begin{description}
\item[{class Pessoa(object):}] \leavevmode
nome = ‘’
idade = 0
\begin{description}
\item[{def saudacao(self):}] \leavevmode
print(‘Olá, meu nome é \{\}’.format(self.nome))

\end{description}

\item[{class Funcionario(Pessoa):}] \leavevmode
matricula = ‘’

\item[{def cubo(x):}] \leavevmode
return x ** 3

\end{description}

\textgreater{} import pack\_0.foo
Pacote pack\_0 importado
pack\_0.foo

\textgreater{} import pack\_0
Pacote pack\_0 importado

\$ mkdir \sphinxhyphen{}p pack\_0/pack\_1/pack\_2

\$ mv pack\_0/foo.py pack\_0/pack\_1/pack\_2/

\textgreater{} import pack\_0.pack\_1.pack\_2.foo
Pacote pack\_0 importado
pack\_0.pack\_1.pack\_2.foo

\textgreater{} f1 = pack\_0.pack\_1.pack\_2.foo.Funcionario()

Apesar de terem sido declarados, os diretórios pack\_1 e pack\_2 não são pacotes.

Transformando os subdiretórios em pacotes:

\$ \textgreater{} pack\_0/pack\_1/\_\_init\_\_.py
\$ \textgreater{} pack\_0/pack\_1/pack\_2/\_\_init\_\_.py

\textgreater{} from pack\_0.pack\_1.pack\_2 import foo
Pacote pack\_0 importado
pack\_0.pack\_1.pack\_2.foo
\textgreater{} f1 = foo.Funcionario()

\textgreater{} from pack\_0.pack\_1.pack\_2.foo import Funcionario
Pacote pack\_0 importado
pack\_0.pack\_1.pack\_2.foo
\textgreater{} f1 = Funcionario()

\$ echo ‘print(\_\_name\_\_)’ \textgreater{} pack\_0/pack\_1/\_\_init\_\_.py
\$ echo ‘print(\_\_name\_\_)’ \textgreater{} pack\_0/pack\_1/pack\_2/\_\_init\_\_.py

\textgreater{} from pack\_0.pack\_1.pack\_2.foo import Funcionario
Pacote pack\_0 importado
pack\_0.pack\_1
pack\_0.pack\_1.pack\_2
pack\_0.pack\_1.pack\_2.foo

\textgreater{} from pack\_0.pack\_1.pack\_2.foo import *
Pacote pack\_0 importado
pack\_0.pack\_1
pack\_0.pack\_1.pack\_2
pack\_0.pack\_1.pack\_2.foo
\textgreater{} f1 = Funcionario()

\$ python3 pack\_0/\_\_pycache\_\_/foo.cpython\sphinxhyphen{}34.pyc

\_\_main\_\_

Exceções
Exceções
\sphinxstylestrong{****}

\sphinxurl{https://docs.python.org/2/tutorial/errors.html}
\sphinxurl{https://docs.python.org/2/library/exceptions.html}

Blocos de Comandos:

try: Tenta executar o bloco de comandos.

except: Em outras linguagens de programação é conhecido como “catch”, captura e trata a exceção.

else: Só é executado se não tiver nenhuma exceção.

finally: Tendo exceção ou não é executado de qualquer jeito.
\begin{quote}
\begin{quote}
\begin{quote}
\begin{quote}
\begin{description}
\item[{try:}] \leavevmode
\begin{DUlineblock}{0em}
\item[] 
\end{DUlineblock}

\end{description}
\end{quote}

\begin{DUlineblock}{0em}
\item[] Deu erro? |
/ 
\end{DUlineblock}
\end{quote}
\begin{description}
\item[{sim  /   não}] \leavevmode\begin{quote}

/     
\end{quote}

/       

\end{description}
\end{quote}
\begin{description}
\item[{else:         except:}] \leavevmode\begin{description}
\item[{      /}] \leavevmode\begin{description}
\item[{    /}] \leavevmode\begin{description}
\item[{  /}] \leavevmode
/

\end{description}

\end{description}

finally:

\end{description}

\end{description}
\end{quote}


\chapter{raise}
\label{\detokenize{content/except:raise}}\label{\detokenize{content/except::doc}}
The class hierarchy for built\sphinxhyphen{}in exceptions is:
\begin{description}
\item[{BaseException}] \leavevmode
+\textendash{} SystemExit
+\textendash{} KeyboardInterrupt
+\textendash{} GeneratorExit
+\textendash{} Exception
\begin{quote}

+\textendash{} StopIteration
+\textendash{} StandardError
|    +\textendash{} BufferError
|    +\textendash{} ArithmeticError
|    |    +\textendash{} FloatingPointError
|    |    +\textendash{} OverflowError
|    |    +\textendash{} ZeroDivisionError
|    +\textendash{} AssertionError
|    +\textendash{} AttributeError
|    +\textendash{} EnvironmentError
|    |    +\textendash{} IOError
|    |    +\textendash{} OSError
|    |         +\textendash{} WindowsError (Windows)
|    |         +\textendash{} VMSError (VMS)
|    +\textendash{} EOFError
|    +\textendash{} ImportError
|    +\textendash{} LookupError
|    |    +\textendash{} IndexError
|    |    +\textendash{} KeyError
|    +\textendash{} MemoryError
|    +\textendash{} NameError
|    |    +\textendash{} UnboundLocalError
|    +\textendash{} ReferenceError
|    +\textendash{} RuntimeError
|    |    +\textendash{} NotImplementedError
|    +\textendash{} SyntaxError
|    |    +\textendash{} IndentationError
|    |         +\textendash{} TabError
|    +\textendash{} SystemError
|    +\textendash{} TypeError
|    +\textendash{} ValueError
|         +\textendash{} UnicodeError
|              +\textendash{} UnicodeDecodeError
|              +\textendash{} UnicodeEncodeError
|              +\textendash{} UnicodeTranslateError
+\textendash{} Warning
\begin{quote}

+\textendash{} DeprecationWarning
+\textendash{} PendingDeprecationWarning
+\textendash{} RuntimeWarning
+\textendash{} SyntaxWarning
+\textendash{} UserWarning
+\textendash{} FutureWarning
+\textendash{} ImportWarning
+\textendash{} UnicodeWarning
+\textendash{} BytesWarning
\end{quote}
\end{quote}

\end{description}

print(1 / 0)

—\sphinxhyphen{}\textgreater{} 1 print(‘Resultado = \%s’ \% (x / y))

ZeroDivisionError: integer division or modulo by zero
\begin{description}
\item[{try:}] \leavevmode
print(1 / 0)

\end{description}

\# catch all exceptions
except:
\begin{quote}

print(‘nnErro: Não dividirás por zero!nn’)
\end{quote}

Erro: Não dividirás por zero!
\begin{description}
\item[{try:}] \leavevmode
print(1 / 0)

\end{description}

\# specified exception
except ZeroDivisionError:
\begin{quote}

print(‘nnErro: Não dividirás por zero!nn’)
\end{quote}

Erro: Não dividirás por zero!

numeros = (‘zero’, ‘um’, ‘dois’)
\begin{description}
\item[{In {[}10{]}: try:}] \leavevmode
print(numeros{[}0{]})
print(numeros{[}1{]})
print(numeros{[}2{]})
print(numeros{[}3{]})

\item[{except IndexError:}] \leavevmode
print(‘nnERRO: Índice não encontradonn’)

\end{description}

zero
um
dois

ERRO: Índice não encontrado
\begin{description}
\item[{try:}] \leavevmode
print(numeros{[}0{]})
print(numeros{[}1{]})
print(numeros{[}2{]})
print(numeros{[}3{]})

\item[{except IndexError, e:}] \leavevmode\begin{quote}

print(‘nnERRO: Índice não encontradonn\%s’ \% e)
\end{quote}

….:

\end{description}

zero
um
dois

ERRO: Índice não encontrado

tuple index out of range

O “e” é a mensagem de erro “tuple index out of range”

\$ vim excecao.py

\#\_*\_ encoding: utf\sphinxhyphen{}8 \_*\_

import sys

numeros = (‘zero’, ‘um’, ‘dois’)
\begin{description}
\item[{try:}] \leavevmode
print(numeros{[}0{]})
print(numeros{[}1{]})
print(numeros{[}2{]})
print(numeros{[}3{]})

\item[{except IndexError, e:}] \leavevmode
print \textgreater{}\textgreater{} sys.stderr, ‘nnERRO: Índice não encontradonn\%s’ \% e
sys.exit(1)

\end{description}

\$ python excecao.py
zero
um
dois

ERRO: Índice não encontrado

tuple index out of range

\$ echo \$?
1

\$ vim excecao3.py

\#\_*\_ encoding: utf\sphinxhyphen{}8 \_*\_

import sys

numeros = (‘zero’, ‘um’, ‘dois’)
\begin{description}
\item[{try:}] \leavevmode
print(numeros{[}0{]})
print(numeros{[}1{]})
print(numeros{[}2{]})
print(numeros{[}3{]})

\item[{except IndexError as e:}] \leavevmode
print(‘nnERRO: Índice não encontradonn\%s’ \% e, file = sys.stderr)
sys.exit(1)

\end{description}

\$ python3 excecao3.py

zero
um
dois

ERRO: Índice não encontrado

tuple index out of range

\$ echo \$?
1

f = open(‘/tmp/blablabla.txt’, ‘r’)

—\sphinxhyphen{}\textgreater{} 1 f = open(‘/tmp/blablabla.txt’, ‘r’)

IOError: {[}Errno 2{]} No such file or directory: ‘/tmp/blablabla.txt’
\begin{description}
\item[{try:}] \leavevmode
f = open(‘/tmp/blablabla.txt’, ‘r’)

\item[{except IOError as e:}] \leavevmode
print(‘O arquivo não existe!’)

\end{description}

O arquivo não existe!

type(e)
IOError

dir(e)
\begin{description}
\item[{{[}‘\_\_class\_\_’,}] \leavevmode
‘\_\_delattr\_\_’,
‘\_\_dict\_\_’,
‘\_\_doc\_\_’,
‘\_\_format\_\_’,
‘\_\_getattribute\_\_’,
‘\_\_getitem\_\_’,
‘\_\_getslice\_\_’,
‘\_\_hash\_\_’,
‘\_\_init\_\_’,
‘\_\_new\_\_’,
‘\_\_reduce\_\_’,
‘\_\_reduce\_ex\_\_’,
‘\_\_repr\_\_’,
‘\_\_setattr\_\_’,
‘\_\_setstate\_\_’,
‘\_\_sizeof\_\_’,
‘\_\_str\_\_’,
‘\_\_subclasshook\_\_’,
‘\_\_unicode\_\_’,
‘args’,
‘errno’,
‘filename’,
‘message’,
‘strerror’{]}

\end{description}

repr(e)

“IOError(2, ‘No such file or directory’)”

e.\textless{}TAB\textgreater{}
e.args      e.errno     e.filename  e.message   e.strerror

print(e.errno)
2

\$ vim excecao.py

\#\_*\_ encoding: utf\sphinxhyphen{}8 \_*\_

import sys
\begin{description}
\item[{try:}] \leavevmode
f = open(‘/tmp/blablabla.txt’, ‘r’)

\item[{except IOError as e:}] \leavevmode
print \textgreater{}\textgreater{} sys.stderr, ‘nnERRO: Arquivo não encontradonn\%s’ \% e
sys.exit(e.errno)

\end{description}

\$ python excecao.py

ERRO: Arquivo não encontrado

{[}Errno 2{]} No such file or directory: ‘/tmp/blablabla.txt’

\$ echo \$?
2

\$ vim excecao3.py

\#\_*\_ encoding: utf\sphinxhyphen{}8 \_*\_

import sys
\begin{description}
\item[{try:}] \leavevmode
f = open(‘/tmp/blablabla.txt’, ‘r’)

\item[{except IOError as e:}] \leavevmode
print(‘nnERRO: Arquivo não encontradonn\%s’ \% e, file = sys.stderr)
sys.exit(e.errno)

\end{description}

\$ python3 excecao3.py

ERRO: Arquivo não encontrado

{[}Errno 2{]} No such file or directory: ‘/tmp/blablabla.txt’

\$ echo \$?
2

else

\$ vim excecao.py

\#\_*\_ encoding: utf\sphinxhyphen{}8 \_*\_

import sys

f = sys.argv{[}1{]}
\begin{description}
\item[{try:}] \leavevmode
f = open(f, ‘r’)

\item[{except IOError:}] \leavevmode
print(‘Não existe o arquivo’)

\item[{else:}] \leavevmode
print(‘Sim, o arquivo existe!’)
f.close()

\end{description}

\$ python excecao.py /tmp/blablabla.txt
Não existe o arquivo

\$ \textgreater{} /tmp/blablabla.txt

\$ python excecao.py /tmp/blablabla.txt
Sim, o arquivo existe!

\$ rm /tmp/blablabla.txt

finally

\$ vim excecao.py

\#\_*\_ encoding: utf\sphinxhyphen{}8 \_*\_

import sys

f = sys.argv{[}1{]}
\begin{description}
\item[{try:}] \leavevmode
f = open(f, ‘r’)

\item[{except IOError:}] \leavevmode
print(‘Não existe o arquivo’)

\item[{else:}] \leavevmode
print(‘Sim, o arquivo existe!’)
f.close()

\item[{finally:}] \leavevmode
print(‘Se deu certo ou errado, o programa termina aqui :/’)

\end{description}

\$ python excecao.py /tmp/blablabla.txt
Não existe o arquivo
Se deu certo ou errado, o programa termina aqui :/

\$ \textgreater{} /tmp/blablabla.txt

\$ python excecao.py /tmp/blablabla.txt
Sim, o arquivo existe!
Se deu certo ou errado, o programa termina aqui :/

raise


\chapter{raise Exception(‘Provocando uma exceção’)}
\label{\detokenize{content/except:raise-exception-provocando-uma-excecao}}
Exception                                 Traceback (most recent call last)
\textless{}ipython\sphinxhyphen{}input\sphinxhyphen{}62\sphinxhyphen{}69ebfeaaf513\textgreater{} in \textless{}module\textgreater{}()
—\sphinxhyphen{}\textgreater{} 1 raise Exception(‘Provocando uma exceção’)

Exception: Provocando uma exceção
\begin{description}
\item[{try:}] \leavevmode
raise ZeroDivisionError()

\item[{except IOError:}] \leavevmode
print(‘Exceção IOError’)

\item[{except IndexError:}] \leavevmode
print(‘Exceção IndexError’)

\item[{except:}] \leavevmode
print(‘Exceção não declarada’)

\end{description}

Exceção não declarada
\begin{description}
\item[{try:}] \leavevmode
raise ZeroDivisionError(‘Erro de divisão por zero’)

\item[{except IOError:}] \leavevmode
print(‘Exceção IOError’)

\item[{except IndexError:}] \leavevmode
print(‘Exceção IndexError’)

\item[{except ZeroDivisionError:}] \leavevmode
print(‘Exceção ZeroDivisionError’)

\item[{except:}] \leavevmode
print(‘Exceção não declarada’)

\end{description}

Exceção ZeroDivisionError
\begin{description}
\item[{try:}] \leavevmode
raise ZeroDivisionError(‘Erro de divisão por zero’)

\item[{except IOError:}] \leavevmode
print(‘Exceção IOError’)

\item[{except IndexError:}] \leavevmode
print(‘Exceção IndexError’)

\item[{except ZeroDivisionError:}] \leavevmode
print(‘Exceção ZeroDivisionError’)

\item[{except Exception:}] \leavevmode
print(‘Exceção não declarada’)

\end{description}

Exceção ZeroDivisionError
\begin{description}
\item[{try:}] \leavevmode
raise ZeroDivisionError(‘Erro de divisão por zero’)

\item[{except IOError:}] \leavevmode
print(‘Exceção IOError’)

\item[{except IndexError:}] \leavevmode
print(‘Exceção IndexError’)

\item[{except ZeroDivisionError:}] \leavevmode
print(‘Exceção ZeroDivisionError’)
raise

\item[{except Exception:}] \leavevmode
print(‘Exceção não declarada’)

\end{description}


\chapter{Exceção ZeroDivisionError}
\label{\detokenize{content/except:excecao-zerodivisionerror}}
ZeroDivisionError                         Traceback (most recent call last)
\textless{}ipython\sphinxhyphen{}input\sphinxhyphen{}77\sphinxhyphen{}be8b35cbd0cc\textgreater{} in \textless{}module\textgreater{}()
\begin{quote}

1 try:
\end{quote}
\begin{description}
\item[{—\sphinxhyphen{}\textgreater{} 2     raise ZeroDivisionError(‘Erro de divisão por zero’)}] \leavevmode
3 except IOError:
4     print(‘Exceção IOError’)
5 except IndexError:

\end{description}

ZeroDivisionError: Erro de divisão por zero

Exceção Personalizada
\begin{quote}

Para criarmos uma exceção personalizada preciamos criar uma classe que herde de uma outra classe de exceção, cuja raiz é Exception.
\end{quote}
\begin{description}
\item[{class FooException(Exception):}] \leavevmode
pass

\item[{try:}] \leavevmode
raise FooException(‘Bla bla bla’)

\item[{except Exception as e:}] \leavevmode
print(‘Erro ——\textgreater{} \%s’ \% e)

\end{description}

Erro ——\textgreater{} Bla bla bla
\begin{description}
\item[{class EggsException(Exception):}] \leavevmode\begin{description}
\item[{def \_\_init\_\_(self, value):}] \leavevmode
self.value = value

\item[{def \_\_str\_\_(self):}] \leavevmode
return self.value

\end{description}

\item[{try:}] \leavevmode
raise EggsException(50)

\item[{except EggsException as e:}] \leavevmode
print(‘Ocorreu um erro da minha exceção, cujo valor é \%s’ \% (e.value))

\end{description}

Ocorreu um erro da minha exceção, cujo valor é 50

raise EggsException(‘bla bla bla’)

—\sphinxhyphen{}\textgreater{} 1 raise EggsException(‘bla bla bla’)

EggsException: bla bla bla


\chapter{Closures}
\label{\detokenize{content/closures:closures}}\label{\detokenize{content/closures::doc}}\begin{quote}

Funções são objetos também.
Closures são funções que constroem funções, retorna outra função.
Sua estrutura é uma função dentro de outra.
A função interna é também conhecida como função auxiliadora (helper).
\end{quote}

Criação de uma closure:
\begin{description}
\item[{\textgreater{} def funcao\_principal(x):}] \leavevmode\begin{description}
\item[{def funcao\_auxiliadora(y):}] \leavevmode
return x ** y

\end{description}

return funcao\_auxiliadora

\end{description}

Criação de uma variável foo e passando o valor x para a função principal:

\textgreater{} foo = funcao\_principal(2)
\begin{quote}

“foo” é uma nova função cujo o “x” de “função\_principal” é 2.
\end{quote}

Qual é o nome da função?:

\textgreater{} print(foo.\_\_name\_\_)

funcao\_auxiliadora

Tipo de “foo”:

\textgreater{} type(foo)

function

Representação de “foo”:

\textgreater{} repr(foo)

‘\textless{}function funcao\_principal.\textless{}locals\textgreater{}.funcao\_auxiliadora at 0x7f9845369950\textgreater{}’

Imprimindo o valor resultante ao passar agora o valor y:

\textgreater{} print(foo(6))

64
\begin{quote}

A operação realizada foi 2 elevado a 6 (x ** y).
\end{quote}

Podemos também chamar a função principal passando o parâmetro da função auxiliar:

\textgreater{} funcao\_principal(5)(2)

25

Closures com Lambda

Criação de uma closure com lambda:
\begin{description}
\item[{\textgreater{} def funcao\_principal(x):}] \leavevmode
return lambda y: x ** y

\end{description}

O “x” será 3:

\textgreater{} bar = funcao\_principal(3)

Exibindo o nome do objeto:

\textgreater{} print(bar.\_\_name\_\_)

\textless{}lambda\textgreater{}

Tipo:

\textgreater{} type(bar)

function

Representação:

\textgreater{} repr(bar)

‘\textless{}function funcao\_principal.\textless{}locals\textgreater{}.\textless{}lambda\textgreater{} at 0x7f9844527730\textgreater{}’

3 elevado a 2:

\textgreater{} print(bar(2))

9

Passando o parâmetro da função principal e de lambda:

\textgreater{} funcao\_principal(2)(5)

32


\chapter{Decoradores em Python}
\label{\detokenize{content/decorators:decoradores-em-python}}\label{\detokenize{content/decorators::doc}}\begin{quote}

É um conceito diferente do conceito decorator de design pattern.
Pode ser implementado como uma classe ou como uma função.
Modificam funções.
É identificado da seguinte forma: @nome\_do\_decorador.
Deve ser inserido na linha anterior da definição.
Um decorador deve ser executável, ou seja, deve ter o método \_\_call\_\_().
São envelopes de função.
Decoradores podem ser definidos como classes ou funções.
\end{quote}

Decoradores de Funções

Classe como decorador:

\textgreater{} class Decorador(object):
\begin{quote}
\begin{description}
\item[{def \_\_init\_\_(self, funcao):}] \leavevmode
‘’’
O parâmetro “funcao” é a função que será decorada.
‘’’

print(‘Método \_\_init\_\_() do decorador’)  \# Mensagem ao instanciar
funcao()  \# Executa a função ao instanciar

\item[{def \_\_call\_\_(self):}] \leavevmode
print(‘Método \_\_call\_\_() do decorador’)

\end{description}
\end{quote}

Definição da classe com decorador:

\textgreater{} @Decorador
def funcao\_decorada():
\begin{quote}

print(‘Dentro da função decorada’)
\end{quote}

Método \_\_init\_\_() do decorador
Dentro da função decorada
\begin{quote}

O método construtor do decorador é invocado logo após da definição da função decorada.
No exemplo, a própria função, que é usada como parâmetro, é invocada pelo método construtor \_\_init\_\_().
\end{quote}

Execução da função decorada:

\textgreater{} funcao\_decorada()

Método \_\_call\_\_() do decorador

Recriação da classe decoradora:
\begin{description}
\item[{\textgreater{} class Decorador(object):}] \leavevmode\begin{quote}
\begin{description}
\item[{def \_\_init\_\_(self, funcao):}] \leavevmode
print(‘Método \_\_init\_\_() do decorador’)
self.funcao = funcao

\item[{def \_\_call\_\_(self):}] \leavevmode
print(‘Método \_\_call\_\_() do decorador’)
self.funcao()

\end{description}
\end{quote}

Diferente da criação da classe anteriormente, aqui a função não é chamada no método construtor, mas sim no método \_\_call\_\_.

\end{description}

Definição da função decorada:

\textgreater{} @Decorador
def funcao\_decorada():
\begin{quote}

print(‘Dentro da função’)
\end{quote}

Método \_\_init\_\_() do decorador

Execução da função decorada:

\textgreater{} funcao\_decorada()

Método \_\_call\_\_() do decorador
Dentro da função

Decorador com Argumentos
\begin{quote}

Para funções decoradas que têm argumentos.
\end{quote}

Criação da classe decoradora:

\textgreater{} class Decorador():
\begin{quote}
\begin{description}
\item[{def \_\_init\_\_(self, f):}] \leavevmode
print(‘Método \_\_init\_\_() do decorador’)
self.f = f

\item[{def \_\_call\_\_(self, {\color{red}\bfseries{}*}args, {\color{red}\bfseries{}**}kargs):}] \leavevmode
print(‘Método \_\_call\_\_() do decorador’)
return self.f({\color{red}\bfseries{}*}args, {\color{red}\bfseries{}**}kargs)

\end{description}
\end{quote}

Definição da classe decorada:

\textgreater{} @Decorador
def soma(x, y):
\begin{quote}

return x + y
\end{quote}

Execução da classe decorada:

\textgreater{} print(soma(2, 5))

Método \_\_call\_\_() do decorador
7

Função como decorador
\begin{quote}

Até então foram vistos exemplos de decoradores definidos como classes, agora serão vistos os definidos como função.
\end{quote}

Criação da função decoradora, o decorador em si:

\textgreater{} def funcao\_decoradora(funcao):
\begin{quote}

\# Função auxiliar
def funcao\_auxiliar():
\begin{quote}

print(‘Antes da função decorada’)
funcao()
print(‘Depois da função decorada’)
\end{quote}

return funcao\_auxiliar
\end{quote}

Definição da função decorada:

\textgreater{} @funcao\_decoradora
def foo():
\begin{quote}

print(‘Função decorada’)
\end{quote}

Execução da função decorada:

\textgreater{} foo()

Antes da função decorada
Função decorada
Depois da função decorada

Podemos também aplicar mais de um decorador a uma função:

\textgreater{} \# Função para itálico em HTML
def to\_italic(funcao):
\begin{quote}
\begin{description}
\item[{def funcao\_auxiliadora({\color{red}\bfseries{}*}args, {\color{red}\bfseries{}**}kargs):}] \leavevmode
return ‘\textless{}i\textgreater{}’ + funcao({\color{red}\bfseries{}*}args, {\color{red}\bfseries{}**}kargs) + ‘\textless{}/i\textgreater{}’

\end{description}

return funcao\_auxiliadora
\end{quote}

\# Função para negrito em HTML
def to\_bold(funcao):
\begin{quote}
\begin{description}
\item[{def funcao\_auxiliadora({\color{red}\bfseries{}*}args, {\color{red}\bfseries{}**}kargs):}] \leavevmode
return ‘\textless{}b\textgreater{}’ + funcao({\color{red}\bfseries{}*}args, {\color{red}\bfseries{}**}kargs) + ‘\textless{}/b\textgreater{}’

\end{description}

return funcao\_auxiliadora
\end{quote}

Definição da função decorada com dois decoradores:

\textgreater{} @to\_italic
@to\_bold
def fc\_msg(msg):
\begin{quote}

return msg
\end{quote}

Execução da função decorada:

\textgreater{} fc\_msg(‘Hello, World!’)

‘\textless{}i\textgreater{}\textless{}b\textgreater{}Hello, World!\textless{}/b\textgreater{}\textless{}/i\textgreater{}’

Memoização
\begin{quote}

Texto…
\end{quote}

Criação do decorador de memoização:
\begin{description}
\item[{\textgreater{} def memoize(f):}] \leavevmode
‘’’
Memoize function (as decorator)
‘’’

\# dictionary (cache)
mem = \{\}

‘’’ Helper function ‘’’
def memoizer({\color{red}\bfseries{}*}param):
\begin{quote}

key = repr(param)
if not key in mem:
\begin{quote}

mem{[}key{]} = f({\color{red}\bfseries{}*}param)
\end{quote}

return mem{[}key{]}
\end{quote}

return memoizer

\end{description}

Criação da função decorada com memoização aplicada para Fibonacci:

\textgreater{} @memoize
def fibo(n):
\begin{quote}

if (n \textless{} 2): return n
else:
\begin{quote}

return fibo(n \sphinxhyphen{} 1) + fibo(n \sphinxhyphen{} 2)
\end{quote}
\end{quote}

Execução da função decorada:

\textgreater{} fibo(39)

63245986


\chapter{Expressões vs Statements}
\label{\detokenize{content/exec_eval_execfile_compile:expressoes-vs-statements}}\label{\detokenize{content/exec_eval_execfile_compile::doc}}\begin{quote}

Expressão: é algo que pode ser reduzido para um valor.

Statement: sinônimo de comando.
\end{quote}

exec
\begin{quote}

É uma função que executa um comando (statement) passado como string.
\end{quote}

bla bla bla:

\textgreater{} exec(‘print(5)’)  \# prints 5.

5

\textgreater{} exec(‘print(5)nprint(6)’)  \# prints 5\{newline\}6.

5
6

\textgreater{} exec(‘if True: print(6)’)  \# prints 6.

6

\textgreater{} exec(‘5’)  \# does nothing and returns nothing.

\textgreater{} exec(‘x = 7’)  \# does nothing and returns nothing.

eval
\begin{quote}

É uma função que retorna o resultado de uma expressão.
\end{quote}

bla bla bla:

\textgreater{} x = eval(‘5’)  \# x \textless{}\sphinxhyphen{} 5

\textgreater{} x = eval(‘\%d + 6’ \% x)  \# x \textless{}\sphinxhyphen{} 11

\textgreater{} x = eval(‘abs(\%d)’ \% \sphinxhyphen{}100)  \# x \textless{}\sphinxhyphen{} 100

\textgreater{} x = eval(‘x = 5’)  \# INVÁLIDO; atribuição não é uma expressão

SyntaxError: invalid syntax

\textgreater{} x = eval(‘if 1: x = 4’)  \# INVÁLIDO; if é um statement e não uma expressão

SyntaxError: invalid syntax

\# \sphinxurl{https://nedbatchelder.com/blog/201206/eval\_really\_is\_dangerous.html}

bla bla bla:

\textgreater{} execfile(file)

bla bla bla:

\$ cat \textless{}\textless{} EOF \textgreater{} /tmp/foo.py
print(‘hello, world!’)
EOF

bla bla bla:

\$ python /tmp/foo.py
hello, world!

bla bla bla:

\textgreater{} execfile(‘/tmp/foo.py’)

hello, world!

compile

compile is a lower level version of exec and eval. It does not execute or evaluate your statements or expressions, but returns a code object that can do it. The modes are as follows:
\begin{quote}

compile(string, ‘’, ‘eval’) returns the code object that would have been executed had you done eval(string). Note that you cannot use statements in this mode; only a (single) expression is valid.
compile(string, ‘’, ‘exec’) returns the code object that would have been executed had you done exec(string). You can use any number of statements here.
compile(string, ‘’, ‘single’) is like the exec mode, but it will ignore everything except for the first statement. Note that an if/else statement with its results is considered a single statement.
\end{quote}

\sphinxurl{http://sahandsaba.com/thirty-python-language-features-and-tricks-you-may-not-know.html}

python3.7 \sphinxhyphen{}mtimeit \sphinxhyphen{}s ‘code = “a = 2; b = 3; c = a * b”’ ‘exec(code)’

python3.7 \sphinxhyphen{}mtimeit \sphinxhyphen{}s ‘code = compile(“a = 2; b = 3; c = a * b”, “\textless{}string\textgreater{}”, “exec”)’ ‘exec(code)’


\chapter{Recursividade}
\label{\detokenize{content/recursive:recursividade}}\label{\detokenize{content/recursive::doc}}\begin{quote}

De uma forma geral, recursividade é uma palavra cujo significado é a capacidade de um objeto fazer uma chamada a si mesmo em profundidade.
Como exemplo podemos citar aqui um espelho de frente para outro.
A palavra recursividade tem como origem o latim, do verbo “recurrere”, cujo significado é algo como correr de volta. Essa característica de correr de volta é algo similar a um bumerangue, em que, obedecendo uma condição, ao chegar no fundo, esse “bumerangue” volta com o resultado desejado.
Em linguagem de programação, recursividade é a capacidade de uma função fazer uma chamada a si mesma dentro de sua definição. Uma função recursiva precisa de uma condição de término para não rodar em um loop infinito, o que causaria efeitos indesejados.
\end{quote}

\# \_*\_ coding: utf\sphinxhyphen{}8 \_*\_

\# Fatoração
\begin{description}
\item[{def fatorial(n):}] \leavevmode\begin{description}
\item[{if (n == 1) or (n == 0):}] \leavevmode
return 1

\item[{else:}] \leavevmode
return n * fatorial(n \sphinxhyphen{} 1)

\end{description}

\end{description}

‘’’

Ida:

n = 5 \sphinxhyphen{}\textgreater{} 5 * fatorial(4)

n = 4 \sphinxhyphen{}\textgreater{} 4 * fatorial(3)

n = 3 \sphinxhyphen{}\textgreater{} 3 * fatorial(2)

n = 2 \sphinxhyphen{}\textgreater{} 2 * fatorial(1)

Volta:

fatorial(1) = 1

fatorial(2) = 2 * 1 = 2

fatorial(3) = 3 * 2 = 6

fatorial(4) = 4 * 6 = 24

fatorial(5) = 5 * 24 = 120

‘’’

print(fatorial(5))

\# Fibonacci
\begin{description}
\item[{def fibo(n):}] \leavevmode\begin{description}
\item[{if (n \textless{} 2):}] \leavevmode
return n

\item[{else:}] \leavevmode
return fibo(n \sphinxhyphen{} 1) + fibo(n \sphinxhyphen{} 2)

\end{description}

\end{description}

‘’’

Sequência de Fibonacci: | 0| 1| 1| 2| 3| 5| 8| 13| 21| 34| 55| …
Elemento (n):           | 0| 1| 2| 3| 4| 5| 6| 7 | 8 | 9 | 10| …

Ida:

n = 7 \sphinxhyphen{}\textgreater{} fib(6) + fib(5)
n = 6 \sphinxhyphen{}\textgreater{} fib(5) + fib(4)
n = 5 \sphinxhyphen{}\textgreater{} fib(4) + fib(3)
n = 4 \sphinxhyphen{}\textgreater{} fib(3) + fib(2)
n = 3 \sphinxhyphen{}\textgreater{} fib(2) + fib(1)
n = 2 \sphinxhyphen{}\textgreater{} fib(1) + fib(0)
n = 1 \sphinxhyphen{}\textgreater{} fib(0)

Volta:

fib(0) = 0

fib

‘’’

print(fib(7))

Memoização
\begin{quote}

É uma técnica de otimização usada principalmente para acelerar aplicativos pelo armazenamento de resultados de chamadas de função que têm custo alto de processamento e retornando o resultado do cache quando as mesmas entradas acontecerem novamente.
\end{quote}

Para testarmos vamos criar o arquivo memoizacao.py com o seguinte conteúdo:

\#\_*\_ encoding: utf\sphinxhyphen{}8 \_*\_

import time

‘’’ Fibonacci function ‘’’
def fibo(n):
\begin{quote}

if (n \textless{} 2): return n
else:
\begin{quote}

return fibo(n \sphinxhyphen{} 1) + fibo(n \sphinxhyphen{} 2)
\end{quote}
\end{quote}

‘’’ Memoize function ‘’’
def memoize(f):
\begin{quote}

\# dictionary (cache)
mem = \{\}

‘’’ Helper function ‘’’
def memoizer({\color{red}\bfseries{}*}param):
\begin{quote}

key = repr(param)
if not key in mem:
\begin{quote}

mem{[}key{]} = f({\color{red}\bfseries{}*}param)
\end{quote}

return mem{[}key{]}
\end{quote}

return memoizer
\end{quote}

\# Start time
t1 = time.time()

\# Loop
for i in range(35):
\begin{quote}

print(‘fib(\%s) = \%s’ \% (i, fibo(i)))
\end{quote}

\# End time
t2 = time.time()

\# Total time
print(‘Tempo de execução: \%.3fs’ \% (t2 \sphinxhyphen{} t1))

\# Take a pause
raw\_input(‘Pressione \textless{}ENTER\textgreater{} para continuarn’)

\# Memoization of fibo
fibo = memoize(fibo)

\# Start time
t1 = time.time()

\# loop after memoization
for i in range(40):
\begin{quote}

print(‘fib(\%s) = \%s’ \% (i, fibo(i)))
\end{quote}

\# End time
t2 = time.time()

\# Total time
print(‘Tempo de execução: \%.3fs’ \% (t2 \sphinxhyphen{} t1))

Execute da seguinte forma:

\$ python memoizacao.py

Antes da memoização:

Tempo de execução: 5.107s

Depois da memoização:

Tempo de execução: 0.001s

Podemos ver na prática como é eficiente e como agiliza na execução.


\chapter{Threads}
\label{\detokenize{content/threads:threads}}\label{\detokenize{content/threads::doc}}
import threading
import time
\begin{description}
\item[{def funcao1():}] \leavevmode
linha = (‘\_’ * 79)
while (True):
\begin{quote}

print(linha)
time.sleep(10)
\end{quote}

\item[{def funcao2(tempo):}] \leavevmode\begin{description}
\item[{while (True):}] \leavevmode
print(‘funcao2’)
time.sleep(tempo)

\end{description}

\end{description}

t1 = threading.Thread(target=funcao1, name=’primeira\_thread’)

t2 = threading.Thread(target=funcao2, name=’segunda\_thread’, args=(3,))

t1.start()

t2.start()


\section{funcao2}
\label{\detokenize{content/threads:funcao2}}
…

\textgreater{} t1.getName()
‘primeira\_thread’

E se quisermos parar uma thread?

import threading
import os
import time
\begin{description}
\item[{class FileChecker(threading.Thread):}] \leavevmode\begin{description}
\item[{def \_\_init\_\_(self, file\_path):}] \leavevmode
super(FileChecker, self).\_\_init\_\_()
self.\_kill = False
self.\_file\_path = file\_path

\item[{def run(self):}] \leavevmode\begin{description}
\item[{while (not os.path.isfile(self.\_file\_path)):}] \leavevmode
print(“File \{\} not found!”.format(self.\_file\_path))
time.sleep(5)
if self.\_kill:
\begin{quote}

break
\end{quote}

\end{description}

\item[{def stop(self):}] \leavevmode
self.\_kill = True
self.join()

\end{description}

\end{description}

t = FileChecker(‘/tmp/teste.txt’)
t.start()

time.sleep(10)

t.stop()

print(‘fim’)


\chapter{pytest}
\label{\detokenize{content/pytest:pytest}}\label{\detokenize{content/pytest::doc}}
\$ vim /tmp/test\_pytest.py

\#=============================================================================

\# Função fatorial
def fatorial(x):
\begin{quote}
\begin{description}
\item[{if x == 0:}] \leavevmode
return 1

\end{description}

return x * fatorial(x \sphinxhyphen{} 1)
\end{quote}

\# Função que testa se um número é par
def e\_par(x):
\begin{quote}
\begin{description}
\item[{if x \% 2 == 0:}] \leavevmode
return True

\end{description}

return False
\end{quote}

\# Função para testar a função fatorial
def test\_fatorial():
\begin{quote}

assert fatorial(0) == 1
\end{quote}

\# Função para testar a função e\_par
def test\_par():
\begin{quote}

assert e\_par(8)
\end{quote}

\#=============================================================================

\$ py.test /tmp/test\_pytest.py

================================================= test session starts =================================================
platform linux \textendash{} Python 3.4.3+, pytest\sphinxhyphen{}2.8.7, py\sphinxhyphen{}1.4.31, pluggy\sphinxhyphen{}0.3.1
rootdir: /tmp, inifile:
collected 2 items

../../tmp/test\_pytest.py .F

====================================================== FAILURES =======================================================
\_\_\_\_\_\_\_\_\_\_\_\_\_\_\_\_\_\_\_\_\_\_\_\_\_\_\_\_\_\_\_\_\_\_\_\_\_\_\_\_\_\_\_\_\_\_\_\_\_\_\_\_\_\_ test\_par \_\_\_\_\_\_\_\_\_\_\_\_\_\_\_\_\_\_\_\_\_\_\_\_\_\_\_\_\_\_\_\_\_\_\_\_\_\_\_\_\_\_\_\_\_\_\_\_\_\_\_\_\_\_\_
\begin{quote}

def test\_par():
\end{quote}

\textgreater{}       assert e\_par(8)
E       assert e\_par(8)

/tmp/test\_pytest.py:19: AssertionError
========================================= 1 failed, 1 passed in 0.01 seconds ==========================================

\$ vim /tmp/test\_pytest.py

\#=============================================================================

\# Função fatorial
def fatorial(x):
\begin{quote}
\begin{description}
\item[{if x == 0:}] \leavevmode
return 1

\end{description}

return x * fatorial(x \sphinxhyphen{} 1)
\end{quote}

\# Função que testa se um número é par
def e\_par(x):
\begin{quote}
\begin{description}
\item[{if x \% 2 == 0:}] \leavevmode
return True

\end{description}

return False
\end{quote}

\# Função para testar a função fatorial
def test\_fatorial():
\begin{quote}

assert fatorial(0) == 1
\end{quote}

\# Função para testar a função e\_par
def test\_par():
\begin{quote}

assert e\_par(8)
\end{quote}

\#=============================================================================

\$ py.test /tmp/test\_pytest.py

================================================= test session starts =================================================
platform linux \textendash{} Python 3.4.3+, pytest\sphinxhyphen{}2.8.7, py\sphinxhyphen{}1.4.31, pluggy\sphinxhyphen{}0.3.1
rootdir: /tmp, inifile:
collected 2 items

../../tmp/test\_pytest.py ..

============================================== 2 passed in 0.00 seconds ===============================================


\chapter{Indices and tables}
\label{\detokenize{index:indices-and-tables}}\begin{itemize}
\item {} 
\DUrole{xref,std,std-ref}{genindex}

\item {} 
\DUrole{xref,std,std-ref}{modindex}

\item {} 
\DUrole{xref,std,std-ref}{search}

\end{itemize}



\renewcommand{\indexname}{Índice}
\printindex
\end{document}